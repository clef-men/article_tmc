\begin{acks}
  This work was initiated by Frédéric Bour who implemented TMC in an
  experimental variant of the OCaml compiler and submitted his work
  for inclusion in 2015, but the proposal remained stalled due to lack
  of review and integration effort among maintainers. The broad lines
  of the final implementation, in term of generated code, were already
  in Frédéric's initial version, as well as the idea to have an opt-in
  transformation controlled by an attribute
  (\cref{subsec:optin}). This experimental also generated the
  performance data that motivated us to push further. (One notable
  technical difference is that instead of each destination-passing
  style taking two parameters \ocaml|dst| and |ofs|, it would generate
  several versions specializing the \ocaml|ofs| variable to
  a constant. Frédéric found that this did not noticeably improve
  performance, and changed back to a parameter to simplify the
  implementation and reduce code size.)
  
  In 2020, Gabriel Scherer restarted Frédéric Bour's effort with
  a review followed by a partial reimplementation of the
  transformation that introduced the applicative style discussed in
  \cref{subsec:applicative-impl}, and much of the current
  attribute-based user interface to control the transformation
  (\cref{subsec:user-control}). Basile Clément in turn reviewed
  Gabriel's version, and introduced constructor compression
  (\cref{subsec:constructor-compression}). Xavier Leroy implemented
  a change to the \OCaml calling convention to remove parameter-number
  restrictions on tail calls on some architecture
  (\cref{subsec:PR-history}). The work was finally reviewed by Pierre
  Chambart and merged in the upstream \OCaml compiler in November
  2021.

  Clément Allain started working on a mechanized soundness proof for
  the TMC transformation in Iris in Summer 2022, as a master
  internship supervised by François Pottier. They discovered that the
  question of defining simulations in Iris is surprisingly
  interesting, and Clément wrote the bulk of the correctness proof at
  this point. The mechanization work continued over 2023, and the
  research paper itself was written by Clément Allain and Gabriel
  Scherer in Summer-Winter 2023.

  % TODO: grateful for review comments of François Pottier, Anton
  % Lorenzen, and anonymous reviewers.
\end{acks}

%%% Local Variables:
%%% mode: latex
%%% TeX-master: "main"
%%% End:
