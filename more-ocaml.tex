\section{More on TMC in \OCaml}
\label{app:more-ocaml}

\subsection{Alternatives}
\label{subsec:alternative-impls}

We mention other approaches than TMC to solve the problem of overflowing the call stack, and explain why they were less suitable for \OCaml. We also discuss a significant implementation change between \OCaml 4 and \OCaml 5.

\subsubsection{Doing a CPS transformation}

Instead of a program transformation in \emph{destination-passing} style, we could perform a more general program transformation that can make more functions tail-recursive, for example a generic \emph{continuation-passing} style (CPS) transformation.

We have three arguments for implementing the TMC transformation:
\begin{itemize}
\item The TMC transformation generates more efficient code, using mutation instead of function calls. On the OCaml runtime, the difference is a large constant factor.\footnote{On a toy benchmark with large-sized lists, the CPS version is 100\% slower and has 130\% more allocations than the non-tail-recursive version.}

\item The CPS transformation can be expressed at the source level, and can be made reasonably nice-looking using some monadic-binding syntactic sugar. TMC can only be done by the compiler, or using safety-breaking features.
\end{itemize}

\subsubsection{Lazy data}

Lazy (call-by-need) languages will also often avoid running into stack overflows: as soon as a lazy datastructure is returned, which is the default, functions such as \ocaml{map} will return immediately, with recursive calls frozen in a lazy thunk, waiting to be evaluated on-demand as the user traverses the result structure.
User still need to worry about tail-recursivity for their strict functions; strict functions are often preferred when writing performant code.

\subsubsection{Unlimiting the system stack}

Some operating systems can provide an unlimited system stack; such as \texttt{ulimit -s unlimited} on Linux systems -- the system stack is then resized on-demand.
Frustratingly, unlimited stacks are not available on all systems, and not the default on any system in wide use.
Convincing all users to setup their system in a non-standard way would be \emph{much} harder than performing a program transformation or accepting the CPS overhead for some programs.

\subsubsection{Using another stack}

Using the native system stack is a choice of the OCaml 4 implementation.
Some other implementations of functional languages, such as SML/NJ, use a different stack (the OCaml bytecode interpreter does this), or directly allocate stack frames on their GC-managed heap.
This approach can make ``stack overflow'' go away completely, and it also makes it very simple to implement stack-capture control operators, such as continuations, or other stack operations such as continuation marks.

On the other hand, using the native stack brings compatibility benefits (coherent stack traces for mixed OCaml+C programs), and seems to improve the performance of function calls (on benchmarks that are only testing function calls and return, such as Ackermann or the naive Fibonacci, OCaml can be 4x, 5x faster than SML/NJ.)

\paragraph{OCaml 5}

OCaml 5.0.0 was released in December 2022, about a year after we landed our TMC implementation in the OCaml 4 compiler.
OCaml 5 uses a different runtime to support multicore programming, and a different calling convention to support algebraic effects.
In particular, it only uses the system stack for C calls, and a ``cactus stack'' for OCaml calls.

OCaml 5 manages its own stack, but the implementors still decided to have a stack limit.
It is sensibly higher by default (1Gio at the time of writing) than the system stack (8Mio), so many ``small'' can now use non-tail-recursive functions without fears of stack overflows, but overflows remain possible and likely in practice on large inputs.
The reason to keep a (user-settable) limit for the OCaml call stack is usability in the face of buggy programs.
When programmers write recursive functions, they occasionally go through incorrect versions with a faulty base case that fail to terminate.
In this case, we want the system to fail quickly with an error, instead of remaining silent for a few minutes, crashing once all the machine memory has been consumed.

We were curious about whether users would still see a need for TMC in OCaml 5, or whether they would stop using the feature in practice.
The feedback we got from expert users is that stack overflows is still an issue they worry about.
We observe that TMC adoption in OCaml codebases keep growing after the transition to OCaml 5.

\subsection{Implementation history}
\label{subsec:PR-history}

We first proposed adding TMC as an optional program transformation to
the OCaml compiler in
May 2015.\footnote{\nonanon{URL}{\url{https://github.com/ocaml/ocaml/pull/181}}}
The proposal was received favorably, but it never received an in-depth
review and a detailed performance evaluation and remained unmerged for
years.

We restarted the implementation in 2020, resulting in a new pull July 2020 request with a modified implementation, and a careful performance
evaluation\footnote{\nonanon{URL}{\url{https://github.com/ocaml/ocaml/pull/9760}}}. The
new implementation put a larger focus on producing readable code,
giving more control to the user through annotations. It also removed
an optimization of the previous implementation that would specialize
the DPS version, generating a distinct definition for each block
offset. The new design was carefully reviewed by Basile Clément and
Pierre Chambart, and was finally merged in November 2021, available to
users with OCaml 4.14, released in March 2022.

The TMC transformation is that it adds extra parameters to functions
(two parameters, the block and the offset). At the time the OCaml
compiler had a limitation on some supported architectures, where it
would not optimze tail calls above a certain number of arguments
(enough that they cannot be all passed by registers), breaking the
tail-call promises of TMC on those systems. Xavier Leroy implemented
a change to the OCaml calling convention for those architectures in
May 2021,\footnote{\url{https://github.com/ocaml/ocaml/pull/10595}},
which was motivated by the TMC work.

\subsection{Implementation: an applicative functor}
\label{subsec:applicative-impl}

The computation of choices described in \cref{subsec:implementation} can be expressed elegantly. Instead of a monomorphic type \ocaml|Choice.t| that represents a choice of how to transform a term, we use a polymorphic type \ocaml|'a Choice.t| that represents how to transform zero, one or several subterms that occur in the source; for example transforming two subterms in the same context relies on a choice of type \ocaml|(lambda * lambda) Choice.t|, where \ocaml|lambda| is the type of terms in the intermediate representation we are working on. This \ocaml|'a Choice.t| type can be equipped with an applicative functor interface:

\begin{minipage}{0.4\linewidth}
\begin{Ocaml}
module Choice : sig
  type 'a t = {
    dps : 'a Dps.t;
    direct : unit -> 'a;
    tmc_calls : tmc_call_info list;
    benefits_from_dps : bool;
    explicit_tailcall_request : bool;
  }





end
\end{Ocaml}
\end{minipage}
\hfill
\begin{minipage}{0.5\linewidth}
\begin{Ocaml}
  (* construct a choice from an arbitrary term *)
  val lambda : lambda -> lambda t

  (* applicative functor interface *)
  val unit : unit t
  val map : ('a -> 'b) -> 'a t -> 'b t
  val pair : 'a t * 'b t -> ('a * 'b) t

  (* extract the direct-style and DPS transforms *)
  val direct : lambda t -> lambda
  val dps :
    lambda t -> tail:bool -> dst:offset dst ->
    lambda
\end{Ocaml}
\end{minipage}

With this interface, the easy cases of the transformation can be expressed compactly:
\begin{Ocaml}
  let rec choice ctx ~tail t =
    match t with
    [...]
    | Lifthenelse (l1, l2, l3) ->
        let l1 = traverse ctx l1 in
        let+ l2 = choice ctx ~tail l2
        and+ l3 = choice ctx ~tail l3
        in Lifthenelse (l1, l2, l3)
    [...]
\end{Ocaml}
The binding operators \ocaml|let+|, \ocaml|and+| are standard syntax for applicative-like structures in OCaml: \ocaml|let+| desugars to \ocaml|Choice.map| and \ocaml|and+| desugars to \ocaml|Choice.pair|. \ocaml|traverse| performs a direct-style translation, and \ocaml|traverse_letrec| may also transform \ocaml|let|-bound functions marked with \ocaml|[@tail_mod_constr]| and add them to the local transformation environment.

\subsection{Evaluation: adoption}
\label{subsec:adoption}

% [@tail_mod_cons] usage on Github

% Search URL:
%   https://github.com/search?q=tail_mod_cons+lang%3Aocaml&type=code

% Results (November 13th 2023):
%   - in the OCaml stdlib
%     + Melange
%       (reused from stdlib)
%     + janestreet/base
%       (with manual unfolding)
%   - in other general utility modules
%     https://github.com/RedPRL/asai/blob/ac523674579772e5929c8d912d407b8b7db89c74/src/Utils.ml#L33
%     https://github.com/jonathan-laurent/KaTie/blob/2db0d2cf223fae003b26bdf4c82a9788b5804bc7/src/utils.ml#L95
%     https://github.com/jamsidedown/ocaml-cll/blob/394effa65b5d3f883e2de8322ed60fb7aa495bec/examples/crab_cups.ml#L61
%   - in other user code
%     at list type:
%       https://github.com/brendanzab/language-garden/blob/9cc21e2f57228556cf0b867ca2874ceb4a4250ec/compile-arith/lib/StackLang.ml#L63
%       https://github.com/abella-prover/abella/blob/ceee13822001137898ca5de9c76a315da3d1f0e3/src/extensions.ml#L421
%       https://github.com/bikallem/spring/blob/6d10fef0b21caea3f975ff13132f5994e7c37db5/lib_spring/response.ml#L99
%       https://github.com/bikallem/spring/blob/6d10fef0b21caea3f975ff13132f5994e7c37db5/lib_spring/headers.ml#L150
%       https://github.com/polytypic/io/blob/dbe4d37a98ef958178fe18bb4dae396a5537392e/src/io/io.ml#L8
%       https://github.com/avsm/eeww/blob/5b6c0617306f4c9d8b44bf07ef4d45ec9668dd0c/lib/cohttp/cohttp-eio/src/rwer.ml#L43
%       https://github.com/just-max/less-power/blob/96f8e88f72275246b6bc175e44c732aa6e08ce7f/src/common/path_util.ml#L18
%       https://github.com/dalps/ocaml-challenge/blob/8fe115023457b6b4359ef736c78ba980cd871717/3/extract/solve.ml#L4
%       https://github.com/Octachron/ocaml-changelog-analyzer/blob/d6757b9576b5070ea3cfe6d4f4f79aaa3cc6e5d4/parse/release.ml#L4
%       https://github.com/Skyb0rg007/PL-Reading-Group/blob/7002ae35ba69fb9010e5049eee88ab475bc79ec3/list-optimizations/lib/adt.ml#L44
%       https://github.com/jfeser/symetric/blob/f49a57afc90a2c1e38f1097f98bd74725c074b9b/lib/tower.ml#L75
%       https://github.com/verse-lab/sisyphus/blob/a08addae06bbccb1e6ea68bac3d7f9eeea4197be/lib/dynamic/utils.ml#L65
%       https://github.com/jaymody/ocaml-tokenizers/blob/06cbce75d2865690cb39017222efff5c284bc721/lib/utils.ml#L18
%       https://github.com/Bannerets/ocaml-kdl/blob/687c404ebeceef7fd905935c23665294133f99e6/src/lens.ml#L83
%       https://github.com/OCADml/OCADml/blob/f5dfbf55f9c19ad808daa0df4d76a55e58756e5f/lib/poly3.ml#L33
%     other type
%       https://github.com/johnridesabike/acutis/blob/4113fb516d9c5dd6f2dbfd9657f52c5ae7a5dcae/lib/matching.ml#L161
%         (on-demand stream as a record of functions)
%       https://github.com/RedPRL/ocaml-bwd/blob/fbf496b29532085b38073eaa62ba3d22ac619d5d/src/BwdNoLabels.ml#L60
%         (snoc list)
%       https://github.com/Skyb0rg007/PL-Reading-Group/blob/7002ae35ba69fb9010e5049eee88ab475bc79ec3/effects/lib/free/tseq.ml#L44
%         (difference list gadt)

%   - Misuse
%     https://github.com/chengsun/simd-sexp/commit/e714a8205c6df512d920a31a4dd2448975703c55
%       (already tail-recursive)
%     https://github.com/gridbugs/llama/blob/96d1c96c31ff136f465b5f37c981fff591bac6fd/src/midi/byte_array_parser.ml#L50
%       (needless complexity)
%     https://github.com/LimitEpsilon/lambda-interpreter/blob/ab8e81de7f896aa64eba14e23d964b9705e94161/lib/evaluate.ml#L46
%       (already tail-recursive)

The TMC transformation was first made available to users in OCaml 4.14.0, released in March 2022.
At the time there were no users of the feature.
Notably, the OCaml standard library had intentionally not been modified to start using it: before the release, we did not want the stdlib codebase to depend on a not-available-yet feature, and after the release we decided not to push for adoption (we may be biased about the benefits/importance of TMC), and let other contributors propose its use, after doing their own evaluation of performance impact and code-clarity benefits.

Over time, \ocaml|[@tail_mod_cons]| was adopted in a few places in the OCaml standard library, either to make some functions tail-recursive that previously were not, or simplify some complex tail-recursive implementations. As of spring 2024, the feature is now used in \ocaml|Hashtbl.find_all| and in \ocaml|List| module (\ocaml|init, map, mapi, map2, find_all, filteri, filter_map, take, take_while, of_seq|). \ocaml|init| and the \ocaml|map*| functions, which are the most heavily used, were unrolled once -- the minimal amount observed to preserve the non-tailrec performance for small lists. Other functions are written in the simplest possible way. (All \texttt{stdlib} uses so far build lists rather than another datatype.)

We performed a systematic search for \ocaml|[@tail_mod_cons]| usage on November 2023, and we found that it was also used
\begin{itemize}
\item in a few utility modules (general-purpose functions designed to extend or replace the standard library), notably in Jane Street's \texttt{base} library.
\item in the middle of domain-specific user code, to build lists, in 14 different projects
\item in the middle of user code, to build other types than lists, in three projects:
  \begin{itemize}
  \item in \texttt{RedPRL}\footnote{\url{https://github.com/RedPRL/ocaml-bwd/blob/fbf496b29532085b38073eaa62ba3d22ac619d5d/src/BwdNoLabels.ml\#L60}},
    it is used to build snoc lists
  \item in a project named \texttt{PL-reading-group}\footnote{\url{https://github.com/Skyb0rg007/PL-Reading-Group/blob/7002ae35ba69fb9010e5049eee88ab475bc79ec3/effects/lib/free/tseq.ml\#L44}},
    it is used to build a GADT of difference lists
  \item in \texttt{acutis}\footnote{\url{https://github.com/johnridesabike/acutis/blob/4113fb516d9c5dd6f2dbfd9657f52c5ae7a5dcae/lib/matching.ml\#L161}},
    in a group of mutually-recursive functions that builds a complex AST structure used options and nested records
  \end{itemize}
\end{itemize}

We also found three cases where the annotation is (in our opinion) misused: in two cases, the function is already tail-recursive, so there is no need for the annotation, and in a third case\footnote{\url{https://github.com/gridbugs/llama/blob/96d1c96c31ff136f465b5f37c981fff591bac6fd/src/midi/byte_array_parser.ml\#L50}} we consider that the use is gratuitous and can be replaced by already-available standard library functions.

%%% Local Variables:
%%% mode: latex
%%% TeX-master: "main"
%%% End:
