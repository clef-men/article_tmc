\section{Extended republication}
\label{app:extended-republication}

A preliminary, work-in-progress version of this work was published at
a national conference -- all talks given in a non-English
language. This makes the present submission to a PACMPL-published
conference a partial republication.

This is unusual, so we decided to write this appendix section to
explain this choice. Our explanation is carefully worded to maintain
anonymity. (Note: there is overlap between the authors of both
presentations, but they are not equal sets, so guessing which previous
version we are talking about would be only partially de-anonimizing.)

\subsection{The letter of the law.}

We believe that partial republication fully satisfies the
\href{http://www.sigplan.org/Resources/Policies/Republication/}{SIGPLAN
  Republication Policy}, which we quote below in full
(with our emphasis):

\begin{quote}
  \textbf{Prior Publication of Closely Related Work.}  If a closely
  related paper was previously accepted at a conference or a workshop
  with proceedings of any kind, the original paper must be cited, its
  relationship to the current paper explained, and the program chair
  must be explicitly informed. The second paper will be judged on the
  additional value of its publication in the new venue. Such value
  might come from the \textbf{wider audience for the new venue} or
  from subjecting the work to a \textbf{higher standard of
    review}. Program committees should consider factors such as the
  following in deciding whether to publish the second paper:
    \begin{itemize}
    \item The call for papers for the first venue clearly states that publication in the venue is not intended to preclude later publication.
    \item The original proceedings are not easily accessible to the SIGPLAN community.
    \item The first venue \textbf{targeted a geographically limited audience}.
    \end{itemize}
\end{quote}

The present submission will be subject to higher standards
(the acceptance rate of the national venue is in the
80-100\% range). If accepted, it would reach a wider audience. The
past submission targeted a geographically limited audience. (If we
remember correctly, everyone attending the conference that year was
working in the same country it was located in.)

\subsection{The spirit of the law.}

We believe that a partial republication mirrors the established
practice of publishing a first version of a paper in conference
proceedings, and then an extended version in a journal. Here the
situation is different but similar, we published a work-in-progress
presentation of our work in a minor/national conference, and we are
submitting a full presentation of the same work
(and additional contributions) to a major/international
conference-journal hybrid.

Common/accepted conference-to-journal republication typically expect
``enhanced versions'' with, according to the Journal of Functional
Programming guidelines for example, ``should contain additional
discussion, examples, or proofs''. We believe that our new
presentation of our work is well above this bar, as it includes major
new scientific contributions.

\subsection{Comparing the two versions.}

The work-in-progress paper was written in October 2020, after we
implemented TMC for OCaml, but before it was integrated into the OCaml
compiler. It focuses on the implementation in OCaml, and does not
contain any correctness argument for the program transformation --
a major missing piece for a high-quality publication in an
international conference.

The majority of the presentation in the present paper is new. The
following parts are reused from the previous paper --
integrated by the same author who originally wrote them:
\begin{itemize}
\item Two thirds of the introduction (the first two pages).
\item \cref{subsec:ocaml-examples} on examples of TRMC functions in
  OCaml (1.5 pages).
\end{itemize}
We thus estimate the reused content at about 3.5 pages, the rest of
the content was written from scratch for the present submission.

The major scientific difference between the two papers is the
correctness proof in Iris, which is a significant contribution of its
own. To our knowledge, it is the first mechanized proof of correctness
for the tail-modulo-constructor program transformation, and the first
verification of a program transformation (for all inputs) in an
Iris-based relational separation logic. This constitutes the majority
of the content of the current presentation.

More minor differences include an analysis of the adoption of TMC in
OCaml libraries post-upstreaming, and a comparison with the Koka
work~\citep*{tmc-koka-2023} which was developed simultaneously with
ours.

\subsection{Motivating our choice.}

An alternative that we of course considered would be to present only
the correctness proof for TMC in the present paper, without reusing
any contributions and presentation from the work-in-progress,
implementation-focused paper.

This approach would however result in a ``small'' publication about
the (non-trivial) implementation work in the OCaml compiler, and
a more prestigious international submission for the correctness proof
alone. It would deprive from symbolic/academic reward the contributors
who lead the implementation and upstreaming work, without which none
of the present work would exist in the first place. (This approach
would also weaken national conferences, by making researchers think
twice about submitting implementation-focused work in progress.)

This is in fact a common dynamic in programming-language research that
combines delicate implementation work with delicate meta-theory
work. The latter is more easily identified as challenging research,
easier to evaluate for quality and novelty, and tends to gather more
scientific credit. The implementation work also typically comes
before, and it is natural to do the verification work separately, with
a different (sometimes entirely disjoint) group of contributors. We
are trying to avoid this dynamic by grouping the two sorts of
contributions together in the same publication.

%%% Local Variables:
%%% mode: latex
%%% TeX-master: "main"
%%% End:
