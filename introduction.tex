\section{Introduction}

In \OCaml, the natural implementation of the \ocaml|map| function on polymorphic lists is the following:

\begin{OCaml}
let rec map f xs =
  match xs with
  | [] =>
      []
  | x :: xs =>
      let y = f x in
      y :: map f xs
\end{OCaml}

Yet, calling \ocaml|map| on a large input list may crash with the dreaded \ocaml|Stack_overflow| exception.
%
Indeed, as functions calls consume stack space, the stack usage of this implementation is linear in the size of the input list.
%
This is the reason why \OCaml programmers have to be careful, when they write recursive functions, to remain in the so-called \emph{tail-recursive} fragment: every recursive call must be in tail position, that is the last thing the function does before returning.
%
As there is nothing to do after a tail call, the stack frame can be directly reused.
%
Therefore, tail calls do not consume stack space.
%
Knowing this, we can provide another tail-recursive implementation of \ocaml|map|:

\begin{minipage}{0.5\linewidth}
\begin{Ocaml}
let rec rev_map f ys xs =
  match xs with
  | [] ->
      ys
  | x :: xs ->
      let y = f x in
      rev_map f (y :: ys) xs
\end{Ocaml}
\end{minipage}
\hfill
\begin{minipage}{0.5\linewidth}
\begin{Ocaml}
let map f xs =
  let ys = rev_map f [] xs in
  List.rev ys
\end{Ocaml}
\end{minipage}

This second version computes the reverse result list with \ocaml|rev_map| in the accumulator-passing style and applies \ocaml|List.rev|.
%
As both \ocaml|rev_map| and \ocaml|rev| are tail-recursive, it is stack-safe.
%
However, it is not as efficient on small lists since it does two traversals instead of one.
%
There is yet another implementation:

\begin{minipage}{0.65\linewidth}
\begin{Ocaml}
/* [dst] is a partially initialized list cell. */
let rec map_dps dst f xs =
  match xs with
  | [] ->
      /* Write result empty list to [dst]. */
      set_field dst 1 []
  | x :: xs ->
      let y = f x in
      /* New partially initialized list cell. */
      let dst' = y :: [] in
      /* Write result [dst'] to [dst]. */
      set_field dst 1 dst' ;
      /* Continue with [dst'] as new destination. */
      map_dps dst' f xs
\end{Ocaml}
\end{minipage}
\hfill
\begin{minipage}{0.45\linewidth}
\begin{Ocaml}
let map f xs =
  match xs with
  | [] ->
      []
  | x :: xs ->
      let y = f x in
      let dst = y :: [] in
      map_dps dst f xs ;
      dst
\end{Ocaml}
\end{minipage}

This third implementation is much less natural.
%
It takes advantage of the memory representation of lists in \OCaml: some constant integer for the empty list and a heap-allocated cell containing one tag followed by two fields for the list constructor.
%
The \ocaml|set_field| function allows us to modify these fields.
%
It can be implemented using the unsafe \ocaml|Obj| module.

The core of this implementation is the tail-recursive \ocaml|map_dps| auxiliary function.
%
It is the \emph{destination-passing style} version of \ocaml|map|: it writes its result to the tail field of destination \ocaml|dst| instead of returning it.
%
On the empty list, it just writes the empty list.
%
On non-empty lists, it allocates a new partially initialized list cell \ocaml|dst'|, writes \ocaml|dst'| to \ocaml|dst| and calls itself recursively with \ocaml|dst|' as its new destination.
%
\ocaml|map| calls \ocaml|map_dps| on non-empty lists.

What happens is \ocaml|map_dps| constructs the final transformed list by modifying the last cell of the current partial transformed list.
%
Not only is it stack-safe but it also does only one traversal, as opposed the previous stack-safe implementation.
%
Unfortunately, it cannot be implemented using only safe \OCaml features, as list cells are normally immutable.
%
In fact, most strongly typed languages reject it.
%
A notable exception is \Mezzo \cite{DBLP:journals/toplas/BalabonskiPP16}, which is based on a permission system inspired by \emph{separation logic}~\cite{DBLP:journals/cacm/OHearn19}.

We can do better.
%
If we cannot write it, the compiler may be able to do so.
%
In fact, the manual transformation of the natural implementation of \ocaml|map| we have performed is an instance of a more general transformation known as \emph{tail call optimization modulo constructor} (TMC)~\cite{risch-73,friedman-wise-75}.
%
TMC has been implemented in Lisp~[TODO], Prolog~[TODO] and Koka~\cite{DBLP:journals/pacmpl/LeijenL23}.
%
The first main contribution of this work is an implementation of TMC in the \OCaml compiler as an optional optimization.
%
Various functions in the standard library and third-party codebases have been rewritten to use it.
%
For instance, the efficient stack-safe version of \ocaml|map| is obtained just by annotating the natural definition with the attribute \ocaml|[@tail_mod_cons]|:

\begin{Ocaml}
let[@tail_mod_cons] rec map f xs =
  match xs with
  | [] -> []
  | x :: xs -> f x :: map f xs
\end{Ocaml}

The second main contribution of this work is a mechanized proof of correctness for the core of this transformation on a small untyped calculus.
%
We established a \emph{behavioral refinement} between the source program and the transformed program: any behavior of the transformed program, be it converging, diverging or stuck, is a behavior of the source program.
%
Our proof relies on \Simuliris~\cite{DBLP:journals/pacmpl/GaherSSJDKKD22}, a separation logic framework built on top of the \Iris base logic.
%
We found that their proof technique, a simulation relation, is not expressive enough to reason about program transformations that introduce new function calling conventions.
%
We had to generalize the \Simuliris handling of function calls by parameterizing the simulation relation over an abstract protocol inspired by the work of \citeauthor{DBLP:journals/pacmpl/VilhenaP21}~\cite{DBLP:journals/pacmpl/VilhenaP21}.
%
This contribution of our work is independent from the TMC transformation and our small calculus.
%
It could be reused in future work on verifying compiler transformations.

Interestingly, the core of the soundness proof is the specification of the two variants of each TMC-transformed function --- direct style and destination-passing style.
%
It conveys the essence of destination-passing style: computing the same thing and writing it to an owned destination.
%
For instance, delaying the definition of the simulation relation $\iSimv{\Phi}{e_s}{e_t}$ and the similarity relation $\datalangVal_s \iSimilar \datalangVal_t$, the specification of the variants of \ocaml|map| is:
%
\begin{align*}
        \iSimvHoare{
            \iTrue
        }{
            \iSimilar
        }{
            & \datalangCall{\datalangFnptr{\ocamlText{map}}}{\datalangVal_s}
        }{
            \datalangCall{\datalangFnptr{\ocamlText{map}}}{\datalangVal_t}
        }
    \\
        \iSimvHoare{
            (\datalangLoc + \datalangIdx) \iPointsto \datalangHole
        }{
            \lambdaAbs (\datalangVal_s, \datalangVal_t') \ldotp
            \exists \datalangVal_t \ldotp
            \datalangVal_t' = \datalangUnit \iSep
            (\datalangLoc + \datalangIdx) \iPointsto \datalangVal_t \iSep
            \datalangVal_s \iSimilar \datalangVal_t
        }{
            & \datalangCall{\datalangFnptr{\ocamlText{map}}}{\datalangVal_s}
        }{
            \datalangCall{\datalangFnptr{\ocamlText{map_dps}}}{\datalangPair{\datalangPair{\datalangLoc}{\datalangIdx}}{\datalangVal_t}}
        }
\end{align*}

To sum up, our main contributions are:
\begin{enumerate}
    \item an implementation of the TMC transformation in the \OCaml compiler;
    \item a mechanized proof of soundness for a simplified transformation on a small calculus;
    \item a generalization of \Simuliris handling of function calls to reason about different calling conventions.
\end{enumerate}
