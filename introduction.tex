\section{Introduction}

\subsection{Why TMC in OCaml?}

OCaml function calls consume stack space. If a program runs out of stack space, it crashes with the dreaded \texttt{Stack\_overflow} exception. As a result, OCaml programmers have to be careful, when they write recursive functions, to remain in the so-called \emph{tail-recursive} fragment, using \emph{tail} calls that do not consume stack space.

This discipline is a source of difficulties for both beginners and experts. Beginners have to be taught recursion, and then tail-recursion. Experts disagree on the "right" way to write `List.map`. The direct version is beautiful but not tail-recursive, so it crashes on larger inputs. The naive tail-recursive transformation is (slightly) slower than the direct version, and experts may want to avoid that cost. Some libraries propose complex implementations to compensate for this performance loss. In general, tail-recursion requires the programmer to manually perform sophisticated program transformations.

In this work we propose an implementation of the "Tail Modulo Constructor" (TMC) transformation for OCaml. TMC is a program transformation for a fragment of non-tail-recursive functions, that rewrites them in \emph{destination-passing style}. The supported fragment is smaller than other approaches such as continuation-passing-style, but the performance of the transformed code is on par with the direct, non-tail-recursive version. Many useful functions that traverse a recursive datastructure and rebuild another recursive structure are in the TMC fragment, in particular `List.map` (and `List.filter`, `List.append`, etc.). Finally those functions can be written in a way that is beautiful, correct on all inputs, and efficient.

After we introduced this TMC transformation in the OCaml compiler (merged in OCaml 4.... in ...), various functions in the standard library (...) and third-party codebases have been rewritten to use it.

\subsection{TMC on an example}

\newcommand{\captionlabel}[2]{\caption{#1}\label{#2}}
\newcommand{\fref}[1]{Figure~\ref{#1}}

\begin{figure}[tp]
\begin{OCaml}
let rec map f xs =
  match xs with
  | [] -> []
  | x :: xs ->
      let y = f x in y :: map f xs
\end{OCaml}
\captionlabel{\ocaml|map| in direct style}{fig:map}
\end{figure}

% I am assuming
% let set_field x i v = Obj.set_field (Obj.repr x) i (Obj.repr v)
\begin{figure}[tp]
\begin{OCaml}
(* [dst] points to a partially initialized list cell. *)
let rec map_dps dst f xs =
  match xs with
  | [] ->
      (* Write our result, the empty list, to [dst]. *)
      set_field dst 1 []
  | x :: xs ->
      let y = f x in
      (* Allocate a new partially initialized list cell. *)
      let dst' = y :: [] in
      (* Write our result, [dst'], to [dst]. *)
      set_field dst 1 dst';
      (* Continue with [dst'] as new destination. *)
      map_dps dst' f xs

let map f xs =
  match xs with
  | [] -> []
  | x :: xs ->
      let y = f x in
      let dst = y :: [] in
      map_dps dst f xs;
      dst
\end{OCaml}
% TODO the call from map to map_dps is not a tail call.
%      is this what our translation does? check
\captionlabel{\ocaml|map| in destination-passing style}{fig:map_dps}
\end{figure}

A natural%
%
% A comment on evaluation order.
\footnote{The sequencing construct \ocaml|let y = f x| ensures that the call \ocaml|f x| takes place before the recursive call \ocaml|map f xs|. OCaml does not specify in which order the arguments of a constructor are evaluated. Writing \ocaml|f x :: map f xs| would allow the compiler to choose an evaluation order.  Here, left-to-right evaluation is imposed by the author of \ocaml|map|.}
%
implementation of the \ocaml|map| function on lists appears in \fref{fig:map}.
%
This code is simple, but suffers from a well-known problem: because the recursive call \ocaml|map f xs| is not a tail call, \ocaml|map| requires an amount of stack space that is linear in the size of the list~\ocaml|xs|.
%
The OCaml runtime limits the size of the stack, so \ocaml|map| fails (by raising a ``stack overflow'' exception) when applied to very long lists.

To avoid this problem, a possible workaround is to write a variant of \ocaml|map|
accumulator-passing style (\fref{fig:rev_map}). The function \ocaml|rev_map|
takes an extra argument \ocaml|ys| and inserts the value \ocaml|y| \emph{in front
  of} the list \ocaml|ys|.
%
It is easy to see that \ocaml|rev_map f ys xs| computes
%
\ocaml|ys @ rev (map f xs)|,
%
where \ocaml|rev| is list reversal.
%
This implies that
%
\ocaml|map f xs|
%
can be defined as \ocaml|rev (rev_map f [] xs)|.
%
Both \ocaml|rev| and \ocaml|rev_map| are tail-recursive functions, so this new
definition of \ocaml|map| requires constant stack space and is no longer subject
to a ``stack overflow'' issue.
%
Unfortunately, this implementation of \ocaml|map| is still somewhat
unsatisfactory. It involves building an intermediate list \ocaml|ys|, which is
immediately thrown away. Furthermore, it requires two list traversals, to~wit,
a traversal of the original list~\ocaml|xs| and a traversal of the intermediate
list~\ocaml|ys|.

To remedy this remaining deficiency, another approach is to write a variant of
\ocaml|map| in destination-passing style.
% TODO should we explicitly say anything about this paper?
\nocite{shaikha-17}
%
The function \ocaml|map_dps| (\fref{fig:map_dps}) takes an extra parameter
\ocaml|dst|, which must be a list cell. Its effect is to construct the list
\ocaml|map f xs| and to write the address of this list into the ``tail''
field of the list cell \ocaml|dst|.
%
Then, \ocaml|map| can be defined in terms of \ocaml|map_dps|.
%
Because \ocaml|map_dps| is tail-recursive, \ocaml|map_dps| and \ocaml|map| require
constant stack space. Furthermore, the original list is traversed only once,
and the new list is constructed in the desired order; no reversal is required.

The code of \ocaml|map_dps| uses \ocaml|set_field|, an unsafe primitive
operation. \ocaml|set_field dst 1 dst'| writes the value \ocaml|dst'| into the field
at offset 1 in the memory block at address~\ocaml|dst|. This operation is unsafe
insofar as it can be used to modify a list cell, even though list cells are
normally immutable.

To sum up, the definition of \ocaml|map| in \fref{fig:map_dps} has desirable
properties, but it is not pretty -- in fact it cannot be expressed without using safety escape hatches. The definition of \ocaml|map| in \fref{fig:map}
is much more elegant, but is subject to the ``stack overflow'' issue. We would
like the best of both worlds: to write the code in \fref{fig:map} and to let
the compiler transform it into the destination-passing style implementation in
\fref{fig:map_dps}. This is the purpose of the \emph{tail modulo cons} (TMC)
transformation. As of OCaml 4.14, this transformation is available upon
request: the user must annotate the definition of \ocaml|map| with the attribute
\ocaml|[@tail_mod_cons]|:

% TODO
\nocite{sobel-friedman-98}
\nocite{minamide-98}

%%%%

\begin{OCaml}
let[@tail_mod_cons] rec map f = function
| [] -> []
| x :: xs -> f x :: map f xs
\end{OCaml}

This transformation is opt-in, only applied by the compiler on function definitions annotated with this new \ocaml{[@tail_mod_cons]} attribute. The OCaml compiler transforms this program at the level of the Lambda intermediate representation; the result of the transformation cannot be expressed as a source OCaml program, but it can be approximated by the following:

\hspace{-1.6em}
\begin{minipage}{0.5\linewidth}
\begin{OCaml}
let rec map f = function
| [] -> []
| x::xs ->
  let y = f x in
  let dst = y :: [] in
  map_dps dst 1 f xs;
  dst
\end{OCaml}
\end{minipage}
\hfill
\begin{minipage}{0.5\linewidth}
\begin{OCaml}
and map_dps dst i f = function
| [] ->
  dst.i <- []
| x::xs ->
  let y = f x in
  let dst' = y :: [] in
  set_field dst i dst';
  map_dps dst' 1 f xs
\end{OCaml}
\end{minipage}

COMMENT (Gabriel): it is a bit unfortunate that this example is redundant with the nice DPS version from François. But note that we are parametrized over the destination index -- this is necessary in the general case, for binary trees for example. We could sweep the difference under the rug here, but not for example when comparing with the work of Daan Leijen.

\subsection{Proving the correctness of the TMC transformation}

The TMC transformation crucially relies on the introduction of uniquely-owned mutable state. Our mechanized correctness proof establishes a backward simulation using a relational separation logic built on top of the Iris base logic.

It is in fact difficult to define a notion of simulation inside Iris -- natural definitions using the ``later'' modality do not give the expected definition. We build on top of Simuliris~\citep*{TODO-simuliris}, which provides a carefully-chosen definition of simulations that works well in Iris.

Following the proof technique proposed by Simuliris, we found that their simulation definition is not expressive enough to reason about program transformations that introduce new function calling conventions. We had to generalize the Simuliris handling of function calls by parameterizing the simulation relation over an abstract protocol $\Chi$ inspired by the work of \citet*{TODO-paulo}. This contribution of our work is independent from the TMC transformation and our simplified programming language, and could be reused in future work on verifying compiler transformations.

Once this machinery is applied, what remains to be proven are two crisp specifications for the two variants of each TMC-transformed function -- direct style and destination-passing style.

- What remains to prove are two key specifications:
  + direct spec
  + dps spec

- We prove these two specifications using a relational program logic for our simplified language.

\subsection{Contributions}


describe TMC

OCaml compiler

Daan Leijen

annoncer éléments décisifs (specs dir/dps) ?

contributions :
generalization of Simuliris simulation (more calling conventions)
mechanized proof of soundness in separation logic
intuitive specification (direct and DPS calling conventions)
global optimization

\begin{figure}[tp]
\begin{OCaml}
let rec map f xs =
  match xs with
  | [] ->
      []
  | x :: xs ->
      let y = f x in
      y :: map f xs
\end{OCaml}
\caption{\ocaml|map| in direct style (\OCamlLang)}
\label{fig:map}
\end{figure}
\begin{figure}[tp]
\begin{OCaml}
let rec rev_map f ys xs =
  match xs with
  | [] ->
      ys
  | x :: xs ->
      let y = f x in
      rev_map f (y :: ys) xs
      
let map f xs =
  let ys = rev_map f [] xs in
  rev ys
\end{OCaml}
\caption{\ocaml|map| in accumulator-passing style (\OCamlLang)}
\label{fig:rev_map}
\end{figure}
\begin{figure}[tp]
\begin{OCaml}
(* [dst] points to a partially initialized list cell. *)
let rec map_dps dst f xs =
  match xs with
  | [] ->
      (* Write our result, the empty list, to [dst]. *)
      set_field dst 1 []
  | x :: xs ->
      let y = f x in
      (* Allocate a new partially initialized list cell. *)
      let dst' = y :: [] in
      (* Write our result, [dst'], to [dst]. *)
      set_field dst 1 dst' ;
      (* Continue with [dst'] as new destination. *)
      map_dps dst' f xs

let map f xs =
  match xs with
  | [] ->
      []
  | x :: xs ->
      let y = f x in
      let dst = y :: [] in
      map_dps dst f xs ;
      dst
\end{OCaml}
\caption{\ocaml|map| in destination-passing style (\OCamlLang)}
\label{fig:map_dps}
\end{figure}
