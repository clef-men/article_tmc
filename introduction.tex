\section{Introduction}

\subsection{Prologue}

``OCaml'', we teach our students, ``is a functional programming language. We can write the beautiful function \ocaml{List.map} as follows:''
\begin{Ocaml}
let rec map f = function
| [] -> []
| x :: xs -> f x :: map f xs
\end{Ocaml}

``Well, actually'', we continue, ``OCaml is an effectful language, so we need to be careful about the evaluation order. We want \ocaml{map} to process elements from the beginning to the end of the input list, and the evaluation order of \ocaml{f x :: map f xs} is unspecified. So we write:''
\begin{Ocaml}
let rec map f = function
| [] -> []
| x :: xs ->
  let y = f x in
  y :: map f xs
\end{Ocaml}

``Well, actually, this version fails with a \ocaml{Stack_overflow}
exception on large input lists. If you want your \ocaml{map} to behave
correctly on all inputs, you should write a \emph{tail-recursive}
version. For this you can use the accumulator-passing style:''
\begin{Ocaml}
let map f li =
  let rec map_ acc = function
  | [] -> List.rev acc
  | x :: xs -> map_ (f x :: acc) xs
  in map_ [] f li
\end{Ocaml}

``Well, actually, this version works fine on large lists, but it is
less efficient than the original version. One approach is to start with
a non-tail-recursive version, and switch to a tail-recursive version
for large inputs; even there you can use some manual unrolling to
reduce the overhead of the accumulator. For example, the nice
\href{https://github.com/c-cube/ocaml-containers}{Containers} library
does it as follows:''.

%\hspace{-3em}
\begin{minipage}{0.6\linewidth}
\begin{Ocaml}[basicstyle=\ttfamily\tiny]
let tail_map f l =
  (* Unwind the list of tuples, reconstructing the full list front-to-back.
     @param tail_acc a suffix of the final list; we append tuples' content
     at the front of it *)
  let rec rebuild tail_acc = function
    | [] -> tail_acc
    | (y0, y1, y2, y3, y4, y5, y6, y7, y8) :: bs ->
      rebuild (y0 :: y1 :: y2 :: y3 :: y4 :: y5 :: y6 :: y7 :: y8 :: tail_acc) bs
  in
  (* Create a compressed reverse-list representation using tuples
     @param tuple_acc a reverse list of chunks mapped with [f] *)
  let rec dive tuple_acc = function
    | x0 :: x1 :: x2 :: x3 :: x4 :: x5 :: x6 :: x7 :: x8 :: xs ->
      let y0 = f x0 in let y1 = f x1 in let y2 = f x2 in
      let y3 = f x3 in let y4 = f x4 in let y5 = f x5 in
      let y6 = f x6 in let y7 = f x7 in let y8 = f x8 in
      dive ((y0, y1, y2, y3, y4, y5, y6, y7, y8) :: tuple_acc) xs
    | xs ->
      (* Reverse direction, finishing off with a direct map *)
      let tail = List.map f xs in
      rebuild tail tuple_acc
  in
  dive [] l
\end{Ocaml}
\end{minipage}
\hfill
\begin{minipage}{0.4\linewidth}
\begin{Ocaml}[basicstyle=\ttfamily\tiny]
let direct_depth_default_ = 1000

let map f l =
  let rec direct f i l = match l with
    | [] -> []
    | [x] -> [f x]
    | [x1;x2] -> let y1 = f x1 in [y1; f x2]
    | [x1;x2;x3] ->
      let y1 = f x1 in let y2 = f x2 in [y1; y2; f x3]
    | _ when i=0 -> tail_map f l
    | x1::x2::x3::x4::l' ->
      let y1 = f x1 in
      let y2 = f x2 in
      let y3 = f x3 in
      let y4 = f x4 in
      y1 :: y2 :: y3 :: y4 :: direct f (i-1) l'
  in
  direct f direct_depth_default_ l
\end{Ocaml}
\end{minipage}
\lstset{basicstyle=\small\ttfamily}

At this point, unfortunately, some students leave the class and never
come back.

We propose a new feature for the OCaml compiler, an explicit, opt-in
``Tail Modulo Cons'' transformation, to retain our students. After the
first version (or maybe, if we are teaching an advanced class, after
the second version), we could show them the following version:
\begin{Ocaml}
let[@tail_mod_cons] rec map f = function
| [] -> []
| x :: xs -> f x :: map f xs
\end{Ocaml}

This version is as fast as the simple implementation, tail-recursive,
and easy to write.

The catch, of course, is to teach when this \ocaml{[@tail_mod_cons]}
annotation can be used. Maybe we would not show it at all, and pretend
that the direct \ocaml{map} version with \ocaml{let y} is fine. This
would be a much smaller lie than it currently is,
a \ocaml{[@tail_mod_cons]}-sized lie.

Finally, experts should be very happy. They know about all these
versions, but they do not have to write them by hand anymore. Have
a program perform (some of) the program transformations that they are
currently doing manually.

\subsection{TMC transformation example}

A function call is in \emph{tail position} within a function
definition if the definition has ``nothing to do'' after evaluating
the function call -- the result of the call is the result of the whole
function at this point of the program. (A precise definition will be
given in Section~\ref{sec:tcmc}.) A function is \emph{tail recursive}
if all its recursive calls are tail calls.

In the definition of \ocaml{map}, the recursive call is not in tail
position: after computing the result of \ocaml{map f xs} we still have
to compute the final list cell, \ocaml{y :: ?}. We say that a call is
\emph{tail modulo cons} when the work remaining is formed of data
\emph{constructors} only, such as \ocaml{(::)} here.

\begin{Ocaml}
let[@tail_mod_cons] rec map f = function
| [] -> []
| x :: xs ->
  let y = f x in
  y :: map f xs
\end{Ocaml}

Other datatype constructors may also be used; the following example is
also tail-recursive \emph{modulo cons}:

\begin{Ocaml}
let[@tail_mod_cons] rec tree_of_list = function
| [] -> Empty
| x :: xs -> Node(Empty, x, tree_of_list xs)
\end{Ocaml}

The TMC transformation returns an equivalent function in
\emph{destination-passing} style where the calls in \emph{tail modulo
  cons} position have been turned into \emph{tail} calls. In
particular, for \ocaml{map} it gives a tail-recursive function, which
runs in constant stack space; many other list functions also become
tail-recursive. This works for other datatypes as well, such as binary trees, but in this case some recursive calls may remain non-tail-recursive.

The transformed code of \ocaml{map} can be described as
follows:\\
\hspace{-1.6em}
\begin{minipage}{0.5\linewidth}
\begin{Ocaml}
let rec map f = function
| [] -> []
| x::xs ->
  let y = f x in
  let dst = y :: ? in
  map_dps dst 1 f xs;
  dst
\end{Ocaml}
\end{minipage}
\hfill
\begin{minipage}{0.5\linewidth}
\begin{Ocaml}
and map_dps dst i f = function
| [] ->
  dst.i <- []
| x::xs ->
  let y = f x in
  let dst' = y :: ? in
  dst.i <- dst';
  map_dps dst' 1 f xs
\end{Ocaml}
\end{minipage}

The transformed code has two variants of the \ocaml{map} function. The \ocaml{map_dps} variant is in \emph{destination-passing style}, it expects additional parameters that specify a memory location, a \emph{destination}, and will write its result to this \emph{destination} instead of returning it. It is tail-recursive, and it performs a single traversal of the list. The \ocaml{map} variant provides the same interface as the non-transformed function, we say that it is in \emph{direct style}. It is not tail-recursive, but it does not call itself recursively, it calls the tail-recursive \ocaml{map_dps} on non-empty lists.

The key idea of the transformation is that the expression \ocaml{y :: map f xs}, which contained a non-tail-recursive call, is transformed into first the computation of a \emph{partial} list cell, written \ocaml{y :: ?}, followed by a call to \ocaml{map_dps} that is asked to write its result in the position of the \ocaml{?}. The recursive call thus happens after the cell creation (instead of before), in tail-recursive position in the \ocaml{map_dps} variant. In the direct variant, the value of the destination \ocaml{dst} has to be returned after the call.

The transformed code is pseudo-OCaml, it is not a valid OCaml
program: we use a magical \ocaml{?} constant, and our notation
\ocaml{dst.i <- ...} to update constructor parameters in-place is also
invalid in source programs. The transformation is implemented on
a lower-level, untyped intermediate representation of the OCaml
compiler (Lambda), where those operations do exist. The OCaml type
system is not expressive enough to type-check the transformed program:
the list cell is only partially-initialized at first, each partial
cell is mutated exactly once, and in the end the whole result is
returned as an \emph{immutable} list. Some type system are expressive
enough to represent this transformed code, notably \Mezzo~\citep*{mezzo-2016}, based on a permission system inspired by \emph{separation logic}~\citep*{seplog-cacm-2019}, or the linear types used in \citet*{minamide-98}.

TMC has been implemented in Lisp~[TODO], Prolog~[TODO] and (simultaneously to our work) Koka~\citep*{tmc-koka-2023}. A simpler variant of TMC, where constructors are replaced by associative operations and transformed to \emph{accumulator-passing style}, is in \texttt{gcc} and \texttt{clang}.

The first main contribution of our work is an implementation of TMC in the \OCaml compiler as an on-demand program transformation.
%
We describe the non-trivial design choices in terms of user interface, and evaluate performance through microbenchmarks.
%
Various functions in the standard library and third-party codebases have been rewritten to use it, to become tail-recursive, gain in performance, or (when an efficient but complex tail-recursive version was used) simplify considerably the implementation.

The second main contribution of this work is a mechanized proof of correctness for the core of this transformation on a small untyped calculus.
%
For any input source program, we establish a termination-preserving \emph{behavioral refinement} between the source program and the corresponding transformed program: any behavior of the transformed program, be it converging, diverging or stuck, is a behavior of the source program.

Our proof technique is to define a relational program logic for our small untyped calculus, to show the correctness of the TMC transformation using this program logic, and to get the behavioral refinement by proving adequacy of our program logic. We build on top of \Simuliris~\citep*{simuliris-2022}, a framework for simulations in separation logic over the \Iris base logic. The use of separation logic nicely captures certain aspects of the proof argument, in particular the fact that the the destination-passing-style function uniquely owns the destination location it receives.

To our knowledge, previous works on \Iris-based relational separation program logics have verified \emph{examples} of interesting optimizations and program transformations, by formally proving relations between pairs of concrete programs. Our work may be the first proof of correctness of a \emph{program transformation} (as a function or relation) using an \Iris-based relational separation logic, establishing correctness for all input programs.

At this level of generality, we found that the \Simuliris simulation is not expressive enough to reason about transformations that introduce new function calling conventions.
%
We generalize the \Simuliris handling of function calls by parameterizing the simulation relation over an \emph{abstract protocol}, inspired by \citet*{protocols-2021}.
%
This sub-contribution of our work is independent from the TMC transformation and our small calculus,
and we tried to express it in general terms, beyond the specific needs of TMC.
%
In particular, we believe that \Iris-based relational separation logics could be a powerful yet pleasant proof technique for compiler verification.

The core of the soundness proof is the specification of the two variants of each TMC-transformed function --- direct style and destination-passing style.
%
It concisely conveys the essence of destination-passing style: computing the same thing and writing it to an owned destination.
%
For instance, to give a taste of the formalism, the specification of the variants of \ocaml|map| looks as follows:
%
\begin{align*}
        \iSimvHoare{
            \iTrue
        }{
            \iSimilar
        }{
          & \datalangCall
              {\datalangFnptr{\ocamlText{map}}}
              {\datalangPair{\datalangFnptr{\datalangFn}}{\datalangVal_s}}
        }{
            \datalangCall
              {\datalangFnptr{\ocamlText{map}}}
              {\datalangPair{\datalangFnptr{\datalangFn}}{\datalangVal_t}}
        }
    \\
        \iSimvHoare{
            (\datalangLoc + \datalangIdx) \iPointsto \datalangHole
        }{
            \datalangVal'_s, \datalangUnit \ldotp
            \exists \datalangVal'_t \ldotp\ %
            (\datalangLoc + \datalangIdx) \iPointsto \datalangVal'_t
            \ \iSep\ %
            \datalangVal'_s \iSimilar \datalangVal'_t
        }{
          & \datalangCall
              {\datalangFnptr{\ocamlText{map}}}
              {\datalangPair{\datalangFnptr{\datalangFn}}{\datalangVal_s}}
        }{
            \datalangCall
              {\datalangFnptr{\ocamlText{map_dps}}}
              {\datalangTriple
                {\datalangPair{\datalangLoc}{\datalangIdx}}
                {\datalangFnptr{\datalangFn}}
                {\datalangVal_t}}
        }
\end{align*}

To sum up, our main contributions are:
\begin{enumerate}
    \item an implementation of the TMC transformation in the \OCaml compiler, with a discussion of the user interface, a performance evaluation, and a survey of its early usage;
    \item a mechanized proof of soundness for an idealized TMC transformation on a small calculus, using a relational separation program logic;
    \item a generalization of the \Simuliris handling of function calls with abstract \emph{protocols} to reason about different calling conventions.
\end{enumerate}


\paragraph{Remark}
A preliminary, work-in-progress version of this work was presented in
a previous publication at a national
conference.\footnote{\nonanon{Citation}{\citet*{tmc-ocaml-2021}}} We
explain our choice to submit a full paper to an international
conference in \cref{app:extended-republication}.

%%% Local Variables:
%%% mode: latex
%%% TeX-master: "main"
%%% End:
