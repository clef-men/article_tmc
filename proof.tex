\section{Proof of the specification}
\label{sec:proof}

%\subsection{Logical relation on open expressions}
%
%The assertion $\iSim[\iProt]{\iPred}{\datalangExpr_s}{\datalangExpr_t}$ relates closed terms that may contain locations, as may be produced by our operational semantics. To prove the correctness of a program transformation, we proceed by induction on its definition on well-formed source terms, that do not contain locations and contain free variables.
%
%To bridge the gap between the two sorts of terms, we define in Figure~\ref{fig:rsim} a \emph{runtime relation} $\iRsimv[\iProt]{\iPred}{\datalangExpr_s}{\datalangExpr_t}$ that closes over all bisubstitutions $\datalangBisubst$ of related values for free variables. This sort of definition is standard (considering our simulation as an untyped logical relation), it is simply called the ``logical relation'' in \Simuliris.
%
%Our theorems state that terms related by the TMC transformation are in the runtime relation $\iRsimv[\iProt]{\iPred}{\datalangExpr_s}{\datalangExpr_t}$, and their proofs expand its definition to introduce an arbitrary bisubstitution $\datalangBisubst$ and work directly on $\iSim[\iProt]{\iPred}{\datalangExpr_s}{\datalangExpr_t}$ using our reasoning rules.


% Those two statements are proven together.
% (First by induction on \datalangExpr_s, then on the two transformation relations mutually-inductively.)

% Note: the \geq here is a "parametric" simulation, we assume that all
% free variables of \datalangExpr_s, \datalangExpr_t (they must be the same) are replaced by
% externally-equivalent values.

% Taking a step back: to prove the contextual refinements between programs,
% we have to check each function,
%   @f vs refined by @f vt  (for vs externally-equiv. vt)
% for this we can use the adequacy lemma
%   f vs simulated by @f vt (for the empty protocol) into internal value equivalence
% and for this we use the simulation closure theorem
%   f vs simulated by @f vt (for the Xtmc protocol)
% assuming the Xtmc protocol is valid/adequate.
%
% But then this is trivial, just use case (4) right away.
% So the meat of the proof is in showing that the Xtmc protocol is valid.

% To prove that the Xtmc propocol is valid:
% - in the direct case, we have a function application on both sides,
%   so we make a pure step on both sides and we have to show that the function bodies
%   (with a bisubstitution (x mapsto (vs, vt)) applied) are in the relation.


\begin{figure}[tp]
    \begin{align*}
    		\wf{\datalangBisubst}
    		&\coloneqq
    		\forall \datalangVar \ldotp
    		\exists \datalangVal_s, \datalangVal_t \ldotp
    		\datalangBisubst (\datalangVar) = (\datalangVal_s, \datalangVal_t) \iSep
    		\datalangVal_s \iSimilar \datalangVal_t
    	\\
    		\iRsimv[\iProt]{\iPred}{\datalangExpr_s}{\datalangExpr_t}
    		&\coloneqq
    		\forall \datalangBisubst \ldotp
    		\wf{\datalangBisubst} \iWand
    		\iSimv[\iProt]{\iPred}{\datalangBisubst_s (\datalangExpr_s)}{\datalangBisubst_t (\datalangExpr_t)}
    	\\
    	   \iRsimvHoare[\iProt]{P}{\iPred}{\datalangExpr_s}{\datalangExpr_t}
    	   &\coloneqq
    	   \iPersistent \left( P \iWand \iRsimv[\iProt]{\iPred}{\datalangExpr_s}{\datalangExpr_t} \right)
    \end{align*}
    \caption{Logical relation on open expressions}
    \label{fig:rsim}
\end{figure}