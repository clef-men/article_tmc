\section{TMC formalized: transformation}

To define the TMC transformation, we draw inspiration from the work of \citeauthor{DBLP:journals/corr/abs-2102-09823}~\cite{DBLP:journals/corr/abs-2102-09823}.
In our presentation, formalized in \cref{fig:tmc}, transforming a program $\datalangProg$ consists in:

\paragraph{1) Choosing a subset of toplevel functions to be TMC-transformed}
In \OCaml, the programmer has to annotate these.
For each such function $\datalangFn$, we also require a fresh function name (that is not defined in $\datalangProg$) that will be defined to the \emph{destination-passing style} (DPS) version of $\datalangFn$ in the transformed program.
Formally, this is given by a renaming $\datalangRenaming$ as a parameter of the auxiliary transformations we describe next.

\paragraph{2) For each TMC-transformed function $\datalangFn$, computing its DPS transform}
Actually, it is not unique.
In \OCaml, the ambiguity is resolved by annotations and rules described in~\cite{DBLP:journals/corr/abs-2102-09823} (section 4).
To represent all possible DPS transformations, we introduce in \cref{fig:tmc_dps} the relations $\datalangDef_s \tmcDps{\datalangRenaming} \datalangDef_t$ for definitions and $(\datalangExpr_\mathit{dst}, \datalangExpr_\mathit{idx}, \datalangExpr_s) \tmcDps{\datalangRenaming} \datalangExpr_t$ for expressions.

$\datalangDef_s \tmcDps{\datalangRenaming} \datalangDef_t$ expresses that:
1) $\datalangDef_t$ has an additional parameter representing the destination where it must write its result: the location of a memory block $\datalangVar_\mathit{dst}$ along with the index $\datalangVar_\mathit{idx}$ of a particular field in this block.
2) The body of $\datalangDef_t$ is the DPS transform of the body of $\datalangDef_s$ under the given destination.

$(\datalangExpr_\mathit{dst}, \datalangExpr_\mathit{idx}, \datalangExpr_s) \tmcDps{\datalangRenaming} \datalangExpr_t$ expresses that $\datalangExpr_t$ is the DPS transform of $\datalangExpr_s$ under destination $(\datalangExpr_\mathit{dst}, \datalangExpr_\mathit{idx})$.
Intuitively, this means $\datalangExpr_t$ computes the same thing as $\datalangExpr_s$ but writes it into the destination instead of returning it.
We will formalize this intuition in \cref{sec:sketch}.

Most rules are straightforward.
Whenever it is possible, we take the direct transformation as described next of sub-expressions that are not inductively DPS-transformed.

The core of the TMC transformation lies in the symmetric \RefTirName{DPSBlock1} and \RefTirName{DPSBlock2} rules:
1) We choose a side to evaluate first and partially initialize a new memory block with it.
2) We write this block to the destination.
3) We fill the uninitialized field by passing it as destination to the DPS-transformed corresponding source sub-expression.

\paragraph{3) For each function $\datalangFn$ defined in $\datalangProg$, computing its direct transform.}
As before, as it is not unique, we introduce in \cref{fig:tmc_dir} the relations $\datalangDef_s \tmcDir{\datalangRenaming} \datalangDef_t$ for definitions and $\datalangExpr_s \tmcDir{\datalangRenaming} \datalangDef_t$ for expressions.

$\datalangDef_s \tmcDir{\datalangRenaming} \datalangDef_t$ expresses that:
1) $\datalangDef_t$ has the same calling convention as $\datalangDef_s$.
2) The body of $\datalangDef_t$ is the direct transform of the body of $\datalangDef_s$.

$\datalangExpr_s \tmcDir{\datalangRenaming} \datalangDef_t$ expresses that $\datalangExpr_t$ is the direct transform of $\datalangExpr_s$.
Intuitively, this means $\datalangExpr_t$ computes the same thing as $\datalangExpr_s$.

Again, most cases are straightforward.
The symmetric rules \RefTirName{DirBlockDPS1} and \RefTirName{DirBlockDPS2} are the equivalent of \RefTirName{DPSBlock1} and \RefTirName{DPSBlock2}:
1) We choose a side to evaluate first and partially initialize a new memory block with it.
2) We trigger the DPS transformation on the other side, passing the uninitialized field as destination.
3) We return the now fully initialized block.

\medskip

As an example, we can show that the two \DataLang functions defined in \cref{fig:map_tmc} are respectively the direct and DPS transforms of the original \datalang|map| function defined in \cref{fig:map}.
Thanks to the rules \RefTirName{DirBlockDPS1} and \RefTirName{DPSCall}, the non-tail recursive call is replaced by the allocation of a partially initialized block followed by a call to \datalang|map_dps|.

\begin{figure}[tp]
    \[
        \datalangProg_s \tmc \datalangProg_t
        \coloneqq
        \exists \datalangRenaming \ldotp
        \bigwedge \left[ \begin{array}{l}
                \dom{\datalangRenaming} \subseteq \dom{\datalangProg_s}
            \\
                \dom{\datalangProg_t} = \dom{\datalangProg_s} \cup \codom{\datalangRenaming}
            \\
                \forall \datalangFn, \datalangDef_s \ldotp
                \datalangProg_s [\datalangFn] = \datalangDef_s \implies
                \exists \datalangDef_t \ldotp
                \datalangProg_t [\datalangFn] = \datalangDef_t \wedge \datalangDef_s \tmcDir{\datalangRenaming} \datalangDef_t
            \\
                \forall \datalangFn, \datalangFn_\mathit{dps}, \datalangDef_s \ldotp
                \datalangProg_s [\datalangFn] = \datalangDef_s \wedge \datalangRenaming [\datalangFn] = \datalangFn_{\mathit{dps}} \implies
                \exists \datalangDef_t \ldotp
                \datalangProg_t [\datalangFn_\mathit{dps}] = \datalangDef_t \wedge \datalangDef_s \tmcDps{\datalangRenaming} \datalangDef_t
        \end{array} \right.
    \]
    \caption{TMC transformation}
    \label{fig:tmc}
\end{figure}
\begin{figure}[tp]
    \begin{mathparpagebreakable}
        \inferrule*
            {}{
                \boxed{- \tmcDps{\xi} - : \nterm{Expr} \times \nterm{Expr} \times \nterm{Expr} \rightarrow \nterm{Expr} \rightarrow \Prop}
            }
        \and
        \inferrule*
            {}{
                \boxed{- \tmcDps{\xi} - : \nterm{Def} \rightarrow \nterm{Def} \rightarrow \Prop}
            }
        \\
        \inferrule*[lab=DPSBase]
            {
                e_s \tmcDir{\xi} e_t
            }{
                \left(
                    \mathit{dst},
                    \mathit{idx},
                    e_s
                \right)
                \tmcDps{\xi}
                \lambdStore{\mathit{dst}}{i}{e_t}
            }
        \and
        \inferrule*[lab=DPSLet]
            {
                e_{s1} \tmcDir{\xi} e_{t1}
            \and
                (\mathit{dst}, \mathit{idx}, e_{s2}) \tmcDps{\xi} e_{t2}
            }{
                \left(
                    \mathit{dst},
                    \mathit{idx},
                    \lambdLet{x}{e_{s1}}{e_{s2}}
                \right)
                \tmcDps{\xi}
                \lambdLet{x}{e_{t1}}{e_{t2}}
            }
        \and
        \inferrule*[lab=DPSCall]
            {
                f \in \dom{\xi}
            \and
                e_s \tmcDir{\xi} e_t
            }{
                \left(
                    \mathit{dst},
                    \mathit{idx},
                    \lambdCall{\lambdFunc{f}}{e_s}
                \right)
                \tmcDps{\xi}
                \lambdCall{\lambdFunc{\xi [f]}}{e_t}
            }
        \and
        \inferrule*[lab=DPSIf]
            {
                e_{s0} \tmcDir{\xi} e_{t0}
            \and
                (\mathit{dst}, \mathit{idx}, e_{s1}) \tmcDps{\xi} e_{s2}
            \and
                (\mathit{dst}, \mathit{idx}, e_{s2}) \tmcDps{\xi} e_{t2}
            }{
                \left(
                    \mathit{dst},
                    \mathit{idx},
                    \lambdIf{e_{s0}}{e_{s1}}{e_{s2}}
                \right)
                \tmcDps{\xi}
                \lambdIf{e_{t0}}{e_{t1}}{e_{t2}}
            }
        \\
        \inferrule*[lab=DPSConstr1]
            {
                e_{s1} \tmcDir{\xi} e_{t1}
            \and
                (\mathit{dst}, \lambdTwo, e_{s2}) \tmcDir{\xi} e_{t2}
            }{
                \left(
                    \mathit{dst},
                    \mathit{idx},
                    \lambdConstr{c}{e_{s1}}{e_{s2}}
                \right)
                \tmcDps{\xi}
                \begin{array}{l}
                    \lambdLet{\mathit{dst'}}{\lambdConstr{c}{e_{t1}}{\lambdHole}}{\\
                    \lambdSeq{\lambdStore{\mathit{dst}}{\mathit{idx}}{\mathit{dst'}}}{e_{t2}}
                    }
                \end{array}
            }
        \and
        \inferrule*[lab=DPSConstr2]
            {
                e_{s2} \tmcDir{\xi} e_{t2}
            \and
                (\mathit{dst}, \lambdOne, e_{s1}) \tmcDir{\xi} e_{t1}
            }{
                \left(
                    \mathit{dst},
                    \mathit{idx},
                    \lambdConstr{c}{e_{s1}}{e_{s2}}
                \right)
                \tmcDps{\xi}
                \begin{array}{l}
                    \lambdLet{\mathit{dst'}}{\lambdConstr{c}{\lambdHole}{e_{t2}}}{\\
                    \lambdSeq{\lambdStore{\mathit{dst}}{\mathit{idx}}{\mathit{dst'}}}{e_{t1}}
                    }
                \end{array}
            }
        \\
        \inferrule*[lab=DPSDef]
            {
                (\mathit{dst}, \mathit{idx}, e_s) \tmcDps{\xi} e_t
            }{
                \lambdDef{x}{e_s}
                \tmcDps{\xi}
                {\begin{array}{l}
                    \lambdDef{y}{\\ \ \begin{array}{l}
                        \lambdLet{z}{\lambdLoad{y}{\lambdOne}}{\\
                        \lambdLet{\mathit{dst}}{\lambdLoad{z}{\lambdOne}}{\\
                        \lambdLet{\mathit{idx}}{\lambdLoad{z}{\lambdTwo}}{\\
                        \lambdLet{x}{\lambdLoad{y}{\lambdTwo}}{\\
                        e_t
                        }}}}
                    \end{array}}
                \end{array}}
            }
    \end{mathparpagebreakable}
    \caption{Destination-passing style TMC transformation (omitting freshness conditions)}
    \label{fig:tmc_dps}
\end{figure}
\begin{figure}[tp]
    \begin{mathparpagebreakable}
        \inferrule*
            {}{
                \boxed{- \tmcDir{\datalangRenaming} - : \datalangExpr[] \rightarrow \datalangExpr[] \rightarrow \Prop}
            }
        \and
        \inferrule*
            {}{
                \boxed{- \tmcDir{\datalangRenaming} - : \datalangDef[] \rightarrow \datalangDef[] \rightarrow \Prop}
            }
        \\
        \inferrule*[lab=DirVal]
            {}{
                \datalangVal
                \tmcDir{\datalangRenaming}
                \datalangVal
            }
        \and
        \inferrule*[lab=DirVar]
            {}{
                \datalangVar
                \tmcDir{\datalangRenaming}
                \datalangVar
            }
        \and
        \inferrule*[lab=DirLet]
            {
                \datalangExpr_{s1} \tmcDir{\datalangRenaming} \datalangExpr_{t1}
            \and
                \datalangExpr_{s2} \tmcDir{\datalangRenaming} \datalangExpr_{t2}
            }{
                \datalangLet{\datalangVar}{\datalangExpr_{s1}}{\datalangExpr_{s2}}
                \tmcDir{\datalangRenaming}
                \datalangLet{\datalangVar}{\datalangExpr_{t1}}{\datalangExpr_{t2}}
            }
        \\
        \inferrule*[lab=DirBlockDPS1]
            {
                \datalangExpr_{s1} \tmcDir{\datalangRenaming} \datalangExpr_{t1}
            \and
                (\datalangVar, \datalangTwo, \datalangExpr_{s2}) \tmcDps{\datalangRenaming} \datalangExpr_{t2}
            }{
                \datalangBlock{\datalangTag}{\datalangExpr_{s1}}{\datalangExpr_{s2}}
                \tmcDir{\datalangRenaming}
                \begin{array}{l}
                    \datalangLet{\datalangVar}{\datalangBlock{\datalangTag}{\datalangExpr_{t1}}{\datalangHole}}{\\
                    \datalangSeq{\datalangExpr_{t2}}{\datalangVar}}
                \end{array}
            }
        \and
        \inferrule*[lab=DirBlockDPS2]
            {
                \datalangExpr_{s2} \tmcDir{\datalangRenaming} \datalangExpr_{t2}
            \and
                (\datalangVar, \datalangOne, \datalangExpr_{s1}) \tmcDps{\datalangRenaming} \datalangExpr_{t1}
            }{
                \datalangBlock{\datalangTag}{\datalangExpr_{s1}}{\datalangExpr_{s2}}
                \tmcDir{\datalangRenaming}
                \begin{array}{l}
                    \datalangLet{\datalangVar}{\datalangBlock{\datalangTag}{\datalangHole}{\datalangExpr_{t2}}}{\\
                    \datalangSeq{\datalangExpr_{t1}}{\datalangVar}}
                \end{array}
            }
        \\
        \inferrule*[lab=DirDef]
            {
                \datalangExpr_s \tmcDir{\datalangRenaming} \datalangExpr_t
            }{
                \datalangRec{\datalangVar}{\datalangExpr_s}
                \tmcDir{\datalangRenaming}
                \datalangRec{\datalangVar}{\datalangExpr_t}
            }
    \end{mathparpagebreakable}
    \caption{Direct TMC transformation (omitting congruence rules similar to \TirName{DirLet} and freshness conditions)}
    \label{fig:tmc_dir}
\end{figure}
\begin{figure}[tp]
\begin{minipage}{.40\columnwidth}
\begin{Datalang}
map |-> rec (fn, xs) =
  match xs with
  | [] ->
      []
  | x :: xs ->
      let y = fn x in
      let dst = y :: ? in
      @map_dps ((dst, 2), fn, xs) ;
      dst
\end{Datalang}
\end{minipage}
\hfill
\begin{minipage}{.52\columnwidth}
\begin{Datalang}
map_dps |-> rec ((dst, idx), fn, xs) =
  match xs with
  | [] ->
      dst.(idx) <- []
  | x :: xs ->
      let y = fn x in
      let dst' = y :: ? in
      dst.(idx) <- dst' ;
      @map_dps ((dst', 2), fn, xs)
\end{Datalang}
\end{minipage}
\caption{Direct and DPS transforms of \datalang|map| (\cref{fig:map})}
\label{fig:map_tmc}
\end{figure}