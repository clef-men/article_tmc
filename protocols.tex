\section{Abstract protocols} \label{sec:protocols} \label{subsec:protocols}

\begin{figure}[tp]
    \begin{tabular}{rcl}
            $\iProt_\mathrm{dir} (\iPredTwo, \datalangExpr_s, \datalangExpr_t)$
            & $\coloneqq$ &
            $\exists \datalangFn, \datalangVal_s, \datalangVal_t \ldotp$
        \\
            &&
            $\datalangFn \in \dom{\datalangProg_s} \iSep
            \datalangExpr_s = \datalangCall{\datalangFnptr{\datalangFn}}{\datalangVal_s} \iSep
            \datalangExpr_t = \datalangCall{\datalangFnptr{\datalangFn}}{\datalangVal_t} \iSep {}$
        \\
            &&
            $\datalangVal_s \iSimilar \datalangVal_t \iSep {}$
        \\
            &&
            $\forall \datalangValTwo_s, \datalangValTwo_t \ldotp
            \datalangValTwo_s \iSimilar \datalangValTwo_t \iWand
            \iPredTwo (\datalangValTwo_s, \datalangValTwo_t)$
        \\
            $\iProt_\mathrm{DPS} (\iPredTwo, \datalangExpr_s, \datalangExpr_t)$
            & $\coloneqq$ &
            $\exists \datalangFn, \datalangFn_\mathit{dps}, \datalangVal_s, \datalangLoc_1, \datalangLoc_2, \datalangLoc, \datalangIdx, \datalangVal_t \ldotp$
        \\
            &&
            $\datalangFn \in \dom{\datalangProg_s} \iSep
            \datalangRenaming [\datalangFn] = \datalangFn_\mathit{dps} \iSep
            \datalangExpr_s = \datalangCall{\datalangFnptr{\datalangFn}}{\datalangVal_s} \iSep
            \datalangExpr_t = \datalangCall{\datalangFnptr{\datalangFn_\mathit{dps}}}{\datalangLoc_1} \iSep {}$
        \\
            &&
            $(\datalangLoc_1 + 1) \iPointsto_t (\datalangLoc_2, \datalangVal_t) \iSep
            (\datalangLoc_2 + 1) \iPointsto_t (\datalangLoc, \datalangIdx) \iSep
            (\datalangLoc + \datalangIdx) \iPointsto \datalangHole \iSep
            \datalangVal_s \iSimilar \datalangVal_t$
        \\
            &&
            $\forall \datalangValTwo_s, \datalangValTwo_t \ldotp
            (\datalangLoc + \datalangIdx) \iPointsto \datalangValTwo_t \iSep
            \datalangValTwo_s \iSimilar \datalangValTwo_t \iWand
            \iPredTwo (\datalangValTwo_s, \datalangUnit)$
        \\
            $\iProt_\mathrm{TMC}$
            & $\coloneqq$ &
            $\iProt_\mathrm{dir} \sqcup \iProt_\mathrm{DPS}
            =
            \lambdaAbs (\iPredTwo, \datalangExpr_s, \datalangExpr_t) \ldotp \iProt_\mathrm{dir} (\iPredTwo, \datalangExpr_s, \datalangExpr_t) \iOr \iProt_\mathrm{DPS} (\iPredTwo, \datalangExpr_s, \datalangExpr_t)$
    \end{tabular}
    \caption{TMC protocol ($\iProt_\mathrm{TMC}$)}
    \label{fig:protocol}
\end{figure}

In \cref{sec:simulation}, we explain how our relation is defined coinductively and the first step of the proof essentially amounts to coinduction.
To internalize the coinduction hypothesis into the program logic, we introduce an additional parameter $\iProt$, a \emph{protocol}~\citep*{protocols-2021}, which is a general proof-state transformer of type
\begin{mathline}
(\datalangExpr[] \to \datalangExpr[] \to \iProp) \to \datalangExpr[] \to \datalangExpr[] \to \iProp
\end{mathline}

Protocols are used in the logic via the $\RefTirName{RelProtocol}$ rule.
A pair of expressions $\datalangExpr_s$ and $\datalangExpr_t$ is supported by the protocol when it relates them to a postcondition $\iPredTwo$, capturing the possible results of an abstract/axiomatic transition from $\datalangExpr_s$ and $\datalangExpr_t$.
To conclude that $\datalangExpr_s$ and $\datalangExpr_t$ are related, one must prove that any two $\datalangExpr_s'$ and $\datalangExpr_t'$ accepted by this postcondition $\iPredTwo$ remain related.

\subsection{TMC protocols}

In our correctness proof for the TMC transformation, we use a specific protocol $\iProt_\mathrm{TMC}$ defined in \cref{fig:protocol} by combining two sub-protocols $\iProt_\mathrm{dir}$ and $\iProt_\mathrm{DPS}$ for the direct-style and DPS-style functions. Our coinduction hypothesis assumes toplevel function calls to be compatible with the direct and DPS specifications that we want to prove, and allows to reason about recursive calls to those functions inside the function bodies we are trying to relate.

$\iProt_\mathrm{dir}$ specifies the \emph{direct calling convention} induced by the direct transformation.
It requires $\datalangExpr_s$ and $\datalangExpr_t$ to be function calls to the same function with similar arguments.
To apply it, users can choose any postcondition implied by value similarity.
This rule is equivalent to the \TirName{Sim-Call} rule of \Simuliris.
Most useful protocols are formed by combining $\iProt_\mathrm{dir}$ with other, more specialized protocols.

$\iProt_\mathrm{DPS}$ specifies the \emph{DPS calling convention} induced by the DPS transformation.
It requires $\datalangExpr_s$ to be a function call to a TMC-transformed function $\datalangFn$ and $\datalangExpr_t$ to be a function call to the DPS transform of $\datalangFn$.
As in the DPS specification, ownership of the destination must be passed to the protocol.
To apply it, users can choose any postcondition implied by the postcondition of the DPS specification, including the recovered ownership of the modified destination.

% In the instantiated program logic, the application of the $\iProt_\mathrm{TMC}$ protocol is equivalent to the following rules:
% \begin{mathpar}
%     \inferrule*[lab=RelDir]
%         {
%             \datalangFn \in \dom{\datalangProg_s}
%         \\\\
%             \datalangVal_s \iSimilar \datalangVal_t
%         \\\\
%             \forall \datalangValTwo_s, \datalangValTwo_t \ldotp
%             \datalangValTwo_s \iSimilar \datalangValTwo_t \iWand
%             \iPred (\datalangValTwo_s, \datalangValTwo_t)
%         }{
%             \iSimv{
%                 \iPred
%             }{
%                 \datalangCall{\datalangFnptr{\datalangFn}}{\datalangVal_s}
%             }{
%                 \datalangCall{\datalangFnptr{\datalangFn}}{\datalangVal_t}
%             }
%         }
%     \and
%     \inferrule*[lab=RelDPS]
%         {
%             \datalangRenaming [\datalangFn] = \datalangFn_\mathit{dps}
%         \\\\
%             (\datalangLoc_1 + 1) \iPointsto_t (\datalangLoc_2, \datalangVal_t)
%         \\\\
%             (\datalangLoc_2 + 1) \iPointsto_t (\datalangLoc, \datalangIdx)
%         \\\\
%             (\datalangLoc + \datalangIdx) \iPointsto_t \datalangHole
%         \\\\
%             \datalangVal_s \iSimilar \datalangVal_t
%         \\\\
%             \forall \datalangValTwo_s, \datalangValTwo_t \ldotp
%             (\datalangLoc + \datalangIdx) \iPointsto_t \datalangValTwo_t \iWand
%             \datalangValTwo_s \iSimilar \datalangValTwo_t \iWand
%             \iPred (\datalangValTwo_s, \datalangUnit)
%         }{
%             \iSimv{
%                 \iPred
%             }{
%                 \datalangCall{\datalangFnptr{\datalangFn}}{\datalangVal_s}
%             }{
%                 \datalangCall{\datalangFnptr{\datalangFn_\mathit{dps}}}{\datalangLoc_1}
%             }
%         }
% \end{mathpar}

\subsection{Other examples of protocols}

Our program logic can be instantiated with other protocols to reason other program transformations.
To demonstrate this generality, we have also verified an inlining and an accumulator-passing-style (APS) transformation --- both included in our mechanization.

\paragraph{Inlining:} Here, the relation $\datalangExpr_s \rightsquigarrow \datalangExpr_t$ allows $\datalangExpr_t$ to recursively inline functions in $\datalangExpr_s$.
As with TMC, it captures all possible inlining strategies.

This relation can be proved correct by using a fairly simple protocol (combined with $\iProt_\mathrm{dir}$) relating a source function and its body:

\begin{tabular}{rcl}
        $\iProt_\mathrm{inline} (\iPredTwo, \datalangExpr_s, \datalangExpr_t)$
        & $\coloneqq$ &
        $\exists \datalangFn, \datalangVar, \datalangExpr_s', \datalangExpr_t', \datalangVal_s, \datalangVal_t \ldotp$
    \\
        &&
        $
        \datalangExpr_s = \datalangCall{\datalangFnptr{\datalangFn}}{\datalangVal_s}
        \ \iSep\ %
        \datalangVal_s \iSimilar \datalangVal_t
        \ \iSep {}$
    \\
        &&
        $\datalangProg_s [\datalangFn] = (\datalangRec{\datalangVar}{\datalangExpr_s'})
        \ \iSep\ %
        \datalangExpr_s' \rightsquigarrow \datalangExpr_t'
        \ \iSep\ %
        \datalangExpr_t = (\datalangLet{\datalangVar}{\datalangVal_t}{\datalangExpr_t'})
        \ \iSep {}$
    \\
        &&
        $\forall \datalangValTwo_s, \datalangValTwo_t \ldotp
        \datalangValTwo_s \iSimilar \datalangValTwo_t \iWand
        \iPredTwo (\datalangValTwo_s, \datalangValTwo_t)$
\end{tabular}
\medskip

\paragraph{Accumulator-passing style}: The APS transformation is a variant of the TMC transformation where the contexts that are made tail-recursive are applications of associative arithmetic operators, typically of the form $(\datalangExpr + \datalangCtxHole)$ or $\datalangExpr_1 + (\datalangExpr_2 \times \datalangCtxHole)$. (See the discussion by \citet*{tmc-koka-2023}.)

We define an APS transformation, after extending \DataLang with integers and arithmetic operations. We verify it with a protocol similar to $\iProt_\mathrm{DPS}$ that allows calling the APS transform of a source function with an integer accumulator: if $\datalangCall{\datalangFn}{\datalangVal_s}$ returns $n$, then $\datalangCall{\datalangFn_\mathit{aps}}{\datalangPair{\datalangVal_{\mathit{acc}}}{\datalangVal_t}}$ returns $\datalangVal_\mathit{acc} + n$.

\medskip
\begin{tabular}{rcl}
        $\iProt_\mathrm{APS} (\iPredTwo, \datalangExpr_s, \datalangExpr_t)$
        & $\coloneqq$ &
        $\exists \datalangFn, \datalangFn_\mathit{aps}, \datalangVal_s, \datalangVal_\mathit{acc}, \datalangVal_t \ldotp$
    \\
        &&
        $\datalangFn \in \dom{\datalangProg_s}
        \ \iSep\ %
        \datalangRenaming [\datalangFn] = \datalangFn_\mathit{aps}
        \iSep {}$
    \\
        &&
        $\datalangVal_s \iSimilar \datalangVal_t
        \ \iSep\ %
        \datalangExpr_s = \datalangCall{\datalangFnptr{\datalangFn}}{\datalangVal_s}
        \ \iSep\ %
        \datalangExpr_t = \datalangCall{\datalangFnptr{\datalangFn_\mathit{aps}}}{\datalangPair{\datalangVal_\mathit{acc}}{\datalangVal_t}}
        \iSep {}$
    \\
        &&
        $\forall \datalangVal_s', \datalangExpr_t' \ldotp$
    \\
        &&
        $\bm{\mathrm{match}}\ \datalangVal_s'\ \bm{\mathrm{with}}\ \datalangNat \Rightarrow \datalangExpr_t' = \datalangVal_\mathit{acc} + \datalangNat \mid \_ \Rightarrow \mathrm{strongly \mathhyphen stuck}_{\datalangProg_t} (\datalangExpr_t')\ \bm{\mathrm{end}} \iWand$
    \\
        &&
        $\iPredTwo (\datalangVal_s', \datalangExpr_t')$
\end{tabular}
\medskip
\Xfrancois{Explain why this guarantee (``$f_\mathit{aps}$ will get stuck'') is needed. Do we want to prove that if the source crashes then the target crashes?}

One subtlety is that our \DataLang language is untyped, so arithmetic operations (here addition) may get stuck on non-integer values. If the function call $\datalangCall {\datalangFnptr \datalangFn} {\datalangVal_s}$ in the source program returns a non-integer value, then the outer context $\datalangVal_\mathit{acc} + \datalangCtxHole$ gets stuck. But\Xfrancois{not easy to follow your reasoning here} in the transformed program, this failure happens inside the body of the APS-transformed function $\datalangFnptr {\datalangFn_\mathit{aps}}$. To represent this failure case in our protocol, the postcondition $\iPredTwo$ relates a non-integer source return value $\datalangVal_s'$ with any strongly stuck expression $\datalangExpr_t'$ in the target. This relies on the generality of our protocols being predicate transformers on expressions, not just values.

%%% Local Variables:
%%% mode: latex
%%% TeX-master: "main"
%%% End:
