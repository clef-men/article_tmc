\section{Simulation}
\label{sec:simulation}

% ---------------------------------------------------------

%\subsection{The quest for the right notion of simulation}
%\label{sec:howto-relation}
%
%The final theorem we want to establish is a behaviour refinement $\datalangProg_s \refined \datalangProg_t$: the behaviours of the transformed program are included in the behaviour of the source program.
%%
%We propose a notion of behaviour refinement that gives a form of total correctness.
%%
%It is termination-preserving, but also (this is less common in the body of work around Iris) safety-preserving and divergence-preserving.
%
%We prove this using a backward simulation argument for our transformations $\datalangExpr_s \tmc \datalangExpr_t$.
%%
%A direct approach, ultimately unsucessful, would be to prove that the relation is in the simulation by induction, on $\datalangExpr_s$ for example.
%%
%This works well for all constructions except for function calls $\datalangCall{\datalangFnptr{\datalangFn}}{\datalangExpr}$: to reason on the relation between a call to $\datalangFnptr{\datalangFn}$ and its transformation (in direct or DPS style), we would need to inline the function definition, which is not structurally decreasing.
%
%This is a common problem to establish simulations, and the solution is to use a form of co-induction.
%%
%In the Iris logic, a natural way to do this would be to use a ``later'' modality.
%%
%But defining simulations using a ``later'' modality is in fact non-trivial; simple definitions do not give the excepted notion of simulation, due to non-intuitive aspects of the step-indexing semantics of ``later'' in the Iris model. This problem is discussed in details in the previous work on Transfinite Iris~\citep*{TODO-transfinite-Iris}, an alternative logic with different axioms and models.
%%
%Sticking to the main Iris logic, a good definition of simulation has been worked out in Simuliris~\citep*{TODO-Simuliris}.
%%
%It avoids using the ``later'' modality by using the notion of coinduction native to Coq. In Simuliris, coinductive reasoning steps correspond to an ``abstract'' case in the simulation relation, used for function applications; an ``open simulation'' allows these abstract steps, while a ``closed simulation'' does not allow them.
%%
%The coinduction principle is encapsulated in a ``simulation closure'' theorem, which states if two terms are related in the open simulation, they are in fact also related in the closed simulation if we (coinductively) assume that foreign function bodies are correct.
%
%Our proof is heavily inspired by the Simuliris work, but we cannot reuse their results directly because the notion of simulation that they provide is not expressive enough for our proof: we need to support transformations that change the calling conventions between source and target programs.
%%
%(Along the way we also simplified the simulation relation by removing proof rules related to concurrency, which we do not consider in this work.)
%
%To support different calling conventions, we make our simulation relation $\iSim[\iProt]{\iPred}{\datalangExpr_s}{\datalangExpr_t}$ parametric over abstract protocols $\iProt$. This extends the technique of \citet*{TODO-paulo} from the unary setting of safety predicates to the binary setting of a relational separation logic.
%
%\subsection{Proof outline}
%
%Once we have decided to reuse and extend the Simuliris work, our argument can be split in three separate parts:
%
%\begin{enumerate}
%
%\item Define our notion of simulation parametric over protocols (calling conventions), prove its adequacy (it implies a behaviour refinement) and establish a ``closure theorem'' as in the Simuliris work.
%%
%  This technical contribution is independent of the TMC language or transformation, and could be reused in other works.
%
%\item Build a relational program logic for our programming language, as a body of lemmas to establish simulations between expressions.
%
%\item Use this program logic to establish the soundness of the TMC transformation.
%%
%The key ingredients are two specifications for the direct and DPS transformations that are proved in a mutually-inductive way, and are used to prove the hypothesis of the closure theorem of our simulation.
%\end{enumerate}

% ---------------------------------------------------------

%Simuliris~\citep*{TODO-simuliris} suggests a definition of simulation relations in Iris. As we explained in \cref{sec:howto-relation}, it carefully avoids using the ``later'' modality by using Coq's native notion of coinduction. Simuliris has several desirable properties for us: it can be defined in the un-modified Iris base logic (unlike Transfinite Iris~\citep*{transfinite-iris}), it is defined in a way that makes it easy to specialize to other programming languages, and it supports showing the preservation of termination. On the other hand, the original definition is complex as it supports concurrency and notions of fairness. We reused the definition of Simuliris, simplified for the sequential setting. We then extended the resulting definition to support our treatment of stuck states as failures, and generalize the notion of external calls to abstract protocols $\iProt$. Most of the ideas below come from the Simuliris work directly, we will explicitly point out the parts that differ.
%
%\paragraph{Relating states} The \emph{state interpretation} $I(\datalangState_s, \datalangState_t)$ relates source and target states. It extends the unary notion of state interpretation predicate $I(\datalangState)$. Recall that in separation logic, formulas contain both pure propositions and ressources, which include in particular logical points-to assertions $\datalangLoc \iPointsto v$. $I(\datalangState)$ enforces the consistency between those logical assertions and the physical state $\datalangState$.
%
%In the relational setting, the state interpretation becomes a relation $I(\datalangState_s, \datalangState_t)$ tracking points-to predicates on either states $\datalangLoc_s \iPointsto_s \datalangVal_s$ and $\datalangLoc_t \iPointsto_t \datalangVal_t$. It is additionally in charge of maintaining the heap bijection mentioned in \cref{TODO} by tracking predicates $\datalangLoc_s \iInBij \datalangLoc_t$. This assertion is persistent: one cannot remove locations from the bijection.
%
%\paragraph{Relating values} TODO relation between values.
%
%\paragraph{Relating expressions} The definition of the expression relation $\iSim[\iProt]{\iPred}{\datalangExpr_s}{\datalangExpr_t}$ is shown in \cref{fig:sim}; it reads as ``$\datalangExpr_s$ simulates $\datalangExpr_t$ under the postcondition $\iPred$ and protocol $\iProt$''. It expresses that the source $\datalangExpr_s$ can simulate any computation of the target $\datalangExpr_t$ until they both get stuck, both diverge, or reach two expressions in the relational post-condition $\iPred$. The parameter $\iProt$ is the relational protocol that must be followed by function calls in $\datalangExpr_s$ and $\datalangExpr_t$ -- generalizing the Simuliris rule for external calls.
%
%\paragraph{Mixed induction and coinduction}
%The definition of the relation itself starts with a double fixpoint, a greatest fixpoint $\nuAbs sim .$ (coinduction) of smallest fixpoint $\muAbs {sim \mathhyphen inner} .$ (induction) of a relation $\mathrm{sim \mathhyphen body}_\iProt$. Inside the relation definition, the cases that use the inductive $sim \mathhyphen inner$ in their recursive occurrence can only be repeated finitely many times, while the cases that use the coinductive $sim$ in their recursive occurrence can be repeated infinitely, but can only do so under a form of guardedness condition.
%
%\paragraph{Simulation clauses} The relation between $\datalangExpr_s$ and $\datalangExpr_t$, at a given post-condition $\iPred$, must be parametric over all physical states in the state interpretation relation $I(\datalangState_s, \datalangState_t)$. It is established by one of the following clauses:
%
%\begin{enumerate}
%\item[\circled{1}] Halting clause: $\datalangExpr_s$, $\datalangExpr_t$ can stop if they are already in the post-condition $\iPred$ -- and the states are still related.
%\item[\circled{2}]
%\item[\circled{3}]
%\item[\circled{4}]
%\item[\circled{5}]
%\end{enumerate}
%
%% sim-body:
%% cas 1, 2, 3, 4: on raconte
%% "ouverte" comprend un autre cas.
%% cas 5: les appels externes, protocoles (on raconte)
%
%\begin{theorem}[Close simulation]
%    \begin{align*}
%            &
%            \iPersistent \left(
%                \forall \iPredTwo, \datalangExpr_s, \datalangExpr_t \ldotp
%                \iProt (\iPredTwo, \datalangExpr_s, \datalangExpr_t) \iSepImp
%                \mathrm{sim \mathhyphen inner}_\bot (\lambdaAbs (\_, \datalangExpr_s', \datalangExpr_t') \ldotp \iSim[\iProt]{\iPredTwo}{\datalangExpr_s'}{\datalangExpr_t'}) (\bot, \datalangExpr_s, \datalangExpr_t)
%            \right) \iSepImp
%        \\
%            &
%            \iSim[\iProt]{\iPred}{\datalangExpr_s}{\datalangExpr_t} \iSepImp
%            \iSim[\bot]{\iPred}{\datalangExpr_s}{\datalangExpr_t}
%    \end{align*}
%\end{theorem}

%\subsection{Soundness theorem}
%
%The TMC transformation preserves the behaviours of programs in a strong sense.
%We expect terminating programs to remain terminating, diverging programs to remain diverging and failing programs (those that get stuck) to remain failing.
%
%To formalize it, we first define the notion of behaviour.
%
%we define in \cref{fig:refinement} the notions of behaviours and behaviour refinement as in \Simuliris~\cite{DBLP:journals/pacmpl/GaherSSJDKKD22}.
%TODO
%
%We define the set $\mathrm{behaviours}_{\datalangProg} (\datalangExpr)$ of behaviours of an expression $\datalangExpr$ within a program $\datalangProg$, starting from an empty store.
%The behaviour $\constr{Conv}(\datalangVal_s)$ indicates that $\datalangExpr$ can evaluate to $\datalangVal_s$.
%The behaviour $\constr{Conv}(\datalangExpr')$, when $\datalangExpr'$ is not a value, indicates that $\datalangExpr$ can reduce to a stuck configuration $(\datalangExpr', \datalangState)$.
%Finally, $\constr{Div}$ indicates that $\datalangProg$ can diverge.
%
%The refinement relation on behaviours $b_s \refined b_t$ then follows in a natural way: divergence refines divergence, any stuck expression refines any other stuck expression, and a value $\datalangVal_t$ refines $\datalangVal_s$ when they are related for a ground equivalence $\datalangVal_s \similar \datalangVal_t$, which is the equality for all value kinds, except for locations where it gives no information, as done in Simuliris.
%This equivalence of ground values could be used, for example, to argue for the correctness of a \texttt{main} function that takes and returns integer arguments.
%
%Note a difference to Simuliris: our refinement between stuck/failing behaviours is $\constr{Div} \refined \constr{Div}$, whereas Simuliris uses $\constr{Div} \refined b_t$: the Simuliris relation assumes that the input program is safe, never gets stuck, and the behaviour refinement thus allows a stuck source term to be refined by any target program. With our definition, a source program that only gets stuck must be transformed into a target program that only gets stuck.
%This property would not be desirable for C compilers that want to optimize aggressively assuming the absence of undefined behaviors, but it does capture a finer-grained property of our program transformation.
%
%Finally, an expression $\datalangExpr_t$ refines $\datalangExpr_s$ when all its behaviours are refined by a behaviour of $\datalangExpr_s$, and a program $\datalangProg_t$ refines $\datalangProg_s$ when calls to source functions on equivalent inputs are in the refinement relation.
%
%We can now state the main soundness theorem of our work, whose proof is spread over the rest of the present \cref{sec:soundness}.
%
%\begin{theorem}[Soundness]
%    $
%        \wf{\datalangProg_s} \wedge \datalangProg_s \tmc \datalangProg_t \implies
%        \datalangProg_s \refined \datalangProg_t
%    $
%\end{theorem}
%
%The condition $\wf{\datalangProg_s}$ guarantees that the input program $\datalangProg_s$ is well-scoped. Under this condition, we can consider the target programs $\datalangProg_t$ that are TMC transformations $\datalangProg_s \tmc \datalangProg_s$, and our theorem states that they refine $\datalangProg_s$ as expected.

%\begin{theorem}[Adequacy]
%    $
%        \left( \vdash \iSimv{\iSimilar}{\datalangExpr_s}{\datalangExpr_t} \right) \implies
%        \datalangExpr_s \refined \datalangExpr_t
%    $
%\end{theorem}

% ---------------------------------------------------------

% I(sigma, sigma´) is as in Simuliris, not given explicitly in the current document.
% the \approx^bij relation refers to a component of I (a bijection between addresses)
% that is an implicit parameter of the definitions.

% ---------------------------------------------------------

% If two expressions refine internally (thy are in the
% simulation relation), and go to the internal equivalence between
% values, then they refine externally (they are in the denotational
% refinement relation).

% This is similar to the Simuliris proof, simplified thanks to the absence
% of concurrency.

% ---------------------------------------------------------

\begin{figure}[tp]
    \begin{align*}
    		\mathrm{sim \mathhyphen body}_\iProt
    		&\coloneqq
    		\begin{array}{l}
    				\lambdaAbs sim \ldotp
    				\lambdaAbs sim \mathhyphen inner \ldotp
    				\lambdaAbs {(\iPred, \datalangExpr_s, \datalangExpr_t)} \ldotp
    				\forall \datalangState_s, \datalangState_s \ldotp
    				\mathrm{I} (\datalangState_s, \datalangState_t)
    				\iSepImp \iBupd
    			\\
    				\bigvee \left[ \begin{array}{ll}
    							\circled{1}
    						&
    							\mathrm{I} (\datalangState_s, \datalangState_t) \iSep
    							\iPred (\datalangExpr_s, \datalangExpr_t)
    					\\
    					        \circled{2}
                            &
                                \mathrm{I} (\datalangState_s, \datalangState_t) \iSep
    							\mathrm{strongly \mathhyphen stuck}_{\datalangProg_s} (\datalangExpr_s) \iSep
    							\mathrm{strongly \mathhyphen stuck}_{\datalangProg_t} (\datalangExpr_s)
    					\\
    							\circled{3}
    						&
    							\exists \datalangExpr_s', \datalangState_s' \ldotp
    							(\datalangExpr_s, \datalangState_s) \step{\datalangProg_s}^+ (\datalangExpr_s', \datalangState_s') \iSep
    							\mathrm{I} (\datalangState_s', \datalangState_t) \iSep
    							sim \mathhyphen inner (\iPred, \datalangExpr_s', \datalangExpr_t)
    					\\
    							\circled{4}
    						&
								\mathrm{reducible}_{\datalangProg_t} (\datalangExpr_t, \datalangState_t) \iSep
								\forall \datalangExpr_t', \datalangState_t' \ldotp
								(\datalangExpr_t, \datalangState_t) \step{\datalangProg_t} (\datalangExpr_t', \datalangState_t')
								\iSepImp \iBupd
						\\
                            &
								\bigvee \left[ \begin{array}{ll}
											\circled{A}
										&
											\mathrm{I} (\datalangState_s, \datalangState_t') \iSep
											sim \mathhyphen inner (\iPred, \datalangExpr_s, \datalangExpr_t')
									\\
											\circled{B}
										&
											\exists \datalangExpr_s', \datalangState_s' \ldotp
											(\datalangExpr_s, \datalangState_s) \step{\datalangProg_s}^+ (\datalangExpr_s', \datalangState_s') \iSep
											\mathrm{I} (\datalangState_s', \datalangState_t') \iSep
											sim (\iPred, \datalangExpr_s', \datalangExpr_t')
								\end{array} \right.
    					\\
    							\circled{5}
    						&
    							\exists \datalangEctx_s, \datalangExpr_s', \datalangEctx_t, \datalangExpr_t', \iPredTwo \ldotp
    					\\
    					    &
    					        \datalangExpr_s = \datalangEctx_s [\datalangExpr_s'] \iSep
    							\datalangExpr_t = \datalangEctx_t [\datalangExpr_t'] \iSep
    						    \iProt (\iPredTwo, \datalangExpr_s', \datalangExpr_t') \iSep
    						    \mathrm{I} (\datalangState_s, \datalangState_t) \iSep {}
    					\\
                            &
								\forall \datalangExpr_s'', \datalangExpr_t'' \ldotp
								\iPredTwo (\datalangExpr_s'', \datalangExpr_t'') \iSepImp
								sim (\iPred, \datalangEctx_s [\datalangExpr_s''], \datalangEctx_t [\datalangExpr_t''])
    				\end{array} \right.
    		\end{array}
    	\\
    	    \mathrm{sim \mathhyphen inner}_\iProt
    	    &\coloneqq
    	    \lambdaAbs sim \ldotp
    	    \muAbs sim \mathhyphen inner \ldotp
    	    \mathrm{sim \mathhyphen body}_\iProt (sim, sim \mathhyphen inner)
    	\\
    		\mathrm{sim}_\iProt
    		&\coloneqq
    		\nuAbs sim \ldotp
    		\mathrm{sim \mathhyphen inner}_\iProt (sim)
    	\\
    		\iSim[\iProt]{\iPred}{\datalangExpr_s}{\datalangExpr_t}
    		&\coloneqq
    		\mathrm{sim}_\iProt (\iPred, \datalangExpr_s, \datalangExpr_t)
    	\\
    	   \iSimv[\iProt]{\iPred}{\datalangExpr_s}{\datalangExpr_t}
    	   &\coloneqq
    	   \iSim[\iProt]{\lambdaAbs (\datalangExpr_s', \datalangExpr_t') \ldotp \exists \datalangVal_s, \datalangVal_t \ldotp \datalangExpr_s' = \datalangVal_s \iSep \datalangExpr_t' = \datalangVal_t \iSep \iPred (\datalangVal_s, \datalangVal_t)}{\datalangExpr_s}{\datalangExpr_t}
    \end{align*}
    \caption{Simulation with protocol}
    \label{fig:sim}
\end{figure}
\begin{figure}[tp]
    \centering
    \begin{mathparpagebreakable}
        \inferrule*
            {}{
                \datalangUnit \similar \datalangUnit
            }
        \and
        \inferrule*
            {}{
                \datalangIdx \similar \datalangIdx
            }
        \and
        \inferrule*
            {}{
                \datalangTag \similar \datalangTag
            }
        \and
        \inferrule*
            {}{
                \datalangBool \similar \datalangBool
            }
        \and
        \inferrule*
            {}{
                \datalangLoc_s \similar \datalangLoc_t
            }
        \and
        \inferrule*
            {}{
                \datalangFnptr{\datalangFn} \similar \datalangFnptr{\datalangFn}
            }
    \end{mathparpagebreakable}
    \caption{Similarity in $\Prop$}
    \label{fig:similarity}
\end{figure}
\begin{figure}[tp]
    \centering
    \begin{mathparpagebreakable}
        \inferrule*
            {
                v_s \similar v_t
            }{
                \constr[v_s]{Conv} \refined \constr[v_t]{Conv}
            }
        \and
        \inferrule*
            {
                e_s \notin \nterm{Val}
            \and
                e_t \notin \nterm{Val}
            }{
                \constr[e_s]{Conv} \refined \constr[e_t]{Conv}
            }
        \and
        \inferrule*
            {}{
                \constr{Div} \refined \constr{Div}
            }
    \end{mathparpagebreakable}
    \\
    \begin{align*}
        	\mathrm{behaviours}_p (e)
        	&\coloneqq
        	\begin{array}{l}
        			\{ \constr[e']{Conv} \mid \exists \sigma \ldotp (e, \emptyset) \step{p}^* (e', \sigma) \wedge \mathrm{irreducible}_p (e', \sigma) \}\ \uplus
        		\\
        			\{ \constr{Div} \mid\ \diverges{p}{(e, \emptyset)} \}
            \end{array}
        \\
            e_s \refined e_t
            &\coloneqq
            \forall b_t \in \mathrm{behaviours}_{p_t} (e_t) \ldotp
            \exists b_s \in \mathrm{behaviours}_{p_s} (e_s) \ldotp
            b_s \refined b_t
        \\
            p_s \refined p_t
            &\coloneqq
            \forall f \in \dom{p_t}, v_s, v_t \ldotp
            \wf{v_s} \wedge v_s \similar v_t \implies
            \lambdCall{\lambdFunc{f}}{v_s} \refined \lambdCall{\lambdFunc{f}}{v_t}
    \end{align*}
    \caption{Program refinement}
    \label{fig:refinement}
\end{figure}