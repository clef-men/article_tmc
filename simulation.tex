\section{Simulation}
\label{sec:simulation}

So far, we assumed a program logic satisfying a set of reasoning rules.
In this section, we prove that our rules are sound: they imply a \emph{simulation} à la \Simuliris~\citep*{simuliris-2022}.
This simulation comes with an \emph{adequacy theorem} that allows to extract a \emph{behavioral refinement} in the meta-logic (Coq, without \Iris), our final soundness theorem.

\subsection{Definition}

\begin{figure}[tp]
    \begin{align*}
    		\mathrm{sim \mathhyphen body}_\iProt
    		&\coloneqq
    		\begin{array}{l}
    				\lambdaAbs sim \ldotp
    				\lambdaAbs sim \mathhyphen inner \ldotp
    				\lambdaAbs {(\iPred, \datalangExpr_s, \datalangExpr_t)} \ldotp
    				\forall \datalangState_s, \datalangState_s \ldotp
    				\mathrm{I} (\datalangState_s, \datalangState_t)
    				\iSepImp \iBupd
    			\\
    				\bigvee \left[ \begin{array}{ll}
    							\circled{1}
    						&
    							\mathrm{I} (\datalangState_s, \datalangState_t) \iSep
    							\iPred (\datalangExpr_s, \datalangExpr_t)
    					\\
    					        \circled{2}
                            &
                                \mathrm{I} (\datalangState_s, \datalangState_t) \iSep
    							\mathrm{strongly \mathhyphen stuck}_{\datalangProg_s} (\datalangExpr_s) \iSep
    							\mathrm{strongly \mathhyphen stuck}_{\datalangProg_t} (\datalangExpr_s)
    					\\
    							\circled{3}
    						&
    							\exists \datalangExpr_s', \datalangState_s' \ldotp
    							(\datalangExpr_s, \datalangState_s) \step{\datalangProg_s}^+ (\datalangExpr_s', \datalangState_s') \iSep
    							\mathrm{I} (\datalangState_s', \datalangState_t) \iSep
    							sim \mathhyphen inner (\iPred, \datalangExpr_s', \datalangExpr_t)
    					\\
    							\circled{4}
    						&
								\mathrm{reducible}_{\datalangProg_t} (\datalangExpr_t, \datalangState_t) \iSep
								\forall \datalangExpr_t', \datalangState_t' \ldotp
								(\datalangExpr_t, \datalangState_t) \step{\datalangProg_t} (\datalangExpr_t', \datalangState_t')
								\iSepImp \iBupd
						\\
                            &
								\bigvee \left[ \begin{array}{ll}
											\circled{A}
										&
											\mathrm{I} (\datalangState_s, \datalangState_t') \iSep
											sim \mathhyphen inner (\iPred, \datalangExpr_s, \datalangExpr_t')
									\\
											\circled{B}
										&
											\exists \datalangExpr_s', \datalangState_s' \ldotp
											(\datalangExpr_s, \datalangState_s) \step{\datalangProg_s}^+ (\datalangExpr_s', \datalangState_s') \iSep
											\mathrm{I} (\datalangState_s', \datalangState_t') \iSep
											sim (\iPred, \datalangExpr_s', \datalangExpr_t')
								\end{array} \right.
    					\\
    							\circled{5}
    						&
    							\exists \datalangEctx_s, \datalangExpr_s', \datalangEctx_t, \datalangExpr_t', \iPredTwo \ldotp
    					\\
    					    &
    					        \datalangExpr_s = \datalangEctx_s [\datalangExpr_s'] \iSep
    							\datalangExpr_t = \datalangEctx_t [\datalangExpr_t'] \iSep
    						    \iProt (\iPredTwo, \datalangExpr_s', \datalangExpr_t') \iSep
    						    \mathrm{I} (\datalangState_s, \datalangState_t) \iSep {}
    					\\
                            &
								\forall \datalangExpr_s'', \datalangExpr_t'' \ldotp
								\iPredTwo (\datalangExpr_s'', \datalangExpr_t'') \iSepImp
								sim (\iPred, \datalangEctx_s [\datalangExpr_s''], \datalangEctx_t [\datalangExpr_t''])
    				\end{array} \right.
    		\end{array}
    	\\
    	    \mathrm{sim \mathhyphen inner}_\iProt
    	    &\coloneqq
    	    \lambdaAbs sim \ldotp
    	    \muAbs sim \mathhyphen inner \ldotp
    	    \mathrm{sim \mathhyphen body}_\iProt (sim, sim \mathhyphen inner)
    	\\
    		\mathrm{sim}_\iProt
    		&\coloneqq
    		\nuAbs sim \ldotp
    		\mathrm{sim \mathhyphen inner}_\iProt (sim)
    	\\
    		\iSim[\iProt]{\iPred}{\datalangExpr_s}{\datalangExpr_t}
    		&\coloneqq
    		\mathrm{sim}_\iProt (\iPred, \datalangExpr_s, \datalangExpr_t)
    	\\
    	   \iSimv[\iProt]{\iPred}{\datalangExpr_s}{\datalangExpr_t}
    	   &\coloneqq
    	   \iSim[\iProt]{\lambdaAbs (\datalangExpr_s', \datalangExpr_t') \ldotp \exists \datalangVal_s, \datalangVal_t \ldotp \datalangExpr_s' = \datalangVal_s \iSep \datalangExpr_t' = \datalangVal_t \iSep \iPred (\datalangVal_s, \datalangVal_t)}{\datalangExpr_s}{\datalangExpr_t}
    \end{align*}
    \caption{Simulation with protocol}
    \label{fig:sim}
\end{figure}
\begin{figure}[tp]
    \centering
    \begin{mathparpagebreakable}
        \inferrule*
            {
                v_s \similar v_t
            }{
                \constr[v_s]{Conv} \refined \constr[v_t]{Conv}
            }
        \and
        \inferrule*
            {
                e_s \notin \nterm{Val}
            \and
                e_t \notin \nterm{Val}
            }{
                \constr[e_s]{Conv} \refined \constr[e_t]{Conv}
            }
        \and
        \inferrule*
            {}{
                \constr{Div} \refined \constr{Div}
            }
    \end{mathparpagebreakable}
    \\
    \begin{align*}
        	\mathrm{behaviours}_p (e)
        	&\coloneqq
        	\begin{array}{l}
        			\{ \constr[e']{Conv} \mid \exists \sigma \ldotp (e, \emptyset) \step{p}^* (e', \sigma) \wedge \mathrm{irreducible}_p (e', \sigma) \}\ \uplus
        		\\
        			\{ \constr{Div} \mid\ \diverges{p}{(e, \emptyset)} \}
            \end{array}
        \\
            e_s \refined e_t
            &\coloneqq
            \forall b_t \in \mathrm{behaviours}_{p_t} (e_t) \ldotp
            \exists b_s \in \mathrm{behaviours}_{p_s} (e_s) \ldotp
            b_s \refined b_t
        \\
            p_s \refined p_t
            &\coloneqq
            \forall f \in \dom{p_t}, v_s, v_t \ldotp
            \wf{v_s} \wedge v_s \similar v_t \implies
            \lambdCall{\lambdFunc{f}}{v_s} \refined \lambdCall{\lambdFunc{f}}{v_t}
    \end{align*}
    \caption{Program refinement}
    \label{fig:refinement}
\end{figure}

Our simulation in defined in \cref{fig:sim}.
It is largely inspired by the \Simuliris simulation --- simplified due to the absence of concurrency.
The main difference lies in the protocol clause \circled{5}, which is a generalization of the function call clause of \Simuliris.

\newcommand{\iSimGfp}{\textcolor{\iSimGfpColor}{sim}\xspace}
\newcommand{\iSimLfp}{\textcolor{\iSimLfpColor}{sim-inner}\xspace}

The definition consists of two nested fixpoints \iSimGfp and \iSimLfp.
\iSimGfp is a greatest fixpoint (coinduction) allowing expressions to diverge (in a controlled way, see clause \circled{4}\circled{B}).
\iSimLfp is a least fixpoint (induction) allowing source and target stuttering (see clauses \circled{3} and \circled{4}\circled{A}).
The \emph{state interpretation} $\mathrm{I} (\datalangState_s, \datalangState_t)$ intuitively materializes the invariant of the simulation, including the heap bijection (see \cref{sec:specification}); it is systematically maintained.
We now review the six clauses.

\paragraph{\circled{1} Postcondition.}
The simulation can stop when the postcondition is satisfied.

\paragraph{\circled{2} Stuck expressions.}
The simulation can also stop on simultaneously stuck expressions.

\paragraph{\circled{3} Target stuttering.}
The source expression can angelically take some steps.
\Xfrancois{Could this be just one step?}
\Xgabriel{I think the present form allows to temporarily break the state invariant.}
This can only happen finitely many times, as we continue with \iSimLfp.
If we used \iSimGfp instead, a silent loop in the source could be simulated by anything, breaking preservation of divergence.

\paragraph{\circled{4}\circled{A} Source stuttering.}
The target expression can demonically take one step.
This can also only happen finitely many times, as we continue with \iSimLfp.
If we used \iSimGfp instead, a silent loop in the target could simulate any source expression, breaking preservation of termination.

\paragraph{\circled{4}\circled{B} Synchronization.}
Alternatively, both expressions can simultaneously take one step.
This can happen infinitely many times, as we continue with \iSimGfp.
If we used \iSimLfp instead, we would be unable to relate divergent programs.

\paragraph{\circled{5} Protocol application.}
Finally, we can apply the protocol under evaluation contexts.
We can choose any postcondition $\iPredTwo$ accepted by the protocol and assume it to prove the continuation.
(We justify separately that a protocol is admissible, see \cref{subsec:closure}.)

\subsection{Simulation closure}\label{subsec:closure}

If a protocol $\iProt$ respects a certain admissibility condition, then program relations established using this protocol are also in the \emph{closed} simulation, using the empty protocol $\bot$.

\begin{definition}[Admissibility]
  \label{def:protocol-admissibility}
  A protocol $\iProt$ is admissible, written $\Admissible{\iProt}$, when we have:
  \begin{mathline}
    \iPersistent \left(
      \forall \iPredTwo, \datalangExpr_s, \datalangExpr_t \ldotp
      \iProt (\iPredTwo, \datalangExpr_s, \datalangExpr_t) \iWand
      \mathrm{sim \mathhyphen inner}_\bot (\lambdaAbs (\_, \datalangExpr_s', \datalangExpr_t') \ldotp \iSim[\iProt]{\iPredTwo}{\datalangExpr_s'}{\datalangExpr_t'}) (\bot, \datalangExpr_s, \datalangExpr_t)
    \right)
  \end{mathline}
\end{definition}

In simple terms, the admissibility condition $\Admissible{\iProt}$ states that every triple $(\iPredTwo, \datalangExpr_s, \datalangExpr_t)$ is justified, that is, that $\datalangExpr_s$ and $\datalangExpr_t$ are related.
But the protocol $\iProt$ cannot be used right away to establish this relation (this would allow cyclic, vacuous proofs). Our use of $\mathrm{sim \mathhyphen inner}_\bot$ forces a proof of admissibility to perform some ``productive'' simulation steps with an empty protocol: in this instantiation of the simulation, \iSimLfp uses the empty protocol and \iSimGfp uses $\iProt$, so we have to perform at least one reduction in the source before we can use the protocol again.

\begin{theorem}[Simulation closure]
\label{thm:closure}
  For any protocol $\iProt$, we have:
    \begin{mathline}
            \Admissible{\iProt} \iWand
            \iSim[\iProt]{\iPred}{\datalangExpr_s}{\datalangExpr_t} \iWand
            \iSim[\bot]{\iPred}{\datalangExpr_s}{\datalangExpr_t}
    \end{mathline}
\end{theorem}

Actually, for the TMC protocol, we prove a simpler condition that implies admissibility:
\begin{mathline}
            \iPersistent \left(
                \forall \iPredTwo, \datalangExpr_s, \datalangExpr_t \ldotp
                \iProt (\iPredTwo, \datalangExpr_s, \datalangExpr_t) \iWand
                \exists \datalangExpr_s', \datalangExpr_t' \ldotp
                \datalangExpr_s \pureStep{\datalangProg_s} \datalangExpr_s' \iSep
                \datalangExpr_t \pureStep{\datalangProg_t} \datalangExpr_t' \iSep
                \iSim[\iProt]{\iPredTwo}{\datalangExpr_s'}{\datalangExpr_t'}
            \right)
\end{mathline}
With this weaker version, an admissibility proof must perform exactly one pure step on both sides before the protocol $\iProt$ becomes available again. Other users of our program logic may want to reuse this simpler definition, unless they need the full generality of the $\Admissible{\iProt}$ definition.

\subsection{Adequacy}

Informally, our closed simulation is \emph{adequate} in the sense that if $\datalangExpr_s$ simulates $\datalangExpr_t$, then $\datalangExpr_t$ refines $\datalangExpr_s$, \ie the behaviors of $\datalangExpr_t$ are included in the behaviors of $\datalangExpr_s$:

\begin{theorem}[Simulation adequacy] \label{thm:adequacy}
    $
        \left( \vdash \iSimv[\bot]{\iSimilar}{\datalangExpr_s}{\datalangExpr_t} \right) \implies
        \datalangExpr_s \refined \datalangExpr_t
    $
\end{theorem}

The notions of \emph{behaviors} and \emph{refinement} are defined in \cref{fig:refinement}.
We consider not only converging behaviors (resulting in values or stuck expressions) but also diverging behaviors.
Expression refinement $\datalangExpr_s \refined \datalangExpr_t$ is \emph{termination-preserving}: if $\datalangExpr_s$ always terminates, so does $\datalangExpr_t$.
Program refinement $\datalangProg_s \refined \datalangProg_t$, also defined in \cref{fig:refinement}, requires any source function call in $\datalangProg_t$ to behave as in $\datalangProg_s$.\Xfrancois{ambiguous}

\subsection{Transformation soundness}

We can finally express the soundness of the TMC transformation: if program $\datalangProg_s$ is well-formed and transforms into program $\datalangProg_t$, then $\datalangProg_t$ refines $\datalangProg_s$.
A program is well-formed when its function definitions are well-formed and well-scoped.
%
Note that this statement does not use separation logic. (In our mechanization, it is a pure Coq statement without Iris propositions.)

\begin{theorem}[Transformation soundness] \label{thm:soundness}
    $
        \wf{\datalangProg_s} \wedge \datalangProg_s \tmc \datalangProg_t \implies
        \datalangProg_s \refined \datalangProg_t
    $
\end{theorem}

%The proof uses \cref{thm:closure} to connect \cref{thm:dir} and \cref{thm:adequacy}.

%%% Local Variables:
%%% mode: latex
%%% TeX-master: "main"
%%% End:
