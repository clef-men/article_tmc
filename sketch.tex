\section{Sketch of soundness proof}
\label{sec:sketch}

In this section, we give the soundness theorem for the \DataLang TMC transformation and sketch its proof.
The main idea is to use a \emph{relational separation logic} in the style of \Simuliris~\cite{DBLP:journals/pacmpl/GaherSSJDKKD22} that supports \emph{multiple calling conventions} in the transformed program.

\subsection{Soundness theorem}

The TMC transformation preserves the behaviours of programs in a strong sense.
We expect terminating programs to remain terminating, diverging programs to remain diverging and failing programs (those that get stuck) to remain failing.

To formalize it, we define in \cref{fig:refinement} the notions of behaviours and behaviour refinement as in \Simuliris~\cite{DBLP:journals/pacmpl/GaherSSJDKKD22}.
TODO

We define the set $\mathrm{behaviours}_{\datalangProg} (\datalangExpr)$ of behaviours of an expression $\datalangExpr$ within a program $\datalangProg$, starting from an empty store.
The behaviour $\constr{Conv}(\datalangVal_s)$ indicates that $\datalangExpr$ can evaluate to $\datalangVal_s$.
The behaviour $\constr{Conv}(\datalangExpr')$, when $\datalangExpr'$ is not a value, indicates that $\datalangExpr$ can reduce to a stuck configuration $(\datalangExpr', \datalangState)$.
Finally, $\constr{Div}$ indicates that $\datalangProg$ can diverge.

The refinement relation on behaviours $b_s \refined b_t$ then follows in a natural way: divergence refines divergence, any stuck expression refines any other stuck expression, and a value $\datalangVal_t$ refines $\datalangVal_s$ when they are related for a ground equivalence $\datalangVal_s \similar \datalangVal_t$, which is the equality for all value kinds, except for locations where it gives no information, as done in Simuliris.
This equivalence of ground values could be used, for example, to argue for the correctness of a \texttt{main} function that takes and returns integer arguments.

Note a difference to Simuliris: our refinement between stuck/failing behaviours is $\constr{Div} \refined \constr{Div}$, whereas Simuliris uses $\constr{Div} \refined b_t$: the Simuliris relation assumes that the input program is safe, never gets stuck, and the behaviour refinement thus allows a stuck source term to be refined by any target program. With our definition, a source program that only gets stuck must be transformed into a target program that only gets stuck.
This property would not be desirable for C compilers that want to optimize aggressively assuming the absence of undefined behaviors, but it does capture a finer-grained property of our program transformation.

Finally, an expression $\datalangExpr_t$ refines $\datalangExpr_s$ when all its behaviours are refined by a behaviour of $\datalangExpr_s$, and a program $\datalangProg_t$ refines $\datalangProg_s$ when calls to source functions on equivalent inputs are in the refinement relation.

We can now state the main soundness theorem of our work, whose proof is spread over the rest of the present \cref{sec:soundness}.

\begin{theorem}[Soundness]
    $
        \wf{\datalangProg_s} \wedge \datalangProg_s \tmc \datalangProg_t \implies
        \datalangProg_s \refined \datalangProg_t
    $
\end{theorem}

The condition $\wf{\datalangProg_s}$ guarantees that the input program $\datalangProg_s$ is well-scoped. Under this condition, we can consider the target programs $\datalangProg_t$ that are TMC transformations $\datalangProg_s \tmc \datalangProg_s$, and our theorem states that they refine $\datalangProg_s$ as expected.

\begin{figure}[tp]
    \centering
    \begin{mathparpagebreakable}
        \inferrule*
            {}{
                \datalangUnit \similar \datalangUnit
            }
        \and
        \inferrule*
            {}{
                \datalangIdx \similar \datalangIdx
            }
        \and
        \inferrule*
            {}{
                \datalangTag \similar \datalangTag
            }
        \and
        \inferrule*
            {}{
                \datalangBool \similar \datalangBool
            }
        \and
        \inferrule*
            {}{
                \datalangLoc_s \similar \datalangLoc_t
            }
        \and
        \inferrule*
            {}{
                \datalangFnptr{\datalangFn} \similar \datalangFnptr{\datalangFn}
            }
    \end{mathparpagebreakable}
    \caption{Similarity in $\Prop$}
    \label{fig:similarity}
\end{figure}
\begin{figure}[tp]
    \centering
    \begin{mathparpagebreakable}
        \inferrule*
            {
                v_s \similar v_t
            }{
                \constr[v_s]{Conv} \refined \constr[v_t]{Conv}
            }
        \and
        \inferrule*
            {
                e_s \notin \nterm{Val}
            \and
                e_t \notin \nterm{Val}
            }{
                \constr[e_s]{Conv} \refined \constr[e_t]{Conv}
            }
        \and
        \inferrule*
            {}{
                \constr{Div} \refined \constr{Div}
            }
    \end{mathparpagebreakable}
    \\
    \begin{align*}
        	\mathrm{behaviours}_p (e)
        	&\coloneqq
        	\begin{array}{l}
        			\{ \constr[e']{Conv} \mid \exists \sigma \ldotp (e, \emptyset) \step{p}^* (e', \sigma) \wedge \mathrm{irreducible}_p (e', \sigma) \}\ \uplus
        		\\
        			\{ \constr{Div} \mid\ \diverges{p}{(e, \emptyset)} \}
            \end{array}
        \\
            e_s \refined e_t
            &\coloneqq
            \forall b_t \in \mathrm{behaviours}_{p_t} (e_t) \ldotp
            \exists b_s \in \mathrm{behaviours}_{p_s} (e_s) \ldotp
            b_s \refined b_t
        \\
            p_s \refined p_t
            &\coloneqq
            \forall f \in \dom{p_t}, v_s, v_t \ldotp
            \wf{v_s} \wedge v_s \similar v_t \implies
            \lambdCall{\lambdFunc{f}}{v_s} \refined \lambdCall{\lambdFunc{f}}{v_t}
    \end{align*}
    \caption{Program refinement}
    \label{fig:refinement}
\end{figure}

\subsection{Relational separation logic}

cellules partiellement initialisées
logique de séparation (relationnelle)
bijection entre tas source et tas cible

spécifications directe/DPS

% ----------

The separation logic in which our reasoning rules are formulated is relational: it describes resources from a source and target programs and relations between them. We use the following primite assertions:

\begin{itemize}
\item $\datalangLoc_s \iPointsto_s \datalangVal_s$, a points-to assertion for the source program
\item $\datalangLoc_t \iPointsto_t \datalangVal_t$, for the target program
\item $\datalangLoc_s \iInBij \datalangLoc_t$, which asserts that the source location
  $\datalangLoc_s$ and the target location $\datalangLoc_t$ are in the ``heap
  bijection'' mentioned in \cref{TODO}. This assertion is
  persistent: one cannot remove locations from the bijection.
\end{itemize}

On top of these assertions we define a \emph{value similarity} relation $\datalangVal_s \iSimilar \datalangVal_t$, given in Figure~\ref{fig:isimilar}.  Except for locations, this is exactly the equality. For locations, notice that locations appearing during the reduction of our programs are locations of valid blocks (always of size 3 in our language). We relate them by asking their field locations to be in the heap bijection.

% ----------

\begin{figure}[tp]
	\centering
	\begin{tabular}{|c|ccc||c|ccc|}
		\hline
				\multicolumn{4}{|c||}{\bf{Source program}}
			&
				\multicolumn{4}{c|}{\bf{Transformed program}}
		\\ \hline
				\bf{Heap}
			&
				\multicolumn{3}{c||}{\bf{Position}}
			&
				\bf{Heap}
			&
				\multicolumn{3}{c|}{\bf{Position}}
		\\ \hline
				$\emptyset$
			&
				$\rightarrow$
			&
				$\datalangCall{\datalangText{map}}{\datalangLoc_1}$
			&
			&
				$\emptyset$
			&
				$\rightarrow$
			&
				$\datalangCall{\datalangText{map}}{\datalangLoc_1}$
			&
		\\ \hline
				$\emptyset$
			&
				$\rightarrow$
			&
				$\datalangCall{\datalangText{map}}{\datalangLoc_2}$
			&
			&
				$\datalangLoc_1' \iPointsto (\mathrm{CONS}, \datalangVal_1', \datalangHole)$
			&
				$\rightarrow$
			&
				$\datalangCall{\datalangText{map\_dps}}{\datalangPair{\datalangPair{\datalangLoc_1'}{\datalangTwo}}{\datalangLoc_2}}$
			&
		\\ \hline
				$\emptyset$
			&
				$\rightarrow$
			&
				$\datalangCall{\datalangText{map}}{\datalangLoc_3}$
			&
			&
				$\begin{array}{l}
						\datalangLoc_1' \iPointsto (\mathrm{CONS}, \datalangVal_1', \datalangLoc_2') \iSep {}
					\\
						\datalangLoc_2' \iPointsto (\mathrm{CONS}, \datalangVal_2', \datalangHole)
				\end{array}$
			&
				$\rightarrow$
			&
				$\datalangCall{\datalangText{map_dps}}{\datalangPair{\datalangPair{\datalangLoc_2'}{\datalangTwo}}{\datalangLoc_3}}$
			&
		\\ \hline
				$\textcolor{teal}{\datalangLoc_3'} \iPointsto (\mathrm{NIL}, \datalangUnit, \datalangUnit)$
			&
			&
				$\datalangCall{\datalangText{map}}{\datalangLoc_3}$
			&
				$\rightarrow$
			&
				$\begin{array}{l}
						\datalangLoc_1' \iPointsto (\mathrm{CONS}, \datalangVal_1', \datalangLoc_2') \iSep {}
					\\
						\datalangLoc_2' \iPointsto (\mathrm{CONS}, \datalangVal_2', \datalangLoc_3') \iSep {}
					\\
                        \textcolor{teal}{\datalangLoc_3'} \iPointsto (\mathrm{NIL}, \datalangUnit, \datalangUnit)
				\end{array}$
			&
			&
				$\datalangCall{\datalangText{map\_dps}}{\datalangPair{\datalangPair{\datalangLoc_2'}{\datalangTwo}}{\datalangLoc_3}}$
			&
				$\rightarrow$
		\\ \hline
				$\begin{array}{l}
				        \textcolor{teal}{\datalangLoc_3'} \iPointsto (\mathrm{NIL}, \datalangUnit, \datalangUnit) \iSep {}
                    \\
				        \textcolor{blue}{\datalangLoc_2'} \iPointsto (\mathrm{CONS}, \datalangVal_2', \datalangLoc_3')
				\end{array}$
			&
			&
				$\datalangCall{\datalangText{map}}{\datalangLoc_2}$
			&
				$\rightarrow$
			&
				$\begin{array}{l}
						\datalangLoc_1' \iPointsto (\mathrm{CONS}, \datalangVal_1', \datalangLoc_2') \iSep {}
					\\
						\textcolor{blue}{\datalangLoc_2'} \iPointsto (\mathrm{CONS}, \datalangVal_2', \datalangLoc_3')
					\\
                        \textcolor{teal}{\datalangLoc_3'} \iPointsto (\mathrm{NIL}, \datalangUnit, \datalangUnit)
				\end{array}$
			&
			&
				$\datalangCall{\datalangText{map\_dps}}{\datalangPair{\datalangPair{\datalangLoc_1'}{\datalangTwo}}{\datalangLoc_2}}$
			&
				$\rightarrow$
		\\ \hline
				$\begin{array}{l}
				        \textcolor{teal}{\datalangLoc_3'} \iPointsto (\mathrm{NIL}, \datalangUnit, \datalangUnit) \iSep {}
                    \\
						\textcolor{blue}{\datalangLoc_2'} \iPointsto (\mathrm{CONS}, \datalangVal_2', \datalangLoc_3') \iSep {}
					\\
						\textcolor{red}{\datalangLoc_1'} \iPointsto (\mathrm{CONS}, \datalangVal_1', \datalangLoc_2')
				\end{array}$
			&
			&
				$\datalangCall{\datalangText{map}}{\datalangLoc_1}$
			&
				$\rightarrow$
			&
				$\begin{array}{l}
						\textcolor{red}{\datalangLoc_1'} \iPointsto (\mathrm{CONS}, \datalangVal_1', \datalangLoc_2') \iSep {}
					\\
						\textcolor{blue}{\datalangLoc_2'} \iPointsto (\mathrm{CONS}, \datalangVal_2', \datalangLoc_3')
					\\
                        \textcolor{teal}{\datalangLoc_3'} \iPointsto (\mathrm{NIL}, \datalangUnit, \datalangUnit)
				\end{array}$
			&
			&
				$\datalangCall{\datalangText{map}}{\datalangLoc_1}$
			&
				$\rightarrow$
		\\ \hline
	\end{tabular}
	\captionsetup{format=hang}
	\caption{Execution of \datalang{map} in the source program and the transformed program from the heap ${\datalangLoc_1 \iPointsto (\mathrm{CONS}, \datalangVal_1, \datalangLoc_2) \iSep \datalangLoc_2 \iPointsto (\mathrm{CONS}, \datalangVal_2, \datalangLoc_3) \iSep \datalangLoc_3 \iPointsto (\mathrm{NIL}, \datalangUnit, \datalangUnit)}$}
	\label{fig:heap_bij}
\end{figure}
\begin{figure}[tp]
    \centering
    \begin{mathparpagebreakable}
        \inferrule*
            {}{
                \lambdUnit \iSimilar \lambdUnit
            }
        \and
        \inferrule*
            {}{
                i \iSimilar i
            }
        \and
        \inferrule*
            {}{
                n \iSimilar n
            }
        \and
        \inferrule*
            {}{
                b \iSimilar b
            }
        \and
        \inferrule*
            {
                \forall i \in \{0,1,2\} \ldotp
                (\loc_s + i) \iInBij (\loc_t + i)
            }{
                \loc_s \iSimilar \loc_t
            }
        \and
        \inferrule*
            {
                f \in \dom{p_s}
            }{
                \lambdFunc{f} \iSimilar \lambdFunc{f}
            }
    \end{mathparpagebreakable}
    \caption{Similarity in $\iProp$}
    \label{fig:isimilar}
\end{figure}

\subsection{Coinduction with multiple calling conventions}

coinduction pour gérer cas appels (directs et DPS)
protocoles

\begin{figure}[tp]
    \begin{tabular}{rcl}
            $\iProt_\mathrm{dir} (\iPredTwo, \datalangExpr_s, \datalangExpr_t)$
            & $\coloneqq$ &
            $\exists \datalangFn, \datalangVal_s, \datalangVal_t \ldotp$
        \\
            &&
            $\datalangFn \in \dom{\datalangProg_s} \iSep
            \datalangExpr_s = \datalangCall{\datalangFnptr{\datalangFn}}{\datalangVal_s} \iSep
            \datalangExpr_t = \datalangCall{\datalangFnptr{\datalangFn}}{\datalangVal_t} \iSep {}$
        \\
            &&
            $\datalangVal_s \iSimilar \datalangVal_t \iSep {}$
        \\
            &&
            $\forall \datalangValTwo_s, \datalangValTwo_t \ldotp
            \datalangValTwo_s \iSimilar \datalangValTwo_t \iWand
            \iPredTwo (\datalangValTwo_s, \datalangValTwo_t)$
        \\
            $\iProt_\mathrm{DPS} (\iPredTwo, \datalangExpr_s, \datalangExpr_t)$
            & $\coloneqq$ &
            $\exists \datalangFn, \datalangFn_\mathit{dps}, \datalangVal_s, \datalangLoc_1, \datalangLoc_2, \datalangLoc, \datalangIdx, \datalangVal_t \ldotp$
        \\
            &&
            $\datalangFn \in \dom{\datalangProg_s} \iSep
            \datalangRenaming [\datalangFn] = \datalangFn_\mathit{dps} \iSep
            \datalangExpr_s = \datalangCall{\datalangFnptr{\datalangFn}}{\datalangVal_s} \iSep
            \datalangExpr_t = \datalangCall{\datalangFnptr{\datalangFn_\mathit{dps}}}{\datalangLoc_1} \iSep {}$
        \\
            &&
            $(\datalangLoc_1 + 1) \iPointsto_t (\datalangLoc_2, \datalangVal_t) \iSep
            (\datalangLoc_2 + 1) \iPointsto_t (\datalangLoc, \datalangIdx) \iSep
            (\datalangLoc + \datalangIdx) \iPointsto \datalangHole \iSep
            \datalangVal_s \iSimilar \datalangVal_t$
        \\
            &&
            $\forall \datalangValTwo_s, \datalangValTwo_t \ldotp
            (\datalangLoc + \datalangIdx) \iPointsto \datalangValTwo_t \iSep
            \datalangValTwo_s \iSimilar \datalangValTwo_t \iWand
            \iPredTwo (\datalangValTwo_s, \datalangUnit)$
        \\
            $\iProt_\mathrm{TMC}$
            & $\coloneqq$ &
            $\iProt_\mathrm{dir} \sqcup \iProt_\mathrm{DPS}
            =
            \lambdaAbs (\iPredTwo, \datalangExpr_s, \datalangExpr_t) \ldotp \iProt_\mathrm{dir} (\iPredTwo, \datalangExpr_s, \datalangExpr_t) \iOr \iProt_\mathrm{DPS} (\iPredTwo, \datalangExpr_s, \datalangExpr_t)$
    \end{tabular}
    \caption{TMC protocol ($\iProt_\mathrm{TMC}$)}
    \label{fig:protocol}
\end{figure}