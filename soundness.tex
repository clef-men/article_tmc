\section{Soundness}

(This paragraph below probably rather belongs to the introduction. Abort.)

The formal contribution of our work is a mechanized soundness proof for the TMC transformation on a simplified programming language, exhibiting the salient parts of the actual transformation for OCaml.
%
The destination-passing style used by the TMC transformation critically relies on uniquely-owned mutable state.
%
It is thus natural to use separation logic as our specification language.

\subsection{Proof outline}

The final theorem we want to establish is a behaviour refinement $p_s \refined p_t$: the behaviours of the transformed program are included in the behaviour of the source program.
%
We propose a notion of behaviour refinement that gives a form of total correctness.
%
It is termination-preserving, but also (this is less common in the body of work around Iris) safety-preserving and divergence-preserving.

Our argument can be split in three separate parts:

\begin{enumerate}

\item The proof technique is a simulation argument, following the approach of Simuliris~\citep*{TODO-simuliris}.
%
We cannot however use Simuliris directly, because it assumes that functions calls have the same structure in the source and target programs.
%
We had to build a modified version of Simuliris, simplified to remove concurrency which does not play a role in our work, but extended to support transformations that change the calling conventions between source and target programs.
%
More precisely, we make our simulation relation $\iSim[\Chi]{\phi}{e_s}{e_t}$ parametric over abstract protocols $\Chi$, extending the technique of \citet*{TODO-paulo} from the unary setting of safety predicates to the binary setting of a relational separation logic.
%
This definition of a simulation relation with protocols is independent
of the TMC language or transformation, and could be reused in other
works.

\item We build a program logic on the TMC language, as a body of lemmas to establish simulations between expressions.

\item Finally we use this program logic to establish the soundness of the TMC transformation.
%
The key ingredents are two specifications for the direct and DPS transformations that are proved in a mutually-inductive way.
\end{enumerate}

\subsection{End goal: behaviour refinement}

The TMC transformation preserves the behavior of programs in a strong sense.
%
We expect terminating programs to remain terminating, diverging programs to remain diverging and failing programs (those that get stuck) to remain failing.

ReLoC~\citep*{TODO-reloc} is focused on terminating behaviors for correct programs.
%
Its behaviour refinement guarantees that if the target program $e_t$ evaluates to $v_t$, then the source program $e_s$ evaluates to a $v_s$ related to $v_t$.

Recent works extend the notion of behaviour refinement to preserve termination.
%
If $e_t$ diverges, then $e_s$ diverges; said otherwise, if $e_s$ reduces to a value,
then $e_t$ reduces to a value or gets stuck.

We further extend behaviour refinement to be safety-preserving: if $e_t$ gets stuck, then $e_s$ can get stuck.
%
As we cover all three possible reduction behaviours, we preserve divergence in the following sense: if a source program $e_s$ can only diverge, then the transformed program must diverge.
%
Note that our language is non-deterministic; if a source program $e_s$ can either converge or diverge, then a $e_t$ that refines $e_s$ could exhibit only a subset of those behaviours.

We define this notion of behaviour refinement $e_s \refined e_t$ in Figure~\ref{fig:refinement}.
%
We define the set $\mathrm{behaviours}_p(e)$ of behaviours of an expression $e$ within a program $p$, starting from an empty store.
%
The behaviour $\constr{Conv}(v_s)$ indicates that $e$ can evaluate to $v_s$.
%
The behaviour $\constr{Conv}(e')$, when $e'$ is not a value, indicates that $e$ can reduce to a stuck configuration $(e', \sigma)$.
%
Finally, $\constr{Div}$ indicates that $e$ can diverge.

We define a refinement relation on behaviours in a natural way: divergence refines divergence, any stuck expression refines any other stuck expression, and a value $v_t$ refines $v_s$ when they are related for a ground equivalence $v_s \similar v_t$, which is the equality for all value kinds, except for locations where it gives no information, as done in Simuliris.
%
This equivalence of ground values could be used, for example, to argue for the correctness of a \texttt{main} function that takes and returns integer arguments.

Finally, an expresion $e_t$ refines $e_s$ when all its behaviours are refined by a behaviour of $e_s$, and a program $p_t$ refines $p_s$ when calls to source functions on equivalent inputs are in the refinement relation.

\begin{figure}[tp]
    \centering
    \begin{mathparpagebreakable}
        \inferrule*
            {}{
                \datalangUnit \similar \datalangUnit
            }
        \and
        \inferrule*
            {}{
                \datalangIdx \similar \datalangIdx
            }
        \and
        \inferrule*
            {}{
                \datalangTag \similar \datalangTag
            }
        \and
        \inferrule*
            {}{
                \datalangBool \similar \datalangBool
            }
        \and
        \inferrule*
            {}{
                \datalangLoc_s \similar \datalangLoc_t
            }
        \and
        \inferrule*
            {}{
                \datalangFnptr{\datalangFn} \similar \datalangFnptr{\datalangFn}
            }
    \end{mathparpagebreakable}
    \caption{Similarity in $\Prop$}
    \label{fig:similarity}
\end{figure}
\begin{figure}[tp]
    \centering
    \begin{mathparpagebreakable}
        \inferrule*
            {
                v_s \similar v_t
            }{
                \constr[v_s]{Conv} \refined \constr[v_t]{Conv}
            }
        \and
        \inferrule*
            {
                e_s \notin \nterm{Val}
            \and
                e_t \notin \nterm{Val}
            }{
                \constr[e_s]{Conv} \refined \constr[e_t]{Conv}
            }
        \and
        \inferrule*
            {}{
                \constr{Div} \refined \constr{Div}
            }
    \end{mathparpagebreakable}
    \\
    \begin{align*}
        	\mathrm{behaviours}_p (e)
        	&\coloneqq
        	\begin{array}{l}
        			\{ \constr[e']{Conv} \mid \exists \sigma \ldotp (e, \emptyset) \step{p}^* (e', \sigma) \wedge \mathrm{irreducible}_p (e', \sigma) \}\ \uplus
        		\\
        			\{ \constr{Div} \mid\ \diverges{p}{(e, \emptyset)} \}
            \end{array}
        \\
            e_s \refined e_t
            &\coloneqq
            \forall b_t \in \mathrm{behaviours}_{p_t} (e_t) \ldotp
            \exists b_s \in \mathrm{behaviours}_{p_s} (e_s) \ldotp
            b_s \refined b_t
        \\
            p_s \refined p_t
            &\coloneqq
            \forall f \in \dom{p_t}, v_s, v_t \ldotp
            \wf{v_s} \wedge v_s \similar v_t \implies
            \lambdCall{\lambdFunc{f}}{v_s} \refined \lambdCall{\lambdFunc{f}}{v_t}
    \end{align*}
    \caption{Program refinement}
    \label{fig:refinement}
\end{figure}

We can now state the main soundness theorem of our work, whose proof is spread over the rest of this section.

\begin{theorem}[Soundness]
    $
        \wf{p_s} \wedge p_s \tmc p_t \implies
        p_s \refined p_t
    $
\end{theorem}

\subsection{Proof technique: Simuliris with protocols}

TODO

\begin{theorem}[Close simulation]
    \begin{align*}
            &
            \iPersistent \left(
                \forall \Psi, e_s, e_t \ldotp
                \Chi (\Psi, e_s, e_t) \iSepImp
                \mathrm{sim \mathhyphen inner}_\bot (\lambdaAbs (\_, e_s', e_t') \ldotp \iSim[\Chi]{\Psi}{e_s'}{e_t'}) (\bot, e_s, e_t)
            \right) \iSepImp
        \\
            &
            \iSim[\Chi]{\Phi}{e_s}{e_t} \iSepImp
            \iSim[\bot]{\Phi}{e_s}{e_t}
    \end{align*}
\end{theorem}

\begin{theorem}[Adequacy]
    $
        \left( \vdash \iSimv{\iSimilar}{e_s}{e_t} \right) \implies
        e_s \refined e_t
    $
\end{theorem}

relational separation logic = simulation

problem: direct calling convention in Simuliris, we need two calling conventions (direct + DPS)

remedy: simulation parameterized by protocol, generalization of Simuliris (but: sequential, without source safety but closed at toplevel)

% X_tmc := X_dir \uplus X_dps

% 1. when we are done
% 2. target stuttering (inductive)
% 3. all one-step source movees
%    3a. no source change (inductive)
%    3b. at least one source reduction (coinductive / guarded)
% 4. "open" simulation; X (\Khi) characterizes admissible foreign calls

% A trick from Simuliris:
% sim-after-progress: instance of sim-inner with bottom in well-chosen places
% closed simulation theorem: if the protocol X is "closable",
%   then the simulation relation for X "collapses" into a closed simulation (bottom for X, no case 4)
%
% MYSTERY: from a high-level perspective this "closd simulation theorem" feels a bit obvious:
% its hypothesis says that your protocol X can always be replaced by making progress on both sides,
% so it should be "easy" to transform a simulation proof using 1-2-3-4 into a simulation proof using only 1-2-3.
%
% Question: if it is "easy", cannot you do this right away (in the proof of simulation for a specific X that satisfies
% this property), proving a closed simulation (without 4) directly?
%
% Clément thinks that the answer is that this would require a super
% tricky coinduction, just like the one used in the proof of this
% theorem.
%
% Clément: this theorem is a sort of coinduction principle. You can
% prove an open simulation by induction (no coinduction
% apparently required), and then applying this theorem gives a result
% for which a direct proof would require a coinduction.

DPS calling convention

TMC protocol

% The X for TMC is essentially a relation/first-order rephrasing of
% the relation specs for direct-style and destination-passing-style
% transformations.

heap bijection

% I(sigma, sigma´) is as in Simuliris, not given explicitly in the current document.
% the \approx^bij relation refers to a component of I (a bijection between addresses)
% that is an implicit parameter of the definitions.

adequacy

% If two expressions refine internally (thy are in the
% simulation relation), and go to the internal equivalence between
% values, then they refine externally (they are in the denotational
% refinement relation).

% This is similar to the Simuliris proof, simplified thanks to the absence
% of concurrency.

simulation rules

% Note: those inference rules are in fact Iris propositions
% (the separating conjunction of the premises magic-wands the conclusion)

% Pure reductions: those that do not need state (neither write nor read)

% Gabriel: we could introduc the non-deterministic pair rules into the "pure" relation
% if we changed the pure-target rule to quantify over all e'_t such that e_t -{pure}-> e'_t.
% This would please Gabriel (existential on the left, universal on the right) and be more compact,
% but Clément prefers the current presentation -- it is more standard for Iris folks -- and closer
% to the mechanization.

% SimBijInsert: \approx^bij: Simuliris calls this the escaped/public
% locations, those that have been published.

% Clément now mentions Figure 12, which shows how stores evolve on an
% example in the TMC case. The shape of the relation between the
% source and target stores is sort of implicit in the program logic
% simulations rules, but it is clear on this example.
%
% Clément would explain this to give an informal sense of the proof,
% before even talking about specification logic etc.

closing simulation

proof


\begin{theorem}[Specification of direct transformation]
    \[
        \iCsimvHoare{
            \wf{e_s} \iSep
            e_s \tmcDir{\xi} e_t
        }{
            \iSimilar
        }{
            e_s
        }{
            e_t
        }
    \]
\end{theorem}

\begin{theorem}[Specification of DPS transformation]
    \[
        \iCsimvHoare{
            \wf{e_s} \iSep
            (\loc, i, e_s) \tmcDps{\xi} e_t \iSep
            (\loc + i) \mapsto \lambdHole
        }{
            \lambdaAbs (v_s, w_t) \ldotp
            \exists v_t \ldotp
            w_t = \lambdUnit \iSep
            (\loc + i) \mapsto v_t \iSep
            v_s \iSimilar v_t
        }{
            e_s
        }{
            e_t
        }
    \]
\end{theorem}

% Those two statements are proven together.
% (First by induction on e_s, then on the two transformation relations mutually-inductively.)

% Note: the \geq here is a "parametric" simulation, we assume that all
% free variables of e_s, e_t (they must be the same) are replaced by
% externally-equivalent values.

% Taking a step back: to prove the contextual refinements between programs,
% we have to check each function,
%   @f vs refined by @f vt  (for vs externally-equiv. vt)
% for this we can use the adequacy lemma
%   f vs simulated by @f vt (for the empty protocol) into internal value equivalence
% and for this we use the simulation closure theorem
%   f vs simulated by @f vt (for the Xtmc protocol)
% assuming the Xtmc protocol is valid/adequate.
%
% But then this is trivial, just use case (4) right away.
% So the meat of the proof is in showing that the Xtmc protocol is valid.

% To prove that the Xtmc propocol is valid:
% - in the direct case, we have a function application on both sides,
%   so we make a pure step on both sides and we have to show that the function bodies
%   (with a bisubstitution (x mapsto (vs, vt)) applied) are in the relation.



\begin{figure}[tp]
    \begin{align*}
    		\mathrm{sim \mathhyphen inner}_\Chi
    		&\coloneqq
    		\begin{array}{l}
    				\lambdaAbs sim \ldotp
    				\muAbs sim \mathhyphen inner \ldotp
    				\lambdaAbs{(\Phi, e_s, e_t)} \ldotp
    				\forall \sigma_s, \sigma_s \ldotp
    				\mathrm{I} (\sigma_s, \sigma_t)
    				\iSepImp \iBupd
    			\\
    				\bigvee \left[ \begin{array}{ll}
    							\circled{1}
    						&
    							\Phi (e_s, e_t) \iSep \mathrm{I} (\sigma_s, \sigma_t)
    					\\
    							\circled{2}
    						&
    							\exists e_s', \sigma_s' \ldotp
    							(e_s, \sigma_s) \step{p_s}^+ (e_s', \sigma_s') \iSep
    							\mathrm{I} (\sigma_s', \sigma_t) \iSep
    							sim \mathhyphen inner (\Phi, e_s', e_t)
    					\\
    							\circled{3}
    						&
								\mathrm{reducible} (e_t, \sigma_t) \iSep
								\forall e_t', \sigma_t' \ldotp
								(e_t, \sigma_t) \step{p_t} (e_t', \sigma_t')
								\iSepImp \iBupd
						\\
                            &
								\bigvee \left[ \begin{array}{ll}
											\circled{A}
										&
											\mathrm{I} (\sigma_s, \sigma_t') \iSep
											sim \mathhyphen inner (\Phi, e_s, e_t')
									\\
											\circled{B}
										&
											\exists e_s', \sigma_s' \ldotp
											(e_s, \sigma_s) \step{p_s}^+ (e_s', \sigma_s') \iSep
											\mathrm{I} (\sigma_s', \sigma_t') \iSep
											sim (\Phi, e_s', e_t')
								\end{array} \right.
    					\\
    							\circled{4}
    						&
    							\exists K_s, e_s', K_t, e_t', \Psi \ldotp
    					\\
    					    &
    					        e_s = K_s [e_s'] \iSep
    							e_t = K_t [e_t'] \iSep
    						    \Chi (\Psi, e_s', e_t') \iSep
    						    \mathrm{I} (\sigma_s, \sigma_t) \iSep {}
    					\\
                            &
								\forall e_s'', e_t'' \ldotp
								\Psi (e_s'', e_t'') \iSepImp
								sim (\Phi, K_s [e_s''], K_t [e_t''])
    				\end{array} \right.
    		\end{array}
    	\\
    		\mathrm{sim}_\Chi
    		&\coloneqq
    		\nuAbs{\mathrm{sim \mathhyphen inner}_\Chi}
    	\\
    		\iSim[\Chi]{\Phi}{e_s}{e_t}
    		&\coloneqq
    		\mathrm{sim}_\Chi (\Phi, e_s, e_t)
    	\\
    	   \iSimv[\Chi]{\Phi}{e_s}{e_t}
    	   &\coloneqq
    	   \iSim[\Chi]{\lambdaAbs (e_s', e_t') \ldotp \exists v_s, v_t \ldotp e_s' = v_s \iSep e_t' = v_t \iSep \Phi (v_s, v_t)}{e_s}{e_t}
    \end{align*}
    \caption{Simulation with protocol}
    \label{fig:sim}
\end{figure}
\begin{figure}[tp]
    \begin{mathparpagebreakable}
        \inferrule*[lab=SimPost]
            {
                \iPred (\datalangExpr_s, \datalangExpr_t)
            }{
                \iSim[\iProt]{\iPred}{\datalangExpr_s}{\datalangExpr_t}
            }
        \and
        \inferrule*[lab=SimBind]
            {
                \iSim[\iProt]{\lambdaAbs (\datalangExpr_s', \datalangExpr_t') \ldotp \iSim[\iProt]{\iPred}{\datalangEctx_s [\datalangExpr_s']}{\datalangEctx_t [\datalangExpr_t']}}{\datalangExpr_s}{\datalangExpr_t}
            }{
                \iSim[\iProt]{\iPred}{\datalangEctx_s [\datalangExpr_s]}{\datalangEctx_t [\datalangExpr_t]}
            }
        \\
        \inferrule*[lab=SimSrcPure]
            {
                \datalangExpr_s \pureStep{\datalangProg_s} \datalangExpr_s'
            \and
                \iSim[\iProt]{\iPred}{\datalangExpr_s'}{\datalangExpr_t}
            }{
                \iSim[\iProt]{\iPred}{\datalangExpr_s}{\datalangExpr_t}
            }
        \and
        \inferrule*[lab=SimTgtPure]
            {
                \datalangExpr_t \pureStep{\datalangProg_t} \datalangExpr_t'
            \and
                \iSim[\iProt]{\iPred}{\datalangExpr_s}{\datalangExpr_t'}
            }{
                \iSim[\iProt]{\iPred}{\datalangExpr_s}{\datalangExpr_t}
            }
        \\
        \inferrule*[lab=SimApplyProtocol]
            {
              \iProt (\iPredTwo, \datalangExpr_s, \datalangExpr_t)
            \and
               \forall \datalangExpr_s', \datalangExpr_t' \ldotp
               \iPredTwo (\datalangExpr_s', \datalangExpr_t') \iSepImp
               \iSim[\iProt]{\iPred}{\datalangExpr_s'}{\datalangExpr_t'}
            }{
                \iSim[\iProt]{\iPred}{\datalangExpr_s}{\datalangExpr_t}
            }
        \\
        \inferrule*[lab=SimSrcBlock1]
            {
                \iSim[\iProt]{\iPred}{
                    \begin{array}{l}
                        \datalangLet{\datalangVar_1}{\datalangExpr_{s1}}{\\
                        \datalangLet{\datalangVar_2}{\datalangExpr_{s2}}{\\
                        \datalangBlockDet{\datalangTag}{\datalangVar_1}{\datalangVar_2}}}
                    \end{array}
                }{\datalangExpr_t}
            }{
                \iSim[\iProt]{\iPred}{\datalangBlock{\datalangTag}{\datalangExpr_{s1}}{\datalangExpr_{s2}}}{\datalangExpr_t}
            }
        \and
        \inferrule*[lab=SimSrcBlock2]
            {
                \iSim[\iProt]{\iPred}{
                    \begin{array}{l}
                        \datalangLet{\datalangVar_2}{\datalangExpr_{s2}}{\\
                        \datalangLet{\datalangVar_1}{\datalangExpr_{s1}}{\\
                        \datalangBlockDet{\datalangTag}{\datalangVar_1}{\datalangVar_2}}}
                    \end{array}
                }{\datalangExpr_t}
            }{
                \iSim[\iProt]{\iPred}{\datalangBlock{\datalangTag}{\datalangExpr_{s1}}{\datalangExpr_{s2}}}{\datalangExpr_t}
            }
        \and
        \inferrule*[lab=SimTgtBlock]
            {
                \iSim[\iProt]{\iPred}{\datalangExpr_s}{
                    \begin{array}{l}
                        \datalangLet{\datalangVar_1}{\datalangExpr_{t1}}{\\
                        \datalangLet{\datalangVar_2}{\datalangExpr_{t2}}{\\
                        \datalangBlockDet{\datalangTag}{\datalangVar_1}{\datalangVar_2}}}
                    \end{array}
                }
            \and
                \iSim[\iProt]{\iPred}{\datalangExpr_s}{
                    \begin{array}{l}
                        \datalangLet{\datalangVar_2}{\datalangExpr_{t2}}{\\
                        \datalangLet{\datalangVar_1}{\datalangExpr_{t1}}{\\
                        \datalangBlockDet{\datalangTag}{\datalangVar_1}{\datalangVar_2}}}
                    \end{array}
                }
            }{
                \iSim[\iProt]{\iPred}{\datalangExpr_s}{\datalangBlock{\datalangTag}{\datalangExpr_{t1}}{\datalangExpr_{t2}}}
            }
        \\
        \inferrule*[lab=SimSrcBlockDet]
            {
                \forall \datalangLoc_s \ldotp
                \datalangLoc_s \iPointsto_s (\datalangTag, \datalangVal_{s1}, \datalangVal_{s2}) \iSepImp
                \iSim[\iProt]{\iPred}{\datalangLoc_s}{\datalangExpr_t}
            }{
                \iSim[\iProt]{\iPred}{\datalangBlockDet{\datalangTag}{\datalangVal_{s1}}{\datalangVal_{s2}}}{\datalangExpr_t}
            }
        \and
        \inferrule*[lab=SimTgtBlockDet]
            {
                \forall \datalangLoc_t \ldotp
                \datalangLoc_t \iPointsto_t (\datalangTag, \datalangVal_{t1}, \datalangVal_{t2}) \iSepImp
                \iSim[\iProt]{\iPred}{\datalangExpr_s}{\datalangLoc_t}
            }{
                \iSim[\iProt]{\iPred}{\datalangExpr_s}{\datalangBlockDet{\datalangTag}{\datalangVal_{t1}}{\datalangVal_{t2}}}
            }
        \\
        \inferrule*[lab=SimSrcLoad]
            {
                (\datalangLoc_s + \datalangIdx) \iPointsto_s \datalangVal_s
            \\\\
                (\datalangLoc_s + \datalangIdx) \iPointsto_s \datalangVal_s \iSepImp
                \iSim[\iProt]{\iPred}{\datalangVal_s}{\datalangExpr_t}
            }{
                \iSim[\iProt]{\iPred}{\datalangLoad{\datalangLoc_s}{\datalangIdx}}{\datalangExpr_t}
            }
        \and
        \inferrule*[lab=SimSrcStore]
            {
                (\datalangLoc_s + \datalangIdx) \iPointsto_s \datalangVal_s
            \\\\
                (\datalangLoc_s + \datalangIdx) \iPointsto_s \datalangVal_s' \iSepImp
                \iSim[\iProt]{\iPred}{\datalangUnit}{\datalangExpr_t}
            }{
                \iSim[\iProt]{\iPred}{\datalangStore{\datalangLoc_s}{\datalangIdx}{\datalangVal_s'}}{\datalangExpr_t}
            }
        \\
        \inferrule*[lab=SimTgtLoad]
            {
                (\datalangLoc_t + \datalangIdx) \iPointsto_t \datalangVal_t
            \\\\
                (\datalangLoc_s + \datalangIdx) \iPointsto_t \datalangVal_t \iSepImp
                \iSim[\iProt]{\iPred}{\datalangExpr_s}{\datalangVal_t}
            }{
                \iSim[\iProt]{\iPred}{\datalangExpr_s}{\datalangLoad{\datalangLoc_t}{\datalangIdx}}
            }
        \and
        \inferrule*[lab=SimTgtStore]
            {
                (\datalangLoc_t + \datalangIdx) \iPointsto_t \datalangVal_t
                \\\\
                (\datalangLoc_t + \datalangIdx) \iPointsto_t \datalangVal_t' \iSepImp
                \iSim[\iProt]{\iPred}{\datalangExpr_s}{\datalangUnit}
            }{
                \iSim[\iProt]{\iPred}{\datalangExpr_s}{\datalangStore{\datalangLoc_t}{\datalangIdx}{\datalangVal_t'}}
            }
        \\
        \inferrule*[lab=SimBijInsert]
            {
                \datalangLoc_s \iPointsto_s \datalangVal_s
            \and
                \datalangLoc_t \iPointsto_t \datalangVal_t
            \and
                \datalangVal_s \iSimilar \datalangVal_t
            \and
                \datalangLoc_s \iInBij \datalangLoc_t \iSepImp
                \iSim[\iProt]{\iPred}{\datalangExpr_s}{\datalangExpr_t}
            }{
                \iSim[\iProt]{\iPred}{\datalangExpr_s}{\datalangExpr_t}
            }
        \\
        \inferrule*[lab=SimLoad]
            {
                \datalangLoc_s \iSimilar \datalangLoc_t
            \and
                \forall \datalangVal_s, \datalangVal_t \ldotp
                \datalangVal_s \iSimilar \datalangVal_t
                \iSepImp
                \iPred (\datalangVal_s, \datalangVal_t)
            }{
                \iSim[\iProt]{\iPred}{\datalangLoad{\datalangLoc_s}{\datalangIdx}}{\datalangLoad{\datalangLoc_t}{\datalangIdx}}
            }
        \and
        \inferrule*[lab=SimStore]
            {
                \datalangLoc_s \iSimilar \datalangLoc_t
            \and
                \datalangVal_s \iSimilar \datalangVal_t
            \and
                \iPred (\datalangUnit, \datalangUnit)
            }{
                \iSim[\iProt]{\iPred}{\datalangStore{\datalangLoc_s}{\datalangIdx}{\datalangVal_s}}{\datalangStore{\datalangLoc_t}{\datalangIdx}{\datalangVal_t}}
            }
    \end{mathparpagebreakable}
    \caption{Reasoning rules for simulation (excerpt, omitting freshness conditions)}
    \label{fig:sim_rules}
\end{figure}
\begin{figure}[tp]
    \centering
    \begin{mathparpagebreakable}
        \inferrule*
            {}{
                \lambdUnit \iSimilar \lambdUnit
            }
        \and
        \inferrule*
            {}{
                i \iSimilar i
            }
        \and
        \inferrule*
            {}{
                t \iSimilar t
            }
        \and
        \inferrule*
            {}{
                b \iSimilar b
            }
        \and
        \inferrule*
            {
                \forall i \in \{0,1,2\} \ldotp
                (\loc_s + i) \iInBij (\loc_t + i)
            }{
                \loc_s \iSimilar \loc_t
            }
        \and
        \inferrule*
            {
                f \in \dom{p_s}
            }{
                \lambdFunc{f} \iSimilar \lambdFunc{f}
            }
    \end{mathparpagebreakable}
    \caption{Similarity in $\iProp$}
    \label{fig:isimilar}
\end{figure}
\begin{figure}[tp]
    \begin{tabular}{rcl}
            $\iProt_\mathrm{dir} (\iPredTwo, \datalangExpr_s, \datalangExpr_t)$
            & $\coloneqq$ &
            $\exists \datalangFn, \datalangVal_s, \datalangVal_t \ldotp$
        \\
            &&
            $\datalangFn \in \dom{\datalangProg_s} \iSep
            \datalangExpr_s = \datalangCall{\datalangFnptr{\datalangFn}}{\datalangVal_s} \iSep
            \datalangExpr_t = \datalangCall{\datalangFnptr{\datalangFn}}{\datalangVal_t} \iSep {}$
        \\
            &&
            $\datalangVal_s \iSimilar \datalangVal_t \iSep {}$
        \\
            &&
            $\forall \datalangValTwo_s, \datalangValTwo_t \ldotp
            \datalangValTwo_s \iSimilar \datalangValTwo_t \iWand
            \iPredTwo (\datalangValTwo_s, \datalangValTwo_t)$
        \\
            $\iProt_\mathrm{DPS} (\iPredTwo, \datalangExpr_s, \datalangExpr_t)$
            & $\coloneqq$ &
            $\exists \datalangFn, \datalangFn_\mathit{dps}, \datalangVal_s, \datalangLoc_1, \datalangLoc_2, \datalangLoc, \datalangIdx, \datalangVal_t \ldotp$
        \\
            &&
            $\datalangFn \in \dom{\datalangProg_s} \iSep
            \datalangRenaming [\datalangFn] = \datalangFn_\mathit{dps} \iSep
            \datalangExpr_s = \datalangCall{\datalangFnptr{\datalangFn}}{\datalangVal_s} \iSep
            \datalangExpr_t = \datalangCall{\datalangFnptr{\datalangFn_\mathit{dps}}}{\datalangLoc_1} \iSep {}$
        \\
            &&
            $(\datalangLoc_1 + 1) \iPointsto_t (\datalangLoc_2, \datalangVal_t) \iSep
            (\datalangLoc_2 + 1) \iPointsto_t (\datalangLoc, \datalangIdx) \iSep
            (\datalangLoc + \datalangIdx) \iPointsto \datalangHole \iSep
            \datalangVal_s \iSimilar \datalangVal_t$
        \\
            &&
            $\forall \datalangValTwo_s, \datalangValTwo_t \ldotp
            (\datalangLoc + \datalangIdx) \iPointsto \datalangValTwo_t \iSep
            \datalangValTwo_s \iSimilar \datalangValTwo_t \iWand
            \iPredTwo (\datalangValTwo_s, \datalangUnit)$
        \\
            $\iProt_\mathrm{TMC}$
            & $\coloneqq$ &
            $\iProt_\mathrm{dir} \sqcup \iProt_\mathrm{DPS}
            =
            \lambdaAbs (\iPredTwo, \datalangExpr_s, \datalangExpr_t) \ldotp \iProt_\mathrm{dir} (\iPredTwo, \datalangExpr_s, \datalangExpr_t) \iOr \iProt_\mathrm{DPS} (\iPredTwo, \datalangExpr_s, \datalangExpr_t)$
    \end{tabular}
    \caption{TMC protocol ($\iProt_\mathrm{TMC}$)}
    \label{fig:protocol}
\end{figure}
\begin{figure}[tp]
    \begin{align*}
    		\wf{\gamma}
    		&\coloneqq
    		\forall x \ldotp
    		\exists v_s, v_t \ldotp
    		\gamma (x) = (v_s, v_t) \iSep
    		v_s \iSimilar v_t
    	\\
    		\iCsimv[\Chi]{\Phi}{e_s}{e_t}
    		&\coloneqq
    		\forall \gamma \ldotp
    		\wf{\gamma} \iSepImp
    		\iSimv[\Chi]{\Phi}{\gamma_s (e_s)}{\gamma_t (e_t)}
    	\\
    	   \iCsimvHoare[\Chi]{P}{\Phi}{e_s}{e_t}
    	   &\coloneqq
    	   \iPersistent \left( P \iSepImp \iCsimv[\Chi]{\Phi}{e_s}{e_t} \right)
    \end{align*}
    \caption{Closed simulation}
    \label{fig:csim}
\end{figure}
