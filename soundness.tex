\section{A relational separation logic for \LambdaLang programs}
\label{sec:reasoning_rules}

The proof technique we propose to verify program transformations such
as TMC is to capture a relational specification between inputs and
outputs of the translation using a relational separation logic.

We use a quadruple of the form $\iSim[\Chi]{\Phi}{e_s}{e_t}$, which can be read as ``$e_t$ refines $e_s$ under the postcondition $\Phi$ and protocol $\Chi$''. Intuitively, it states that $e_s$ and $e_t$ behave in a related way, and eventually (both get stuck or) reduce to expressions in the relational post-condition $\Phi$. The parameter $\Chi$ is a \emph{protocol} that specifies ``external'' transitions, typically used to model function calls.

Traditionally this quadruple would be a judgment in a bespoke system of inference rules, proved correct by establishing a soundness or adequacy theorem. Our work instead uses the standard Iris approach of defining it directly as an assertion in a separation logic. The inference rules that we will provide are admissible lemmas, and they use separation logic as their meta-logic.

We defer the definition of the assertion as a simulation to Section~\ref{sec:adequacy}, where it will also be proved adequate with respect to a notion of behavior refinement expressed outside Iris.

\subsection{Relation assertions}

The separation logic in which our reasoning rules are formulated is relational: it describes resources from a source and target programs and relations between them. We use the following primite assertions:

\begin{itemize}
\item $\loc_s \mapsto_s v_s$, a points-to assertion for the source program
\item $\loc_t \mapsto_t v_t$, for the target program
\item $\loc_s \iInBij \loc_t$, which asserts that the source location
  $\loc_s$ and the target location $\loc_t$ are in the ``heap
  bijection'' mentioned in Section~\ref{TODO}. This assertion is
  persistent: one cannot remove locations from the bijection.
\end{itemize}

\begin{figure}[tp]
    \centering
    \begin{mathparpagebreakable}
        \inferrule*
            {}{
                \lambdUnit \iSimilar \lambdUnit
            }
        \and
        \inferrule*
            {}{
                i \iSimilar i
            }
        \and
        \inferrule*
            {}{
                n \iSimilar n
            }
        \and
        \inferrule*
            {}{
                b \iSimilar b
            }
        \and
        \inferrule*
            {
                \forall i \in \{0,1,2\} \ldotp
                (\loc_s + i) \iInBij (\loc_t + i)
            }{
                \loc_s \iSimilar \loc_t
            }
        \and
        \inferrule*
            {
                f \in \dom{p_s}
            }{
                \lambdFunc{f} \iSimilar \lambdFunc{f}
            }
    \end{mathparpagebreakable}
    \caption{Similarity in $\iProp$}
    \label{fig:isimilar}
\end{figure}

On top of these assertions we define a \emph{value similarity} relation $\iSimilar {v_s} {v_t}$, given in Figure~\ref{fig:isimilar}.  Except for locations, this is exactly the equality. For locations, notice that locations appearing during the reduction of our programs are locations of valid blocks (always of size 3 in our language). We relate them by asking their field locations to be in the heap bijection.

\subsection{Reasoning rules}

\begin{figure}[tp]
    \begin{mathparpagebreakable}
        \inferrule*[lab=SimPost]
            {
                \Phi (e_s, e_t)
            }{
                \iSim[\Chi]{\Phi}{e_s}{e_t}
            }
        \and
        \inferrule*[lab=SimBind]
            {
                \iSim[\Chi]{\lambdaAbs (e_s', e_t') \ldotp \iSim[\Chi]{\Phi}{K_t [e_s']}{K_t [e_t']}}{e_s}{e_t}
            }{
                \iSim[\Chi]{\Phi}{K_t [e_s]}{K_t [e_t]}
            }
        \\
        \inferrule*[lab=SimSrcPure]
            {
                e_s \pureStep{p_s} e_s'
            \and
                \iSim[\Chi]{\Phi}{e_s'}{e_t}
            }{
                \iSim[\Chi]{\Phi}{e_s}{e_t}
            }
        \and
        \inferrule*[lab=SimTgtPure]
            {
                e_t \pureStep{p_t} e_t'
            \and
                \iSim[\Chi]{\Phi}{e_s}{e_t'}
            }{
                \iSim[\Chi]{\Phi}{e_s}{e_t}
            }
        \\
        \inferrule*[lab=SimBijInsert]
            {
                \loc_s \mapsto_s v_s
            \and
                \loc_t \mapsto_t v_t
            \and
                v_s \iSimilar v_t
            \and
                \loc_s \iInBij \loc_t \iSepImp
                \iSim[\Chi]{\Phi}{e_s}{e_t}
            }{
                \iSim[\Chi]{\Phi}{e_s}{e_t}
            }
        \\
        \inferrule*[lab=SimSrcConstr1]
            {
                \iSim[\Chi]{\Phi}{
                    \begin{array}{l}
                        \lambdLet{x_1}{e_{s1}}{\\
                        \lambdLet{x_2}{e_{s2}}{\\
                        \lambdConstrDet{t}{x_1}{x_2}}}
                    \end{array}
                }{e_t}
            }{
                \iSim[\Chi]{\Phi}{\lambdConstr{t}{e_{s1}}{e_{s2}}}{e_t}
            }
        \and
        \inferrule*[lab=SimSrcConstr2]
            {
                \iSim[\Chi]{\Phi}{
                    \begin{array}{l}
                        \lambdLet{x_2}{e_{s2}}{\\
                        \lambdLet{x_1}{e_{s1}}{\\
                        \lambdConstrDet{t}{x_1}{x_2}}}
                    \end{array}
                }{e_t}
            }{
                \iSim[\Chi]{\Phi}{\lambdConstr{t}{e_{s1}}{e_{s2}}}{e_t}
            }
        \and
        \inferrule*[lab=SimTgtConstr]
            {
                \iSim[\Chi]{\Phi}{e_s}{
                    \begin{array}{l}
                        \lambdLet{x_1}{e_{t1}}{\\
                        \lambdLet{x_2}{e_{t2}}{\\
                        \lambdConstrDet{t}{x_1}{x_2}}}
                    \end{array}
                }
            \and
                \iSim[\Chi]{\Phi}{e_s}{
                    \begin{array}{l}
                        \lambdLet{x_2}{e_{t2}}{\\
                        \lambdLet{x_1}{e_{t1}}{\\
                        \lambdConstrDet{t}{x_1}{x_2}}}
                    \end{array}
                }
            }{
                \iSim[\Chi]{\Phi}{e_s}{\lambdConstr{t}{e_{t1}}{e_{t2}}}
            }
        \\
        \inferrule*[lab=SimSrcConstrDet]
            {
                \forall \loc_s \ldotp
                \loc_s \mapsto_s (t, v_{s1}, v_{s2}) \iSepImp
                \iSim[\Chi]{\Phi}{\loc_s}{e_t}
            }{
                \iSim[\Chi]{\Phi}{\lambdConstrDet{t}{v_{s1}}{v_{s2}}}{e_t}
            }
        \and
        \inferrule*[lab=SimTgtConstrDet]
            {
                \forall \loc_t \ldotp
                \loc_t \mapsto_t (t, v_{t1}, v_{t2}) \iSepImp
                \iSim[\Chi]{\Phi}{e_s}{\loc_t}
            }{
                \iSim[\Chi]{\Phi}{e_s}{\lambdConstrDet{t}{v_{t1}}{v_{t2}}}
            }
        \\
        \inferrule*[lab=SimSrcStore]
            {
                (\loc_s + i) \mapsto_s w_s
            \and
                (\loc_s + i) \mapsto_s v_s \iSepImp
                \iSim[\Chi]{\Phi}{\lambdUnit}{e_t}
            }{
                \iSim[\Chi]{\Phi}{\lambdStore{\loc_s}{i}{v_s}}{e_t}
            }
        \and
        \inferrule*[lab=SimTgtStore]
            {
                (\loc_t + i) \mapsto_t w_t
            \and
                (\loc_t + i) \mapsto_t v_t \iSepImp
                \iSim[\Chi]{\Phi}{e_s}{\lambdUnit}
            }{
                \iSim[\Chi]{\Phi}{e_s}{\lambdStore{\loc_t}{i}{v_t}}
            }
        \\
        \inferrule*[lab=SimStore]
            {
                \loc_s \iSimilar \loc_t
            \and
                v_s \iSimilar v_t
            \and
                \Phi (\lambdUnit, \lambdUnit)
            }{
                \iSim[\Chi]{\Phi}{\lambdStore{\loc_s}{i}{v_s}}{\lambdStore{\loc_t}{i}{v_t}}
            }
    \end{mathparpagebreakable}
    \caption{Reasoning rules for simulation (excerpt, omitting freshness conditions)}
    \label{fig:sim_rules}
\end{figure}

% TODO add the rule that uses X


Figure~\ref{fig:sim_rules} defines the reasoning rules for our relational program logic. We omitted some standard rules for brevity.

\paragraph{Language-independent rules} The following rules are
independent of \LambdaLang and could be reused as-is in further works.

$\RefTirName{SimPost}$ states that two terms are related when they are in the relational postcondition.

$\RefTirName{SimBind}$ is a standard bind rule sequencing computations on both sides.

$\RefTirName{SimSrcPure}$ and $\RefTirName{SimTgtPure}$ let us take pure reduction steps in either the source or target and remain related. Pure steps (omitted for space) are the reduction steps that are deterministic and do not depend on the state.

\paragraph{Abstract protocols}
$\RefTirName{SimApplyProtocol}$ is the reasoning rule for external transitions specified by the protocol parameter $\Chi$. The protocol is a relation that axiomatizes the ability of certain pairs of expression $e_s$ and $e_t$ to evolve in an abstract way specified by a relational postcondition $\Psi$ -- they may reduce to arbitrary expressions $e_s'$ and $e_t'$ related by $\Psi$. To conclude that $e_s$ and $e_t$ are related, one must prove that any such $e_s'$ and $e_t'$ remain related.

Here is a simple protocol to relate calls to the same functions in the source and target expressions:
\begin{align*}
  \Chi_\mathrm{dir} (\Psi, e_s, e_t)
  &\coloneqq
  \begin{array}{l}
          \exists f, v_s, v_t \ldotp
      \\
          f \in \dom{p_s} \iSep
          e_s = \lambdCall{\lambdFunc{f}}{v_s} \iSep
          e_t = \lambdCall{\lambdFunc{f}}{v_t} \iSep
          v_s \iSimilar v_t \iSep {}
      \\
          \forall v_s', v_t' \ldotp
          v_s' \iSimilar v_t' \iSepImp
          \Psi (v_s', v_t')
  \end{array}
\end{align*}
This protocol relates two calls of the same function $\lambdFunc{f}$
with related arguments $v_s \iSimilar v_t$. Informally, it then
guarantees that the results of the calls will be related values
$v_s' \iSimilar v_t'$; precisely, it accepts any choice of post-condition $\Psi$ contains the similarity relation on values.

If we use this protocol in the reasoning rule
$\RefTirName{SimApplyProtocol}$ and instiatiate the postcondition $\Phi$ to be the value similarity relation $(\iSimilar)$ itself, we get the following specialized rule:
\begin{mathpar}
\infer
{
  f \in \dom{p_s}
  \and
  v_s \iSimilar v_t
}{
  \iSim[\Chi]{\iSimilar}
    {\lambdCall{\lambdFunc{f}}{v_s}}
    {\lambdCall{\lambdFunc{f}}{v_t}}
}
\end{mathpar}
This is exactly the \TirName{Sim-Call} rule of \Simuliris~\citep*{TODO-simuliris}. Our generalized rule $\RefTirName{SimApplyProtocol}$ is a contribution of our work, inspired by \citet*{TODO-paulo}. The extra expressivity is required to support program transformations such as TMC that change function calling conventions.

\paragraph{Language-specific rules: non-determinism}
%
Our relation $\iSim[\Chi]{\Phi}{e_s}{e_t}$ asserts that the target expression $e_t$ refines the source expression $e_s$: any behavior of $e_t$ is a valid behavior of $e_s$, all reductions of $e_t$ can be matched by some reduction of $e_s$. Consequently, non-determinism is treated differently in the source and target: we treat non-determinism as \emph{angelic} in source reductions and \emph{demonic} in target reductions.

Our operational semantics uses non-deterministic rule for reduction of constructors: the non-deterministic constructor $\lambdConstr{c}{e_1}{e_2}$ reduces to a deterministic constructor $\lambdConstrDet{c}{x_1}{x_2}$, where the $x_1$ and $x_2$ are bound to $e_1$ and $e_2$ in some non-determistic order.
%
In the program logic, the user may \emph{choose} an order for the source reduction, by using one of the rules $\RefTirName{SimSrcConstr1}$ or $\RefTirName{SimSrcConstr2}$, and it has to prove that the expressions are related against \emph{any} target order, by proving the two premises of the rule $\RefTirName{SimTgtConstr}$.

Once we have a deterministic constructor $\lambdConstrDet{c}{v_1}{v_2}$, the source and target reasoning rules $\RefTirName{SimSrcConstrDet}$ and $\RefTirName{SimTgtConstrDet}$ are symmetric.
%
They allocate the block at a new location in the source or target state; the user can assume a points-to predicate and has to prove that the location is related to the other expression.

\paragraph{Language-specific rules: private locations}
%
If the user owns a points-to assertion $(\loc + i) \mapsto v$ on the source or target side, they can use it to reason about loads and stores on the location $\loc$ at index $i$. The rules $\RefTirName{SimSrcLoad}$ and $\RefTirName{SimTgtLoad}$ let us replace the load expression $\lambdLoad{\loc}{i}$ by the pointed value $v$, and the rules $\RefTirName{SimSrcStore}$ and $\RefTirName{SimTgtStore}$ let us store a new value $w$, replacing the points-to assertion by $(\loc + i) \mapsto w$.

\paragraph{Language-specific rules: locations in the bijection}
%
In addition to points-to predicate on source and target locations, we maintain a heap bijection formed by the resource assertions $\loc_s \iInBij \loc_t$.

At any point the user can add two locations to the heap bijection using the rule $\RefTirName{SimBijInsert}$. They need to provide private points-to predicates $\loc_s \mapsto_s v_s$ and $\loc_t \mapsto_t v_t$, and prove that the content of the locations are related: $v_s \iSimilar v_t$.
%
In exchange they receive a heap bijection assertion $\loc_s \iInBij \loc_t$. These locations will remain in the heap bijection forever ($\loc_s \iInBij \loc_t$ is a persistent assertion), and we will preserve the invariant that their values are related.

To add locations $\loc_s, \loc_t$ to the bijections, we \emph{consume} their points-to assertions. At this point we cannot use the one-sided reasoning rules for load and store: we lost ownership of $\loc_s \mapsto_s v_s$, $\loc_t \mapsto_t v_t$.
%
The user must instead use the synchronized rules $\RefTirName{SimLoad}$ and $\RefTirName{SimStore}$, which relate two loads or two stores on $\loc_s$ and $\loc_t$.
%
$\RefTirName{SimLoad}$ proves that two loads $\lambdLoad{\loc_s}{i}$ and $\lambdLoad{\loc_t}{i}$ are related. The user must prove that $\loc_s \iSimilar \loc_t$, that is, that all three pairs of fields $(\loc_s + i), (\loc_t + i)$ are in the bijection; it knows nothing of the current values $v_s, v_t$ at these locations, except that they are related by the bijection invariant.
%
$\RefTirName{SimStore}$ proves that two stores $\lambdStore{\loc_s}{i}{v_s}$ and $\lambdStore{\loc_t}{i}{v_t}$ are related. The user must show that $\loc_s, \loc_t$ are in the bijection, and preserve the bijection invariant by proving that the values to be stored are related.

\subsection{Runtime relation}

\begin{figure}[tp]
    \begin{align*}
    		\wf{\Gamma}
    		&\coloneqq
    		\forall x \ldotp
    		\exists v_s, v_t \ldotp
    		\Gamma (x) = (v_s, v_t) \iSep
    		v_s \iSimilar v_t
    	\\
    		\iCsimv{\Phi}{e_s}{e_t}
    		&\coloneqq
    		\forall \Gamma \ldotp
    		\wf{\Gamma} \iSepImp
    		\iSimv[\Chi_\mathrm{TMC}]{\Phi}{\Gamma_s (e_s)}{\Gamma_t (e_t)}
    	\\
    	   \iCsimvHoare{P}{\Phi}{e_s}{e_t}
    	   &\coloneqq
    	   \iPersistent \left( P \iSepImp \iCsimv{\Phi}{e_s}{e_t} \right)
    \end{align*}
    \caption{Closed simulation}
    \label{fig:csim}
\end{figure}

The assertion $\iSim[\Chi]{\Phi}{e_s}{e_t}$ relates closed terms that may contain locations, as may be produced by our operational semantics. To prove the correctness of a program transformation, we proceed by induction on its definition on well-formed source terms, that do not contain locations and contain free variables.

To bridge the gap between the two sorts of terms, we define in Figure~\ref{fig:csim} a \emph{runtime relation} $\iCsimv[\Chi]{\Phi}{e_s}{e_t}$ that closes over all bisubstitutions $\gamma$ of related values for free variables. This sort of definition is standard (considering our simulation as an untyped logical relation), it is simply called the ``logical relation'' in \Simuliris.

Our theorems state that terms related by the TMC transformation are in the runtime relation $\iCsimv[\Chi]{\Phi}{e_s}{e_t}$, and their proofs expand its definition to introduce an arbitrary bisubstitution $\gamma$ and work directly on $\iSim[\Chi]{\Phi}{e_s}{e_t}$ using our reasoning rules.

\section{Reasoning about TMC: protocols and specifications}
\label{sec:TMC_protocols_and_specs}

TODO (see sketch)

\section{Adequacy}
\label{sec:soundness}
\label{sec:adequacy}

(This paragraph below probably rather belongs to the introduction. Abort.)

The formal contribution of our work is a mechanized soundness proof for the TMC transformation on a simplified programming language, exhibiting the salient parts of the actual transformation for OCaml.
%
The destination-passing style used by the TMC transformation critically relies on uniquely-owned mutable state.
%
It is thus natural to use separation logic as our specification language.

\subsection{The quest for the right notion of simulation}
\label{sec:howto-relation}

The final theorem we want to establish is a behaviour refinement $p_s \refined p_t$: the behaviours of the transformed program are included in the behaviour of the source program.
%
We propose a notion of behaviour refinement that gives a form of total correctness.
%
It is termination-preserving, but also (this is less common in the body of work around Iris) safety-preserving and divergence-preserving.

We prove this using a backward simulation argument for our transformations $e_s \tmc e_t$.
%
A direct approach, ultimately unsucessful, would be to prove that the relation is in the simulation by induction, on $e_s$ for example.
%
This works well for all constructions except for function calls $\lambdCall{\lambdFunc{f}}{e}$: to reason on the relation between a call to $\lambdFunc{f}$ and its transformation (in direct or DPS style), we would need to inline the function definition, which is not structurally decreasing.

This is a common problem to establish simulations, and the solution is to use a form of co-induction.
%
In the Iris logic, a natural way to do this would be to use a ``later'' modality.
%
But defining simulations using a ``later'' modality is in fact non-trivial; simple definitions do not give the excepted notion of simulation, due to non-intuitive aspects of the step-indexing semantics of ``later'' in the Iris model. This problem is discussed in details in the previous work on Transfinite Iris~\citep*{TODO-transfinite-Iris}, an alternative logic with different axioms and models.
%
Sticking to the main Iris logic, a good definition of simulation has been worked out in Simuliris~\citep*{TODO-Simuliris}.
%
It avoids using the ``later'' modality by using the notion of coinduction native to Coq. In Simuliris, coinductive reasoning steps correspond to an ``abstract'' case in the simulation relation, used for function applications; an ``open simulation'' allows these abstract steps, while a ``closed simulation'' does not allow them.
%
The coinduction principle is encapsulated in a ``simulation closure'' theorem, which states if two terms are related in the open simulation, they are in fact also related in the closed simulation if we (coinductively) assume that foreign function bodies are correct.

Our proof is heavily inspired by the Simuliris work, but we cannot reuse their results directly because the notion of simulation that they provide is not expressive enough for our proof: we need to support transformations that change the calling conventions between source and target programs.
%
(Along the way we also simplified the simulation relation by removing proof rules related to concurrency, which we do not consider in this work.)

To support different calling conventions, we make our simulation relation $\iSim[\Chi]{\Phi}{e_s}{e_t}$ parametric over abstract protocols $\Chi$. This extends the technique of \citet*{TODO-paulo} from the unary setting of safety predicates to the binary setting of a relational separation logic.

\subsection{Proof outline}

Once we have decided to reuse and extend the Simuliris work, our argument can be split in three separate parts:

\begin{enumerate}

\item Define our notion of simulation parametric over protocols (calling conventions), prove its adequacy (it implies a behaviour refinement) and establish a ``closure theorem'' as in the Simuliris work.
%
  This technical contribution is independent of the TMC language or transformation, and could be reused in other works.

\item Build a relational program logic for our programming language, as a body of lemmas to establish simulations between expressions.

\item Use this program logic to establish the soundness of the TMC transformation.
%
The key ingredients are two specifications for the direct and DPS transformations that are proved in a mutually-inductive way, and are used to prove the hypothesis of the closure theorem of our simulation.
\end{enumerate}

The rest of this section expands each step of the proof, providing the necessary technical details to follow the Coq formalization.

\subsection{End goal: behaviour refinement}

The TMC transformation preserves the behaviour of programs in a strong sense.
%
We expect terminating programs to remain terminating, diverging programs to remain diverging and failing programs (those that get stuck) to remain failing.

Our definitions of behaviours and behaviour refinement are given in \fref{fig:refinement}.
%
Our notion of behaviours follows the definitions of Simuliris~\citep*{TODO-simuliris}.
%
We define the set $\mathrm{behaviours}_p(e)$ of behaviours of an expression $e$ within a program $p$, starting from an empty store.
%
The behaviour $\constr{Conv}(v_s)$ indicates that $e$ can evaluate to $v_s$.
%
The behaviour $\constr{Conv}(e')$, when $e'$ is not a value, indicates that $e$ can reduce to a stuck configuration $(e', \sigma)$.
%
Finally, $\constr{Div}$ indicates that $e$ can diverge.

The refinement relation on behaviours $b_s \refined b_t$ then follows in a natural way: divergence refines divergence, any stuck expression refines any other stuck expression, and a value $v_t$ refines $v_s$ when they are related for a ground equivalence $v_s \similar v_t$, which is the equality for all value kinds, except for locations where it gives no information, as done in Simuliris.
%
This equivalence of ground values could be used, for example, to argue for the correctness of a \texttt{main} function that takes and returns integer arguments.

Note a difference to Simuliris: our refinement between stuck/failing behaviours is $\constr{Div} \refined \constr{Div}$, whereas Simuliris uses $\constr{Div} \refined b_t$: the Simuliris relation assumes that the input program is safe, never gets stuck, and the behaviour refinement thus allows a stuck source term to be refined by any target program. With our definition, a source program that only gets stuck must be transformed into a target program that only gets stuck.
%
This property would not be desirable for C compilers that want to optimize aggressively assuming the absence of undefined behaviors, but it does capture a finer-grained property of our program transformation.

Finally, an expression $e_t$ refines $e_s$ when all its behaviours are refined by a behaviour of $e_s$, and a program $p_t$ refines $p_s$ when calls to source functions on equivalent inputs are in the refinement relation.

\begin{figure}[tp]
    \centering
    \begin{mathparpagebreakable}
        \inferrule*
            {}{
                \datalangUnit \similar \datalangUnit
            }
        \and
        \inferrule*
            {}{
                \datalangIdx \similar \datalangIdx
            }
        \and
        \inferrule*
            {}{
                \datalangTag \similar \datalangTag
            }
        \and
        \inferrule*
            {}{
                \datalangBool \similar \datalangBool
            }
        \and
        \inferrule*
            {}{
                \datalangLoc_s \similar \datalangLoc_t
            }
        \and
        \inferrule*
            {}{
                \datalangFnptr{\datalangFn} \similar \datalangFnptr{\datalangFn}
            }
    \end{mathparpagebreakable}
    \caption{Similarity in $\Prop$}
    \label{fig:similarity}
\end{figure}
\begin{figure}[tp]
    \centering
    \begin{mathparpagebreakable}
        \inferrule*
            {
                v_s \similar v_t
            }{
                \constr[v_s]{Conv} \refined \constr[v_t]{Conv}
            }
        \and
        \inferrule*
            {
                e_s \notin \nterm{Val}
            \and
                e_t \notin \nterm{Val}
            }{
                \constr[e_s]{Conv} \refined \constr[e_t]{Conv}
            }
        \and
        \inferrule*
            {}{
                \constr{Div} \refined \constr{Div}
            }
    \end{mathparpagebreakable}
    \\
    \begin{align*}
        	\mathrm{behaviours}_p (e)
        	&\coloneqq
        	\begin{array}{l}
        			\{ \constr[e']{Conv} \mid \exists \sigma \ldotp (e, \emptyset) \step{p}^* (e', \sigma) \wedge \mathrm{irreducible}_p (e', \sigma) \}\ \uplus
        		\\
        			\{ \constr{Div} \mid\ \diverges{p}{(e, \emptyset)} \}
            \end{array}
        \\
            e_s \refined e_t
            &\coloneqq
            \forall b_t \in \mathrm{behaviours}_{p_t} (e_t) \ldotp
            \exists b_s \in \mathrm{behaviours}_{p_s} (e_s) \ldotp
            b_s \refined b_t
        \\
            p_s \refined p_t
            &\coloneqq
            \forall f \in \dom{p_t}, v_s, v_t \ldotp
            \wf{v_s} \wedge v_s \similar v_t \implies
            \lambdCall{\lambdFunc{f}}{v_s} \refined \lambdCall{\lambdFunc{f}}{v_t}
    \end{align*}
    \caption{Program refinement}
    \label{fig:refinement}
\end{figure}

We can now state the main soundness theorem of our work, whose proof is spread over the rest of the present Section~\ref{sec:soundness}.

\begin{theorem}[Soundness]
    $
        \wf{p_s} \wedge p_s \tmc p_t \implies
        p_s \refined p_t
    $
\end{theorem}

The condition $\wf{p_s}$ guarantees that the input program $p_s$ is well-scoped. Under this condition, we can consider the target programs $p_t$ that are TMC transformations $p_s \tmc p_s$, and our theorem states that they refine $p_s$ as expected.

\subsection{Proof technique: Simuliris with protocols}

Simuliris~\citep*{TODO-simuliris} suggests a definition of simulation relations in Iris. As we explained in Section~\ref{sec:howto-relation}, it carefully avoids using the ``later'' modality by using Coq's native notion of coinduction. Simuliris has several desirable properties for us: it can be defined in the un-modified Iris base logic (unlike Transfite Iris~\citep*{transfinite-iris}), it is defined in a way that makes it easy to specialize to other programming languages, and it supports showing the preservation of termination. On the other hand, the original definition is complex as it supports concurrency and notions of fairness. We reused the definition of Simuliris, simplified for the sequential setting. We then extended the resulting definition to support our treatment of stuck states as failures, and generalize the notion of external calls to abstract protocols $\Chi$. Most of the ideas below come from the Simuliris work directly, we will explicitly point out the parts that differ.

\paragraph{Relating states} The \emph{state interpertation} $I(\sigma_s, \sigma_t)$ relates source and target states. It extends the unary notion of state intereptation predicate $I(\sigma)$. Recall that in separation logic, formulas contain both pure propositions and ressources, which include in particular logical points-to assertions $\loc \mapsto v$. $I(\sigma)$ enforces the consistency between those logical assertions and the physical state $\sigma$.

In the relational setting, the state interpretation becomes a relation $I(\sigma_s, \sigma_t)$ tracking points-to predicates on either states $\loc_s \mapsto_s v_s$ and $\loc_t \mapsto_t v_t$. It is additionally in charge of maintaing the heap bijection mentioned in Section~\ref{TODO} by tracking predicates $\loc_s \iInBij \loc_t$. This assertion is persistent: one cannot remove locations from the bijection.

\paragraph{Relating values} TODO relation between values.

\paragraph{Relating expressions} The definition of the expression relation $\iSim[\Chi]{\Phi}{e_s}{e_t}$ is shown in \fref{fig:sim}; it reads as ``$e_s$ simulates $e_t$ under the postcondition $\Phi$ and protocol $\Chi$''. It expresses that the source $e_s$ can simulate any computation of the target $e_t$ until they both get stuck, both diverge, or reach two expressions in the relational post-condition $\Phi$. The parameter $\Chi$ is the relational protocol that must be followed by function calls in $e_s$ and $e_t$ -- generalizing the Simuliris rule for external calls.

\paragraph{Mixed induction and coinduction}
The definition of the relation itself starts with a double fixpoint, a greatest fixpoint $\nuAbs sim .$ (coinduction) of smallest fixpoint $\muAbs {sim \mathhyphen inner} .$ (induction) of a relation $\mathrm{sim \mathhyphen body}_\Chi$. Inside the relation definition, the cases that use the inductive $sim \mathhyphen inner$ in their recursive occurrence can only be repeated finitely many times, while the cases that use the coinductive $sim$ in their recursive occurrence can be repeated infinitely, but can only do so under a form of guardedness condition.

\clearpage
\begin{figure}[tp]
    \begin{align*}
    		\mathrm{sim \mathhyphen body}_\iProt
    		&\coloneqq
    		\begin{array}{l}
    				\lambdaAbs sim \ldotp
    				\lambdaAbs sim \mathhyphen inner \ldotp
    				\lambdaAbs {(\iPred, \datalangExpr_s, \datalangExpr_t)} \ldotp
    				\forall \datalangState_s, \datalangState_s \ldotp
    				\mathrm{I} (\datalangState_s, \datalangState_t)
    				\iSepImp \iBupd
    			\\
    				\bigvee \left[ \begin{array}{ll}
    							\circled{1}
    						&
    							\mathrm{I} (\datalangState_s, \datalangState_t) \iSep
    							\iPred (\datalangExpr_s, \datalangExpr_t)
    					\\
    					        \circled{2}
                            &
                                \mathrm{I} (\datalangState_s, \datalangState_t) \iSep
    							\mathrm{strongly \mathhyphen stuck}_{\datalangProg_s} (\datalangExpr_s) \iSep
    							\mathrm{strongly \mathhyphen stuck}_{\datalangProg_t} (\datalangExpr_s)
    					\\
    							\circled{3}
    						&
    							\exists \datalangExpr_s', \datalangState_s' \ldotp
    							(\datalangExpr_s, \datalangState_s) \step{\datalangProg_s}^+ (\datalangExpr_s', \datalangState_s') \iSep
    							\mathrm{I} (\datalangState_s', \datalangState_t) \iSep
    							sim \mathhyphen inner (\iPred, \datalangExpr_s', \datalangExpr_t)
    					\\
    							\circled{4}
    						&
								\mathrm{reducible}_{\datalangProg_t} (\datalangExpr_t, \datalangState_t) \iSep
								\forall \datalangExpr_t', \datalangState_t' \ldotp
								(\datalangExpr_t, \datalangState_t) \step{\datalangProg_t} (\datalangExpr_t', \datalangState_t')
								\iSepImp \iBupd
						\\
                            &
								\bigvee \left[ \begin{array}{ll}
											\circled{A}
										&
											\mathrm{I} (\datalangState_s, \datalangState_t') \iSep
											sim \mathhyphen inner (\iPred, \datalangExpr_s, \datalangExpr_t')
									\\
											\circled{B}
										&
											\exists \datalangExpr_s', \datalangState_s' \ldotp
											(\datalangExpr_s, \datalangState_s) \step{\datalangProg_s}^+ (\datalangExpr_s', \datalangState_s') \iSep
											\mathrm{I} (\datalangState_s', \datalangState_t') \iSep
											sim (\iPred, \datalangExpr_s', \datalangExpr_t')
								\end{array} \right.
    					\\
    							\circled{5}
    						&
    							\exists \datalangEctx_s, \datalangExpr_s', \datalangEctx_t, \datalangExpr_t', \iPredTwo \ldotp
    					\\
    					    &
    					        \datalangExpr_s = \datalangEctx_s [\datalangExpr_s'] \iSep
    							\datalangExpr_t = \datalangEctx_t [\datalangExpr_t'] \iSep
    						    \iProt (\iPredTwo, \datalangExpr_s', \datalangExpr_t') \iSep
    						    \mathrm{I} (\datalangState_s, \datalangState_t) \iSep {}
    					\\
                            &
								\forall \datalangExpr_s'', \datalangExpr_t'' \ldotp
								\iPredTwo (\datalangExpr_s'', \datalangExpr_t'') \iSepImp
								sim (\iPred, \datalangEctx_s [\datalangExpr_s''], \datalangEctx_t [\datalangExpr_t''])
    				\end{array} \right.
    		\end{array}
    	\\
    	    \mathrm{sim \mathhyphen inner}_\iProt
    	    &\coloneqq
    	    \lambdaAbs sim \ldotp
    	    \muAbs sim \mathhyphen inner \ldotp
    	    \mathrm{sim \mathhyphen body}_\iProt (sim, sim \mathhyphen inner)
    	\\
    		\mathrm{sim}_\iProt
    		&\coloneqq
    		\nuAbs sim \ldotp
    		\mathrm{sim \mathhyphen inner}_\iProt (sim)
    	\\
    		\iSim[\iProt]{\iPred}{\datalangExpr_s}{\datalangExpr_t}
    		&\coloneqq
    		\mathrm{sim}_\iProt (\iPred, \datalangExpr_s, \datalangExpr_t)
    	\\
    	   \iSimv[\iProt]{\iPred}{\datalangExpr_s}{\datalangExpr_t}
    	   &\coloneqq
    	   \iSim[\iProt]{\lambdaAbs (\datalangExpr_s', \datalangExpr_t') \ldotp \exists \datalangVal_s, \datalangVal_t \ldotp \datalangExpr_s' = \datalangVal_s \iSep \datalangExpr_t' = \datalangVal_t \iSep \iPred (\datalangVal_s, \datalangVal_t)}{\datalangExpr_s}{\datalangExpr_t}
    \end{align*}
    \caption{Simulation with protocol}
    \label{fig:sim}
\end{figure}

\paragraph{Simulation clauses} The relation between $e_s$ and $e_t$, at a given post-condition $\Phi$, must be parametric over all physical states in the state interpretation relation $I(\sigma_s, \sigma_t)$. It is established by one of the following clauses:

\begin{enumerate}
\item[\circled{1}] Halting clause: $e_s$, $e_t$ can stop if they are already in the post-condition $\Phi$ -- and the states are still related.
\item[\circled{2}]
\item[\circled{3}]
\item[\circled{4}]
\item[\circled{5}]
\end{enumerate}

% sim-body:
% cas 1, 2, 3, 4: on raconte
% "ouverte" comprend un autre cas.
% cas 5: les appels externes, protocoles (on raconte)



TODO

\clearpage

\begin{theorem}[Close simulation]
    \begin{align*}
            &
            \iPersistent \left(
                \forall \Psi, e_s, e_t \ldotp
                \Chi (\Psi, e_s, e_t) \iSepImp
                \mathrm{sim \mathhyphen inner}_\bot (\lambdaAbs (\_, e_s', e_t') \ldotp \iSim[\Chi]{\Psi}{e_s'}{e_t'}) (\bot, e_s, e_t)
            \right) \iSepImp
        \\
            &
            \iSim[\Chi]{\Phi}{e_s}{e_t} \iSepImp
            \iSim[\bot]{\Phi}{e_s}{e_t}
    \end{align*}
\end{theorem}

\begin{theorem}[Adequacy]
    $
        \left( \vdash \iSimv{\iSimilar}{e_s}{e_t} \right) \implies
        e_s \refined e_t
    $
\end{theorem}

relational separation logic = simulation

problem: direct calling convention in Simuliris, we need two calling conventions (direct + DPS)

remedy: simulation parameterized by protocol, generalization of Simuliris (but: sequential, without source safety but closed at toplevel)

% X_tmc := X_dir \uplus X_dps

% 1. when we are done
% 2. target stuttering (inductive)
% 3. all one-step source movees
%    3a. no source change (inductive)
%    3b. at least one source reduction (coinductive / guarded)
% 4. "open" simulation; X (\Chi) characterizes admissible foreign calls

% A trick from Simuliris:
% sim-after-progress: instance of sim-inner with bottom in well-chosen places
% closed simulation theorem: if the protocol X is "closable",
%   then the simulation relation for X "collapses" into a closed simulation (bottom for X, no case 4)
%
% MYSTERY: from a high-level perspective this "closd simulation theorem" feels a bit obvious:
% its hypothesis says that your protocol X can always be replaced by making progress on both sides,
% so it should be "easy" to transform a simulation proof using 1-2-3-4 into a simulation proof using only 1-2-3.
%
% Question: if it is "easy", cannot you do this right away (in the proof of simulation for a specific X that satisfies
% this property), proving a closed simulation (without 4) directly?
%
% Clément thinks that the answer is that this would require a super
% tricky coinduction, just like the one used in the proof of this
% theorem.
%
% Clément: this theorem is a sort of coinduction principle. You can
% prove an open simulation by induction (no coinduction
% apparently required), and then applying this theorem gives a result
% for which a direct proof would require a coinduction.

DPS calling convention

TMC protocol

% The X for TMC is essentially a relation/first-order rephrasing of
% the relation specs for direct-style and destination-passing-style
% transformations.

heap bijection

% I(sigma, sigma´) is as in Simuliris, not given explicitly in the current document.
% the \approx^bij relation refers to a component of I (a bijection between addresses)
% that is an implicit parameter of the definitions.

adequacy

% If two expressions refine internally (thy are in the
% simulation relation), and go to the internal equivalence between
% values, then they refine externally (they are in the denotational
% refinement relation).

% This is similar to the Simuliris proof, simplified thanks to the absence
% of concurrency.

simulation rules

% Note: those inference rules are in fact Iris propositions
% (the separating conjunction of the premises magic-wands the conclusion)

% Pure reductions: those that do not need state (neither write nor read)

% Gabriel: we could introduc the non-deterministic pair rules into the "pure" relation
% if we changed the pure-target rule to quantify over all e'_t such that e_t -{pure}-> e'_t.
% This would please Gabriel (existential on the left, universal on the right) and be more compact,
% but Clément prefers the current presentation -- it is more standard for Iris folks -- and closer
% to the mechanization.

% SimBijInsert: \approx^bij: Simuliris calls this the escaped/public
% locations, those that have been published.

% Clément now mentions Figure 12, which shows how stores evolve on an
% example in the TMC case. The shape of the relation between the
% source and target stores is sort of implicit in the program logic
% simulations rules, but it is clear on this example.
%
% Clément would explain this to give an informal sense of the proof,
% before even talking about specification logic etc.

closing simulation

proof


\begin{theorem}[Specification of direct transformation]
    \[
        \iCsimvHoare{
            \wf{e_s} \iSep
            e_s \tmcDir{\xi} e_t
        }{
            \iSimilar
        }{
            e_s
        }{
            e_t
        }
    \]
\end{theorem}

\begin{theorem}[Specification of DPS transformation]
    \[
        \iCsimvHoare{
            \wf{e_s} \iSep
            (\loc, i, e_s) \tmcDps{\xi} e_t \iSep
            (\loc + i) \mapsto \lambdHole
        }{
            \lambdaAbs (v_s, w_t) \ldotp
            \exists v_t \ldotp
            w_t = \lambdUnit \iSep
            (\loc + i) \mapsto v_t \iSep
            v_s \iSimilar v_t
        }{
            e_s
        }{
            e_t
        }
    \]
\end{theorem}

% Those two statements are proven together.
% (First by induction on e_s, then on the two transformation relations mutually-inductively.)

% Note: the \geq here is a "parametric" simulation, we assume that all
% free variables of e_s, e_t (they must be the same) are replaced by
% externally-equivalent values.

% Taking a step back: to prove the contextual refinements between programs,
% we have to check each function,
%   @f vs refined by @f vt  (for vs externally-equiv. vt)
% for this we can use the adequacy lemma
%   f vs simulated by @f vt (for the empty protocol) into internal value equivalence
% and for this we use the simulation closure theorem
%   f vs simulated by @f vt (for the Xtmc protocol)
% assuming the Xtmc protocol is valid/adequate.
%
% But then this is trivial, just use case (4) right away.
% So the meat of the proof is in showing that the Xtmc protocol is valid.

% To prove that the Xtmc propocol is valid:
% - in the direct case, we have a function application on both sides,
%   so we make a pure step on both sides and we have to show that the function bodies
%   (with a bisubstitution (x mapsto (vs, vt)) applied) are in the relation.

\begin{figure}[tp]
    \begin{tabular}{rcl}
            $\iProt_\mathrm{dir} (\iPredTwo, \datalangExpr_s, \datalangExpr_t)$
            & $\coloneqq$ &
            $\exists \datalangFn, \datalangVal_s, \datalangVal_t \ldotp$
        \\
            &&
            $\datalangFn \in \dom{\datalangProg_s} \iSep
            \datalangExpr_s = \datalangCall{\datalangFnptr{\datalangFn}}{\datalangVal_s} \iSep
            \datalangExpr_t = \datalangCall{\datalangFnptr{\datalangFn}}{\datalangVal_t} \iSep {}$
        \\
            &&
            $\datalangVal_s \iSimilar \datalangVal_t \iSep {}$
        \\
            &&
            $\forall \datalangValTwo_s, \datalangValTwo_t \ldotp
            \datalangValTwo_s \iSimilar \datalangValTwo_t \iWand
            \iPredTwo (\datalangValTwo_s, \datalangValTwo_t)$
        \\
            $\iProt_\mathrm{DPS} (\iPredTwo, \datalangExpr_s, \datalangExpr_t)$
            & $\coloneqq$ &
            $\exists \datalangFn, \datalangFn_\mathit{dps}, \datalangVal_s, \datalangLoc_1, \datalangLoc_2, \datalangLoc, \datalangIdx, \datalangVal_t \ldotp$
        \\
            &&
            $\datalangFn \in \dom{\datalangProg_s} \iSep
            \datalangRenaming [\datalangFn] = \datalangFn_\mathit{dps} \iSep
            \datalangExpr_s = \datalangCall{\datalangFnptr{\datalangFn}}{\datalangVal_s} \iSep
            \datalangExpr_t = \datalangCall{\datalangFnptr{\datalangFn_\mathit{dps}}}{\datalangLoc_1} \iSep {}$
        \\
            &&
            $(\datalangLoc_1 + 1) \iPointsto_t (\datalangLoc_2, \datalangVal_t) \iSep
            (\datalangLoc_2 + 1) \iPointsto_t (\datalangLoc, \datalangIdx) \iSep
            (\datalangLoc + \datalangIdx) \iPointsto \datalangHole \iSep
            \datalangVal_s \iSimilar \datalangVal_t$
        \\
            &&
            $\forall \datalangValTwo_s, \datalangValTwo_t \ldotp
            (\datalangLoc + \datalangIdx) \iPointsto \datalangValTwo_t \iSep
            \datalangValTwo_s \iSimilar \datalangValTwo_t \iWand
            \iPredTwo (\datalangValTwo_s, \datalangUnit)$
        \\
            $\iProt_\mathrm{TMC}$
            & $\coloneqq$ &
            $\iProt_\mathrm{dir} \sqcup \iProt_\mathrm{DPS}
            =
            \lambdaAbs (\iPredTwo, \datalangExpr_s, \datalangExpr_t) \ldotp \iProt_\mathrm{dir} (\iPredTwo, \datalangExpr_s, \datalangExpr_t) \iOr \iProt_\mathrm{DPS} (\iPredTwo, \datalangExpr_s, \datalangExpr_t)$
    \end{tabular}
    \caption{TMC protocol ($\iProt_\mathrm{TMC}$)}
    \label{fig:protocol}
\end{figure}

% François: the relational separation logic captures the right
% specification for functions. This is key to compositionality. We
% cannot do this with just a refinement relation. We should emphasize
% that this is a good approach.

%%% Local Variables:
%%% mode: latex
%%% TeX-master: "main"
%%% End:
