\section{Soundness}

\begin{figure}[tp]
    \centering
    \begin{mathparpagebreakable}
        \inferrule*
            {}{
                \datalangUnit \similar \datalangUnit
            }
        \and
        \inferrule*
            {}{
                \datalangIdx \similar \datalangIdx
            }
        \and
        \inferrule*
            {}{
                \datalangTag \similar \datalangTag
            }
        \and
        \inferrule*
            {}{
                \datalangBool \similar \datalangBool
            }
        \and
        \inferrule*
            {}{
                \datalangLoc_s \similar \datalangLoc_t
            }
        \and
        \inferrule*
            {}{
                \datalangFnptr{\datalangFn} \similar \datalangFnptr{\datalangFn}
            }
    \end{mathparpagebreakable}
    \caption{Similarity in $\Prop$}
    \label{fig:similarity}
\end{figure}
\begin{figure}[tp]
    \centering
    \begin{mathparpagebreakable}
        \inferrule*
            {
                v_s \similar v_t
            }{
                \constr[v_s]{Conv} \refined \constr[v_t]{Conv}
            }
        \and
        \inferrule*
            {
                e_s \notin \nterm{Val}
            \and
                e_t \notin \nterm{Val}
            }{
                \constr[e_s]{Conv} \refined \constr[e_t]{Conv}
            }
        \and
        \inferrule*
            {}{
                \constr{Div} \refined \constr{Div}
            }
    \end{mathparpagebreakable}
    \\
    \begin{align*}
        	\mathrm{behaviours}_p (e)
        	&\coloneqq
        	\begin{array}{l}
        			\{ \constr[e']{Conv} \mid \exists \sigma \ldotp (e, \emptyset) \step{p}^* (e', \sigma) \wedge \mathrm{irreducible}_p (e', \sigma) \}\ \uplus
        		\\
        			\{ \constr{Div} \mid\ \diverges{p}{(e, \emptyset)} \}
            \end{array}
        \\
            e_s \refined e_t
            &\coloneqq
            \forall b_t \in \mathrm{behaviours}_{p_t} (e_t) \ldotp
            \exists b_s \in \mathrm{behaviours}_{p_s} (e_s) \ldotp
            b_s \refined b_t
        \\
            p_s \refined p_t
            &\coloneqq
            \forall f \in \dom{p_t}, v_s, v_t \ldotp
            \wf{v_s} \wedge v_s \similar v_t \implies
            \lambdCall{\lambdFunc{f}}{v_s} \refined \lambdCall{\lambdFunc{f}}{v_t}
    \end{align*}
    \caption{Program refinement}
    \label{fig:refinement}
\end{figure}
\begin{figure}[tp]
    \begin{align*}
    		\mathrm{sim \mathhyphen body}_\iProt
    		&\coloneqq
    		\begin{array}{l}
    				\lambdaAbs sim \ldotp
    				\lambdaAbs sim \mathhyphen inner \ldotp
    				\lambdaAbs {(\iPred, \datalangExpr_s, \datalangExpr_t)} \ldotp
    				\forall \datalangState_s, \datalangState_s \ldotp
    				\mathrm{I} (\datalangState_s, \datalangState_t)
    				\iSepImp \iBupd
    			\\
    				\bigvee \left[ \begin{array}{ll}
    							\circled{1}
    						&
    							\mathrm{I} (\datalangState_s, \datalangState_t) \iSep
    							\iPred (\datalangExpr_s, \datalangExpr_t)
    					\\
    					        \circled{2}
                            &
                                \mathrm{I} (\datalangState_s, \datalangState_t) \iSep
    							\mathrm{strongly \mathhyphen stuck}_{\datalangProg_s} (\datalangExpr_s) \iSep
    							\mathrm{strongly \mathhyphen stuck}_{\datalangProg_t} (\datalangExpr_s)
    					\\
    							\circled{3}
    						&
    							\exists \datalangExpr_s', \datalangState_s' \ldotp
    							(\datalangExpr_s, \datalangState_s) \step{\datalangProg_s}^+ (\datalangExpr_s', \datalangState_s') \iSep
    							\mathrm{I} (\datalangState_s', \datalangState_t) \iSep
    							sim \mathhyphen inner (\iPred, \datalangExpr_s', \datalangExpr_t)
    					\\
    							\circled{4}
    						&
								\mathrm{reducible}_{\datalangProg_t} (\datalangExpr_t, \datalangState_t) \iSep
								\forall \datalangExpr_t', \datalangState_t' \ldotp
								(\datalangExpr_t, \datalangState_t) \step{\datalangProg_t} (\datalangExpr_t', \datalangState_t')
								\iSepImp \iBupd
						\\
                            &
								\bigvee \left[ \begin{array}{ll}
											\circled{A}
										&
											\mathrm{I} (\datalangState_s, \datalangState_t') \iSep
											sim \mathhyphen inner (\iPred, \datalangExpr_s, \datalangExpr_t')
									\\
											\circled{B}
										&
											\exists \datalangExpr_s', \datalangState_s' \ldotp
											(\datalangExpr_s, \datalangState_s) \step{\datalangProg_s}^+ (\datalangExpr_s', \datalangState_s') \iSep
											\mathrm{I} (\datalangState_s', \datalangState_t') \iSep
											sim (\iPred, \datalangExpr_s', \datalangExpr_t')
								\end{array} \right.
    					\\
    							\circled{5}
    						&
    							\exists \datalangEctx_s, \datalangExpr_s', \datalangEctx_t, \datalangExpr_t', \iPredTwo \ldotp
    					\\
    					    &
    					        \datalangExpr_s = \datalangEctx_s [\datalangExpr_s'] \iSep
    							\datalangExpr_t = \datalangEctx_t [\datalangExpr_t'] \iSep
    						    \iProt (\iPredTwo, \datalangExpr_s', \datalangExpr_t') \iSep
    						    \mathrm{I} (\datalangState_s, \datalangState_t) \iSep {}
    					\\
                            &
								\forall \datalangExpr_s'', \datalangExpr_t'' \ldotp
								\iPredTwo (\datalangExpr_s'', \datalangExpr_t'') \iSepImp
								sim (\iPred, \datalangEctx_s [\datalangExpr_s''], \datalangEctx_t [\datalangExpr_t''])
    				\end{array} \right.
    		\end{array}
    	\\
    	    \mathrm{sim \mathhyphen inner}_\iProt
    	    &\coloneqq
    	    \lambdaAbs sim \ldotp
    	    \muAbs sim \mathhyphen inner \ldotp
    	    \mathrm{sim \mathhyphen body}_\iProt (sim, sim \mathhyphen inner)
    	\\
    		\mathrm{sim}_\iProt
    		&\coloneqq
    		\nuAbs sim \ldotp
    		\mathrm{sim \mathhyphen inner}_\iProt (sim)
    	\\
    		\iSim[\iProt]{\iPred}{\datalangExpr_s}{\datalangExpr_t}
    		&\coloneqq
    		\mathrm{sim}_\iProt (\iPred, \datalangExpr_s, \datalangExpr_t)
    	\\
    	   \iSimv[\iProt]{\iPred}{\datalangExpr_s}{\datalangExpr_t}
    	   &\coloneqq
    	   \iSim[\iProt]{\lambdaAbs (\datalangExpr_s', \datalangExpr_t') \ldotp \exists \datalangVal_s, \datalangVal_t \ldotp \datalangExpr_s' = \datalangVal_s \iSep \datalangExpr_t' = \datalangVal_t \iSep \iPred (\datalangVal_s, \datalangVal_t)}{\datalangExpr_s}{\datalangExpr_t}
    \end{align*}
    \caption{Simulation with protocol}
    \label{fig:sim}
\end{figure}
\begin{figure}[tp]
    \begin{mathparpagebreakable}
        \inferrule*[lab=SimPost]
            {
                \Phi (e_s, e_t)
            }{
                \iSim[\Chi]{\Phi}{e_s}{e_t}
            }
        \and
        \inferrule*[lab=SimBind]
            {
                \iSim[\Chi]{\lambdaAbs (e_s', e_t') \ldotp \iSim[\Chi]{\Phi}{K_t [e_s']}{K_t [e_t']}}{e_s}{e_t}
            }{
                \iSim[\Chi]{\Phi}{K_t [e_s]}{K_t [e_t]}
            }
        \\
        \inferrule*[lab=SimSrcPure]
            {
                e_s \pureStep{p_s} e_s'
            \and
                \iSim[\Chi]{\Phi}{e_s'}{e_t}
            }{
                \iSim[\Chi]{\Phi}{e_s}{e_t}
            }
        \and
        \inferrule*[lab=SimTgtPure]
            {
                e_t \pureStep{p_t} e_t'
            \and
                \iSim[\Chi]{\Phi}{e_s}{e_t'}
            }{
                \iSim[\Chi]{\Phi}{e_s}{e_t}
            }
        \\
        \inferrule*[lab=SimBijInsert]
            {
                \loc_s \mapsto_s v_s
            \and
                \loc_t \mapsto_t v_t
            \and
                v_s \iSimilar v_t
            \and
                \loc_s \iInBij \loc_t \iSepImp
                \iSim[\Chi]{\Phi}{e_s}{e_t}
            }{
                \iSim[\Chi]{\Phi}{e_s}{e_t}
            }
        \\
        \inferrule*[lab=SimSrcConstr1]
            {
                \iSim[\Chi]{\Phi}{
                    \begin{array}{l}
                        \lambdLet{x_1}{e_{s1}}{\\
                        \lambdLet{x_2}{e_{s2}}{\\
                        \lambdConstrDet{t}{x_1}{x_2}}}
                    \end{array}
                }{e_t}
            }{
                \iSim[\Chi]{\Phi}{\lambdConstr{t}{e_{s1}}{e_{s2}}}{e_t}
            }
        \and
        \inferrule*[lab=SimSrcConstr2]
            {
                \iSim[\Chi]{\Phi}{
                    \begin{array}{l}
                        \lambdLet{x_2}{e_{s2}}{\\
                        \lambdLet{x_1}{e_{s1}}{\\
                        \lambdConstrDet{t}{x_1}{x_2}}}
                    \end{array}
                }{e_t}
            }{
                \iSim[\Chi]{\Phi}{\lambdConstr{t}{e_{s1}}{e_{s2}}}{e_t}
            }
        \and
        \inferrule*[lab=SimTgtConstr]
            {
                \iSim[\Chi]{\Phi}{e_s}{
                    \begin{array}{l}
                        \lambdLet{x_1}{e_{t1}}{\\
                        \lambdLet{x_2}{e_{t2}}{\\
                        \lambdConstrDet{t}{x_1}{x_2}}}
                    \end{array}
                }
            \and
                \iSim[\Chi]{\Phi}{e_s}{
                    \begin{array}{l}
                        \lambdLet{x_2}{e_{t2}}{\\
                        \lambdLet{x_1}{e_{t1}}{\\
                        \lambdConstrDet{t}{x_1}{x_2}}}
                    \end{array}
                }
            }{
                \iSim[\Chi]{\Phi}{e_s}{\lambdConstr{t}{e_{t1}}{e_{t2}}}
            }
        \\
        \inferrule*[lab=SimSrcConstrDet]
            {
                \forall \loc_s \ldotp
                \loc_s \mapsto_s (t, v_{s1}, v_{s2}) \iSepImp
                \iSim[\Chi]{\Phi}{\loc_s}{e_t}
            }{
                \iSim[\Chi]{\Phi}{\lambdConstrDet{t}{v_{s1}}{v_{s2}}}{e_t}
            }
        \and
        \inferrule*[lab=SimTgtConstrDet]
            {
                \forall \loc_t \ldotp
                \loc_t \mapsto_t (t, v_{t1}, v_{t2}) \iSepImp
                \iSim[\Chi]{\Phi}{e_s}{\loc_t}
            }{
                \iSim[\Chi]{\Phi}{e_s}{\lambdConstrDet{t}{v_{t1}}{v_{t2}}}
            }
        \\
        \inferrule*[lab=SimSrcStore]
            {
                (\loc_s + i) \mapsto_s w_s
            \and
                (\loc_s + i) \mapsto_s v_s \iSepImp
                \iSim[\Chi]{\Phi}{\lambdUnit}{e_t}
            }{
                \iSim[\Chi]{\Phi}{\lambdStore{\loc_s}{i}{v_s}}{e_t}
            }
        \and
        \inferrule*[lab=SimTgtStore]
            {
                (\loc_t + i) \mapsto_t w_t
            \and
                (\loc_t + i) \mapsto_t v_t \iSepImp
                \iSim[\Chi]{\Phi}{e_s}{\lambdUnit}
            }{
                \iSim[\Chi]{\Phi}{e_s}{\lambdStore{\loc_t}{i}{v_t}}
            }
        \\
        \inferrule*[lab=SimStore]
            {
                \loc_s \iSimilar \loc_t
            \and
                v_s \iSimilar v_t
            \and
                \Phi (\lambdUnit, \lambdUnit)
            }{
                \iSim[\Chi]{\Phi}{\lambdStore{\loc_s}{i}{v_s}}{\lambdStore{\loc_t}{i}{v_t}}
            }
    \end{mathparpagebreakable}
    \caption{Reasoning rules for simulation (excerpt, omitting freshness conditions)}
    \label{fig:sim_rules}
\end{figure}

% TODO add the rule that uses X

\begin{figure}[tp]
    \centering
    \begin{mathparpagebreakable}
        \inferrule*
            {}{
                \lambdUnit \iSimilar \lambdUnit
            }
        \and
        \inferrule*
            {}{
                i \iSimilar i
            }
        \and
        \inferrule*
            {}{
                n \iSimilar n
            }
        \and
        \inferrule*
            {}{
                b \iSimilar b
            }
        \and
        \inferrule*
            {
                \forall i \in \{0,1,2\} \ldotp
                (\loc_s + i) \iInBij (\loc_t + i)
            }{
                \loc_s \iSimilar \loc_t
            }
        \and
        \inferrule*
            {
                f \in \dom{p_s}
            }{
                \lambdFunc{f} \iSimilar \lambdFunc{f}
            }
    \end{mathparpagebreakable}
    \caption{Similarity in $\iProp$}
    \label{fig:isimilar}
\end{figure}
\begin{figure}[tp]
    \begin{tabular}{rcl}
            $\iProt_\mathrm{dir} (\iPredTwo, \datalangExpr_s, \datalangExpr_t)$
            & $\coloneqq$ &
            $\exists \datalangFn, \datalangVal_s, \datalangVal_t \ldotp$
        \\
            &&
            $\datalangFn \in \dom{\datalangProg_s} \iSep
            \datalangExpr_s = \datalangCall{\datalangFnptr{\datalangFn}}{\datalangVal_s} \iSep
            \datalangExpr_t = \datalangCall{\datalangFnptr{\datalangFn}}{\datalangVal_t} \iSep {}$
        \\
            &&
            $\datalangVal_s \iSimilar \datalangVal_t \iSep {}$
        \\
            &&
            $\forall \datalangValTwo_s, \datalangValTwo_t \ldotp
            \datalangValTwo_s \iSimilar \datalangValTwo_t \iWand
            \iPredTwo (\datalangValTwo_s, \datalangValTwo_t)$
        \\
            $\iProt_\mathrm{DPS} (\iPredTwo, \datalangExpr_s, \datalangExpr_t)$
            & $\coloneqq$ &
            $\exists \datalangFn, \datalangFn_\mathit{dps}, \datalangVal_s, \datalangLoc_1, \datalangLoc_2, \datalangLoc, \datalangIdx, \datalangVal_t \ldotp$
        \\
            &&
            $\datalangFn \in \dom{\datalangProg_s} \iSep
            \datalangRenaming [\datalangFn] = \datalangFn_\mathit{dps} \iSep
            \datalangExpr_s = \datalangCall{\datalangFnptr{\datalangFn}}{\datalangVal_s} \iSep
            \datalangExpr_t = \datalangCall{\datalangFnptr{\datalangFn_\mathit{dps}}}{\datalangLoc_1} \iSep {}$
        \\
            &&
            $(\datalangLoc_1 + 1) \iPointsto_t (\datalangLoc_2, \datalangVal_t) \iSep
            (\datalangLoc_2 + 1) \iPointsto_t (\datalangLoc, \datalangIdx) \iSep
            (\datalangLoc + \datalangIdx) \iPointsto \datalangHole \iSep
            \datalangVal_s \iSimilar \datalangVal_t$
        \\
            &&
            $\forall \datalangValTwo_s, \datalangValTwo_t \ldotp
            (\datalangLoc + \datalangIdx) \iPointsto \datalangValTwo_t \iSep
            \datalangValTwo_s \iSimilar \datalangValTwo_t \iWand
            \iPredTwo (\datalangValTwo_s, \datalangUnit)$
        \\
            $\iProt_\mathrm{TMC}$
            & $\coloneqq$ &
            $\iProt_\mathrm{dir} \sqcup \iProt_\mathrm{DPS}
            =
            \lambdaAbs (\iPredTwo, \datalangExpr_s, \datalangExpr_t) \ldotp \iProt_\mathrm{dir} (\iPredTwo, \datalangExpr_s, \datalangExpr_t) \iOr \iProt_\mathrm{DPS} (\iPredTwo, \datalangExpr_s, \datalangExpr_t)$
    \end{tabular}
    \caption{TMC protocol ($\iProt_\mathrm{TMC}$)}
    \label{fig:protocol}
\end{figure}
\begin{figure}[tp]
    \begin{align*}
    		\wf{\Gamma}
    		&\coloneqq
    		\forall x \ldotp
    		\exists v_s, v_t \ldotp
    		\Gamma (x) = (v_s, v_t) \iSep
    		v_s \iSimilar v_t
    	\\
    		\iCsimv{\Phi}{e_s}{e_t}
    		&\coloneqq
    		\forall \Gamma \ldotp
    		\wf{\Gamma} \iSepImp
    		\iSimv[\Chi_\mathrm{TMC}]{\Phi}{\Gamma_s (e_s)}{\Gamma_t (e_t)}
    	\\
    	   \iCsimvHoare{P}{\Phi}{e_s}{e_t}
    	   &\coloneqq
    	   \iPersistent \left( P \iSepImp \iCsimv{\Phi}{e_s}{e_t} \right)
    \end{align*}
    \caption{Closed simulation}
    \label{fig:csim}
\end{figure}

\begin{theorem}[Soundness]
    $
        \wf{p_s} \wedge p_s \tmc p_t \implies
        p_s \refined p_t
    $
\end{theorem}

\begin{theorem}[Close simulation]
    \begin{align*}
            &
            \iPersistent \left(
                \forall \Psi, e_s, e_t \ldotp
                \Chi (\Psi, e_s, e_t) \iSepImp
                \mathrm{sim \mathhyphen inner}_\bot (\lambdaAbs (\_, e_s', e_t') \ldotp \iSim[\Chi]{\Psi}{e_s'}{e_t'}) (\bot, e_s, e_t)
            \right) \iSepImp
        \\
            &
            \iSim[\Chi]{\Phi}{e_s}{e_t} \iSepImp
            \iSim[\bot]{\Phi}{e_s}{e_t}
    \end{align*}
\end{theorem}

\begin{theorem}[Adequacy]
    $
        \left( \vdash \iSimv{\iSimilar}{e_s}{e_t} \right) \implies
        e_s \refined e_t
    $
\end{theorem}

\begin{theorem}[Specification of direct transformation]
    \[
        \iCsimvHoare{
            \wf{e_s} \iSep
            e_s \tmcDir{\xi} e_t
        }{
            \iSimilar
        }{
            e_s
        }{
            e_t
        }
    \]
\end{theorem}

\begin{theorem}[Specification of DPS transformation]
    \[
        \iCsimvHoare{
            \wf{e_s} \iSep
            (\loc, i, e_s) \tmcDps{\xi} e_t \iSep
            (\loc + i) \mapsto \lambdHole
        }{
            \lambdaAbs (v_s, w_t) \ldotp
            \exists v_t \ldotp
            w_t = \lambdUnit \iSep
            (\loc + i) \mapsto v_t \iSep
            v_s \iSimilar v_t
        }{
            e_s
        }{
            e_t
        }
    \]
\end{theorem}

behaviour refinement

termination preserving (and safety preserving) refinement

state of the art: Simuliris

% Dans ReLoC on regarde seulement les comportements convergeants vers une valeur.
% termination-preserving refinement, tu rajoutes le Div -- la 3e règle mais pas la 2e

% 1.  et value => es value  (ReLoC)
% 3.  et diverges => es diverges  (Simuliris)
% 2.  et fails => es fails (this work)
%
% Simulation that is:
% - termination-preserving: no extra divergence appears in the target (already with Simuliris)
% - safety-preserving: no extra failure appears in the target (new compared to ReLoC and Simuliris)
%
% We are also divergence-preserving: if a source programs only diverges, then the target must diverge.
%   With Simuliris the target program could also fail (have no behaviors for their notion of behaviors)
%
% But: if a source program non-deterministically diverges or converge,
% the target may choose only a subset of those behaviors.

relational separation logic = simulation

problem: direct calling convention in Simuliris, we need two calling conventions (direct + DPS)

remedy: simulation parameterized by protocol, generalization of Simuliris (but: sequential, without source safety but closed at toplevel)

% X_tmc := X_dir \uplus X_dps

% 1. when we are done
% 2. target stuttering (inductive)
% 3. all one-step source movees
%    3a. no source change (inductive)
%    3b. at least one source reduction (coinductive / guarded)
% 4. "open" simulation; X (\Khi) characterizes admissible foreign calls

% A trick from Simuliris:
% sim-after-progress: instance of sim-inner with bottom in well-chosen places
% closed simulation theorem: if the protocol X is "closable",
%   then the simulation relation for X "collapses" into a closed simulation (bottom for X, no case 4)
%
% MYSTERY: from a high-level perspective this "closd simulation theorem" feels a bit obvious:
% its hypothesis says that your protocol X can always be replaced by making progress on both sides,
% so it should be "easy" to transform a simulation proof using 1-2-3-4 into a simulation proof using only 1-2-3.
%
% Question: if it is "easy", cannot you do this right away (in the proof of simulation for a specific X that satisfies
% this property), proving a closed simulation (without 4) directly?
%
% Clément thinks that the answer is that this would require a super
% tricky coinduction, just like the one used in the proof of this
% theorem.
%
% Clément: this theorem is a sort of coinduction principle. You can
% prove an open simulation by induction (no coinduction
% apparently required), and then applying this theorem gives a result
% for which a direct proof would require a coinduction.

DPS calling convention

TMC protocol

% The X for TMC is essentially a relation/first-order rephrasing of
% the relation specs for direct-style and destination-passing-style
% transformations.

heap bijection

% I(sigma, sigma´) is as in Simuliris, not given explicitly in the current document.
% the \approx^bij relation refers to a component of I (a bijection between addresses)
% that is an implicit parameter of the definitions.

adequacy

% If two expressions refine internally (thy are in the
% simulation relation), and go to the internal equivalence between
% values, then they refine externally (they are in the denotational
% refinement relation).

% This is similar to the Simuliris proof, simplified thanks to the absence
% of concurrency.

simulation rules

% Pure reductions: those that do not need state (neither write nor read)

% Gabriel: we could introduc the non-deterministic pair rules into the "pure" relation
% if we changed the pure-target rule to quantify over all e'_t such that e_t -{pure}-> e'_t.
% This would please Gabriel (existential on the left, universal on the right) and be more compact,
% but Clément prefers the current presentation -- it is more standard for Iris folks -- and closer
% to the mechanization.

% SimBijInsert: \approx^bij: Simuliris calls this the escaped/public
% locations, those that have been published.

% Clément now mentions Figure 12, which shows how stores evolve on an
% example in the TMC case. The shape of the relation between the
% source and target stores is sort of implicit in the program logic
% simulations rules, but it is clear on this example.
%
% Clément would explain this to give an informal sense of the proof,
% before even talking about specification logic etc.

closing simulation

proof