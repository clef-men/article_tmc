\section{Soundness proof}

The proof technique we propose to verify program transformations such
as TMC is to capture a relational specification between inputs and
outputs of the translation using a relational separation logic.

We use a quadruple of the form $\iSim[\iProt]{\iPred}{\datalangExpr_s}{\datalangExpr_t}$, which can be read as ``$\datalangExpr_t$ refines $\datalangExpr_s$ under the postcondition $\iPred$ and protocol $\iProt$''. Intuitively, it states that $\datalangExpr_s$ and $\datalangExpr_t$ behave in a related way, and eventually (both get stuck or) reduce to expressions in the relational post-condition $\iPred$. The parameter $\iProt$ is a \emph{protocol} that specifies ``external'' transitions, typically used to model function calls.

Traditionally this quadruple would be a judgment in a bespoke system of inference rules, proved correct by establishing a soundness or adequacy theorem. Our work instead uses the standard Iris approach of defining it directly as an assertion in a separation logic. The inference rules that we will provide are admissible lemmas, and they use separation logic as their meta-logic.

We defer the definition of the assertion as a simulation to \cref{sec:adequacy}, where it will also be proved adequate with respect to a notion of behavior refinement expressed outside Iris.

\section{Adequacy}
\label{sec:soundness}
\label{sec:adequacy}

(This paragraph below probably rather belongs to the introduction. Abort.)

The formal contribution of our work is a mechanized soundness proof for the TMC transformation on a simplified programming language, exhibiting the salient parts of the actual transformation for OCaml.
%
The destination-passing style used by the TMC transformation critically relies on uniquely-owned mutable state.
%
It is thus natural to use separation logic as our specification language.

\subsection{The quest for the right notion of simulation}
\label{sec:howto-relation}

The final theorem we want to establish is a behaviour refinement $\datalangProg_s \refined \datalangProg_t$: the behaviours of the transformed program are included in the behaviour of the source program.
%
We propose a notion of behaviour refinement that gives a form of total correctness.
%
It is termination-preserving, but also (this is less common in the body of work around Iris) safety-preserving and divergence-preserving.

We prove this using a backward simulation argument for our transformations $\datalangExpr_s \tmc \datalangExpr_t$.
%
A direct approach, ultimately unsucessful, would be to prove that the relation is in the simulation by induction, on $\datalangExpr_s$ for example.
%
This works well for all constructions except for function calls $\datalangCall{\datalangFnptr{\datalangFn}}{\datalangExpr}$: to reason on the relation between a call to $\datalangFnptr{\datalangFn}$ and its transformation (in direct or DPS style), we would need to inline the function definition, which is not structurally decreasing.

This is a common problem to establish simulations, and the solution is to use a form of co-induction.
%
In the Iris logic, a natural way to do this would be to use a ``later'' modality.
%
But defining simulations using a ``later'' modality is in fact non-trivial; simple definitions do not give the excepted notion of simulation, due to non-intuitive aspects of the step-indexing semantics of ``later'' in the Iris model. This problem is discussed in details in the previous work on Transfinite Iris~\citep*{TODO-transfinite-Iris}, an alternative logic with different axioms and models.
%
Sticking to the main Iris logic, a good definition of simulation has been worked out in Simuliris~\citep*{TODO-Simuliris}.
%
It avoids using the ``later'' modality by using the notion of coinduction native to Coq. In Simuliris, coinductive reasoning steps correspond to an ``abstract'' case in the simulation relation, used for function applications; an ``open simulation'' allows these abstract steps, while a ``closed simulation'' does not allow them.
%
The coinduction principle is encapsulated in a ``simulation closure'' theorem, which states if two terms are related in the open simulation, they are in fact also related in the closed simulation if we (coinductively) assume that foreign function bodies are correct.

Our proof is heavily inspired by the Simuliris work, but we cannot reuse their results directly because the notion of simulation that they provide is not expressive enough for our proof: we need to support transformations that change the calling conventions between source and target programs.
%
(Along the way we also simplified the simulation relation by removing proof rules related to concurrency, which we do not consider in this work.)

To support different calling conventions, we make our simulation relation $\iSim[\iProt]{\iPred}{\datalangExpr_s}{\datalangExpr_t}$ parametric over abstract protocols $\iProt$. This extends the technique of \citet*{TODO-paulo} from the unary setting of safety predicates to the binary setting of a relational separation logic.

\subsection{Proof outline}

Once we have decided to reuse and extend the Simuliris work, our argument can be split in three separate parts:

\begin{enumerate}

\item Define our notion of simulation parametric over protocols (calling conventions), prove its adequacy (it implies a behaviour refinement) and establish a ``closure theorem'' as in the Simuliris work.
%
  This technical contribution is independent of the TMC language or transformation, and could be reused in other works.

\item Build a relational program logic for our programming language, as a body of lemmas to establish simulations between expressions.

\item Use this program logic to establish the soundness of the TMC transformation.
%
The key ingredients are two specifications for the direct and DPS transformations that are proved in a mutually-inductive way, and are used to prove the hypothesis of the closure theorem of our simulation.
\end{enumerate}

The rest of this section expands each step of the proof, providing the necessary technical details to follow the Coq formalization.

relational separation logic = simulation

problem: direct calling convention in Simuliris, we need two calling conventions (direct + DPS)

remedy: simulation parameterized by protocol, generalization of Simuliris (but: sequential, without source safety but closed at toplevel)

% X_tmc := X_dir \uplus X_dps

% 1. when we are done
% 2. target stuttering (inductive)
% 3. all one-step source movees
%    3a. no source change (inductive)
%    3b. at least one source reduction (coinductive / guarded)
% 4. "open" simulation; X (\iProt) characterizes admissible foreign calls

% A trick from Simuliris:
% sim-after-progress: instance of sim-inner with bottom in well-chosen places
% closed simulation theorem: if the protocol X is "closable",
%   then the simulation relation for X "collapses" into a closed simulation (bottom for X, no case 4)
%
% MYSTERY: from a high-level perspective this "closd simulation theorem" feels a bit obvious:
% its hypothesis says that your protocol X can always be replaced by making progress on both sides,
% so it should be "easy" to transform a simulation proof using 1-2-3-4 into a simulation proof using only 1-2-3.
%
% Question: if it is "easy", cannot you do this right away (in the proof of simulation for a specific X that satisfies
% this property), proving a closed simulation (without 4) directly?
%
% Clément thinks that the answer is that this would require a super
% tricky coinduction, just like the one used in the proof of this
% theorem.
%
% Clément: this theorem is a sort of coinduction principle. You can
% prove an open simulation by induction (no coinduction
% apparently required), and then applying this theorem gives a result
% for which a direct proof would require a coinduction.

DPS calling convention

TMC protocol

% The X for TMC is essentially a relation/first-order rephrasing of
% the relation specs for direct-style and destination-passing-style
% transformations.

heap bijection

% I(sigma, sigma´) is as in Simuliris, not given explicitly in the current document.
% the \approx^bij relation refers to a component of I (a bijection between addresses)
% that is an implicit parameter of the definitions.

adequacy

% If two expressions refine internally (thy are in the
% simulation relation), and go to the internal equivalence between
% values, then they refine externally (they are in the denotational
% refinement relation).

% This is similar to the Simuliris proof, simplified thanks to the absence
% of concurrency.

simulation rules

% Note: those inference rules are in fact Iris propositions
% (the separating conjunction of the premises magic-wands the conclusion)

% Pure reductions: those that do not need state (neither write nor read)

% Gabriel: we could introduce the non-deterministic pair rules into the "pure" relation
% if we changed the pure-target rule to quantify over all e'_t such that \datalangExpr_t -{pure}-> e'_t.
% This would please Gabriel (existential on the left, universal on the right) and be more compact,
% but Clément prefers the current presentation -- it is more standard for Iris folks -- and closer
% to the mechanization.

% SimBijInsert: \approx^bij: Simuliris calls this the escaped/public
% locations, those that have been published.

% Clément now mentions Figure 12, which shows how stores evolve on an
% example in the TMC case. The shape of the relation between the
% source and target stores is sort of implicit in the program logic
% simulations rules, but it is clear on this example.
%
% Clément would explain this to give an informal sense of the proof,
% before even talking about specification logic etc.

closing simulation

proof


\begin{theorem}[Specification of direct transformation]
    \[
        \iRsimvHoare{
            \wf{\datalangExpr_s} \iSep
            \datalangExpr_s \tmcDir{\datalangRenaming} \datalangExpr_t
        }{
            \iSimilar
        }{
            \datalangExpr_s
        }{
            \datalangExpr_t
        }
    \]
\end{theorem}

\begin{theorem}[Specification of DPS transformation]
    \[
        \iRsimvHoare{
            \wf{\datalangExpr_s} \iSep
            (\datalangLoc, \datalangIdx, \datalangExpr_s) \tmcDps{\datalangRenaming} \datalangExpr_t \iSep
            (\datalangLoc + \datalangIdx) \iPointsto \datalangHole
        }{
            \lambdaAbs (\datalangVal_s, \datalangVal_t') \ldotp
            \exists \datalangVal_t \ldotp
            \datalangVal_t' = \datalangUnit \iSep
            (\datalangLoc + \datalangIdx) \iPointsto \datalangVal_t \iSep
            \datalangVal_s \iSimilar \datalangVal_t
        }{
            \datalangExpr_s
        }{
            \datalangExpr_t
        }
    \]
\end{theorem}

% Those two statements are proven together.
% (First by induction on \datalangExpr_s, then on the two transformation relations mutually-inductively.)

% Note: the \geq here is a "parametric" simulation, we assume that all
% free variables of \datalangExpr_s, \datalangExpr_t (they must be the same) are replaced by
% externally-equivalent values.

% Taking a step back: to prove the contextual refinements between programs,
% we have to check each function,
%   @f vs refined by @f vt  (for vs externally-equiv. vt)
% for this we can use the adequacy lemma
%   f vs simulated by @f vt (for the empty protocol) into internal value equivalence
% and for this we use the simulation closure theorem
%   f vs simulated by @f vt (for the Xtmc protocol)
% assuming the Xtmc protocol is valid/adequate.
%
% But then this is trivial, just use case (4) right away.
% So the meat of the proof is in showing that the Xtmc protocol is valid.

% To prove that the Xtmc propocol is valid:
% - in the direct case, we have a function application on both sides,
%   so we make a pure step on both sides and we have to show that the function bodies
%   (with a bisubstitution (x mapsto (vs, vt)) applied) are in the relation.