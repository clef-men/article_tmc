\section{TMC on an idealized language}
\label{sec:formalization}

In this section, we formalize the ``tail modulo cons'' (TMC) transformation in an idealized language, \DataLang, that is expressive enough to account for the main aspects of TMC but does not support all features of \OCaml.
Our proof of correctness covers this idealized fragment.
We intentionally keep the presentation very close to our Coq development, which can be referred to for full details.

\begin{figure}[tp]
    \begin{tabular}{rclclcl}
            $\nterm{Index}$
            & $\ni$ &
            i
            & $\Coloneqq$ &
            $\lambdZero \mid \lambdOne \mid \lambdTwo$
            &&
            index
        \\
            $\nterm{Tag}$
            & $\ni$ &
            $t$
            &&
            &&
            tag
        \\
            $\mathbb{B}$
            & $\ni$ &
            $b$
            &&
            &&
            boolean
    	\\
    		$\mathbb{L}$
    		& $\ni$ &
    		$\loc$
    		&&
    		&&
    		location
        \\
            $\mathbb{F}$
            & $\ni$ &
            $f$
            &&
            &&
            function
        \\
            $\mathbb{X}$
            & $\ni$ &
            $x, y, z$
            &&
            &&
            variable
    	\\
            $\nterm{Val}$
            & $\ni$ &
            $v$
            & $\Coloneqq$ &
            $\lambdUnit \mid i \mid t \mid b \mid \ell \mid \lambdFunc{f}$
            &&
            value
        \\
            $\nterm{Expr}$
            & $\ni$ &
            $e$
            & $\Coloneqq$ &
            $v$
            &&
            expression
        \\
            &&
            & | &
            $x \mid \lambdLet{x}{e_1}{e_2} \mid \lambdCall{e_1}{e_2}$
        \\
            &&
            & | &
            $\lambdEq{e_1}{e_2}$
        \\
            &&
            & | &
            $\lambdIf{e_0}{e_1}{e_2}$
        \\
            &&
            & | &
            $\lambdConstr{t}{e_1}{e_2} \mid \lambdConstrDet{t}{e_1}{e_2}$
        \\
            &&
            & | &
            $\lambdLoad{e_1}{e_2}$
        \\
            &&
            & | &
            $\lambdStore{e_1}{e_2}{e_3}$
        \\
            $\nterm{Ectx}$
            & $\ni$ &
            $K$
            & $\Coloneqq$ &
            $\Box$
            &&
            evaluation context
        \\
            &&
            & | &
            $\lambdLet{x}{K}{e_2} \mid \lambdCall{e_1}{K} \mid \lambdCall{K}{v_2}$
        \\
            &&
            & | &
            $\lambdEq{e_1}{K} \mid \lambdEq{K}{v_2}$
        \\
            &&
            & | &
            $\lambdIf{K}{e_1}{e_2}$
        \\
            &&
            & | &
            $\lambdLoad{e_1}{K} \mid \lambdLoad{K}{v_2}$
        \\
            &&
            & | &
            $\lambdStore{e_1}{e_2}{K} \mid \lambdStore{e_1}{K}{v_3} \mid \lambdStore{K}{v_2}{v_3}$
        \\
            $\nterm{Def}$
            & $\ni$ &
            $d$
            & $\Coloneqq$ &
            $\lambdDef{x}{e}$
            &&
            definition
        \\
            $\nterm{Prog}$
            & $\ni$ &
            $p$
            & $\coloneqq$ &
            $\mathbb{F} \finmap \nterm{Def}$
            &&
            program
        \\
            $\nterm{State}$
            & $\ni$ &
            $\sigma$    
            & $\coloneqq$ &
            $\mathbb{L} \finmap \nterm{Val}$
            &&
            state
        \\
            $\nterm{Config}$
            & $\ni$ &
            $\rho$
            & $\coloneqq$ &
            $\nterm{Expr} \times \nterm{State}$
            &&
            configuration
        \\
            $\nterm{Bisubst}$
            & $\ni$ &
            $\Gamma$
            & $\coloneqq$ &
            $\mathbb{X} \rightarrow \nterm{Expr} \times \nterm{Expr}$
            &&
            bisubstitution
    \end{tabular}
    \caption{Syntax of \LambdaLang}
    \label{fig:syntax}
\end{figure}
\begin{figure}[tp]
    \begin{tabular}{rcl}
            $\datalangSeq{\datalangExpr_1}{\datalangExpr_2}$
            & $\coloneqq$ &
            $\datalangLet{\datalangVar}{\datalangExpr_1}{\datalangExpr_2}$
        \\
            $\datalangPair{\datalangExpr_1}{\datalangExpr_2}$
            & $\coloneqq$ &
            $\datalangBlock{\mathrm{PAIR}}{\datalangExpr_1}{\datalangExpr_2}$
        \\
            $\datalangLet{\datalangPair{\datalangVar_1}{\datalangVar_2}}{\datalangExpr_1}{\datalangExpr_2}$
            & $\coloneqq$ &
            $\datalangLet{\datalangVarTwo}{\datalangExpr_1}{$
        \\
            &&
            $\datalangLet{\datalangVar_1}{\datalangLoad{\datalangVarTwo}{1}}{$
        \\
            &&
            $\datalangLet{\datalangVar_2}{\datalangLoad{\datalangVarTwo}{2}}{$
        \\
            &&
            $\datalangExpr_2}}}$
        \\
            $\datalangRec{\datalangPair{\datalangVar_1}{\datalangVar_2}}{\datalangExpr}$
            & $\coloneqq$ &
            $\datalangRec{\datalangVarTwo}{
            \datalangLet{\datalangPair{\datalangVar_1}{\datalangVar_2}}{\datalangVarTwo}{\datalangExpr}}$
        \\
            $\datalangNil$
            & $\coloneqq$ &
            $\datalangUnit$
        \\
            $\datalangCons{\datalangExpr_1}{\datalangExpr_2}$
            & $\coloneqq$ &
            $\datalangBlock{\mathrm{CONS}}{\datalangExpr_1}{\datalangExpr_2}$
        \\
            $\datalangMatchOneline[]{\datalangExpr_0}{\datalangExpr_1}{\datalangVar}{\mathit{\datalangVar s}}{\datalangExpr_2}$
            & $\coloneqq$ &
            $\datalangLet{\datalangVarTwo}{\datalangExpr_0}{$
        \\
            &&
            $\datalangIf{\datalangEq{\datalangVarTwo}{\datalangNil}$
        \\
            &&
            $}{\datalangExpr_1$
        \\
            &&
            $}{\datalangLet{\datalangPair{\datalangVar}{\mathit{\datalangVar s}}}{\datalangVarTwo}{\datalangExpr_2}}}$
        \\
            $\datalangHole$
            & $\coloneqq$ &
            $\datalangUnit$
    \end{tabular}
    \caption{\DataLang syntactic sugar (omitting freshness conditions)}
    \label{fig:sugar}
\end{figure}
\begin{figure}[tp]
    \begin{mathparpagebreakable}
        \inferrule*
            {}{
                \boxed{- \headStep{\datalangProg} - : \datalangConfig[] \rightarrow \datalangConfig[] \rightarrow \Prop}
            }
        \and
        \inferrule*
            {}{
                \boxed{- \step{\datalangProg} - : \datalangConfig[] \rightarrow \datalangConfig[] \rightarrow \Prop}
            }
        \\
        \inferrule*[lab=StepLet]
            {}{
                \left(
                    \datalangLet{\datalangVar}{\datalangVal}{\datalangExpr},
                    \datalangState
                \right)
                \headStep{\datalangProg}
                \left(
                    \datalangExpr{} [\datalangVar \backslash \datalangVal],
                    \datalangState
                \right)
            }
        \and
        \inferrule*[lab=StepCall]
            {
                \datalangProg{} [\datalangFn] = (\datalangRec{\datalangVar}{\datalangExpr})
            }{
                \left(
                    \datalangCall{\datalangFnptr{\datalangFn}}{\datalangVal},
                    \datalangState
                \right)
                \headStep{\datalangProg}
                \left(
                    \datalangExpr{} [\datalangVar \backslash \datalangVal],
                    \datalangState
                \right)
            }
        \\
        \inferrule*[lab=StepBlock1]
            {}{
                \left(
                    \datalangBlock{\datalangTag}{\datalangExpr_1}{\datalangExpr_2},
                    \datalangState
                \right)
                \headStep{\datalangProg}
                \left(
                    \begin{array}{l}
                        \datalangLet{\datalangVar_1}{\datalangExpr_1}{\\
                        \datalangLet{\datalangVar_2}{\datalangExpr_2}{\\
                        \datalangBlockDet{\datalangTag}{\datalangVar_1}{\datalangVar_2}}}
                    \end{array},
                    \datalangState
                \right)
            }
        \and
        \inferrule*[lab=StepBlock2]
            {}{
                \left(
                    \datalangBlock{\datalangTag}{\datalangExpr_1}{\datalangExpr_2},
                    \datalangState
                \right)
                \headStep{\datalangProg}
                \left(
                    \begin{array}{l}
                        \datalangLet{\datalangVar_2}{\datalangExpr_2}{\\
                        \datalangLet{\datalangVar_1}{\datalangExpr_1}{\\
                        \datalangBlockDet{\datalangTag}{\datalangVar_1}{\datalangVar_2}}}
                    \end{array},
                    \datalangState
                \right)
            }
        \\
        \inferrule*[lab=StepBlockDet]
            {
                \forall \datalangIdx \in \datalangIdx[], \datalangLoc + \datalangIdx \notin \dom{\datalangState}
            }{
                \left(
                    \datalangBlockDet{\datalangTag}{\datalangVal_1}{\datalangVal_2},
                    \datalangState
                \right)
                \headStep{\datalangProg}
                \left(
                    \datalangLoc,
                    \datalangState{} [\datalangLoc \mapsto \datalangTag, \datalangVal_1, \datalangVal_2]
                \right)
            }
        \and
        \inferrule*[lab=StepLoad]
            {
                \datalangState{} [\datalangLoc] = \datalangVal
            }{
                \left(
                    \datalangLoad{\datalangLoc}{\datalangIdx},
                    \datalangState
                \right)
                \headStep{\datalangProg}
                \left(
                    \datalangVal,
                    \datalangState
                \right)
            }
        \and
        \inferrule*[lab=StepStore]
            {
                \datalangLoc + \datalangIdx \in \dom{\datalangState}
            }{
                \left(
                    \datalangStore{\datalangLoc}{\datalangIdx}{\datalangVal},
                    \datalangState
                \right)
                \headStep{\datalangProg}
                \left(
                    \datalangUnit,
                    \datalangState{} [\datalangLoc + \datalangIdx \mapsto \datalangVal]
                \right)
            }
        \and
        \inferrule*[lab=StepEctx]
            {
                \left(
                    \datalangExpr,
                    \datalangState
                \right)
                \headStep{\datalangProg}
                \left(
                    \datalangExpr',
                    \datalangState'
                \right)
            }{
                \left(
                    \datalangEctx{} [\datalangExpr],
                    \datalangState
                \right)
                \step{\datalangProg}
                \left(
                    \datalangEctx{} [\datalangExpr'],
                    \datalangState'
                \right)
            }
    \end{mathparpagebreakable}
    \caption{\DataLang semantics (excerpt)}
    \label{fig:semantics}
\end{figure}
\begin{figure}[tp]
\begin{tabular}{c}
\begin{Datalang}
map |-> rec λ(f, xs) = match xs with
                       | [] -> []
                       | x :: xs -> let y = f x in y :: @map (f, xs)
\end{Datalang}
\end{tabular}
\caption{Natural implementation of \datalang|map| in \DataLang}
\label{fig:map}
\end{figure}

\subsection{Language}

The syntax of \DataLang is given in \cref{fig:syntax} and its semantics in \cref{fig:semantics}.
We also introduce syntactic sugar in \cref{fig:sugar}, in particular shallow pattern-matching on lists.
Going back to our motivating example, we can define the \datalang|map| function on lists as in \cref{fig:map}.

\DataLang is an untyped sequential calculus with mutable state. A \DataLang program $\datalangProg$ is a finite mapping from function names $\datalangFn \in \datalangFn[]$ to mutually-recursive definitions $d$, which are themselves functions whose body is written $\datalangRec \datalangVar \datalangExpr$. Functions have a single parameter for simplicity, with pairs used to pass several values.

\DataLang has Booleans $b \in \{\datalangTrue, \datalangFalse\}$, an if-then-else construct, and a runtime equality test between (untyped) values $\datalangEq{\datalangExpr_1}{\datalangExpr_2}$.

\DataLang is first-order in the same sense that C is (with function pointers): it does not feature general lambda expressions, its programs correspond to closure-converted or lambda-lifted source programs.\footnote{
The usual definition of TMC that we implement and formalize is essentially first-order. See \cref{subsubsec:first-order}.}
Functions names $\datalangFn$ can be turned into values written $\datalangFnptr{\datalangFn}$, to be used directly in function calls or as parameters to higher-order functions.

To express constructors, \DataLang features mutable memory blocks with an abstract \emph{tag} ($\datalangTag \in \datalangTag[]$), and two \emph{fields} which are arbitrary values ($\datalangExpr_1, \datalangExpr_2$).
One can allocate a block with $\datalangBlock{\datalangTag}{\datalangExpr_1}{\datalangExpr_2}$, access its fields with $\datalangLoad{\datalangExpr_1}{\datalangExpr_2}$ and modify them with $\datalangStore{\datalangExpr_1}{\datalangExpr_2}{\datalangExpr_3}$.
Allocation returns a location $\datalangLoc \in \datalangLoc[]$, which may not appear in source programs.

The evaluation order of subexpressions $\datalangExpr_1$ and $\datalangExpr_2$ in $\datalangBlock{\datalangTag}{\datalangExpr_1}{\datalangExpr_2}$ is unspecified as in \OCaml.
This is crucial to allow the behavior-preserving optimization of more programs, as the TMC transformation may affect the evaluation order of subterms of data constructors.
To model this in the semantics, we introduce a separate, deterministic block construction $\datalangBlockDet{\datalangTag}{\datalangExpr_1}{\datalangExpr_2}$ which cannot appear in source programs. A block expression $\datalangBlock{\datalangTag}{\datalangExpr_1}{\datalangExpr_2}$ first nondeterministically reduces to either $\datalangLet{\datalangVar_1}{\datalangExpr_1}{\datalangLet{\datalangVar_2}{\datalangExpr_2}{\datalangBlockDet{\datalangTag}{\datalangVar_1}{\datalangVar_2}}}$ through \RefTirName{StepBlock1} or $\datalangLet{\datalangVar_2}{\datalangExpr_2}{\datalangLet{\datalangVar_1}{\datalangExpr_1}{\datalangBlockDet{\datalangTag}{\datalangVar_1}{\datalangVar_2}}}$ through \RefTirName{StepBlock2}. $\datalangBlockDet{\datalangTag}{\datalangVal_1}{\datalangVal_2}$ performs the allocation through \RefTirName{StepBlockDet}.

The values in \DataLang are functions $\datalangFnptr{\datalangFn}$, locations $\datalangLoc$ of allocated blocks, Booleans $b$, tags $t$ (taken in an arbitrary, denumerable set), the unit value $\datalangUnit$, and indices $i \in \{\datalangZero, \datalangOne, \datalangTwo\}$ inside blocks. (Our transformation never mutates tags in position $\datalangZero$, nor do \OCaml programs, but for example \Mezzo supports it.)

On top of these basic language features, \cref{fig:sugar} introduces syntactic sugar for pairs $\datalangPair {\datalangExpr_1} {\datalangExpr_2}$ as blocks with a specific tag $\mathrm{PAIR}$, for decomposing blocks in $\term{\color{datalangKeyword1}{let}}$-bindings ($\datalangLet {\datalangPair \datalangVar \datalangVarTwo} {\datalangExpr_1} {\datalangExpr_2}$) and in arguments of toplevel functions ($\datalangFn \mapsto \datalangRec {\datalangPair \datalangVar \datalangVarTwo} \datalangExpr$), and for (untyped) lists by defining the empty list $\datalangNil$ as $\datalangUnit$, and for (mutable) cons-cells as blocks with a specific tag $\mathrm{CONS}$. Pattern-matching on lists can be expressed by comparing the list with $\datalangNil$, using our block-deconstructing $\term{\color{datalangKeyword1}{let}}$ (which ignores the tag) to deconstruct cons-cells.

As a side note, we use named expression variables here but the \Coq mechanization actually adopts de Bruijn syntax, which is better suited to define transformations involving binders.
More precisely, our formalization relies on the \Autosubst library~\cite{autosubst-2015}.
Our definitions respect $\alpha$-equivalence on term variables $\datalangVar, \datalangVarTwo$: we implicitly assume any term variable in bound position to be chosen distinct from all other variables in context.
Function names $\datalangFn$ are not $\alpha$-renamed, as the transformation relates names in the source and target of the transformation.

\subsection{Transformation}
\label{subsec:transformation}

We now define the TMC transformation as a relation $\datalangProg_s \tmc \datalangProg_t$ between programs and their transformation.
The relation is total, in the sense that any \DataLang program $\datalangProg_s$ can be related to at least one transformed program $\datalangProg_t$.
It is not deterministic: for each input program it captures a (finite) set of admissible transformations, which we all prove valid.
This non-determinism captures several choices that have to be done by the user through a user interface to control the transformation, or by the compiler implementation, influencing performance and evaluation order of the result.
In this section, we do not describe how these choices are resolved -- there is a large design space.
We present the choices we made for \OCaml compiler in \cref{sec:implementation}.

As formalized in \cref{fig:tmc}, transforming a \DataLang program $\datalangProg$ consists in:

\begin{figure}[tp]
    \[
        p_s \tmc p_t
        \coloneqq
        \exists \xi \ldotp
        \bigwedge \left[ \begin{array}{l}
                \dom{\xi} \subseteq \dom{p_s}
            \\
                \dom{p_t} = \dom{p_s} \cup \codom{\xi}
            \\
                \forall f, d_s \ldotp
                p_s [f] = d_s \implies
                \exists d_t \ldotp
                p_t [f] = d_t \wedge d_s \tmcDir{\xi} d_t
            \\
                \forall f, f_\mathit{dps}, d_s \ldotp
                p_s [f] = d_s \wedge \xi [f] = f_{\mathit{dps}} \implies
                \exists d_t \ldotp
                p_t [f_\mathit{dps}] = d_t \wedge d_s \tmcDps{\xi} d_t
        \end{array} \right.
    \]
    \caption{TMC transformation}
    \label{fig:tmc}
\end{figure}
\paragraph{1. Choosing a subset of toplevel functions to be TMC-transformed}
In \OCaml, the programmer has to annotate these.
For each such function $\datalangFn$, we also require a fresh function name $\datalangRenaming [\datalangFn]$ (that is not defined in $\datalangProg_s$) that will be the \emph{destination-passing style} (DPS) version of $\datalangFn$ in the transformed program $\datalangProg_t$.

Formally, the subset is determined by the domain of the renaming function $\datalangRenaming$, which is passed as a parameter to the auxiliary transformations that we describe next.

\begin{figure}[tp]
    \begin{mathparpagebreakable}
        \inferrule*
            {}{
                \boxed{- \tmcDir{\datalangRenaming} - : \datalangDef[] \rightarrow \datalangDef[] \rightarrow \Prop}
            }
        \and
        \inferrule*
            {}{
                \boxed{- \tmcDir{\datalangRenaming} - : \datalangExpr[] \rightarrow \datalangExpr[] \rightarrow \Prop}
            }
        \\
        \inferrule*[lab=DirDef]
            {
                \datalangExpr_s \tmcDir{\datalangRenaming} \datalangExpr_t
            }{
                \datalangRec{\datalangVar}{\datalangExpr_s}
                \tmcDir{\datalangRenaming}
                \datalangRec{\datalangVar}{\datalangExpr_t}
            }
        \\
        \inferrule*[lab=DirVal]
            {}{
                \datalangVal
                \tmcDir{\datalangRenaming}
                \datalangVal
            }
        \and
        \inferrule*[lab=DirVar]
            {}{
                \datalangVar
                \tmcDir{\datalangRenaming}
                \datalangVar
            }
        \and
        \inferrule*[lab=DirLet]
            {
                \datalangExpr_{s1} \tmcDir{\datalangRenaming} \datalangExpr_{t1}
            \and
                \datalangExpr_{s2} \tmcDir{\datalangRenaming} \datalangExpr_{t2}
            }{
                \datalangLet{\datalangVar}{\datalangExpr_{s1}}{\datalangExpr_{s2}}
                \tmcDir{\datalangRenaming}
                \datalangLet{\datalangVar}{\datalangExpr_{t1}}{\datalangExpr_{t2}}
            }
        \\
        \inferrule*[lab=DirCall]
            {
                \datalangExpr_{s1} \tmcDir{\datalangRenaming} \datalangExpr_{t1}
            \and
                \datalangExpr_{s2} \tmcDir{\datalangRenaming} \datalangExpr_{t2}
            }{
                \datalangCall {\datalangExpr_{s1}} {\datalangExpr_{s2}}
                \tmcDir{\datalangRenaming}
                \datalangCall {\datalangExpr_{t1}} {\datalangExpr_{t2}}
            }
        \and
        \inferrule*[lab=DirBlock]
            {
                \datalangExpr_{s1} \tmcDir{\datalangRenaming} \datalangExpr_{t1}
            \and
                \datalangExpr_{s2} \tmcDir{\datalangRenaming} \datalangExpr_{t2}
            }{
                \datalangBlock \datalangTag {\datalangExpr_{s1}} {\datalangExpr_{s2}}
                \tmcDir{\datalangRenaming}
                \datalangBlock \datalangTag {\datalangExpr_{t1}} {\datalangExpr_{t2}}
            }
        \\
        \inferrule*[lab=DirBlockDPS1]
            {
                (\datalangVar, \datalangOne, \datalangExpr_{s1}) \tmcDps{\datalangRenaming} \datalangExpr_{t1}
            \and
                \datalangExpr_{s2} \tmcDir{\datalangRenaming} \datalangExpr_{t2}
            }{
                \datalangBlock{\datalangTag}{\datalangExpr_{s1}}{\datalangExpr_{s2}}
                \tmcDir{\datalangRenaming}
                \begin{array}{l}
                    \datalangLet{\datalangVar}{\datalangBlock{\datalangTag}{\datalangHole}{\datalangExpr_{t2}}}{\\
                    \datalangSeq{\datalangExpr_{t1}}{\datalangVar}}
                \end{array}
            }
        \and
        \inferrule*[lab=DirBlockDPS2]
            {
                \datalangExpr_{s1} \tmcDir{\datalangRenaming} \datalangExpr_{t1}
            \and
                (\datalangVar, \datalangTwo, \datalangExpr_{s2}) \tmcDps{\datalangRenaming} \datalangExpr_{t2}
            }{
                \datalangBlock{\datalangTag}{\datalangExpr_{s1}}{\datalangExpr_{s2}}
                \tmcDir{\datalangRenaming}
                \begin{array}{l}
                    \datalangLet{\datalangVar}{\datalangBlock{\datalangTag}{\datalangExpr_{t1}}{\datalangHole}}{\\
                  \datalangSeq{\datalangExpr_{t2}}{\datalangVar}}
                \end{array}
            }
        \and
        \inferrule*[lab=DirBlockDPS12]
            {
                (\datalangVar, \datalangOne, \datalangExpr_{s1}) \tmcDps{\datalangRenaming} \datalangExpr_{t1}
            \and
                (\datalangVar, \datalangTwo, \datalangExpr_{s2}) \tmcDps{\datalangRenaming} \datalangExpr_{t2}
            }{
                \datalangBlock{\datalangTag}{\datalangExpr_{s1}}{\datalangExpr_{s2}}
                \tmcDir{\datalangRenaming}
                \begin{array}{l}
                    \datalangLet{\datalangVar}{\datalangBlock{\datalangTag}{\datalangHole}{\datalangHole}}{\\
                    \datalangSeq{\datalangExpr_{t1}}{
                      \datalangSeq{\datalangExpr_{t2}}{
                        \datalangVar}}}
                \end{array}
            }
    \end{mathparpagebreakable}
    \caption{Direct TMC transformation (omitting congruence rules similar to \TirName{DirCall})}
    \label{fig:tmc_dir}
\end{figure}

%%% Local Variables:
%%% mode: latex
%%% TeX-master: "../main"
%%% End:

\paragraph{2. For each function $\datalangFn$ defined in $\datalangProg$, computing its direct transform.}
We introduce in \cref{fig:tmc_dir} the relations $\datalangDef_s \tmcDir{\datalangRenaming} \datalangDef_t$ for definitions and $\datalangExpr_s \tmcDir{\datalangRenaming} \datalangExpr_t$ for expressions.

$\datalangDef_s \tmcDir{\datalangRenaming} \datalangDef_t$ expresses that:
1)~$\datalangDef_t$ has the same calling convention as $\datalangDef_s$.
2)~The body of $\datalangDef_t$ is the direct transform of the body of $\datalangDef_s$.

$\datalangExpr_s \tmcDir{\datalangRenaming} \datalangExpr_t$ expresses that $\datalangExpr_t$ is the direct transform of $\datalangExpr_s$.
Intuitively, this means $\datalangExpr_t$ computes the same thing as $\datalangExpr_s$.

The direct-style transform corresponds to the case where we do not have a block that can serve as a destination: this version is used in an arbitrary calling context, not necessarily under a constructor. Most rules are straightforward congruences -- we recursively transform subexpressions and preserve the term constructor. We omit the rules for loads $\datalangLoad{\datalangExpr_1}{\datalangExpr_2}$, stores $\datalangStore{\datalangExpr_1}{\datalangExpr_2}{\datalangExpr_3}$, and the deterministic blocks $\datalangBlockDet \datalangTag {\datalangExpr_1} {\datalangExpr_2}$ which are such simple congruences, just like calls $\datalangCall {\datalangExpr_1} {\datalangExpr_2}$.

The key cases are for a block construct $\datalangBlock \datalangTag {\datalangExpr_1} {\datalangExpr_2}$. We can use this block as a destination, and switch to the destination-passing-style calling convention -- these rules are a source of non-determinism, and the only places in the direct-style transformation where destination-passing-style is introduced. \RefTirName{DirBlock} is a simple congruence rule that keeps both arguments in direct style. The rules (\RefTirName{DirBlockDPS1}, \RefTirName{DirBlockDPS2}) choose a block argument to be evaluated in destination-passing style. (It is also possible to transform both arguments in DPS style, and we include extra rules for this in our formalization. \Xgabriel{TODO: measure the performance of this tweak for \ocaml{map} on binary trees.})

In details, the terms produced by these rules proceed as follows:
1)~They partially initialize a new memory block, with a hole for one of their arguments.
2)~They evaluate the DPS transformation of the corresponding argument.
   passing the uninitialized field as destination.
3)~They return the now fully initialized block.

An implementation would typically determine which subexpression would benefit from destination-passing style, that is, contains function calls $\datalangCall {\datalangFnptr \datalangFn} \datalangExpr$ in tail position (relatively to the subexpression) that have a destination-passing variant $\datalangRenaming [\datalangFn]$.

\begin{figure}[tp]
    \begin{mathparpagebreakable}
        \inferrule*
            {}{
                \boxed{- \tmcDps{\datalangRenaming} - : \datalangDef[] \rightarrow \datalangDef[] \rightarrow \Prop}
            }
        \and
        \inferrule*
            {}{
                \boxed{- \tmcDps{\datalangRenaming} - : \datalangExpr[] \times \datalangExpr[] \times \datalangExpr[] \rightarrow \datalangExpr[] \rightarrow \Prop}
            }
        \\
        \inferrule*[lab=DPSDef]
            {
                (\datalangVar_\mathit{dst}, \datalangVar_\mathit{idx}, \datalangExpr_s) \tmcDps{\datalangRenaming} \datalangExpr_t
            }{
                \datalangRec{\datalangVar}{\datalangExpr_s}
                \tmcDps{\datalangRenaming}
                \datalangRec{\datalangPair{\datalangPair{\datalangVar_\mathit{dst}}{\datalangVar_\mathit{idx}}}{\datalangVar}}{\datalangExpr_t}
            }
        \\
        \inferrule*[lab=DPSLet]
            {
                \datalangExpr_{s1} \tmcDir{\datalangRenaming} \datalangExpr_{t1}
            \and
                (\datalangExpr_\mathit{dst}, \datalangExpr_\mathit{idx}, \datalangExpr_{s2}) \tmcDps{\datalangRenaming} \datalangExpr_{t2}
            }{
                \left(
                  {\begin{array}{l}
                    \datalangExpr_\mathit{dst},
                    \datalangExpr_\mathit{idx}, \\
                    \datalangLet{\datalangVar}{\datalangExpr_{s1}}{\datalangExpr_{s2}}
                  \end{array}}
                \right)
                \tmcDps{\datalangRenaming}
                \datalangLet{\datalangVar}{\datalangExpr_{t1}}{\datalangExpr_{t2}}
            }
        \quad
        \inferrule*[lab=DPSIf]
            {
                \datalangExpr_{s0} \tmcDir{\datalangRenaming} \datalangExpr_{t0}
            \and
                (\datalangExpr_\mathit{dst}, \datalangExpr_\mathit{idx}, \datalangExpr_{s1}) \tmcDps{\datalangRenaming} \datalangExpr_{s2}
            \and
                (\datalangExpr_\mathit{dst}, \datalangExpr_\mathit{idx}, \datalangExpr_{s2}) \tmcDps{\datalangRenaming} \datalangExpr_{t2}
            }{
                \left(
                  {\begin{array}{l}
                    \datalangExpr_\mathit{dst},
                    \datalangExpr_\mathit{idx}, \\
                    \datalangIf{\datalangExpr_{s0}}{\datalangExpr_{s1}}{\datalangExpr_{s2}}
                  \end{array}}
                \right)
                \tmcDps{\datalangRenaming}
                \datalangIf{\datalangExpr_{t0}}{\datalangExpr_{t1}}{\datalangExpr_{t2}}
            }
        \\
        \inferrule*[lab=DPSBlock1]
            {
                (\datalangVar, \datalangOne, \datalangExpr_{s1}) \tmcDps{\datalangRenaming} \datalangExpr_{t1}
            \and
                \datalangExpr_{s2} \tmcDir{\datalangRenaming} \datalangExpr_{t2}
            }{
                \left(
                  {\begin{array}{l}
                    \datalangExpr_\mathit{dst},
                    \datalangExpr_\mathit{idx}, \\
                    \datalangBlock{\datalangTag}{\datalangExpr_{s1}}{\datalangExpr_{s2}}
                  \end{array}}
                \right)
                \tmcDps{\datalangRenaming}
                \begin{array}{l}
                    \datalangLet{\datalangVar}{\datalangBlock{\datalangTag}{\datalangHole}{\datalangExpr_{t2}}}{\\
                    \datalangSeq{\datalangStore{\datalangExpr_\mathit{dst}}{\datalangExpr_\mathit{idx}}{\datalangVar}}{\datalangExpr_{t1}}}
                \end{array}
            }
        \quad
        \inferrule*[lab=DPSBlock2]
            {
                \datalangExpr_{s1} \tmcDir{\datalangRenaming} \datalangExpr_{t1}
            \and
                (\datalangVar, \datalangTwo, \datalangExpr_{s2}) \tmcDps{\datalangRenaming} \datalangExpr_{t2}
            }{
                \left(
                  {\begin{array}{l}
                    \datalangExpr_\mathit{dst},
                    \datalangExpr_\mathit{idx},
                    \\
                    \datalangBlock{\datalangTag}{\datalangExpr_{s1}}{\datalangExpr_{s2}}
                  \end{array}}
                \right)
                \tmcDps{\datalangRenaming}
                \begin{array}{l}
                    \datalangLet{\datalangVar}{\datalangBlock{\datalangTag}{\datalangExpr_{t1}}{\datalangHole}}{\\
                    \datalangSeq{\datalangStore{\datalangExpr_\mathit{dst}}{\datalangExpr_\mathit{idx}}{\datalangVar}}{\datalangExpr_{t2}}}
                \end{array}
            }
        % \and
        % \inferrule*[lab=DPSBlock21]
        %     {
        %         (\datalangVar, \datalangTwo, \datalangExpr_{s1}) \tmcDps{\datalangRenaming} \datalangExpr_{t1}
        %     \and\datalangExpr_{t1}
        %         (\datalangVar, \datalangTwo, \datalangExpr_{s2}) \tmcDps{\datalangRenaming} \datalangExpr_{t2}
        %     }{
        %         \left(
        %             \datalangExpr_\mathit{dst},
        %             \datalangExpr_\mathit{idx},
        %             \datalangBlock{\datalangTag}{\datalangExpr_{s1}}{\datalangExpr_{s2}}
        %         \right)
        %         \tmcDps{\datalangRenaming}
        %         \begin{array}{l}
        %             \datalangLet{\datalangVar}{\datalangBlock{\datalangTag}{\datalangHole}{\datalangHole}}{\\
        %             \datalangSeq{\datalangStore{\datalangExpr_\mathit{dst}}{\datalangExpr_\mathit{idx}}{\datalangVar}}{\\
        %             \datalangSeq{\datalangExpr_{t2}}{\datalangExpr_{t1}}}}
        %         \end{array}
        %     }
        % \;
        % \inferrule*[lab=DPSBlock12]
        %     {
        %         (\datalangVar, \datalangTwo, \datalangExpr_{s1}) \tmcDps{\datalangRenaming} \datalangExpr_{t1}
        %     \and\datalangExpr_{t1}
        %         (\datalangVar, \datalangTwo, \datalangExpr_{s2}) \tmcDps{\datalangRenaming} \datalangExpr_{t2}
        %     }{
        %         \left(
        %             \datalangExpr_\mathit{dst},
        %             \datalangExpr_\mathit{idx},
        %             \datalangBlock{\datalangTag}{\datalangExpr_{s1}}{\datalangExpr_{s2}}
        %         \right)
        %         \tmcDps{\datalangRenaming}
        %         \begin{array}{l}
        %             \datalangLet{\datalangVar}{\datalangBlock{\datalangTag}{\datalangHole}{\datalangHole}}{\\
        %             \datalangSeq{\datalangStore{\datalangExpr_\mathit{dst}}{\datalangExpr_\mathit{idx}}{\datalangVar}}{\\
        %             \datalangSeq{\datalangExpr_{t1}}{\datalangExpr_{t2}}}}
        %         \end{array}
        %     }
        \\
        \inferrule*[lab=DPSCall]
            {
                \datalangFn \in \dom{\datalangRenaming}
            \and
                \datalangExpr_s \tmcDir{\datalangRenaming} \datalangExpr_t
            }{
                \left(
                    \datalangExpr_\mathit{dst},
                    \datalangExpr_\mathit{idx},
                    \datalangCall{\datalangFnptr{\datalangFn}}{\datalangExpr_s}
                \right)
                \tmcDps{\datalangRenaming}
                \datalangCall{
                    \datalangFnptr{\datalangRenaming [\datalangFn]}
                }{
                    \datalangPair{\datalangPair{\datalangExpr_\mathit{dst}}{\datalangExpr_\mathit{idx}}}{\datalangExpr_t}
                }
            }
        \and
        \inferrule*[lab=DPSBase]
            {
                \datalangExpr_s \tmcDir{\datalangRenaming} \datalangExpr_t
            }{
                \left(
                    \datalangExpr_\mathit{dst},
                    \datalangExpr_\mathit{idx},
                    \datalangExpr_s
                \right)
                \tmcDps{\datalangRenaming}
                \datalangStore{\datalangExpr_\mathit{dst}}{\datalangIdx}{\datalangExpr_t}
            }
    \end{mathparpagebreakable}
    \caption{Destination-passing style TMC transformation of definitions and expressions (in full)}
    \label{fig:tmc_dps}
\end{figure}

%%% Local Variables:
%%% mode: latex
%%% TeX-master: "../main"
%%% End:

\paragraph{3. For each TMC-transformed function $\datalangFn$, choosing a destination-passing-style transform}
We introduce in \cref{fig:tmc_dps} the relations $\datalangDef_s \tmcDps{\datalangRenaming} \datalangDef_t$ for definitions and $(\datalangExpr_\mathit{dst}, \datalangExpr_\mathit{idx}, \datalangExpr_s) \tmcDps{\datalangRenaming} \datalangExpr_t$ for expressions.

$\datalangDef_s \tmcDps{\datalangRenaming} \datalangDef_t$ expresses that:
1)~The function defined in $\datalangDef_t$ has an additional parameter representing the destination where it must write its result. This parameter is a pair of the location of a memory block $\datalangVar_\mathit{dst}$ along with the index $\datalangVar_\mathit{idx}$ of a particular field in this block.
2)~The body of $\datalangDef_t$ is a DPS transform of the body of $\datalangDef_s$ under the given destination.

$(\datalangExpr_\mathit{dst}, \datalangExpr_\mathit{idx}, \datalangExpr_s) \tmcDps{\datalangRenaming} \datalangExpr_t$ expresses that $\datalangExpr_t$ is a DPS transform of $\datalangExpr_s$ under destination $(\datalangExpr_\mathit{dst}, \datalangExpr_\mathit{idx})$.
Intuitively, this means $\datalangExpr_t$ computes the same thing as $\datalangExpr_s$ but writes it into the destination instead of returning it.
We will formalize this intuition in \cref{sec:specification}.

Note that the rule \RefTirName{DPSDef}, which relates the two judgments, uses the expression-level relation $(\mathrel{\tmcDps{\datalangRenaming}})$ with a term variable $\datalangVar_\mathit{idx}$ to represent the index, not just a constant index $\datalangOne$ or $\datalangTwo$ as in block rules. These are the only two sort of expressions used to represent offsets in the transformation.

In the direct-style relation, congruence rules apply the same direct-style transformation to all subexpressions. The congruence-like rule of the DPS relation, for example \RefTirName{DPSLet} and \RefTirName{DPSIf}, are different. They apply the DPS transformation to sub-expressions which are in tail position relative to the expression, and the direct-style transformation to all other subexpressions. The if-then-else construct has two different subexpressions in tail position, at most one of them is evaluated at runtime.

The rules \RefTirName{DPSBlock1} and \RefTirName{DPSBlock2}
correspond to the rules \RefTirName{DirBlockDPS1} and \RefTirName{DirBlockDPS2} in the direct-style transformation, but the transformed code is different. Consider the translation of $
\left(
  \datalangExpr_\mathit{dst},
  \datalangExpr_\mathit{idx},
  \datalangBlock{\datalangTag}{\datalangExpr_{s1}}{\datalangExpr_{s2}}
\right)
$ into $
\datalangLet{\datalangVar}{\datalangBlock{\datalangTag}{\datalangExpr_{t1}}{\datalangHole}}{
  \datalangSeq{\datalangStore{\datalangExpr_\mathit{dst}}{\datalangExpr_\mathit{idx}}{\datalangVar}}{\datalangExpr_{t2}}}
$ by \RefTirName{DirBlockDPS2}. First we create a new destination $\datalangVar$, with a hole in second position. Then, instead of computing the corresponding subterm, we write this new destination $x$ into the \emph{current} destination $({\datalangExpr_\mathit{dst}}, {\datalangExpr_\mathit{idx}})$. Finally we evaluate $\datalangExpr_{t2}$, which is the DPS transform of the subexpression $\datalangExpr_{s2}$, with the destination $(\datalangVar, \datalangTwo)$. Notice that $\datalangExpr_{t2}$ is in tail position relative to the transformed expression, while it was not in tail position in the source expression. This is the key step of the TMC transformation, that turns non-tail calls into tail calls. Rule \RefTirName{DirBlockDPS2} puts the second subterm in tail position, and \RefTirName{DirBlockDPS1} puts the first subterm in tail position. It is not always obvious which rule should be applied. In the case of lists as in our running example \ocaml|y :: map f xs|, we want the second subterm in tail position, so the transformation only uses \RefTirName{DirBlockDPS2}. But consider a \ocaml|map| function on binary trees \ocaml|Node(map f left, map f right)|: the implementation must choose one subterm to put in tail position and another to keep in non-tail position.

The rule \RefTirName{DPSCall} applies only to calls $\datalangCall {\datalangFnptr \datalangFn} {\datalangExpr_s}$ to a known function $\datalangFn$, on the condition that a DPS variant has been generated for $\datalangFn$: $\datalangFn \in \dom{\datalangRenaming}$. In this case, the function call can be compiled to a call to the DPS variant $\datalangRenaming [\datalangFn]$, transferring to the callee the responsibility to write to the destination. This is the case where the DPS transform is beneficial, as this transformation may turn a non-tail-call into a tail-call -- when it occurs under a block, in a subterm that was moved to tail position. This rule is selected for \ocaml|map| in our \ocaml|y :: map f xs| example.

Finally, there is a catch-all rule \RefTirName{DPSBase} that applies in any case, in particular whenever none of the other rules can be selected. This case trivially realizes the DPS calling convention by evaluating the subterm to a result and writing this result in the desired destination. This is what happens in the base case of \ocaml|map_dps|, where the empty list \ocaml|[]| is transformed into \ocaml|dst.(idx) <- []|.

% \begin{figure}[tp]
\begin{minipage}{.40\columnwidth}
\begin{Datalang}
map |-> rec (fn, xs) =
  match xs with
  | [] ==> 
      []
  | x :: xs ==>
      let y = fn x in
      let dst = y :: () in
      @map_dps ((dst, 2), (fn, xs)) ;
      dst
\end{Datalang}
\end{minipage}
\hfill
\begin{minipage}{.52\columnwidth}
\begin{Datalang}
map_dps |-> rec ((dst, idx), (fn, xs)) =
  match xs with
  | [] ==> 
      dst.(idx) <- []
  | x :: xs ==>
      let y = fn x in
      let dst' = y :: () in
      dst.(idx) <- dst' ;
      @map_dps ((dst', 2), (fn, xs))
\end{Datalang}
\end{minipage}
\caption{Direct and DPS transforms of \datalang|map| (\DataLang)}
\label{fig:map_tmc}
\end{figure}

% As an example, we can show that the two \DataLang functions defined in \cref{fig:map_tmc} are respectively the direct and DPS transforms of the original \datalang|map| function defined in \cref{fig:map}.
% Thanks to the rules \RefTirName{DirBlockDPS2} and \RefTirName{DPSCall}, the non-tail recursive call is replaced by the allocation of a partially initialized block followed by a call to \datalang|map_dps|.

\subsection{Realizing the relation as a function}

Our Coq formalization includes a function that takes an input program and outputs a related program, following the one-pass implementation approach that we introduced in the \OCaml compiler (\cref{subsec:implementation}).

%%% Local Variables:
%%% mode: latex
%%% TeX-master: "main"
%%% End:
