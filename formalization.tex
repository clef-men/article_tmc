\section{Formalization}

In this section, we formalize the TMC transformation in a simplified setting.
More precisely, we define the \DataLang language that is expressive enough to account for the main aspects of TMC but does not support all features of the \LambdaLang intermediate language.
It is meant to study and verify the core of the transformation.

\subsection{Language}

The syntax of \DataLang is given in \cref{fig:syntax} and its semantics in \cref{fig:semantics}.
We also introduce syntactic sugar in \cref{fig:sugar}, including some for lists.
Going back to our motivating example, we can define the \datalang|map| function on lists as in \cref{fig:map}.

\DataLang is an untyped sequential post closure conversion calculus with mutable state.
Therefore, it does not feature lambda expressions, which are already compiled to toplevel closed functions referred to using function pointers ($\datalangFnptr{\datalangFn}$).
The semantics is parameterized by the set of compiled toplevel functions that constitute the program.

There is one noteworthy trait.
To express constructors and closures, \DataLang features tagged mutable memory blocks with two fields.
One can allocate a block with $\datalangBlock{\datalangTag}{\datalangExpr_1}{\datalangExpr_2}$, access its fields with $\datalangLoad{\datalangExpr_1}{\datalangExpr_2}$ and modify them with $\datalangStore{\datalangExpr_1}{\datalangExpr_2}{\datalangExpr_3}$.
As in \OCaml, the evaluation order of subexpressions $\datalangExpr_1$ and $\datalangExpr_2$ in $\datalangBlock{\datalangTag}{\datalangExpr_1}{\datalangExpr_2}$ is unspecified.
To model this in the semantics, such an expression first nondeterministically reduces to either $\datalangLet{\datalangVar_1}{\datalangExpr_1}{\datalangLet{\datalangVar_2}{\datalangExpr_2}{\datalangBlockDet{\datalangTag}{\datalangVar_1}{\datalangVar_2}}}$ through \RefTirName{StepBlock1} or $\datalangLet{\datalangVar_2}{\datalangExpr_2}{\datalangLet{\datalangVar_1}{\datalangExpr_1}{\datalangBlockDet{\datalangTag}{\datalangVar_1}{\datalangVar_2}}}$ through \RefTirName{StepBlock2} and $\datalangBlockDet{\datalangTag}{\datalangVal_1}{\datalangVal_2}$ performs the allocation  through \RefTirName{StepBlockDet}.
This $\datalangBlockDet{\datalangTag}{\datalangExpr_1}{\datalangExpr_2}$ construction cannot appear in source programs.

As a side note, we use named expression variables here but the \Coq mechanization actually adopts de Bruijn syntax better suited to define transformations involving binders.
More precisely, we rely on the \Autosubst library~\cite{DBLP:conf/itp/SchaferTS15}.

\begin{figure}[tp]
    \begin{tabular}{lclcl}
            $\datalangIdx[]$
            & $\ni$ &
            $\datalangIdx$
            & $\Coloneqq$ &
            $\datalangZero \mid \datalangOne \mid \datalangTwo$
        \\
            $\datalangTag[]$
            & $\ni$ &
            $\datalangTag$
        \\
            $\datalangBool[]$
            & $\ni$ &
            $\datalangBool$
    	\\
    		$\datalangLoc[]$
    		& $\ni$ &
    		$\datalangLoc$
        \\
            $\datalangFn[]$
            & $\ni$ &
            $\datalangFn$
        \\
            $\datalangVar[]$
            & $\ni$ &
            $\datalangVar, \datalangVarTwo$
    	\\
            $\datalangVal[]$
            & $\ni$ &
            $\datalangVal$
            & $\Coloneqq$ &
            $\datalangUnit \mid \datalangIdx \mid \datalangTag \mid \datalangBool \mid \datalangLoc \mid \datalangFnptr{\datalangFn}$
        \\
            $\datalangExpr[]$
            & $\ni$ &
            $\datalangExpr$
            & $\Coloneqq$ &
            $\datalangVal$
        \\
            &&
            & | &
            $\datalangVar \mid \datalangLet{\datalangVar}{\datalangExpr_1}{\datalangExpr_2} \mid \datalangCall{\datalangExpr_1}{\datalangExpr_2}$
        \\
            &&
            & | &
            $\datalangEq{\datalangExpr_1}{\datalangExpr_2}$
        \\
            &&
            & | &
            $\datalangIf{\datalangExpr_0}{\datalangExpr_1}{\datalangExpr_2}$
        \\
            &&
            & | &
            $\datalangBlock{\datalangTag}{\datalangExpr_1}{\datalangExpr_2} \mid \datalangBlockDet{\datalangTag}{\datalangExpr_1}{\datalangExpr_2}$
        \\
            &&
            & | &
            $\datalangLoad{\datalangExpr_1}{\datalangExpr_2} \mid \datalangStore{\datalangExpr_1}{\datalangExpr_2}{\datalangExpr_3}$
        \\
            $\datalangEctx[]$
            & $\ni$ &
            $\datalangEctx$
            & $\Coloneqq$ &
            $\Box$
        \\
            &&
            & | &
            $\datalangLet{\datalangVar}{\datalangEctx}{\datalangExpr_2} \mid \datalangCall{\datalangExpr_1}{\datalangEctx} \mid \datalangCall{\datalangEctx}{\datalangVal_2}$
        \\
            &&
            & | &
            $\datalangEq{\datalangExpr_1}{\datalangEctx} \mid \datalangEq{\datalangEctx}{\datalangVal_2}$
        \\
            &&
            & | &
            $\datalangIf{\datalangEctx}{\datalangExpr_1}{\datalangExpr_2}$
        \\
            &&
            & | &
            $\datalangLoad{\datalangExpr_1}{\datalangEctx} \mid \datalangLoad{\datalangEctx}{\datalangVal_2}$
        \\
            &&
            & | &
            $\datalangStore{\datalangExpr_1}{\datalangExpr_2}{K} \mid \datalangStore{\datalangExpr_1}{\datalangEctx}{\datalangVal_3} \mid \datalangStore{\datalangEctx}{\datalangVal_2}{\datalangVal_3}$
        \\
            $\datalangDef[]$
            & $\ni$ &
            $\datalangDef$
            & $\Coloneqq$ &
            $\datalangRec{\datalangVar}{\datalangExpr}$
        \\
            $\datalangProg[]$
            & $\ni$ &
            $\datalangProg$
            & $\coloneqq$ &
            $\datalangFn[] \finmap \datalangDef[]$
        \\
            $\datalangState[]$
            & $\ni$ &
            $\datalangState$    
            & $\coloneqq$ &
            $\datalangLoc[] \finmap \datalangVal[]$
        \\
            $\datalangConfig[]$
            & $\ni$ &
            $\datalangConfig$
            & $\coloneqq$ &
            $\datalangExpr[] \times \datalangState[]$
    \end{tabular}
    \caption{\DataLang syntax}
    \label{fig:syntax}
\end{figure}
\begin{figure}[tp]
    \begin{tabular}{rcl}
            $\datalangSeq{\datalangExpr_1}{\datalangExpr_2}$
            & $\coloneqq$ &
            $\datalangLet{\datalangVar}{\datalangExpr_1}{\datalangExpr_2}$
        \\
            $\datalangPair{\datalangExpr_1}{\datalangExpr_2}$
            & $\coloneqq$ &
            $\datalangBlock{\mathrm{PAIR}}{\datalangExpr_1}{\datalangExpr_2}$
        \\
            $\datalangLet{\datalangPair{\datalangVar_1}{\datalangVar_2}}{\datalangExpr_1}{\datalangExpr_2}$
            & $\coloneqq$ &
            $\datalangLet{\datalangVarTwo}{\datalangExpr_1}{$
        \\
            &&
            $\datalangLet{\datalangVar_1}{\datalangLoad{\datalangVarTwo}{1}}{$
        \\
            &&
            $\datalangLet{\datalangVar_2}{\datalangLoad{\datalangVarTwo}{2}}{$
        \\
            &&
            $\datalangExpr_2}}}$
    \end{tabular}~%
    \begin{tabular}{rcl}
            $\datalangRec{\datalangPair{\datalangVar_1}{\datalangVar_2}}{\datalangExpr}$
            & $\coloneqq$ &
            $\datalangRec{\datalangVarTwo}{
            \datalangLet{\datalangPair{\datalangVar_1}{\datalangVar_2}}{\datalangVarTwo}{\datalangExpr}}$
        \\
            $\datalangNil$
            & $\coloneqq$ &
            $\datalangUnit$
        \\
            $\datalangCons{\datalangExpr_1}{\datalangExpr_2}$
            & $\coloneqq$ &
            $\datalangBlock{\mathrm{CONS}}{\datalangExpr_1}{\datalangExpr_2}$
    \end{tabular}

    \begin{tabular}{rcl}
            $\datalangMatchOneline[]{\datalangExpr_0}{\datalangExpr_1}{\datalangVar}{\mathit{\datalangVar s}}{\datalangExpr_2}$
            & $\coloneqq$ &
            $\datalangLet{\datalangVarTwo}{\datalangExpr_0}{$
        \\
            &&
            $\datalangIf{\datalangEq{\datalangVarTwo}{\datalangNil}$
        \\
            &&
            $}{\datalangExpr_1$
        \\
            &&
            $}{\datalangLet{\datalangPair{\datalangVar}{\mathit{\datalangVar s}}}{\datalangVarTwo}{\datalangExpr_2}}}$
        \\
            $\datalangHole$
            & $\coloneqq$ &
            $\datalangUnit$
    \end{tabular}
    \caption{\DataLang syntactic sugar}
    \label{fig:sugar}
\end{figure}
\begin{figure}[tp]
    \begin{mathparpagebreakable}
        \inferrule*
            {}{
                \boxed{- \headStep{\datalangProg} - : \datalangConfig[] \rightarrow \datalangConfig[] \rightarrow \Prop}
            }
        \and
        \inferrule*
            {}{
                \boxed{- \step{\datalangProg} - : \datalangConfig[] \rightarrow \datalangConfig[] \rightarrow \Prop}
            }
        \\
        \inferrule*[lab=StepLet]
            {}{
                \left(
                    \datalangLet{\datalangVar}{\datalangVal}{\datalangExpr},
                    \datalangState
                \right)
                \headStep{\datalangProg}
                \left(
                    \datalangExpr{} [\datalangVar \backslash \datalangVal],
                    \datalangState
                \right)
            }
        \and
        \inferrule*[lab=StepCall]
            {
                \datalangProg{} [\datalangFn] = (\datalangRec{\datalangVar}{\datalangExpr})
            }{
                \left(
                    \datalangCall{\datalangFnptr{\datalangFn}}{\datalangVal},
                    \datalangState
                \right)
                \headStep{\datalangProg}
                \left(
                    \datalangExpr{} [\datalangVar \backslash \datalangVal],
                    \datalangState
                \right)
            }
        \\
        \inferrule*[lab=StepBlock1]
            {}{
                \left(
                    \datalangBlock{\datalangTag}{\datalangExpr_1}{\datalangExpr_2},
                    \datalangState
                \right)
                \headStep{\datalangProg}
                \left(
                    \begin{array}{l}
                        \datalangLet{\datalangVar_1}{\datalangExpr_1}{\\
                        \datalangLet{\datalangVar_2}{\datalangExpr_2}{\\
                        \datalangBlockDet{\datalangTag}{\datalangVar_1}{\datalangVar_2}}}
                    \end{array},
                    \datalangState
                \right)
            }
        \and
        \inferrule*[lab=StepBlock2]
            {}{
                \left(
                    \datalangBlock{\datalangTag}{\datalangExpr_1}{\datalangExpr_2},
                    \datalangState
                \right)
                \headStep{\datalangProg}
                \left(
                    \begin{array}{l}
                        \datalangLet{\datalangVar_2}{\datalangExpr_2}{\\
                        \datalangLet{\datalangVar_1}{\datalangExpr_1}{\\
                        \datalangBlockDet{\datalangTag}{\datalangVar_1}{\datalangVar_2}}}
                    \end{array},
                    \datalangState
                \right)
            }
        \\
        \inferrule*[lab=StepBlockDet]
            {
                \forall \datalangIdx \in \datalangIdx[], \datalangLoc + \datalangIdx \notin \dom{\datalangState}
            }{
                \left(
                    \datalangBlockDet{\datalangTag}{\datalangVal_1}{\datalangVal_2},
                    \datalangState
                \right)
                \headStep{\datalangProg}
                \left(
                    \datalangLoc,
                    \datalangState{} [\datalangLoc \mapsto \datalangTag, \datalangVal_1, \datalangVal_2]
                \right)
            }
        \and
        \inferrule*[lab=StepLoad]
            {
                \datalangState{} [\datalangLoc + \datalangIdx] = \datalangVal
            }{
                \left(
                    \datalangLoad{\datalangLoc}{\datalangIdx},
                    \datalangState
                \right)
                \headStep{\datalangProg}
                \left(
                    \datalangVal,
                    \datalangState
                \right)
            }
        \and
        \inferrule*[lab=StepStore]
            {
                \datalangLoc + \datalangIdx \in \dom{\datalangState}
            }{
                \left(
                    \datalangStore{\datalangLoc}{\datalangIdx}{\datalangVal},
                    \datalangState
                \right)
                \headStep{\datalangProg}
                \left(
                    \datalangUnit,
                    \datalangState{} [\datalangLoc + \datalangIdx \mapsto \datalangVal]
                \right)
            }
        \and
        \inferrule*[lab=StepEctx]
            {
                \left(
                    \datalangExpr,
                    \datalangState
                \right)
                \headStep{\datalangProg}
                \left(
                    \datalangExpr',
                    \datalangState'
                \right)
            }{
                \left(
                    \datalangEctx{} [\datalangExpr],
                    \datalangState
                \right)
                \step{\datalangProg}
                \left(
                    \datalangEctx{} [\datalangExpr'],
                    \datalangState'
                \right)
            }
    \end{mathparpagebreakable}
    \caption{\DataLang semantics (excerpt)}
    \label{fig:semantics}
\end{figure}
\begin{figure}[tp]
\begin{tabular}{c}
\begin{Datalang}
map |-> rec (fn, xs) =
  match xs with
  | [] ->
      []
  | x :: xs ->
      let y = fn x in
      y :: @map (fn, xs)
\end{Datalang}
\end{tabular}
\caption{Natural implementation of \datalang|map| in \DataLang}
\label{fig:map}
\end{figure}

\subsection{Transformation}

We now define the TMC transformation.
As formalized in \cref{fig:tmc}, transforming a \DataLang program $\datalangProg$ consists in:

\paragraph{1. Choosing a subset of toplevel functions to be TMC-transformed}
In \OCaml, the programmer has to annotate these.
For each such function $\datalangFn$, we also require a fresh function name (that is not defined in $\datalangProg$) that will be defined to the \emph{destination-passing style} (DPS) version of $\datalangFn$ in the transformed program.
Formally, this is given by a renaming $\datalangRenaming$ as a parameter of the auxiliary transformations we describe next.

\paragraph{2. For each TMC-transformed function $\datalangFn$, computing its DPS transform}
Actually, it is not unique.
In \OCaml, the ambiguity is resolved by annotations and rules described in \cref{sec:implementation}.
To represent all possible DPS transformations, we introduce in \cref{fig:tmc_dps} the relations $\datalangDef_s \tmcDps{\datalangRenaming} \datalangDef_t$ for definitions and $(\datalangExpr_\mathit{dst}, \datalangExpr_\mathit{idx}, \datalangExpr_s) \tmcDps{\datalangRenaming} \datalangExpr_t$ for expressions.

$\datalangDef_s \tmcDps{\datalangRenaming} \datalangDef_t$ expresses that:
1)~$\datalangDef_t$ has an additional parameter representing the destination where it must write its result: the location of a memory block $\datalangVar_\mathit{dst}$ along with the index $\datalangVar_\mathit{idx}$ of a particular field in this block.
2)~The body of $\datalangDef_t$ is the DPS transform of the body of $\datalangDef_s$ under the given destination.

$(\datalangExpr_\mathit{dst}, \datalangExpr_\mathit{idx}, \datalangExpr_s) \tmcDps{\datalangRenaming} \datalangExpr_t$ expresses that $\datalangExpr_t$ is the DPS transform of $\datalangExpr_s$ under destination $(\datalangExpr_\mathit{dst}, \datalangExpr_\mathit{idx})$.
Intuitively, this means $\datalangExpr_t$ computes the same thing as $\datalangExpr_s$ but writes it into the destination instead of returning it.
We will formalize this intuition in \cref{sec:specification}.

Most rules are straightforward.
Whenever it is possible, we take the direct transformation --- as described next --- of sub-expressions that are not inductively DPS-transformed.

The core of the TMC transformation lies in the symmetric \RefTirName{DPSBlock1} and \RefTirName{DPSBlock2} rules:
1)~We choose a side to evaluate first and partially initialize a new memory block with it.
2)~We write this block to the destination.
3)~We fill the uninitialized field by passing it as destination to the DPS-transformed corresponding source sub-expression.

\paragraph{3. For each function $\datalangFn$ defined in $\datalangProg$, computing its direct transform.}
As before, as it is not unique, we introduce in \cref{fig:tmc_dir} the relations $\datalangDef_s \tmcDir{\datalangRenaming} \datalangDef_t$ for definitions and $\datalangExpr_s \tmcDir{\datalangRenaming} \datalangDef_t$ for expressions.

$\datalangDef_s \tmcDir{\datalangRenaming} \datalangDef_t$ expresses that:
1)~$\datalangDef_t$ has the same calling convention as $\datalangDef_s$.
2)~The body of $\datalangDef_t$ is the direct transform of the body of $\datalangDef_s$.

$\datalangExpr_s \tmcDir{\datalangRenaming} \datalangDef_t$ expresses that $\datalangExpr_t$ is the direct transform of $\datalangExpr_s$.
Intuitively, this means $\datalangExpr_t$ computes the same thing as $\datalangExpr_s$.

Again, most cases are straightforward.
The symmetric rules \RefTirName{DirBlockDPS1} and \RefTirName{DirBlockDPS2} are the equivalent of \RefTirName{DPSBlock1} and \RefTirName{DPSBlock2}:
1)~We choose a side to evaluate first and partially initialize a new memory block with it.
2)~We trigger the DPS transformation on the other side, passing the uninitialized field as destination.
3)~We return the now fully initialized block.

\medskip

As an example, we can show that the two \DataLang functions defined in \cref{fig:map_tmc} are respectively the direct and DPS transforms of the original \datalang|map| function defined in \cref{fig:map}.
Thanks to the rules \RefTirName{DirBlockDPS1} and \RefTirName{DPSCall}, the non-tail recursive call is replaced by the allocation of a partially initialized block followed by a call to \datalang|map_dps|.

\begin{figure}[tp]
    \[
        \datalangProg_s \tmc \datalangProg_t
        \coloneqq
        \exists \datalangRenaming \ldotp
        \bigwedge \left[ \begin{array}{l}
                \dom{\datalangRenaming} \subseteq \dom{\datalangProg_s}
            \\
                \dom{\datalangProg_t} = \dom{\datalangProg_s} \cup \codom{\datalangRenaming}
            \\
                \forall \datalangFn, \datalangDef_s \ldotp
                \datalangProg_s [\datalangFn] = \datalangDef_s \implies
                \exists \datalangDef_t \ldotp
                \datalangProg_t [\datalangFn] = \datalangDef_t \wedge \datalangDef_s \tmcDir{\datalangRenaming} \datalangDef_t
            \\
                \forall \datalangFn, \datalangFn_\mathit{dps}, \datalangDef_s \ldotp
                \datalangProg_s [\datalangFn] = \datalangDef_s \wedge \datalangRenaming [\datalangFn] = \datalangFn_{\mathit{dps}} \implies
                \exists \datalangDef_t \ldotp
                \datalangProg_t [\datalangFn_\mathit{dps}] = \datalangDef_t \wedge \datalangDef_s \tmcDps{\datalangRenaming} \datalangDef_t
        \end{array} \right.
    \]
    \caption{TMC transformation}
    \label{fig:tmc}
\end{figure}
\begin{figure}[tp]
    \begin{mathparpagebreakable}
        \inferrule*
            {}{
                \boxed{- \tmcDps{\xi} - : \nterm{Expr} \times \nterm{Expr} \times \nterm{Expr} \rightarrow \nterm{Expr} \rightarrow \Prop}
            }
        \and
        \inferrule*
            {}{
                \boxed{- \tmcDps{\xi} - : \nterm{Def} \rightarrow \nterm{Def} \rightarrow \Prop}
            }
        \\
        \inferrule*[lab=DPSBase]
            {
                e_s \tmcDir{\xi} e_t
            }{
                \left(
                    \mathit{dst},
                    \mathit{idx},
                    e_s
                \right)
                \tmcDps{\xi}
                \lambdStore{\mathit{dst}}{i}{e_t}
            }
        \and
        \inferrule*[lab=DPSLet]
            {
                e_{s1} \tmcDir{\xi} e_{t1}
            \and
                (\mathit{dst}, \mathit{idx}, e_{s2}) \tmcDps{\xi} e_{t2}
            }{
                \left(
                    \mathit{dst},
                    \mathit{idx},
                    \lambdLet{x}{e_{s1}}{e_{s2}}
                \right)
                \tmcDps{\xi}
                \lambdLet{x}{e_{t1}}{e_{t2}}
            }
        \and
        \inferrule*[lab=DPSCall]
            {
                f \in \dom{\xi}
            \and
                e_s \tmcDir{\xi} e_t
            }{
                \left(
                    \mathit{dst},
                    \mathit{idx},
                    \lambdCall{\lambdFunc{f}}{e_s}
                \right)
                \tmcDps{\xi}
                \lambdCall{\lambdFunc{\xi [f]}}{e_t}
            }
        \and
        \inferrule*[lab=DPSIf]
            {
                e_{s0} \tmcDir{\xi} e_{t0}
            \and
                (\mathit{dst}, \mathit{idx}, e_{s1}) \tmcDps{\xi} e_{s2}
            \and
                (\mathit{dst}, \mathit{idx}, e_{s2}) \tmcDps{\xi} e_{t2}
            }{
                \left(
                    \mathit{dst},
                    \mathit{idx},
                    \lambdIf{e_{s0}}{e_{s1}}{e_{s2}}
                \right)
                \tmcDps{\xi}
                \lambdIf{e_{t0}}{e_{t1}}{e_{t2}}
            }
        \\
        \inferrule*[lab=DPSConstr1]
            {
                e_{s1} \tmcDir{\xi} e_{t1}
            \and
                (\mathit{dst}, \lambdTwo, e_{s2}) \tmcDir{\xi} e_{t2}
            }{
                \left(
                    \mathit{dst},
                    \mathit{idx},
                    \lambdConstr{c}{e_{s1}}{e_{s2}}
                \right)
                \tmcDps{\xi}
                \begin{array}{l}
                    \lambdLet{\mathit{dst'}}{\lambdConstr{c}{e_{t1}}{\lambdHole}}{\\
                    \lambdSeq{\lambdStore{\mathit{dst}}{\mathit{idx}}{\mathit{dst'}}}{e_{t2}}
                    }
                \end{array}
            }
        \and
        \inferrule*[lab=DPSConstr2]
            {
                e_{s2} \tmcDir{\xi} e_{t2}
            \and
                (\mathit{dst}, \lambdOne, e_{s1}) \tmcDir{\xi} e_{t1}
            }{
                \left(
                    \mathit{dst},
                    \mathit{idx},
                    \lambdConstr{c}{e_{s1}}{e_{s2}}
                \right)
                \tmcDps{\xi}
                \begin{array}{l}
                    \lambdLet{\mathit{dst'}}{\lambdConstr{c}{\lambdHole}{e_{t2}}}{\\
                    \lambdSeq{\lambdStore{\mathit{dst}}{\mathit{idx}}{\mathit{dst'}}}{e_{t1}}
                    }
                \end{array}
            }
        \\
        \inferrule*[lab=DPSDef]
            {
                (\mathit{dst}, \mathit{idx}, e_s) \tmcDps{\xi} e_t
            }{
                \lambdDef{x}{e_s}
                \tmcDps{\xi}
                {\begin{array}{l}
                    \lambdDef{y}{\\ \ \begin{array}{l}
                        \lambdLet{z}{\lambdLoad{y}{\lambdOne}}{\\
                        \lambdLet{\mathit{dst}}{\lambdLoad{z}{\lambdOne}}{\\
                        \lambdLet{\mathit{idx}}{\lambdLoad{z}{\lambdTwo}}{\\
                        \lambdLet{x}{\lambdLoad{y}{\lambdTwo}}{\\
                        e_t
                        }}}}
                    \end{array}}
                \end{array}}
            }
    \end{mathparpagebreakable}
    \caption{Destination-passing style TMC transformation (omitting freshness conditions)}
    \label{fig:tmc_dps}
\end{figure}
\begin{figure}[tp]
    \begin{mathparpagebreakable}
        \inferrule*
            {}{
                \boxed{- \tmcDir{\datalangRenaming} - : \datalangExpr[] \rightarrow \datalangExpr[] \rightarrow \Prop}
            }
        \and
        \inferrule*
            {}{
                \boxed{- \tmcDir{\datalangRenaming} - : \datalangDef[] \rightarrow \datalangDef[] \rightarrow \Prop}
            }
        \\
        \inferrule*[lab=DirVal]
            {}{
                \datalangVal
                \tmcDir{\datalangRenaming}
                \datalangVal
            }
        \and
        \inferrule*[lab=DirVar]
            {}{
                \datalangVar
                \tmcDir{\datalangRenaming}
                \datalangVar
            }
        \and
        \inferrule*[lab=DirLet]
            {
                \datalangExpr_{s1} \tmcDir{\datalangRenaming} \datalangExpr_{t1}
            \and
                \datalangExpr_{s2} \tmcDir{\datalangRenaming} \datalangExpr_{t2}
            }{
                \datalangLet{\datalangVar}{\datalangExpr_{s1}}{\datalangExpr_{s2}}
                \tmcDir{\datalangRenaming}
                \datalangLet{\datalangVar}{\datalangExpr_{t1}}{\datalangExpr_{t2}}
            }
        \\
        \inferrule*[lab=DirBlockDPS1]
            {
                \datalangExpr_{s1} \tmcDir{\datalangRenaming} \datalangExpr_{t1}
            \and
                (\datalangVar, \datalangTwo, \datalangExpr_{s2}) \tmcDps{\datalangRenaming} \datalangExpr_{t2}
            }{
                \datalangBlock{\datalangTag}{\datalangExpr_{s1}}{\datalangExpr_{s2}}
                \tmcDir{\datalangRenaming}
                \begin{array}{l}
                    \datalangLet{\datalangVar}{\datalangBlock{\datalangTag}{\datalangExpr_{t1}}{\datalangHole}}{\\
                    \datalangSeq{\datalangExpr_{t2}}{\datalangVar}}
                \end{array}
            }
        \and
        \inferrule*[lab=DirBlockDPS2]
            {
                \datalangExpr_{s2} \tmcDir{\datalangRenaming} \datalangExpr_{t2}
            \and
                (\datalangVar, \datalangOne, \datalangExpr_{s1}) \tmcDps{\datalangRenaming} \datalangExpr_{t1}
            }{
                \datalangBlock{\datalangTag}{\datalangExpr_{s1}}{\datalangExpr_{s2}}
                \tmcDir{\datalangRenaming}
                \begin{array}{l}
                    \datalangLet{\datalangVar}{\datalangBlock{\datalangTag}{\datalangHole}{\datalangExpr_{t2}}}{\\
                    \datalangSeq{\datalangExpr_{t1}}{\datalangVar}}
                \end{array}
            }
        \\
        \inferrule*[lab=DirDef]
            {
                \datalangExpr_s \tmcDir{\datalangRenaming} \datalangExpr_t
            }{
                \datalangRec{\datalangVar}{\datalangExpr_s}
                \tmcDir{\datalangRenaming}
                \datalangRec{\datalangVar}{\datalangExpr_t}
            }
    \end{mathparpagebreakable}
    \caption{Direct TMC transformation (omitting congruence rules similar to \TirName{DirLet} and freshness conditions)}
    \label{fig:tmc_dir}
\end{figure}
\begin{figure}[tp]
\begin{minipage}{.40\columnwidth}
\begin{Datalang}
map |-> rec (fn, xs) =
  match xs with
  | [] ->
      []
  | x :: xs ->
      let y = fn x in
      let dst = y :: ? in
      @map_dps ((dst, 2), fn, xs) ;
      dst
\end{Datalang}
\end{minipage}
\hfill
\begin{minipage}{.52\columnwidth}
\begin{Datalang}
map_dps |-> rec ((dst, idx), fn, xs) =
  match xs with
  | [] ->
      dst.(idx) <- []
  | x :: xs ->
      let y = fn x in
      let dst' = y :: ? in
      dst.(idx) <- dst' ;
      @map_dps ((dst', 2), fn, xs)
\end{Datalang}
\end{minipage}
\caption{Direct and DPS transforms of \datalang|map| (\cref{fig:map})}
\label{fig:map_tmc}
\end{figure}