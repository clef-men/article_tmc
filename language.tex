\section{Language}

To formalize the TMC transformation, we define the \DataLang language.
Its syntax and semantics are given in \cref{fig:syntax} and \cref{fig:semantics}.
We also introduce syntactic sugar in \cref{fig:sugar}, including some for lists.
For instance, we can define the \datalang|map| function on lists as in \cref{fig:map}.

\DataLang is an untyped sequential post closure conversion lambda calculus with mutable state.
That is, it does not feature lambda expressions, which are already compiled to toplevel closed functions referred to using function pointers ($\datalangFnptr{\datalangFn}$).

There is one noteworthy trait.
To express constructors and closures, \DataLang features tagged mutable memory blocks with two fields.
One can allocate a block with $\datalangBlock{\datalangTag}{\datalangExpr_1}{\datalangExpr_2}$, access its fields with $\datalangLoad{\datalangExpr_1}{\datalangExpr_2}$ and modify them with $\datalangStore{\datalangExpr_1}{\datalangExpr_2}{\datalangExpr_3}$.
As in \OCaml, the evaluation order of subexpressions $\datalangExpr_1$ and $\datalangExpr_2$ in $\datalangBlock{\datalangTag}{\datalangExpr_1}{\datalangExpr_2}$ is unspecified.
To model this in the semantics, such an expression first nondeterministically reduces to either $\datalangLet{\datalangVar_1}{\datalangExpr_1}{\datalangLet{\datalangVar_2}{\datalangExpr_2}{\datalangBlockDet{\datalangTag}{\datalangVar_1}{\datalangVar_2}}}$ through \RefTirName{StepBlock1} or $\datalangLet{\datalangVar_2}{\datalangExpr_2}{\datalangLet{\datalangVar_1}{\datalangExpr_1}{\datalangBlockDet{\datalangTag}{\datalangVar_1}{\datalangVar_2}}}$ through \RefTirName{StepBlock2} and $\datalangBlockDet{\datalangTag}{\datalangVal_1}{\datalangVal_2}$ performs the allocation  through \RefTirName{StepBlockDet}.
This $\datalangBlockDet{\datalangTag}{\datalangExpr_1}{\datalangExpr_2}$ construction cannot appear in source programs.

%Another choice is to take an interleaving semantics.

Obviously, \DataLang does not support all features of the \LambdaLang intermediate language in which the original TMC transformation is written.
It is only meant to study and verify the core of the transformation.

As a side note, we use named expression variables in this article but the \Coq mechanization actually adopts de Bruijn syntax better suited to define transformations involving binders.
More precisely, we rely on the \Autosubst library~\cite{DBLP:conf/itp/SchaferTS15}.

\begin{figure}[tp]
    \begin{tabular}{rclclcl}
            $\nterm{Index}$
            & $\ni$ &
            i
            & $\Coloneqq$ &
            $\lambdZero \mid \lambdOne \mid \lambdTwo$
            &&
            index
        \\
            $\nterm{Tag}$
            & $\ni$ &
            $t$
            &&
            &&
            tag
        \\
            $\mathbb{B}$
            & $\ni$ &
            $b$
            &&
            &&
            boolean
    	\\
    		$\mathbb{L}$
    		& $\ni$ &
    		$\loc$
    		&&
    		&&
    		location
        \\
            $\mathbb{F}$
            & $\ni$ &
            $f$
            &&
            &&
            function
        \\
            $\mathbb{X}$
            & $\ni$ &
            $x, y, z$
            &&
            &&
            variable
    	\\
            $\nterm{Val}$
            & $\ni$ &
            $v$
            & $\Coloneqq$ &
            $\lambdUnit \mid i \mid t \mid b \mid \ell \mid \lambdFunc{f}$
            &&
            value
        \\
            $\nterm{Expr}$
            & $\ni$ &
            $e$
            & $\Coloneqq$ &
            $v$
            &&
            expression
        \\
            &&
            & | &
            $x \mid \lambdLet{x}{e_1}{e_2} \mid \lambdCall{e_1}{e_2}$
        \\
            &&
            & | &
            $\lambdEq{e_1}{e_2}$
        \\
            &&
            & | &
            $\lambdIf{e_0}{e_1}{e_2}$
        \\
            &&
            & | &
            $\lambdConstr{t}{e_1}{e_2} \mid \lambdConstrDet{t}{e_1}{e_2}$
        \\
            &&
            & | &
            $\lambdLoad{e_1}{e_2}$
        \\
            &&
            & | &
            $\lambdStore{e_1}{e_2}{e_3}$
        \\
            $\nterm{Ectx}$
            & $\ni$ &
            $K$
            & $\Coloneqq$ &
            $\Box$
            &&
            evaluation context
        \\
            &&
            & | &
            $\lambdLet{x}{K}{e_2} \mid \lambdCall{e_1}{K} \mid \lambdCall{K}{v_2}$
        \\
            &&
            & | &
            $\lambdEq{e_1}{K} \mid \lambdEq{K}{v_2}$
        \\
            &&
            & | &
            $\lambdIf{K}{e_1}{e_2}$
        \\
            &&
            & | &
            $\lambdLoad{e_1}{K} \mid \lambdLoad{K}{v_2}$
        \\
            &&
            & | &
            $\lambdStore{e_1}{e_2}{K} \mid \lambdStore{e_1}{K}{v_3} \mid \lambdStore{K}{v_2}{v_3}$
        \\
            $\nterm{Def}$
            & $\ni$ &
            $d$
            & $\Coloneqq$ &
            $\lambdDef{x}{e}$
            &&
            definition
        \\
            $\nterm{Prog}$
            & $\ni$ &
            $p$
            & $\coloneqq$ &
            $\mathbb{F} \finmap \nterm{Def}$
            &&
            program
        \\
            $\nterm{State}$
            & $\ni$ &
            $\sigma$    
            & $\coloneqq$ &
            $\mathbb{L} \finmap \nterm{Val}$
            &&
            state
        \\
            $\nterm{Config}$
            & $\ni$ &
            $\rho$
            & $\coloneqq$ &
            $\nterm{Expr} \times \nterm{State}$
            &&
            configuration
        \\
            $\nterm{Bisubst}$
            & $\ni$ &
            $\Gamma$
            & $\coloneqq$ &
            $\mathbb{X} \rightarrow \nterm{Expr} \times \nterm{Expr}$
            &&
            bisubstitution
    \end{tabular}
    \caption{Syntax of \LambdaLang}
    \label{fig:syntax}
\end{figure}
\begin{figure}[tp]
    \begin{tabular}{rcl}
            $\datalangSeq{\datalangExpr_1}{\datalangExpr_2}$
            & $\coloneqq$ &
            $\datalangLet{\datalangVar}{\datalangExpr_1}{\datalangExpr_2}$
        \\
            $\datalangPair{\datalangExpr_1}{\datalangExpr_2}$
            & $\coloneqq$ &
            $\datalangBlock{\mathrm{PAIR}}{\datalangExpr_1}{\datalangExpr_2}$
        \\
            $\datalangLet{\datalangPair{\datalangVar_1}{\datalangVar_2}}{\datalangExpr_1}{\datalangExpr_2}$
            & $\coloneqq$ &
            $\datalangLet{\datalangVarTwo}{\datalangExpr_1}{$
        \\
            &&
            $\datalangLet{\datalangVar_1}{\datalangLoad{\datalangVarTwo}{1}}{$
        \\
            &&
            $\datalangLet{\datalangVar_2}{\datalangLoad{\datalangVarTwo}{2}}{$
        \\
            &&
            $\datalangExpr_2}}}$
        \\
            $\datalangRec{\datalangPair{\datalangVar_1}{\datalangVar_2}}{\datalangExpr}$
            & $\coloneqq$ &
            $\datalangRec{\datalangVarTwo}{
            \datalangLet{\datalangPair{\datalangVar_1}{\datalangVar_2}}{\datalangVarTwo}{\datalangExpr}}$
        \\
            $\datalangNil$
            & $\coloneqq$ &
            $\datalangUnit$
        \\
            $\datalangCons{\datalangExpr_1}{\datalangExpr_2}$
            & $\coloneqq$ &
            $\datalangBlock{\mathrm{CONS}}{\datalangExpr_1}{\datalangExpr_2}$
        \\
            $\datalangMatchOneline[]{\datalangExpr_0}{\datalangExpr_1}{\datalangVar}{\mathit{\datalangVar s}}{\datalangExpr_2}$
            & $\coloneqq$ &
            $\datalangLet{\datalangVarTwo}{\datalangExpr_0}{$
        \\
            &&
            $\datalangIf{\datalangEq{\datalangVarTwo}{\datalangNil}$
        \\
            &&
            $}{\datalangExpr_1$
        \\
            &&
            $}{\datalangLet{\datalangPair{\datalangVar}{\mathit{\datalangVar s}}}{\datalangVarTwo}{\datalangExpr_2}}}$
        \\
            $\datalangHole$
            & $\coloneqq$ &
            $\datalangUnit$
    \end{tabular}
    \caption{\DataLang syntactic sugar (omitting freshness conditions)}
    \label{fig:sugar}
\end{figure}
\begin{figure}[tp]
    \begin{mathparpagebreakable}
        \inferrule*
            {}{
                \boxed{- \headStep{\datalangProg} - : \datalangConfig[] \rightarrow \datalangConfig[] \rightarrow \Prop}
            }
        \and
        \inferrule*
            {}{
                \boxed{- \step{\datalangProg} - : \datalangConfig[] \rightarrow \datalangConfig[] \rightarrow \Prop}
            }
        \\
        \inferrule*[lab=StepLet]
            {}{
                \left(
                    \datalangLet{\datalangVar}{\datalangVal}{\datalangExpr},
                    \datalangState
                \right)
                \headStep{\datalangProg}
                \left(
                    \datalangExpr{} [\datalangVar \backslash \datalangVal],
                    \datalangState
                \right)
            }
        \and
        \inferrule*[lab=StepCall]
            {
                \datalangProg{} [\datalangFn] = (\datalangRec{\datalangVar}{\datalangExpr})
            }{
                \left(
                    \datalangCall{\datalangFnptr{\datalangFn}}{\datalangVal},
                    \datalangState
                \right)
                \headStep{\datalangProg}
                \left(
                    \datalangExpr{} [\datalangVar \backslash \datalangVal],
                    \datalangState
                \right)
            }
        \\
        \inferrule*[lab=StepBlock1]
            {}{
                \left(
                    \datalangBlock{\datalangTag}{\datalangExpr_1}{\datalangExpr_2},
                    \datalangState
                \right)
                \headStep{\datalangProg}
                \left(
                    \begin{array}{l}
                        \datalangLet{\datalangVar_1}{\datalangExpr_1}{\\
                        \datalangLet{\datalangVar_2}{\datalangExpr_2}{\\
                        \datalangBlockDet{\datalangTag}{\datalangVar_1}{\datalangVar_2}}}
                    \end{array},
                    \datalangState
                \right)
            }
        \and
        \inferrule*[lab=StepBlock2]
            {}{
                \left(
                    \datalangBlock{\datalangTag}{\datalangExpr_1}{\datalangExpr_2},
                    \datalangState
                \right)
                \headStep{\datalangProg}
                \left(
                    \begin{array}{l}
                        \datalangLet{\datalangVar_2}{\datalangExpr_2}{\\
                        \datalangLet{\datalangVar_1}{\datalangExpr_1}{\\
                        \datalangBlockDet{\datalangTag}{\datalangVar_1}{\datalangVar_2}}}
                    \end{array},
                    \datalangState
                \right)
            }
        \\
        \inferrule*[lab=StepBlockDet]
            {
                \forall \datalangIdx \in \datalangIdx[], \datalangLoc + \datalangIdx \notin \dom{\datalangState}
            }{
                \left(
                    \datalangBlockDet{\datalangTag}{\datalangVal_1}{\datalangVal_2},
                    \datalangState
                \right)
                \headStep{\datalangProg}
                \left(
                    \datalangLoc,
                    \datalangState{} [\datalangLoc \mapsto \datalangTag, \datalangVal_1, \datalangVal_2]
                \right)
            }
        \and
        \inferrule*[lab=StepLoad]
            {
                \datalangState{} [\datalangLoc] = \datalangVal
            }{
                \left(
                    \datalangLoad{\datalangLoc}{\datalangIdx},
                    \datalangState
                \right)
                \headStep{\datalangProg}
                \left(
                    \datalangVal,
                    \datalangState
                \right)
            }
        \and
        \inferrule*[lab=StepStore]
            {
                \datalangLoc + \datalangIdx \in \dom{\datalangState}
            }{
                \left(
                    \datalangStore{\datalangLoc}{\datalangIdx}{\datalangVal},
                    \datalangState
                \right)
                \headStep{\datalangProg}
                \left(
                    \datalangUnit,
                    \datalangState{} [\datalangLoc + \datalangIdx \mapsto \datalangVal]
                \right)
            }
        \and
        \inferrule*[lab=StepEctx]
            {
                \left(
                    \datalangExpr,
                    \datalangState
                \right)
                \headStep{\datalangProg}
                \left(
                    \datalangExpr',
                    \datalangState'
                \right)
            }{
                \left(
                    \datalangEctx{} [\datalangExpr],
                    \datalangState
                \right)
                \step{\datalangProg}
                \left(
                    \datalangEctx{} [\datalangExpr'],
                    \datalangState'
                \right)
            }
    \end{mathparpagebreakable}
    \caption{\DataLang semantics (excerpt)}
    \label{fig:semantics}
\end{figure}
\begin{figure}[tp]
\begin{tabular}{c}
\begin{Datalang}
map |-> rec λ(f, xs) = match xs with
                       | [] -> []
                       | x :: xs -> let y = f x in y :: @map (f, xs)
\end{Datalang}
\end{tabular}
\caption{Natural implementation of \datalang|map| in \DataLang}
\label{fig:map}
\end{figure}