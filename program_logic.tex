\section{Relational separation logic}
\label{sec:program_logic}

To prove the specifications of \cref{sec:specification}, we need a relational program logic.
In this section, we introduce such a logic by giving the reasoning rules that should hold.
In the next section, we use this logic to prove the specifications.
In section \cref{sec:simulation}, we show how it is actually defined and what notion of soundness supports it.

The reasoning rules are presented in \cref{fig:program_logic}.
We omitted some standard rules for brevity.
Compared to the specifications of \cref{sec:specification}, we introduced an additional parameter $\iProt$.
We will explain it together with the \RefTirName{RelProtocol} rule.

\subsection{Language-independent rules}
The following rules are independent of \DataLang and could be reused as is in further works.

\RefTirName{RelPost} states that two expressions are related when they are in the relational postcondition.

\RefTirName{RelStuck} relates \emph{strongly stuck} expressions.
An expression is strongly stuck when it is stuck for any state.

\RefTirName{RelBind} is a standard bind rule sequencing computations on both sides.

\RefTirName{RelSrcPure} and \RefTirName{RelTgtPure} let us take pure reduction steps in either the source or target.
Pure steps (omitted for brevity) are the reduction steps that are deterministic and do not depend on the state.

\subsection{Language-specific rules: non-determinism}
$\iSimv[\iProt]{\iPred}{\datalangExpr_s}{\datalangExpr_t}$ asserts that $\datalangExpr_t$ refines $\datalangExpr_s$: any behavior of $\datalangExpr_t$ is also a behavior of $\datalangExpr_s$.
Consequently, non-determinism is treated differently in the source and target: we treat non-determinism as \emph{angelic} in source reductions and \emph{demonic} in target reductions.

Our operational semantics uses non-determinism in the reduction of constructors: $\datalangBlock{\datalangTag}{\datalangExpr_1}{\datalangExpr_2}$ reduces to $\datalangBlockDet{\datalangTag}{\datalangVar_1}{\datalangVar_2}$, where $\datalangVar_1$ and $\datalangVar_2$ are bound to $\datalangExpr_1$ and $\datalangExpr_2$ in some non-deterministic order.
In the program logic, the user may \emph{choose} an order for the source reduction, by using one of the rules \RefTirName{RelSrcBlock1} or \RefTirName{RelSrcBlock2}, and it has to prove that the expressions are related against \emph{any} target order, by proving the two premises of the rule \RefTirName{RelTgtBlock}.

Once we have a deterministic constructor $\datalangBlockDet{\datalangTag}{\datalangVal_1}{\datalangVal_2}$, we can apply \RefTirName{RelSrcBlockDet} or \RefTirName{RelTgtBlockDet}, yielding a points-to assertion for the allocated block.

\subsection{Language-specific rules: private locations.}
We can reason on points-to assertions --- that we interpret as locations private to the source or target --- in a standard way.
The rules \RefTirName{RelSrcLoad} and \RefTirName{RelTgtLoad} let us load the pointed value while \RefTirName{RelSrcStore} and \RefTirName{RelTgtStore} let us update it with a new value.

\subsection{Language-specific rules: locations in the bijection.}
Corresponding source and target locations registered in the bijection through \RefTirName{BijInsert} have given up their respective points-to assertions but can still be accessed using the rules \RefTirName{RelLoad} and \RefTirName{RelStore}.

\RefTirName{RelLoad} states that simultaneously loading from two corresponding blocks yields similar values.

\RefTirName{RelStore} let us simultaneously store similar values into the same field of two corresponding blocks.

Together, these two rules enforce the bijection invariant: the contents of corresponding blocks are always similar.

\subsection{Abstract protocols} \label{subsec:protocols}

In \cref{sec:simulation}, we will see that our relation is defined coinductively and the first step of the proof essentially amounts to coinduction.
To internalize the coinduction hypothesis into the program logic, we introduce an additional parameter $\iProt$ that we interpret as a \emph{protocol}~\citep*{de-vilhena-2021}.

The function of the protocol is reflected in the $\RefTirName{RelProtocol}$ rule.
It gives some authorized pairs of expressions $\datalangExpr_s$ and $\datalangExpr_t$ the ability to evolve in an abstract way to arbitrary expressions $\datalangExpr_s'$ and $\datalangExpr_t'$ related by some postcondition $\iPredTwo$.
To conclude that $\datalangExpr_s$ and $\datalangExpr_t$ are related, one must prove that any such $\datalangExpr_s'$ and $\datalangExpr_t'$ remain related.

Our coinduction hypothesis assumes toplevel function calls to be compatible with the direct and DPS specifications that we want to prove.
Therefore, we instantiate the program logic with the $\iProt_\mathrm{TMC}$ protocol of \cref{fig:protocol} combining two sub-protocols $\iProt_\mathrm{dir}$ and $\iProt_\mathrm{DPS}$:
\[
    \iSimv{
        \iPred
    }{
        \datalangExpr_s
    }{
        \datalangExpr_t
    }
    \coloneqq
    \iSimv[
        \iProt_\mathrm{TMC}
    ]{
        \iPred
    }{
        \datalangExpr_s
    }{
        \datalangExpr_t
    }
\]

Abstractly, $\iProt_\mathrm{dir}$ and $\iProt_\mathrm{DPS}$ formalize the two \emph{calling conventions} of the transformed program.

$\iProt_\mathrm{dir}$ specifies the \emph{direct calling convention} induced by the direct transformation.
It requires $\datalangExpr_s$ and $\datalangExpr_t$ to be function calls to the same function with similar arguments.
To apply it, users can choose any postcondition implied by value similarity.

$\iProt_\mathrm{DPS}$ specifies the \emph{DPS calling convention} induced by the DPS transformation.
It requires $\datalangExpr_s$ to be a function call to a TMC-transformed function $\datalangFn$ and $\datalangExpr_t$ to be a function call to the DPS transform of $\datalangFn$.
As in the DPS specification, ownership of the destination must be passed to the protocol.
To apply it, users can choose any postcondition implied by the postcondition of the DPS specification, including the recovered ownership of the modified destination.

In the instantiated program logic, the application of the $\iProt_\mathrm{TMC}$ protocol yields the following rules:
\begin{mathpar}
    \inferrule*[lab=RelDir]
        {
            \datalangFn \in \dom{\datalangProg_s}
        \\\\
            \datalangVal_s \iSimilar \datalangVal_t
        \\\\
            \forall \datalangValTwo_s, \datalangValTwo_t \ldotp
            \datalangValTwo_s \iSimilar \datalangValTwo_t \iWand
            \iPred (\datalangValTwo_s, \datalangValTwo_t)
        }{
            \iSimv{
                \iPred
            }{
                \datalangCall{\datalangFnptr{\datalangFn}}{\datalangVal_s}
            }{
                \datalangCall{\datalangFnptr{\datalangFn}}{\datalangVal_t}
            }
        }
    \and
    \inferrule*[lab=RelDPS]
        {
            \datalangRenaming [\datalangFn] = \datalangFn_\mathit{dps}
        \\\\
            (\datalangLoc_1 + 1) \iPointsto_t (\datalangLoc_2, \datalangVal_t)
        \\\\
            (\datalangLoc_2 + 1) \iPointsto_t (\datalangLoc, \datalangIdx)
        \\\\
            (\datalangLoc + \datalangIdx) \iPointsto_t \datalangHole
        \\\\
            \datalangVal_s \iSimilar \datalangVal_t
        \\\\
            \forall \datalangValTwo_s, \datalangValTwo_t \ldotp
            (\datalangLoc + \datalangIdx) \iPointsto_t \datalangValTwo_t \iWand
            \datalangValTwo_s \iSimilar \datalangValTwo_t \iWand
            \iPred (\datalangValTwo_s, \datalangUnit)
        }{
            \iSimv{
                \iPred
            }{
                \datalangCall{\datalangFnptr{\datalangFn}}{\datalangVal_s}
            }{
                \datalangCall{\datalangFnptr{\datalangFn_\mathit{dps}}}{\datalangLoc_1}
            }
        }
\end{mathpar}

More generally, the technique of parameterizing the program logic by a protocol can be applied to other transformations.
As a matter of fact, we have also verified an inlining and an APS transformation --- both included in our mechanization.

For the APS transformation, we use a protocol similar to $\iProt_\mathrm{DPS}$ (combined with $\iProt_\mathrm{dir}$) that allows calling the APS transform of a source function with an integer accumulator --- assuming we extended \DataLang with integers:

\medskip
\begin{tabular}{rcl}
        $\iProt_\mathrm{APS} (\iPredTwo, \datalangExpr_s, \datalangExpr_t)$
        & $\coloneqq$ &
        $\exists \datalangFn, \datalangFn_\mathit{aps}, \datalangVal_s, acc, \datalangVal_t \ldotp$
    \\
        &&
        $\datalangFn \in \dom{\datalangProg_s} \iSep
        \datalangRenaming [\datalangFn] = \datalangFn_\mathit{aps} \iSep
        \datalangExpr_s = \datalangCall{\datalangFnptr{\datalangFn}}{\datalangVal_s} \iSep
        \datalangExpr_t = \datalangCall{\datalangFnptr{\datalangFn_\mathit{aps}}}{\datalangPair{acc}{\datalangVal_t}} \iSep {}$
    \\
        &&
        $\datalangVal_s \iSimilar \datalangVal_t \iSep {}$
    \\
        &&
        $\forall \datalangVal_s', \datalangExpr_t' \ldotp$
    \\
        &&
        $\bm{\mathrm{match}}\ \datalangVal_s'\ \bm{\mathrm{with}}\ \datalangNat \Rightarrow \datalangExpr_t' = acc + \datalangNat \mid \_ \Rightarrow \mathrm{strongly \mathhyphen stuck}_{\datalangProg_t} (\datalangExpr_t')\ \bm{\mathrm{end}} \iWand$
    \\
        &&
        $\iPredTwo (\datalangVal_s', \datalangExpr_t')$
\end{tabular}
\medskip

For the inlining transformation, we use a fairly simple protocol (combined with $\iProt_\mathrm{dir}$) relating a source function and its body:

\medskip
\begin{tabular}{rcl}
        $\iProt_\mathrm{inline} (\iPredTwo, \datalangExpr_s, \datalangExpr_t)$
        & $\coloneqq$ &
        $\exists \datalangFn, \datalangVar, \datalangExpr_s', \datalangExpr_t', \datalangVal_s, \datalangVal_t \ldotp$
    \\
        &&
        $\datalangProg_s [\datalangFn] = \datalangRec{\datalangVar}{\datalangExpr_s'} \iSep
        \datalangExpr_s' \rightsquigarrow \datalangExpr_t' \iSep
        \datalangExpr_s = \datalangCall{\datalangFnptr{\datalangFn}}{\datalangVal_s} \iSep
        \datalangExpr_t = \datalangLet{\datalangVar}{\datalangVal_t}{\datalangExpr_t'} \iSep {}$
    \\
        &&
        $\datalangVal_s \iSimilar \datalangVal_t \iSep {}$
    \\
        &&
        $\forall \datalangValTwo_s, \datalangValTwo_t \ldotp
        \datalangValTwo_s \iSimilar \datalangValTwo_t \iWand
        \iPredTwo (\datalangValTwo_s, \datalangValTwo_t)$
\end{tabular}
\medskip

Here, the relation $\datalangExpr_s \rightsquigarrow \datalangExpr_t$ allows $\datalangExpr_t$ to recursively inline functions in $\datalangExpr_s$.
As with TMC, it captures all possible inlining strategies.
In practice, the inlining depth would obviously be bounded.

\begin{figure}[tp]
    \begin{mathparpagebreakable}
        \inferrule*[lab=RelPost]
            {
                \iPred (\datalangVal_s, \datalangVal_t)
            }{
                \iSimv[\iProt]{\iPred}{\datalangVal_s}{\datalangVal_t}
            }
        \and
        \inferrule*[lab=RelStuck]
            {
                \mathrm{strongly \mathhyphen stuck}_{\datalangProg_s} (\datalangExpr_s)
            \and
                \mathrm{strongly \mathhyphen stuck}_{\datalangProg_t} (\datalangExpr_t)
            }{
                \iSimv[\iProt]{\iPred}{\datalangExpr_s}{\datalangExpr_t}
            }
        \\
        \inferrule*[lab=RelBind]
            {
                \iSimv[\iProt]{\lambdaAbs (\datalangVal_s, \datalangVal_t) \ldotp \iSimv[\iProt]{\iPred}{\datalangEctx_s [\datalangVal_s]}{\datalangEctx_t [\datalangVal_t]}}{\datalangExpr_s}{\datalangExpr_t}
            }{
                \iSimv[\iProt]{\iPred}{\datalangEctx_s [\datalangExpr_s]}{\datalangEctx_t [\datalangExpr_t]}
            }
        \\
        \inferrule*[lab=RelSrcPure]
            {
                \datalangExpr_s \pureStep{\datalangProg_s} \datalangExpr_s'
            \and
                \iSimv[\iProt]{\iPred}{\datalangExpr_s'}{\datalangExpr_t}
            }{
                \iSimv[\iProt]{\iPred}{\datalangExpr_s}{\datalangExpr_t}
            }
        \and
        \inferrule*[lab=RelTgtPure]
            {
                \datalangExpr_t \pureStep{\datalangProg_t} \datalangExpr_t'
            \and
                \iSimv[\iProt]{\iPred}{\datalangExpr_s}{\datalangExpr_t'}
            }{
                \iSimv[\iProt]{\iPred}{\datalangExpr_s}{\datalangExpr_t}
            }
        \\
        \inferrule*[lab=RelSrcBlock1]
            {
                \iSimv[\iProt]{\iPred}{
                    \begin{array}{l}
                        \datalangLet{\datalangVar_1}{\datalangExpr_{s1}}{\\
                        \datalangLet{\datalangVar_2}{\datalangExpr_{s2}}{\\
                        \datalangBlockDet{\datalangTag}{\datalangVar_1}{\datalangVar_2}}}
                    \end{array}
                }{\datalangExpr_t}
            }{
                \iSimv[\iProt]{\iPred}{\datalangBlock{\datalangTag}{\datalangExpr_{s1}}{\datalangExpr_{s2}}}{\datalangExpr_t}
            }
        \and
        \inferrule*[lab=RelSrcBlock2]
            {
                \iSimv[\iProt]{\iPred}{
                    \begin{array}{l}
                        \datalangLet{\datalangVar_2}{\datalangExpr_{s2}}{\\
                        \datalangLet{\datalangVar_1}{\datalangExpr_{s1}}{\\
                        \datalangBlockDet{\datalangTag}{\datalangVar_1}{\datalangVar_2}}}
                    \end{array}
                }{\datalangExpr_t}
            }{
                \iSimv[\iProt]{\iPred}{\datalangBlock{\datalangTag}{\datalangExpr_{s1}}{\datalangExpr_{s2}}}{\datalangExpr_t}
            }
        \and
        \inferrule*[lab=RelTgtBlock]
            {
                \iSimv[\iProt]{\iPred}{\datalangExpr_s}{
                    \begin{array}{l}
                        \datalangLet{\datalangVar_1}{\datalangExpr_{t1}}{\\
                        \datalangLet{\datalangVar_2}{\datalangExpr_{t2}}{\\
                        \datalangBlockDet{\datalangTag}{\datalangVar_1}{\datalangVar_2}}}
                    \end{array}
                }
            \and
                \iSimv[\iProt]{\iPred}{\datalangExpr_s}{
                    \begin{array}{l}
                        \datalangLet{\datalangVar_2}{\datalangExpr_{t2}}{\\
                        \datalangLet{\datalangVar_1}{\datalangExpr_{t1}}{\\
                        \datalangBlockDet{\datalangTag}{\datalangVar_1}{\datalangVar_2}}}
                    \end{array}
                }
            }{
                \iSimv[\iProt]{\iPred}{\datalangExpr_s}{\datalangBlock{\datalangTag}{\datalangExpr_{t1}}{\datalangExpr_{t2}}}
            }
        \\
        \inferrule*[lab=RelSrcBlockDet]
            {
                \forall \datalangLoc_s \ldotp
                \datalangLoc_s \iPointsto_s (\datalangTag, \datalangVal_{s1}, \datalangVal_{s2}) \iSepImp
                \iSimv[\iProt]{\iPred}{\datalangLoc_s}{\datalangExpr_t}
            }{
                \iSimv[\iProt]{\iPred}{\datalangBlockDet{\datalangTag}{\datalangVal_{s1}}{\datalangVal_{s2}}}{\datalangExpr_t}
            }
        \and
        \inferrule*[lab=RelTgtBlockDet]
            {
                \forall \datalangLoc_t \ldotp
                \datalangLoc_t \iPointsto_t (\datalangTag, \datalangVal_{t1}, \datalangVal_{t2}) \iSepImp
                \iSimv[\iProt]{\iPred}{\datalangExpr_s}{\datalangLoc_t}
            }{
                \iSimv[\iProt]{\iPred}{\datalangExpr_s}{\datalangBlockDet{\datalangTag}{\datalangVal_{t1}}{\datalangVal_{t2}}}
            }
        \\
        \inferrule*[lab=RelSrcLoad]
            {
                (\datalangLoc_s + \datalangIdx) \iPointsto_s \datalangVal_s
            \\\\
                (\datalangLoc_s + \datalangIdx) \iPointsto_s \datalangVal_s \iSepImp
                \iSimv[\iProt]{\iPred}{\datalangVal_s}{\datalangExpr_t}
            }{
                \iSimv[\iProt]{\iPred}{\datalangLoad{\datalangLoc_s}{\datalangIdx}}{\datalangExpr_t}
            }
        \and
        \inferrule*[lab=RelSrcStore]
            {
                (\datalangLoc_s + \datalangIdx) \iPointsto_s \datalangVal_s
            \\\\
                (\datalangLoc_s + \datalangIdx) \iPointsto_s \datalangVal_s' \iSepImp
                \iSimv[\iProt]{\iPred}{\datalangUnit}{\datalangExpr_t}
            }{
                \iSimv[\iProt]{\iPred}{\datalangStore{\datalangLoc_s}{\datalangIdx}{\datalangVal_s'}}{\datalangExpr_t}
            }
        \\
        \inferrule*[lab=RelTgtLoad]
            {
                (\datalangLoc_t + \datalangIdx) \iPointsto_t \datalangVal_t
            \\\\
                (\datalangLoc_s + \datalangIdx) \iPointsto_t \datalangVal_t \iSepImp
                \iSimv[\iProt]{\iPred}{\datalangExpr_s}{\datalangVal_t}
            }{
                \iSimv[\iProt]{\iPred}{\datalangExpr_s}{\datalangLoad{\datalangLoc_t}{\datalangIdx}}
            }
        \and
        \inferrule*[lab=RelTgtStore]
            {
                (\datalangLoc_t + \datalangIdx) \iPointsto_t \datalangVal_t
                \\\\
                (\datalangLoc_t + \datalangIdx) \iPointsto_t \datalangVal_t' \iSepImp
                \iSimv[\iProt]{\iPred}{\datalangExpr_s}{\datalangUnit}
            }{
                \iSimv[\iProt]{\iPred}{\datalangExpr_s}{\datalangStore{\datalangLoc_t}{\datalangIdx}{\datalangVal_t'}}
            }
%        \\
%        \inferrule*[lab=RelBijInsert]
%            {
%                \datalangLoc_s \iPointsto_s \datalangVal_s
%            \and
%                \datalangLoc_t \iPointsto_t \datalangVal_t
%            \and
%                \datalangVal_s \iSimilar \datalangVal_t
%            \and
%                \datalangLoc_s \iInBij \datalangLoc_t \iSepImp
%                \iSimv[\iProt]{\iPred}{\datalangExpr_s}{\datalangExpr_t}
%            }{
%                \iSimv[\iProt]{\iPred}{\datalangExpr_s}{\datalangExpr_t}
%            }
        \\
        \inferrule*[lab=RelLoad]
            {
                \datalangLoc_s \iSimilar \datalangLoc_t
            \and
                \forall \datalangVal_s, \datalangVal_t \ldotp
                \datalangVal_s \iSimilar \datalangVal_t
                \iSepImp
                \iPred (\datalangVal_s, \datalangVal_t)
            }{
                \iSimv[\iProt]{\iPred}{\datalangLoad{\datalangLoc_s}{\datalangIdx}}{\datalangLoad{\datalangLoc_t}{\datalangIdx}}
            }
        \and
        \inferrule*[lab=RelStore]
            {
                \datalangLoc_s \iSimilar \datalangLoc_t
            \and
                \datalangVal_s \iSimilar \datalangVal_t
            \and
                \iPred (\datalangUnit, \datalangUnit)
            }{
                \iSimv[\iProt]{\iPred}{\datalangStore{\datalangLoc_s}{\datalangIdx}{\datalangVal_s}}{\datalangStore{\datalangLoc_t}{\datalangIdx}{\datalangVal_t}}
            }
        \\
        \inferrule*[lab=RelApplyProtocol]
            {
              \iProt (\iPredTwo, \datalangExpr_s, \datalangExpr_t)
            \and
               \forall \datalangExpr_s', \datalangExpr_t' \ldotp
               \iPredTwo (\datalangExpr_s', \datalangExpr_t') \iSepImp
               \iSimv[\iProt]{\iPred}{\datalangExpr_s'}{\datalangExpr_t'}
            }{
                \iSimv[\iProt]{\iPred}{\datalangExpr_s}{\datalangExpr_t}
            }
    \end{mathparpagebreakable}
    \caption{Reasoning rules for simulation (excerpt, omitting freshness conditions)}
    \label{fig:program_logic}
\end{figure}
\begin{figure}[tp]
    \begin{tabular}{rcl}
            $\iProt_\mathrm{dir} (\iPredTwo, \datalangExpr_s, \datalangExpr_t)$
            & $\coloneqq$ &
            $\exists \datalangFn, \datalangVal_s, \datalangVal_t \ldotp$
        \\
            &&
            $\datalangFn \in \dom{\datalangProg_s} \iSep
            \datalangExpr_s = \datalangCall{\datalangFnptr{\datalangFn}}{\datalangVal_s} \iSep
            \datalangExpr_t = \datalangCall{\datalangFnptr{\datalangFn}}{\datalangVal_t} \iSep {}$
        \\
            &&
            $\datalangVal_s \iSimilar \datalangVal_t \iSep {}$
        \\
            &&
            $\forall \datalangValTwo_s, \datalangValTwo_t \ldotp
            \datalangValTwo_s \iSimilar \datalangValTwo_t \iWand
            \iPredTwo (\datalangValTwo_s, \datalangValTwo_t)$
        \\
            $\iProt_\mathrm{DPS} (\iPredTwo, \datalangExpr_s, \datalangExpr_t)$
            & $\coloneqq$ &
            $\exists \datalangFn, \datalangFn_\mathit{dps}, \datalangVal_s, \datalangLoc_1, \datalangLoc_2, \datalangLoc, \datalangIdx, \datalangVal_t \ldotp$
        \\
            &&
            $\datalangFn \in \dom{\datalangProg_s} \iSep
            \datalangRenaming [\datalangFn] = \datalangFn_\mathit{dps} \iSep
            \datalangExpr_s = \datalangCall{\datalangFnptr{\datalangFn}}{\datalangVal_s} \iSep
            \datalangExpr_t = \datalangCall{\datalangFnptr{\datalangFn_\mathit{dps}}}{\datalangLoc_1} \iSep {}$
        \\
            &&
            $(\datalangLoc_1 + 1) \iPointsto_t (\datalangLoc_2, \datalangVal_t) \iSep
            (\datalangLoc_2 + 1) \iPointsto_t (\datalangLoc, \datalangIdx) \iSep
            (\datalangLoc + \datalangIdx) \iPointsto \datalangHole \iSep
            \datalangVal_s \iSimilar \datalangVal_t$
        \\
            &&
            $\forall \datalangValTwo_s, \datalangValTwo_t \ldotp
            (\datalangLoc + \datalangIdx) \iPointsto \datalangValTwo_t \iSep
            \datalangValTwo_s \iSimilar \datalangValTwo_t \iWand
            \iPredTwo (\datalangValTwo_s, \datalangUnit)$
        \\
            $\iProt_\mathrm{TMC}$
            & $\coloneqq$ &
            $\iProt_\mathrm{dir} \sqcup \iProt_\mathrm{DPS}
            =
            \lambdaAbs (\iPredTwo, \datalangExpr_s, \datalangExpr_t) \ldotp \iProt_\mathrm{dir} (\iPredTwo, \datalangExpr_s, \datalangExpr_t) \iOr \iProt_\mathrm{DPS} (\iPredTwo, \datalangExpr_s, \datalangExpr_t)$
    \end{tabular}
    \caption{TMC protocol ($\iProt_\mathrm{TMC}$)}
    \label{fig:protocol}
\end{figure}