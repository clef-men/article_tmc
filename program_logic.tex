\section{Relational separation logic}
\label{sec:program_logic}

\begin{figure}[tp]
    \begin{mathparpagebreakable}
        \inferrule*[lab=RelPost]
            {
                \iPred (\datalangVal_s, \datalangVal_t)
            }{
                \iSimv[\iProt]{\iPred}{\datalangVal_s}{\datalangVal_t}
            }
        \and
        \inferrule*[lab=RelStuck]
            {
                \mathrm{strongly \mathhyphen stuck}_{\datalangProg_s} (\datalangExpr_s)
            \and
                \mathrm{strongly \mathhyphen stuck}_{\datalangProg_t} (\datalangExpr_t)
            }{
                \iSimv[\iProt]{\iPred}{\datalangExpr_s}{\datalangExpr_t}
            }
        \\
        \inferrule*[lab=RelBind]
            {
                \iSimv[\iProt]{\lambdaAbs (\datalangVal_s, \datalangVal_t) \ldotp \iSimv[\iProt]{\iPred}{\datalangEctx_s [\datalangVal_s]}{\datalangEctx_t [\datalangVal_t]}}{\datalangExpr_s}{\datalangExpr_t}
            }{
                \iSimv[\iProt]{\iPred}{\datalangEctx_s [\datalangExpr_s]}{\datalangEctx_t [\datalangExpr_t]}
            }
        \\
        \inferrule*[lab=RelSrcPure]
            {
                \datalangExpr_s \pureStep{\datalangProg_s} \datalangExpr_s'
            \and
                \iSimv[\iProt]{\iPred}{\datalangExpr_s'}{\datalangExpr_t}
            }{
                \iSimv[\iProt]{\iPred}{\datalangExpr_s}{\datalangExpr_t}
            }
        \and
        \inferrule*[lab=RelTgtPure]
            {
                \datalangExpr_t \pureStep{\datalangProg_t} \datalangExpr_t'
            \and
                \iSimv[\iProt]{\iPred}{\datalangExpr_s}{\datalangExpr_t'}
            }{
                \iSimv[\iProt]{\iPred}{\datalangExpr_s}{\datalangExpr_t}
            }
        \\
        \inferrule*[lab=RelSrcBlock1]
            {
                \iSimv[\iProt]{\iPred}{
                    \begin{array}{l}
                        \datalangLet{\datalangVar_1}{\datalangExpr_{s1}}{\\
                        \datalangLet{\datalangVar_2}{\datalangExpr_{s2}}{\\
                        \datalangBlockDet{\datalangTag}{\datalangVar_1}{\datalangVar_2}}}
                    \end{array}
                }{\datalangExpr_t}
            }{
                \iSimv[\iProt]{\iPred}{\datalangBlock{\datalangTag}{\datalangExpr_{s1}}{\datalangExpr_{s2}}}{\datalangExpr_t}
            }
        \and
        \inferrule*[lab=RelSrcBlock2]
            {
                \iSimv[\iProt]{\iPred}{
                    \begin{array}{l}
                        \datalangLet{\datalangVar_2}{\datalangExpr_{s2}}{\\
                        \datalangLet{\datalangVar_1}{\datalangExpr_{s1}}{\\
                        \datalangBlockDet{\datalangTag}{\datalangVar_1}{\datalangVar_2}}}
                    \end{array}
                }{\datalangExpr_t}
            }{
                \iSimv[\iProt]{\iPred}{\datalangBlock{\datalangTag}{\datalangExpr_{s1}}{\datalangExpr_{s2}}}{\datalangExpr_t}
            }
        \and
        \inferrule*[lab=RelTgtBlock]
            {
                \iSimv[\iProt]{\iPred}{\datalangExpr_s}{
                    \begin{array}{l}
                        \datalangLet{\datalangVar_1}{\datalangExpr_{t1}}{\\
                        \datalangLet{\datalangVar_2}{\datalangExpr_{t2}}{\\
                        \datalangBlockDet{\datalangTag}{\datalangVar_1}{\datalangVar_2}}}
                    \end{array}
                }
            \and
                \iSimv[\iProt]{\iPred}{\datalangExpr_s}{
                    \begin{array}{l}
                        \datalangLet{\datalangVar_2}{\datalangExpr_{t2}}{\\
                        \datalangLet{\datalangVar_1}{\datalangExpr_{t1}}{\\
                        \datalangBlockDet{\datalangTag}{\datalangVar_1}{\datalangVar_2}}}
                    \end{array}
                }
            }{
                \iSimv[\iProt]{\iPred}{\datalangExpr_s}{\datalangBlock{\datalangTag}{\datalangExpr_{t1}}{\datalangExpr_{t2}}}
            }
        \\
        \inferrule*[lab=RelSrcBlockDet]
            {
                \forall \datalangLoc_s \ldotp
                \datalangLoc_s \iPointsto_s (\datalangTag, \datalangVal_{s1}, \datalangVal_{s2}) \iSepImp
                \iSimv[\iProt]{\iPred}{\datalangLoc_s}{\datalangExpr_t}
            }{
                \iSimv[\iProt]{\iPred}{\datalangBlockDet{\datalangTag}{\datalangVal_{s1}}{\datalangVal_{s2}}}{\datalangExpr_t}
            }
        \and
        \inferrule*[lab=RelTgtBlockDet]
            {
                \forall \datalangLoc_t \ldotp
                \datalangLoc_t \iPointsto_t (\datalangTag, \datalangVal_{t1}, \datalangVal_{t2}) \iSepImp
                \iSimv[\iProt]{\iPred}{\datalangExpr_s}{\datalangLoc_t}
            }{
                \iSimv[\iProt]{\iPred}{\datalangExpr_s}{\datalangBlockDet{\datalangTag}{\datalangVal_{t1}}{\datalangVal_{t2}}}
            }
        \\
        \inferrule*[lab=RelSrcLoad]
            {
                (\datalangLoc_s + \datalangIdx) \iPointsto_s \datalangVal_s
            \\\\
                (\datalangLoc_s + \datalangIdx) \iPointsto_s \datalangVal_s \iSepImp
                \iSimv[\iProt]{\iPred}{\datalangVal_s}{\datalangExpr_t}
            }{
                \iSimv[\iProt]{\iPred}{\datalangLoad{\datalangLoc_s}{\datalangIdx}}{\datalangExpr_t}
            }
        \and
        \inferrule*[lab=RelSrcStore]
            {
                (\datalangLoc_s + \datalangIdx) \iPointsto_s \datalangVal_s
            \\\\
                (\datalangLoc_s + \datalangIdx) \iPointsto_s \datalangVal_s' \iSepImp
                \iSimv[\iProt]{\iPred}{\datalangUnit}{\datalangExpr_t}
            }{
                \iSimv[\iProt]{\iPred}{\datalangStore{\datalangLoc_s}{\datalangIdx}{\datalangVal_s'}}{\datalangExpr_t}
            }
        \\
        \inferrule*[lab=RelTgtLoad]
            {
                (\datalangLoc_t + \datalangIdx) \iPointsto_t \datalangVal_t
            \\\\
                (\datalangLoc_s + \datalangIdx) \iPointsto_t \datalangVal_t \iSepImp
                \iSimv[\iProt]{\iPred}{\datalangExpr_s}{\datalangVal_t}
            }{
                \iSimv[\iProt]{\iPred}{\datalangExpr_s}{\datalangLoad{\datalangLoc_t}{\datalangIdx}}
            }
        \and
        \inferrule*[lab=RelTgtStore]
            {
                (\datalangLoc_t + \datalangIdx) \iPointsto_t \datalangVal_t
                \\\\
                (\datalangLoc_t + \datalangIdx) \iPointsto_t \datalangVal_t' \iSepImp
                \iSimv[\iProt]{\iPred}{\datalangExpr_s}{\datalangUnit}
            }{
                \iSimv[\iProt]{\iPred}{\datalangExpr_s}{\datalangStore{\datalangLoc_t}{\datalangIdx}{\datalangVal_t'}}
            }
%        \\
%        \inferrule*[lab=RelBijInsert]
%            {
%                \datalangLoc_s \iPointsto_s \datalangVal_s
%            \and
%                \datalangLoc_t \iPointsto_t \datalangVal_t
%            \and
%                \datalangVal_s \iSimilar \datalangVal_t
%            \and
%                \datalangLoc_s \iInBij \datalangLoc_t \iSepImp
%                \iSimv[\iProt]{\iPred}{\datalangExpr_s}{\datalangExpr_t}
%            }{
%                \iSimv[\iProt]{\iPred}{\datalangExpr_s}{\datalangExpr_t}
%            }
        \\
        \inferrule*[lab=RelLoad]
            {
                \datalangLoc_s \iSimilar \datalangLoc_t
            \and
                \forall \datalangVal_s, \datalangVal_t \ldotp
                \datalangVal_s \iSimilar \datalangVal_t
                \iSepImp
                \iPred (\datalangVal_s, \datalangVal_t)
            }{
                \iSimv[\iProt]{\iPred}{\datalangLoad{\datalangLoc_s}{\datalangIdx}}{\datalangLoad{\datalangLoc_t}{\datalangIdx}}
            }
        \and
        \inferrule*[lab=RelStore]
            {
                \datalangLoc_s \iSimilar \datalangLoc_t
            \and
                \datalangVal_s \iSimilar \datalangVal_t
            \and
                \iPred (\datalangUnit, \datalangUnit)
            }{
                \iSimv[\iProt]{\iPred}{\datalangStore{\datalangLoc_s}{\datalangIdx}{\datalangVal_s}}{\datalangStore{\datalangLoc_t}{\datalangIdx}{\datalangVal_t}}
            }
        \\
        \inferrule*[lab=RelApplyProtocol]
            {
              \iProt (\iPredTwo, \datalangExpr_s, \datalangExpr_t)
            \and
               \forall \datalangExpr_s', \datalangExpr_t' \ldotp
               \iPredTwo (\datalangExpr_s', \datalangExpr_t') \iSepImp
               \iSimv[\iProt]{\iPred}{\datalangExpr_s'}{\datalangExpr_t'}
            }{
                \iSimv[\iProt]{\iPred}{\datalangExpr_s}{\datalangExpr_t}
            }
    \end{mathparpagebreakable}
    \caption{Reasoning rules for simulation (excerpt, omitting freshness conditions)}
    \label{fig:program_logic}
\end{figure}

In this section, we describe our relational program logic, presented in \cref{fig:program_logic}.
We omit some congruence rules for brevity.
The relation $\iSimv{\iPred}{\datalangExpr_s}{\datalangExpr_t}$\Xfrancois{analogous to an Iris wp?} relates a source expression $\datalangExpr_s$ with a target expression $\datalangExpr_t$ under a postcondition $\iPred$.
We extend it to support a precondition in the standard way:
\Xfrancois{Where is $\iProt$?}
\begin{mathline}
    \iSimvHoare{P}{\iPred}{\datalangExpr_s}{\datalangExpr_t}
    \coloneqq
    \iPersistent \left( P \iWand \iSimv{\iPred}{\datalangExpr_s}{\datalangExpr_t} \right)
\end{mathline}
%
Compared to the specifications of \cref{sec:specification}, we introduce an additional protocol parameter $\iProt$.
We explain it together with the \RefTirName{RelProtocol} rule in \cref{subsec:protocols}.

\paragraph{Language-independent rules}
The following rules are independent of \DataLang and could be reused as is in further works.

\RefTirName{RelPost} states that two values are related when they are in the relational postcondition.

\RefTirName{RelStuck} relates \emph{strongly stuck} expressions.
\Xfrancois{Explain up front that your wp is non-standard as it does not guarantee safety.}
An expression is strongly stuck when it is stuck for any heap state.
\Xfrancois{Explain why ``strongly'' is needed here.}

\RefTirName{RelBind} is a standard bind rule sequencing computations on both sides.

\RefTirName{RelSrcPure} and \RefTirName{RelTgtPure} let us take pure reduction steps in either the source or target.
Pure steps (definition omitted for brevity) are the reduction steps that are deterministic and do not depend on the state.

\paragraph{Language-specific rules: non-determinism}
$\iSimv[\iProt]{\iPred}{\datalangExpr_s}{\datalangExpr_t}$ asserts that $\datalangExpr_t$ refines $\datalangExpr_s$: any behavior of $\datalangExpr_t$ is also a behavior of $\datalangExpr_s$.
Consequently, non-determinism is treated differently in the source and target: we treat non-determinism as \emph{angelic} in source reductions and \emph{demonic} in target reductions.

Our operational semantics uses non-determinism in the reduction of constructors: $\datalangBlock{\datalangTag}{\datalangExpr_1}{\datalangExpr_2}$ reduces to $\datalangBlockDet{\datalangTag}{\datalangVar_1}{\datalangVar_2}$, where $\datalangVar_1$ and $\datalangVar_2$ are bound to $\datalangExpr_1$ and $\datalangExpr_2$ in some non-deterministic order.
In the program logic, the user may \emph{choose} an order for the source reduction, by using one of the rules \RefTirName{RelSrcBlock1} or \RefTirName{RelSrcBlock2}. On the other hand, they have to prove that the expressions are related against \emph{any} target order, by proving the two premises of the rule \RefTirName{RelTgtBlock}.

\paragraph{Language-specific rules: private locations.}
We can reason on points-to assertions in a standard way.
From a deterministic constructor $\datalangBlockDet{\datalangTag}{\datalangVal_1}{\datalangVal_2}$, we can apply \RefTirName{RelSrcBlockDet} or \RefTirName{RelTgtBlockDet}, yielding a points-to assertion for the allocated block.
The rules \RefTirName{RelSrcLoad} and \RefTirName{RelTgtLoad} let us load the pointed value while \RefTirName{RelSrcStore} and \RefTirName{RelTgtStore} let us update it with a new value.
We interpret locations owned by a points-to assertion as ``private'' to the source or target: they are not registered in the ``public'' partial heap bijection.

\paragraph{Language-specific rules: locations in the bijection.}
Corresponding source and target locations registered in the bijection through \RefTirName{BijInsert} have given up their respective points-to assertions but can still be accessed using the rules \RefTirName{RelLoad} and \RefTirName{RelStore}.

\RefTirName{RelLoad} states that simultaneously loading from two corresponding blocks yields similar values.

\RefTirName{RelStore} lets us store similar values into the same field of two corresponding blocks.

These two rules enforce the bijection invariant: corresponding blocks contain similar values.

%%% Local Variables:
%%% mode: latex
%%% TeX-master: "main"
%%% End: