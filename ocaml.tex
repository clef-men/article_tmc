\section{\OCaml Implementation}
\label{sec:implementation}

\subsection{Implementation history}

We first proposed adding TMC as an optional program transformation to
the OCaml compiler in
May 2015.\footnote{\nonanon{URL}{\url{https://github.com/ocaml/ocaml/pull/181}}}
The proposal was received favorably, but it never received an in-depth
review and a detailed performance evaluation and remained unmerged for
years.

We re-activated the discussion in July 2020 with a modified
implementation, and a careful performance
evaluation\footnote{\nonanon{URL}{\url{https://github.com/ocaml/ocaml/pull/9760}}}. The
new implementation put a larger focus on producing readable code,
giving more control to the user through annotations. It also removed
an optimization of the previous implementation that would specialize
the DPS version, generating a distinct definition for each block
offset. The new design was carefully reviewed by Basile Clément and
Pierre Chambart, and was finally merged in November 2021, available to
users with OCaml 4.14, released in March 2022.

The TMC transformation is that it adds extra parameters to functions
(two parameters, the block and the offset). At the time the OCaml
compiler had a limitation on some supported architectures, where it
would not optimze tail calls above a certain number of arguments
(enough that they cannot be all passed by registers), breaking the
tail-call promises of TMC on those systems. Xavier Leroy implemented
a change to the OCaml calling convention for those architectures in
May 2021,\footnote{\url{https://github.com/ocaml/ocaml/pull/10595}},
which was motivated by the TMC work.

\subsection{Examples}

Many functions that consume and produce lists are
tail-recursive-modulo-cons, in the sense that all they have a TMC
decomposition where all recursive calls are in TMC position. Notable
functions include \ocaml{map}, as already discussed, but also for
example:

\begin{Ocaml}
let[@tail_mod_cons] rec filter p = function
| [] -> []
| x :: xs -> if p x then x :: filter p xs else filter p xs

let[@tail_mod_cons] rec merge cmp l1 l2 =
  match l1, l2 with
  | [], l | l, [] -> l
  | h1 :: t1, h2 :: t2 ->
      if cmp h1 h2 <= 0
      then h1 :: merge cmp t1 l2
      else h2 :: merge cmp l1 t2
\end{Ocaml}

TMC is not useful only for lists or other ``linear'' data types, with
at most one recursive occurrence of the datatype in each
constructor.

\paragraph{A non-example} Consider a \ocaml{map} function on binary
trees:
\begin{Ocaml}
let[@tail_mod_cons] rec map f = function
| Leaf v -> Leaf (f v)
| Node(t1, t2) -> Node(map f t1, (map[@tailcall]) f t2)
\end{Ocaml}
In this function, there are two recursive calls, but only one of them
can be optimized; we used the \ocaml{[@tailcall]} attribute to direct
our implementation to optimize the call to the right child, as we will
discuss later. This is a \emph{bad} example of TMC usage in most
cases, given that
\begin{itemize}
\item If the tree is arbitrary, there is no reason that it would be
  right-leaning rather than left-leaning. Making only the right-child
  calls tail-calls does not protect us from stack overflows.
\item If the tree is known to be balanced, then in practice the depth
  is probably very small in both directions, so the TMC transformation
  is not necessary to have a well-behaved function.
\end{itemize}

\paragraph{Interesting non-linear examples} There \emph{are} interesting
examples of TMC-transformation on functions operating on tree-like
data structures, when there are natural assumptions about which child
is likely to contain a deep subtree. The OCaml compiler itself
contains a number of them; consider for example the following function
from the \ocaml{Cmm} module, one of its lower-level program
representations:
\begin{Ocaml}
let[@tail_mod_cons] rec map_tail f = function
  | Clet(id, exp, body) ->
      Clet(id, exp, map_tail f body)
  | Cifthenelse(cond, ifso, ifnot) ->
      Cifthenelse(cond, map_tail f ifso, (map_tail[@tailcall]) f ifnot)
  | Csequence(e1, e2) ->
      Csequence(e1, map_tail f e2)
  | Cswitch(e, tbl, el) ->
      Cswitch(e, tbl, Array.map (map_tail f) el)
  [...]
  | Cexit _ | Cop (Craise _, _, _) as cmm ->
      cmm
  | Cconst_int _ | Cvar _ | Ctuple _ | Cop _ as c ->
      f c
\end{Ocaml}

This function is traversing the ``tail'' context of an arbitrary
program term -- a meta-example! The \ocaml{Cifthenelse} node acts as
our binary-node constructor, we do not know which side is likely to be
larger, so TMC is not so interesting. The recursive calls for
\ocaml{Cswitch} are not in TMC position. But on the other hand the
\ocaml{Clet}, \ocaml{Csequence} cases are very beneficial to have in
TMC: while they have several recursive subtrees, they are in practice
only deeply nested in the direction that is turned into a tailcall by
the transformation. The OCaml compiler does sometimes encounter
machine-generated programs with a unusually long sequence of either
constructions, and the TMC transformation may very well avoid a stack
overflow in this case.

Another example would be
\href{https://github.com/ocaml/ocaml/pull/9636}{\#9636}, a patch to
the OCaml compiler proposed in June 2020 by Mark Shinwell, to get
a partially-tail-recursive implementation of the ``Common
Subexpression Elimination'' (CSE) pass through a manual
contiation-passing-style transform. Xavier Leroy remarked that the
existing implementation in fact fits the TMC fragment. Not all
recursive calls become tail-calls (this would require a more powerful
transformation or a longer, less readable patch), but the behavior of
TMC on the unchanged code matches the tail-call-ness proposed in the
human-written patch.

\subsection{Design choices}

\subsubsection{Resolving non-determinism}

The main question faced by an implementation is how to resolve the
inherent non-determinism. We identified fairly different kinds of
non-determinism.

\paragraph{Choices with an obviously better alternative.} Some choices
have an obviously better alternative, because they allow to strictly
improve more programs. For example, some implementations of TMC limit
themselves to calls that are immediately inside a constructor: they
would optimize the recursive call to \ocaml|f| in \ocaml|Cons(x, f y)|
but not, for example, in \ocaml|Cons(x1, Cons(x2, f y))|, or in
\ocaml|Cons (x, if p then f y else z)|.

Our implementation optimizes calls to functions under an arbitrary
nesting of tail-recursive contexts and constructor contexts, which is
the most general approach allowed by our relation. This comes at the
cost of some implementation complexity, but is otherwise an easy
choice to make when deciding on a specification.

\paragraph{Choices with a subtly better alternative.} In some cases it
is possible to put a bit more work in the choice heuristics, to
generate slightly better code, in terms of performance or
readability. For example, some natural way to simplify the
implementation would result in more subterms being transformed in
destination-passing-style, only to finally use the
\RefTirName{DPSBase} rule without any DPS function call. The generated
code is slightly less pleasant to read, and in our experience it can
be noticeably slower.

Code readability is important in our context of an on-demand program
transformation used by performance experts: those experts will often
read the intermediate representations produced by the compiler to
check that their performance assumptions hold.

\paragraph{Incomparable choices: force the user to decide} The
remaining choices are between incomparable alternatives, that could
each be better than the other depending on the specific program or
programmer intent. Consider again our binary tree example,
\ocaml|Node(map f left, map f right)|: we could make either the first
or the second argument a tail-call.

Our policy in such cases is to let the user decide, by providing
control over the transformation choices using \ocaml{[@tailcall]}
program annotations, and to force the user to decide by raising an
error in case of ambiguity:
\begin{Ocaml}
  | Cifthenelse(cond, ifso, ifnot) ->
      Cifthenelse(cond, map_tail f ifso, map_tail f ifnot)
      /[^^^^^^^^^^^^^^^^^^^^^^^^^^^^^^^^^^^^^^^^^^^^^^^^^^^^]/
/[Error]/: "[@tail_mod_cons]": this constructor application may be TMC-transformed
       in several different ways. Please disambiguate by adding an explicit
       "[@tailcall]" attribute to the call that should be made tail-recursive,
       or a "[@tailcall false]" attribute on calls that should not be
       transformed.
\end{Ocaml}

This approach would be incompatible with a view of the TMC
transformation as an implicit optimization, applied whenever
possible -- in that case one would rather have the compiler make
arbitrary choices rather than fail. We rather view TMC as a tool for
expert users to better reason about program performance. It should be
predictable and flexible (let the user express
their intent). Failing on the lack of annotations is an acceptable way
to drive user interaction.

\subsubsection{First-order implementation} A notable limitation of the
OCaml implementation is that it is first-order and non-modular in
nature: only direct calls to known functions can be converted into DPS
style, and the availability of a DPS variant is not exposed through
module abstractions.

It would be possible to allow DPS calls through external modules or
higher-order function parameters, by annotating interfaces and
function arguments with a \ocaml{[@tail_mod_cons]} annotation, and
elaborating them into a pair of functions, the direct-style and the DPS
version.

\subsection{Constructor compression}

\subsection{Implementation}

\subsection{Evaluation: benchmarks}

\subsection{Evaluation: adoption}

% [@tail_mod_cons] usage on Github

% Search URL:
%   https://github.com/search?q=tail_mod_cons+lang%3Aocaml&type=code

% Results (November 13th 2023):
%   - in the OCaml stdlib
%     + Melange
%       (reused from stdlib)
%     + janestreet/base
%       (with manual unfolding)
%   - in other general utility modules
%     https://github.com/RedPRL/asai/blob/ac523674579772e5929c8d912d407b8b7db89c74/src/Utils.ml#L33
%     https://github.com/jonathan-laurent/KaTie/blob/2db0d2cf223fae003b26bdf4c82a9788b5804bc7/src/utils.ml#L95
%     https://github.com/jamsidedown/ocaml-cll/blob/394effa65b5d3f883e2de8322ed60fb7aa495bec/examples/crab_cups.ml#L61
%   - in other user code
%     at list type:
%       https://github.com/brendanzab/language-garden/blob/9cc21e2f57228556cf0b867ca2874ceb4a4250ec/compile-arith/lib/StackLang.ml#L63
%       https://github.com/abella-prover/abella/blob/ceee13822001137898ca5de9c76a315da3d1f0e3/src/extensions.ml#L421
%       https://github.com/bikallem/spring/blob/6d10fef0b21caea3f975ff13132f5994e7c37db5/lib_spring/response.ml#L99
%       https://github.com/bikallem/spring/blob/6d10fef0b21caea3f975ff13132f5994e7c37db5/lib_spring/headers.ml#L150
%       https://github.com/polytypic/io/blob/dbe4d37a98ef958178fe18bb4dae396a5537392e/src/io/io.ml#L8
%       https://github.com/avsm/eeww/blob/5b6c0617306f4c9d8b44bf07ef4d45ec9668dd0c/lib/cohttp/cohttp-eio/src/rwer.ml#L43
%       https://github.com/just-max/less-power/blob/96f8e88f72275246b6bc175e44c732aa6e08ce7f/src/common/path_util.ml#L18
%       https://github.com/dalps/ocaml-challenge/blob/8fe115023457b6b4359ef736c78ba980cd871717/3/extract/solve.ml#L4
%       https://github.com/Octachron/ocaml-changelog-analyzer/blob/d6757b9576b5070ea3cfe6d4f4f79aaa3cc6e5d4/parse/release.ml#L4
%       https://github.com/Skyb0rg007/PL-Reading-Group/blob/7002ae35ba69fb9010e5049eee88ab475bc79ec3/list-optimizations/lib/adt.ml#L44
%       https://github.com/jfeser/symetric/blob/f49a57afc90a2c1e38f1097f98bd74725c074b9b/lib/tower.ml#L75
%       https://github.com/verse-lab/sisyphus/blob/a08addae06bbccb1e6ea68bac3d7f9eeea4197be/lib/dynamic/utils.ml#L65
%       https://github.com/jaymody/ocaml-tokenizers/blob/06cbce75d2865690cb39017222efff5c284bc721/lib/utils.ml#L18
%       https://github.com/Bannerets/ocaml-kdl/blob/687c404ebeceef7fd905935c23665294133f99e6/src/lens.ml#L83
%       https://github.com/OCADml/OCADml/blob/f5dfbf55f9c19ad808daa0df4d76a55e58756e5f/lib/poly3.ml#L33
%     other type
%       https://github.com/johnridesabike/acutis/blob/4113fb516d9c5dd6f2dbfd9657f52c5ae7a5dcae/lib/matching.ml#L161
%         (on-demand stream as a record of functions)
%       https://github.com/RedPRL/ocaml-bwd/blob/fbf496b29532085b38073eaa62ba3d22ac619d5d/src/BwdNoLabels.ml#L60
%         (snoc list)
%       https://github.com/Skyb0rg007/PL-Reading-Group/blob/7002ae35ba69fb9010e5049eee88ab475bc79ec3/effects/lib/free/tseq.ml#L44
%         (difference list gadt)

%   - Misuse
%     https://github.com/chengsun/simd-sexp/commit/e714a8205c6df512d920a31a4dd2448975703c55
%       (already tail-recursive)
%     https://github.com/gridbugs/llama/blob/96d1c96c31ff136f465b5f37c981fff591bac6fd/src/midi/byte_array_parser.ml#L50
%       (needless complexity)
%     https://github.com/LimitEpsilon/lambda-interpreter/blob/ab8e81de7f896aa64eba14e23d964b9705e94161/lib/evaluate.ml#L46
%       (already tail-recursive)

\subsection{TMC and OCaml 5}

%%% Local Variables:
%%% mode: latex
%%% TeX-master: "main"
%%% End: