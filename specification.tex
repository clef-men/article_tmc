\section{Specifying TMC}
\label{sec:specification}

In this section, we gradually introduce aspects of our relational separation logic, by introducing our specifications for the direct-style and destination-passing-style transformations of \cref{sec:formalization} in relational separation logic.
% In the next sections, we will unroll the proof of these specifications and extract a soundness theorem that lives outside of separation logic, depending only on the \DataLang semantics.

\subsection{Direct transformation}

Intuitively, the direct transformation $\datalangExpr_s \tmcDir{\datalangRenaming} \datalangExpr_t$ preserves the behaviours of the source expression $\datalangExpr_s$.
Basically, $\datalangExpr_s$ and $\datalangExpr_t$ compute the same thing.
Using \emph{relational Hoare logic}, an assumed extension of standard Hoare logic relating two expressions, we would write:
\[
    \iSimvHoare{
        \datalangExpr_s \tmcDir{\datalangRenaming} \datalangExpr_t
    }{
        \datalangVal_s, \datalangVal_t \ldotp
        \datalangVal_s \iSimilar \datalangVal_t
    }{
        \datalangExpr_s
    }{
        \datalangExpr_t
    }
\]

The informal meaning of this specification is that 1) $\datalangExpr_t$ refines $\datalangExpr_s$ in the sense that any behaviour (converging, diverging or stuck execution) of $\datalangExpr_t$ is also a behaviour of $\datalangExpr_s$ and 2) if $\datalangExpr_t$ converges to value $\datalangVal_t$, $\datalangExpr_s$ also converges to some value $\datalangVal_s$ that is \emph{similar} to $\datalangVal_s$.
We will formalize the notion of \emph{behaviour} in \cref{sec:simulation} and that of \emph{similarity} later in this section.
For the time being, the ready may assume similarity is just equality on values.

\subsection{DPS transformation}

The DPS transformation $(\datalangLoc, \datalangIdx, \datalangExpr_s) \tmcDps{\datalangRenaming} \datalangExpr_t$ is parameterized by a destination $(\datalangLoc, \datalangIdx)$ pointing to an uninitialized field of some block.
Intuitively, $\datalangExpr_t$ computes the same thing as $\datalangExpr_s$ but writes it into the destination instead of returning it.
Using \emph{relational separation logic}, an assumed further extension of relational Hoare logic with the concepts of separation logic, we write:
\[
    \iSimvHoare{
        (\datalangLoc, \datalangIdx, \datalangExpr_s) \tmcDps{\datalangRenaming} \datalangExpr_t \iSep
        (\datalangLoc + \datalangIdx) \iPointsto_t \datalangHole
    }{
        \datalangVal_s, \datalangUnit \ldotp
        \exists \datalangVal_t \ldotp
        (\datalangLoc + \datalangIdx) \iPointsto_t \datalangVal_t \iSep
        \datalangVal_s \iSimilar \datalangVal_t
    }{
        \datalangExpr_s
    }{
        \datalangExpr_t
    }
\]

The \emph{points-to assertion} $(\datalangLoc + \datalangIdx) \iPointsto_t -$ expresses that the destination is uniquely owned by the transformed program.
The specification requires $\datalangExpr_t$ to fill it with some value that is similar to the returned source value.
This captures destination-passing style in simple terms.

\subsection{Heap bijection}

Defining value similarity as just syntactic equality is not sufficient: corresponding source and target block allocations are not done in locksteps, so their location may differ.
For example, consider the \datalang{map} function (\cref{fig:map}) and its DPS transform (\cref{fig:map_tmc}).
In the source program, the cons cell \datalang{y :: @map (fn, xs)} is allocated after the recursive call.
In the transformed program, the corresponding block is allocated before the call.

To deal with this, we introduce a \emph{heap bijection} as in \Simuliris~\citep*{simuliris-2022}.
Corresponding source and target locations --- pointing to block fields in our semantics --- are logically and permanently registered in this bijection. 
We can then refer to a registered pair of locations though the assertion $\datalangLoc_s \iInBij \datalangLoc_t$.
This mechanism is formalized by the \RefTirName{BijInsert} rule:

\begin{mathpar}
    \inferrule*[lab=BijInsert]
        {
            \datalangLoc_s \iPointsto_s \datalangVal_s
        \and
            \datalangLoc_t \iPointsto_t \datalangVal_t
        \and
            \datalangVal_s \iSimilar \datalangVal_t
        }{
            \datalangLoc_s \iInBij \datalangLoc_t
        }
\end{mathpar}

We can formally define value similarity $\datalangVal_s \iSimilar \datalangVal_t$ in \cref{fig:isimilar}.
It coincides with equality except on blocks, for which we require all fields to be registered in the bijection.

\begin{figure}[tp]
    \centering
    \begin{mathparpagebreakable}
        \inferrule*
            {}{
                \lambdUnit \iSimilar \lambdUnit
            }
        \and
        \inferrule*
            {}{
                i \iSimilar i
            }
        \and
        \inferrule*
            {}{
                n \iSimilar n
            }
        \and
        \inferrule*
            {}{
                b \iSimilar b
            }
        \and
        \inferrule*
            {
                \forall i \in \{0,1,2\} \ldotp
                (\loc_s + i) \iInBij (\loc_t + i)
            }{
                \loc_s \iSimilar \loc_t
            }
        \and
        \inferrule*
            {
                f \in \dom{p_s}
            }{
                \lambdFunc{f} \iSimilar \lambdFunc{f}
            }
    \end{mathparpagebreakable}
    \caption{Similarity in $\iProp$}
    \label{fig:isimilar}
\end{figure}

%%% Local Variables:
%%% mode: latex
%%% TeX-master: "main"
%%% End:
