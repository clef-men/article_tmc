\documentclass[acmsmall,screen,review,draft]{acmart} % TODO: remove [review,draft] in the final version
\settopmatter{printfolios=true,printccs=true,printacmref=true}
% \renewcommand\footnotetextcopyrightpermission[1]{} % removes footnote with conference information in first column

% ---------------------------------------------------------

\usepackage[T1]{fontenc}
\usepackage[utf8]{inputenc}
\usepackage[scaled=0.8]{beramono}

% ---------------------------------------------------------

\usepackage{boolean}
\usepackage{reversion}

\newcommand{\Anonymous}{\FALSE}

\newcommand{\nonanon}[2]{\Anonymous{#1 removed for anonymity.}{#2}}

% ---------------------------------------------------------

\usepackage{mybiblio}

% ---------------------------------------------------------

\usepackage{enumitem}

% ---------------------------------------------------------

\usepackage{mathtools}

\let\oldexists\exists
\let\exists\relax\DeclareMathOperator{\exists}{\oldexists}
\let\oldforall\forall
\let\forall\relax\DeclareMathOperator{\forall}{\oldforall}

\DeclareMathOperator{\lambdaAbs}{\lambda}
\DeclareMathOperator{\muAbs}{\mu}
\DeclareMathOperator{\nuAbs}{\nu}

% ---------------------------------------------------------

\usepackage{amsthm}

\newenvironment{myproof}[1][\proofname]{%
  \begin{proof}[#1]$ $\par\nobreak\ignorespaces
}{%
  \end{proof}
}

% ---------------------------------------------------------

\usepackage{bm}

% ---------------------------------------------------------

\usepackage{mathpartir}
\let\TirName\textsc
\renewcommand{\TirNameStyle}[1]{\hypertarget{#1}{\TirName{#1}}}
\newcommand{\RefTirName}[1]{\hyperlink{#1}{\TirName{#1}}\xspace}

\newenvironment{mathline}
   {\abovedisplayskip 0.2em
    \belowdisplayskip 0.2em
    \begin{mathpar}}
   {\end{mathpar}}

% ---------------------------------------------------------

\usepackage[capitalize,noabbrev,nameinlink]{cleveref}

% ---------------------------------------------------------
\usepackage{mycomments}
% % uncomment the following command to hide comments
\UNXXX

% ---------------------------------------------------------

\usepackage{macros}

\usepackage{listings/ocaml}
\usepackage{listings/datalang}

% ---------------------------------------------------------

%% Copyright information
%% Supplied to authors (based on authors' rights management selection;
%% see authors.acm.org) by publisher for camera-ready submission;
%% use 'none' for review submission.

\setcopyright{rightsretained}
\acmDOI{10.1145/3704915}
\acmJournal{PACMPL}
\acmYear{2025}
\acmVolume{9}
\acmNumber{POPL}
\acmArticle{79}
\acmMonth{1}
\received{2024-07-11}
\received[accepted]{2024-11-07}

% ---------------------------------------------------------
% ---------------------------------------------------------

\begin{document}

% ---------------------------------------------------------

\title{Tail Modulo Cons, \OCaml, and Relational Separation Logic}

\author{Clément Allain}
\orcid{0009-0005-2972-5181}
\affiliation{%
  \institution{Inria}
  \city{Paris}
  \country{France}
}
\email{clement.allain@inria.fr}

\author{Frédéric Bour}
\orcid{0009-0007-5268-5784}
\affiliation{%
  \institution{Tarides}
  \city{Paris}
  \country{France}
}
\email{frederic.bour@lakaban.net}

\author{Basile Clément}
\orcid{0000-0002-9126-0937}
\affiliation{%
  \institution{OCamlPro}
  \city{Paris}
  \country{France}
}
\email{bc@ocamlpro.com}

\author{François Pottier}
\orcid{0000-0002-4069-1235}
\affiliation{%
  \institution{Inria}
  \city{Paris}
  \country{France}
}
\email{francois.pottier@inria.fr}

\author{Gabriel Scherer}
\orcid{0000-0003-1758-3938}
\affiliation{%
  \institution{Inria}
  \city{Paris}
  \country{France}
}
\affiliation{%
  \institution{IRIF, Université Paris Cité}
  \city{Paris}
  \country{France}
}
\email{gabriel.scherer@inria.fr}

\begin{CCSXML}
<ccs2012>
   <concept>
       <concept_id>10011007.10011006.10011041</concept_id>
       <concept_desc>Software and its engineering~Compilers</concept_desc>
       <concept_significance>500</concept_significance>
       </concept>
   <concept>
       <concept_id>10011007.10011006.10011008.10011024.10011033</concept_id>
       <concept_desc>Software and its engineering~Recursion</concept_desc>
       <concept_significance>500</concept_significance>
       </concept>
   <concept>
       <concept_id>10003752.10003790.10011742</concept_id>
       <concept_desc>Theory of computation~Separation logic</concept_desc>
       <concept_significance>500</concept_significance>
       </concept>
   <concept>
       <concept_id>10003752.10010124.10010138.10010142</concept_id>
       <concept_desc>Theory of computation~Program verification</concept_desc>
       <concept_significance>500</concept_significance>
       </concept>
 </ccs2012>
\end{CCSXML}

\ccsdesc[500]{Software and its engineering~Compilers}
\ccsdesc[500]{Software and its engineering~Recursion}
\ccsdesc[500]{Theory of computation~Separation logic}
\ccsdesc[500]{Theory of computation~Program verification}

\begin{abstract}
    Common functional languages incentivize tail-recursive functions, as opposed to general recursive functions that consume stack space and may not scale to large inputs.
%
This distinction occasionally requires writing functions in a tail-recursive style that may be more complex and slower than the natural, non-tail-recursive definition.

This work describes our implementation of the \emph{tail modulo constructor} (TMC) transformation in the \OCamlLang compiler, an optimization that provides stack-efficiency for a larger class of functions --- tail-recursive \emph{modulo constructors} --- which includes in particular the natural definition of \ocaml{List.map} and many similar recursive data-constructing functions.

We prove the correctness of this program transformation in a simplified setting --- a small untyped calculus --- that captures the salient aspects of the \OCamlLang implementation. Our proof is mechanized in the \Coq proof assistant, using the \Iris base logic.
%
An independent contribution of our work is an extension of the \Simuliris approach to define simulation relations that support different calling conventions. To our knowledge, this is the first use of \Simuliris to prove the correctness of a compiler transformation.

\end{abstract}

\maketitle

\keywords{compilation, separation logic, program verification}

\begin{CCSXML}
<ccs2012>
<concept>
<concept_id>10003752.10003790.10011742</concept_id>
<concept_desc>Theory of computation~Separation logic</concept_desc>
<concept_significance>500</concept_significance>
</concept>
<concept>
<concept_id>10003752.10010124.10010138.10010142</concept_id>
<concept_desc>Theory of computation~Program verification</concept_desc>
<concept_significance>500</concept_significance>
</concept>
</ccs2012>
\end{CCSXML}

\ccsdesc[500]{Theory of computation~Separation logic}
\ccsdesc[500]{Theory of computation~Program verification}

% ---------------------------------------------------------

\newcommand{\separate}{\FALSE}

\section{Introduction}

\subsection{Prologue}

``OCaml'', we teach our students, ``is a functional programming language. We can write the beautiful function \ocaml{List.map} as follows:''
\begin{Ocaml}
let rec map f = function
| [] -> []
| x :: xs -> f x :: map f xs
\end{Ocaml}

``Well, actually'', we continue, ``OCaml is an effectful language, so we need to be careful about the evaluation order. We want \ocaml{map} to process elements from the beginning to the end of the input list, and the evaluation order of \ocaml{f x :: map f xs} is unspecified. So we write:''
\begin{Ocaml}
let rec map f = function
| [] -> []
| x :: xs ->
  let y = f x in
  y :: map f xs
\end{Ocaml}

``Well, actually, this version fails with a \ocaml{Stack_overflow}
exception on large input lists. If you want your \ocaml{map} to behave
correctly on all inputs, you should write a \emph{tail-recursive}
version. For this you can use the accumulator-passing style:''
\begin{Ocaml}
let map f li =
  let rec map_ acc = function
  | [] -> List.rev acc
  | x :: xs -> map_ (f x :: acc) xs
  in map_ [] f li
\end{Ocaml}

``Well, actually, this version works fine on large lists, but it is
less efficient than the original version. It is noticeably slower on
small lists, which are the most common inputs for most programs. We
measured it 35\% slower on lists of size 10. If you want to write
a robust function for a standard library, you may want to support both
use-cases as well as possible. One approach is to start with
a non-tail-recursive version, and switch to a tail-recursive version
for large inputs; even there you can use some manual optimizations to
reduce the overhead of the accumulator. For example, the nice
\href{https://github.com/c-cube/ocaml-containers}{Containers} library
does it as follows:''.

%\hspace{-3em}
\begin{minipage}{0.6\linewidth}
\begin{Ocaml}[basicstyle=\ttfamily\tiny]
let tail_map f l =
  (* Unwind the list of tuples, reconstructing the full list front-to-back.
     @param tail_acc a suffix of the final list; we append tuples' content
     at the front of it *)
  let rec rebuild tail_acc = function
    | [] -> tail_acc
    | (y0, y1, y2, y3, y4, y5, y6, y7, y8) :: bs ->
      rebuild (y0 :: y1 :: y2 :: y3 :: y4 :: y5 :: y6 :: y7 :: y8 :: tail_acc) bs
  in
  (* Create a compressed reverse-list representation using tuples
     @param tuple_acc a reverse list of chunks mapped with [f] *)
  let rec dive tuple_acc = function
    | x0 :: x1 :: x2 :: x3 :: x4 :: x5 :: x6 :: x7 :: x8 :: xs ->
      let y0 = f x0 in let y1 = f x1 in let y2 = f x2 in
      let y3 = f x3 in let y4 = f x4 in let y5 = f x5 in
      let y6 = f x6 in let y7 = f x7 in let y8 = f x8 in
      dive ((y0, y1, y2, y3, y4, y5, y6, y7, y8) :: tuple_acc) xs
    | xs ->
      (* Reverse direction, finishing off with a direct map *)
      let tail = List.map f xs in
      rebuild tail tuple_acc
  in
  dive [] l
\end{Ocaml}
\end{minipage}
\hfill
\begin{minipage}{0.4\linewidth}
\begin{Ocaml}[basicstyle=\ttfamily\tiny]
let direct_depth_default_ = 1000

let map f l =
  let rec direct f i l = match l with
    | [] -> []
    | [x] -> [f x]
    | [x1;x2] -> let y1 = f x1 in [y1; f x2]
    | [x1;x2;x3] ->
      let y1 = f x1 in let y2 = f x2 in [y1; y2; f x3]
    | _ when i=0 -> tail_map f l
    | x1::x2::x3::x4::l' ->
      let y1 = f x1 in
      let y2 = f x2 in
      let y3 = f x3 in
      let y4 = f x4 in
      y1 :: y2 :: y3 :: y4 :: direct f (i-1) l'
  in
  direct f direct_depth_default_ l
\end{Ocaml}
\end{minipage}
\lstset{basicstyle=\small\ttfamily}

At this point, unfortunately, some students leave the class and never
come back.

We propose a new feature for the OCaml compiler, an explicit, opt-in
``Tail Modulo Cons'' transformation, to retain our students. After the
first version (or maybe, if we are teaching an advanced class, after
the second version), we could show them the following version:
\begin{Ocaml}
let[@tail_mod_cons] rec map f = function
| [] -> []
| x :: xs -> f x :: map f xs
\end{Ocaml}

This version is as fast as the simple implementation, tail-recursive,
and easy to write.

The catch, of course, is to teach when this \ocaml{[@tail_mod_cons]}
annotation can be used. Maybe we would not show it at all, and pretend
that the direct \ocaml{map} version with \ocaml{let y} is fine. This
would be a much smaller lie than it currently is,
a \ocaml{[@tail_mod_cons]}-sized lie.

Finally, experts should be very happy. They know about all these
versions, but they do not have to write them by hand anymore. Have
a program perform (some of) the program transformations that they are
currently doing manually.

\subsection{TMC transformation example}

A function call is in \emph{tail position} within a function
definition if the definition has ``nothing to do'' after evaluating
the function call -- the result of the call is the result of the whole
function at this point of the program. (A precise definition will be
given in Section~\ref{sec:tcmc}.) A function is \emph{tail recursive}
if all its recursive calls are tail calls.

In the definition of \ocaml{map}, the recursive call is not in tail
position: after computing the result of \ocaml{map f xs} we still have
to compute the final list cell, \ocaml{y :: ?}. We say that a call is
\emph{tail modulo cons} when the work remaining is formed of data
\emph{constructors} only, such as \ocaml{(::)} here.

\begin{Ocaml}
let[@tail_mod_cons] rec map f = function
| [] -> []
| x :: xs ->
  let y = f x in
  y :: map f xs
\end{Ocaml}

Other datatype constructors may also be used; the following example is
also tail-recursive \emph{modulo cons}:

\begin{Ocaml}
let[@tail_mod_cons] rec tree_of_list = function
| [] -> Empty
| x :: xs -> Node(Empty, x, tree_of_list xs)
\end{Ocaml}

The TMC transformation returns an equivalent function in
\emph{destination-passing} style where the calls in \emph{tail modulo
  cons} position have been turned into \emph{tail} calls. In
particular, for \ocaml{map} it gives a tail-recursive function, which
runs in constant stack space; many other list functions also become
tail-recursive. The transformed code of \ocaml{map} can be described as
follows:

\hspace{-1.6em}
\begin{minipage}{0.5\linewidth}
\begin{Ocaml}
let rec map f = function
| [] -> []
| x::xs ->
  let y = f x in
  let dst = y :: Hole in
  map_dps dst 1 f xs;
  dst
\end{Ocaml}
\end{minipage}
\hfill
\begin{minipage}{0.5\linewidth}
\begin{Ocaml}
and map_dps dst i f = function
| [] ->
  dst.i <- []
| x::xs ->
  let y = f x in
  let dst' = y :: Hole in
  dst.i <- dst';
  map_dps dst' 1 f xs
\end{Ocaml}
\end{minipage}

The transformed code has two variants of the \ocaml{map} function. The
\ocaml{map_dps} variant is in \emph{destination-passing style}, it
expects additional parameters that specify a memory location,
a \emph{destination}, and will write its result to this
\emph{destination} instead of returning it. It is tail-recursive. The
\ocaml{map} variant provides the same interface as the non-transformed
function, and internally calls \ocaml{map_dps} on non-empty lists. It
is not tail-recursive, but it does not call itself recursively, it
jumps to the tail-recursive \ocaml{map_dps} after one call.

The key idea of the transformation is that the expression \ocaml{y :: map f xs}, which contained a non-tail-recursive call, is transformed into first the computation of a \emph{partial} list cell, written \ocaml{y :: ?}, followed by a call to \ocaml{map_dps} that is asked
to write its result in the position of the \ocaml{?}. The recursive
call thus happens after the cell creation (instead of before), in
tail-recursive position in the \ocaml{map_dps} variant. In the direct
variant, the value of the destination \ocaml{dst} has to be returned
after the call.

The transformed code is in a pseudo-OCaml, it is not a valid OCaml
program: we use a magical \ocaml{?} constant, and our notation
\ocaml{dst.i <- ...} to update constructor parameters in-place is also
invalid in source programs. The transformation is implemented on
a lower-level, untyped intermediate representation of the OCaml
compiler (Lambda), where those operations do exist. The OCaml type
system is not expressive enough to type-check the transformed program:
the list cell is only partially-initialized at first, each partial
cell is mutated exactly once, and in the end the whole result is
returned as an \emph{immutable} list. Some type system are expressive
enough to represent this transformed code, notably Mezzo
\citep*{mezzo}.


TODO: what we have above is a re-import of the JFLA introduction.
What is below is a copy of the previous introduction-in-progress.
What should reread it, and more importantly include a new subsection on the proof,
which can be mostly reused from what is below.

\section{Introduction}

In \OCaml, the natural implementation of the \ocaml|map| function on polymorphic lists is the following:

\begin{Ocaml}
let rec map f xs =
  match xs with
  | [] ->
      []
  | x :: xs ->
      let y = f x in
      y :: map f xs
\end{Ocaml}

Yet, calling \ocaml|map| on a large input list may crash with the dreaded \ocaml|Stack_overflow| exception.
%
Indeed, as functions calls consume stack space, the stack usage of this implementation is linear in the size of the input list.
%
This is the reason why \OCaml programmers have to be careful, when they write recursive functions, to remain in the so-called \emph{tail-recursive} fragment: every recursive call must be in tail position, that is the last thing the function does before returning.
%
As there is nothing to do after a tail call, the stack frame can be directly reused.
%
Therefore, tail calls do not consume stack space.
%
Knowing this, we can provide another tail-recursive implementation of \ocaml|map|:

\begin{minipage}{0.5\linewidth}
\begin{Ocaml}
let rec rev_map f ys xs =
  match xs with
  | [] ->
      ys
  | x :: xs ->
      let y = f x in
      rev_map f (y :: ys) xs
\end{Ocaml}
\end{minipage}
\hfill
\begin{minipage}{0.5\linewidth}
\begin{Ocaml}
let map f xs =
  let ys = rev_map f [] xs in
  List.rev ys
\end{Ocaml}
\end{minipage}

This second version computes the reverse result list with \ocaml|rev_map| in the accumulator-passing style and applies \ocaml|List.rev|.
%
As both \ocaml|rev_map| and \ocaml|rev| are tail-recursive, it is stack-safe.
%
However, it is not as efficient on small lists since it does two traversals instead of one.
%
There is yet another implementation:

\begin{minipage}{0.65\linewidth}
\begin{Ocaml}
/* [dst] is a partially initialized list cell. */
let rec map_dps dst f xs =
  match xs with
  | [] ->
      /* Write result empty list to [dst]. */
      set_field dst 1 []
  | x :: xs ->
      let y = f x in
      /* New partially initialized list cell. */
      let dst' = y :: [] in
      /* Write result [dst'] to [dst]. */
      set_field dst 1 dst' ;
      /* Continue with [dst'] as new destination. */
      map_dps dst' f xs
\end{Ocaml}
\end{minipage}
\hfill
\begin{minipage}{0.45\linewidth}
\begin{Ocaml}
let map f xs =
  match xs with
  | [] ->
      []
  | x :: xs ->
      let y = f x in
      let dst = y :: [] in
      map_dps dst f xs ;
      dst
\end{Ocaml}
\end{minipage}

This third implementation is much less natural.
%
It takes advantage of the memory representation of lists in \OCaml: some constant integer for the empty list and a heap-allocated cell containing one tag followed by two fields for the list constructor.
%
The \ocaml|set_field| function allows us to modify these fields.
%
It can be implemented using the unsafe \ocaml|Obj| module.

The core of this implementation is the tail-recursive \ocaml|map_dps| auxiliary function.
%
It is the \emph{destination-passing style} version of \ocaml|map|: it writes its result to the tail field of destination \ocaml|dst| instead of returning it.
%
On the empty list, it just writes the empty list.
%
On non-empty lists, it allocates a new partially initialized list cell \ocaml|dst'|, writes \ocaml|dst'| to \ocaml|dst| and calls itself recursively with \ocaml|dst|' as its new destination.
%
\ocaml|map| calls \ocaml|map_dps| on non-empty lists.

What happens is \ocaml|map_dps| constructs the final transformed list by modifying the last cell of the current partial transformed list.
%
Not only is it stack-safe but it also does only one traversal, as opposed the previous stack-safe implementation.
%
Unfortunately, it cannot be implemented using only safe \OCaml features, as list cells are normally immutable.
%
In fact, most strongly typed languages reject it.
%
A notable exception is \Mezzo \cite{DBLP:journals/toplas/BalabonskiPP16}, which is based on a permission system inspired by \emph{separation logic}~\cite{DBLP:journals/cacm/OHearn19}.

We can do better.
%
If we cannot write it, the compiler may be able to do so.
%
In fact, the manual transformation of the natural implementation of \ocaml|map| we have performed is an instance of a more general transformation known as \emph{tail call optimization modulo constructor} (TMC)~\cite{risch-73,friedman-wise-75}.
%
TMC has been implemented in Lisp~[TODO], Prolog~[TODO] and Koka~\cite{DBLP:journals/pacmpl/LeijenL23}.
%
The first main contribution of this work is an implementation of TMC in the \OCaml compiler as an optional optimization.
%
Various functions in the standard library and third-party codebases have been rewritten to use it.
%
For instance, the efficient stack-safe version of \ocaml|map| is obtained just by annotating the natural definition with the attribute \ocaml|[@tail_mod_cons]|:

\begin{Ocaml}
let[@tail_mod_cons] rec map f xs =
  match xs with
  | [] -> []
  | x :: xs -> f x :: map f xs
\end{Ocaml}

The second main contribution of this work is a mechanized proof of correctness for the core of this transformation on a small untyped calculus.
%
We established a \emph{behavioral refinement} between the source program and the transformed program: any behavior of the transformed program, be it converging, diverging or stuck, is a behavior of the source program.
%
Our proof relies on \Simuliris~\cite{DBLP:journals/pacmpl/GaherSSJDKKD22}, a separation logic framework built on top of the \Iris base logic.
%
We found that their proof technique, a simulation relation, is not expressive enough to reason about program transformations that introduce new function calling conventions.
%
We had to generalize the \Simuliris handling of function calls by parameterizing the simulation relation over an abstract protocol inspired by the work of \citeauthor{DBLP:journals/pacmpl/VilhenaP21}~\cite{DBLP:journals/pacmpl/VilhenaP21}.
%
This contribution of our work is independent from the TMC transformation and our small calculus.
%
It could be reused in future work on verifying compiler transformations.

Interestingly, the core of the soundness proof is the specification of the two variants of each TMC-transformed function --- direct style and destination-passing style.
%
It conveys the essence of destination-passing style: computing the same thing and writing it to an owned destination.
%
For instance, delaying the definition of the simulation relation $\iSimv{\Phi}{e_s}{e_t}$ and the similarity relation $\datalangVal_s \iSimilar \datalangVal_t$, the specification of the variants of \ocaml|map| is:
%
\begin{align*}
        \iSimvHoare{
            \iTrue
        }{
            \iSimilar
        }{
            & \datalangCall{\datalangFnptr{\ocamlText{map}}}{\datalangVal_s}
        }{
            \datalangCall{\datalangFnptr{\ocamlText{map}}}{\datalangVal_t}
        }
    \\
        \iSimvHoare{
            (\datalangLoc + \datalangIdx) \iPointsto \datalangHole
        }{
            \datalangVal_s, \datalangUnit \ldotp
            \exists \datalangVal_t \ldotp
            (\datalangLoc + \datalangIdx) \iPointsto \datalangVal_t \iSep
            \datalangVal_s \iSimilar \datalangVal_t
        }{
            & \datalangCall{\datalangFnptr{\ocamlText{map}}}{\datalangVal_s}
        }{
            \datalangCall{\datalangFnptr{\ocamlText{map_dps}}}{\datalangPair{\datalangPair{\datalangLoc}{\datalangIdx}}{\datalangVal_t}}
        }
\end{align*}

To sum up, our main contributions are:
\begin{enumerate}
    \item an implementation of the TMC transformation in the \OCaml compiler;
    \item a mechanized proof of soundness for a simplified transformation on a small calculus;
    \item a generalization of \Simuliris handling of function calls to reason about different calling conventions.
\end{enumerate}

%%% Local Variables:
%%% mode: latex
%%% TeX-master: "main"
%%% End:


\separate{\clearpage}{}
\section{TMC on an idealized language}
\label{sec:formalization}

In this section, we formalize the ``tail modulo cons'' (TMC) transformation in an idealized language, \DataLang, that is expressive enough to account for the main aspects of TMC but does not support all features of \OCaml.

This serves as a precise presentation of the transformation, and it is those definitions for which we formally prove correctness. We intentionally kept this presentation very close to our Coq development, which can be referred to for full details.

\begin{figure}[tp]
    \begin{tabular}{rclclcl}
            $\nterm{Index}$
            & $\ni$ &
            i
            & $\Coloneqq$ &
            $\lambdZero \mid \lambdOne \mid \lambdTwo$
            &&
            index
        \\
            $\nterm{Tag}$
            & $\ni$ &
            $t$
            &&
            &&
            tag
        \\
            $\mathbb{B}$
            & $\ni$ &
            $b$
            &&
            &&
            boolean
    	\\
    		$\mathbb{L}$
    		& $\ni$ &
    		$\loc$
    		&&
    		&&
    		location
        \\
            $\mathbb{F}$
            & $\ni$ &
            $f$
            &&
            &&
            function
        \\
            $\mathbb{X}$
            & $\ni$ &
            $x, y, z$
            &&
            &&
            variable
    	\\
            $\nterm{Val}$
            & $\ni$ &
            $v$
            & $\Coloneqq$ &
            $\lambdUnit \mid i \mid t \mid b \mid \ell \mid \lambdFunc{f}$
            &&
            value
        \\
            $\nterm{Expr}$
            & $\ni$ &
            $e$
            & $\Coloneqq$ &
            $v$
            &&
            expression
        \\
            &&
            & | &
            $x \mid \lambdLet{x}{e_1}{e_2} \mid \lambdCall{e_1}{e_2}$
        \\
            &&
            & | &
            $\lambdEq{e_1}{e_2}$
        \\
            &&
            & | &
            $\lambdIf{e_0}{e_1}{e_2}$
        \\
            &&
            & | &
            $\lambdConstr{t}{e_1}{e_2} \mid \lambdConstrDet{t}{e_1}{e_2}$
        \\
            &&
            & | &
            $\lambdLoad{e_1}{e_2}$
        \\
            &&
            & | &
            $\lambdStore{e_1}{e_2}{e_3}$
        \\
            $\nterm{Ectx}$
            & $\ni$ &
            $K$
            & $\Coloneqq$ &
            $\Box$
            &&
            evaluation context
        \\
            &&
            & | &
            $\lambdLet{x}{K}{e_2} \mid \lambdCall{e_1}{K} \mid \lambdCall{K}{v_2}$
        \\
            &&
            & | &
            $\lambdEq{e_1}{K} \mid \lambdEq{K}{v_2}$
        \\
            &&
            & | &
            $\lambdIf{K}{e_1}{e_2}$
        \\
            &&
            & | &
            $\lambdLoad{e_1}{K} \mid \lambdLoad{K}{v_2}$
        \\
            &&
            & | &
            $\lambdStore{e_1}{e_2}{K} \mid \lambdStore{e_1}{K}{v_3} \mid \lambdStore{K}{v_2}{v_3}$
        \\
            $\nterm{Def}$
            & $\ni$ &
            $d$
            & $\Coloneqq$ &
            $\lambdDef{x}{e}$
            &&
            definition
        \\
            $\nterm{Prog}$
            & $\ni$ &
            $p$
            & $\coloneqq$ &
            $\mathbb{F} \finmap \nterm{Def}$
            &&
            program
        \\
            $\nterm{State}$
            & $\ni$ &
            $\sigma$    
            & $\coloneqq$ &
            $\mathbb{L} \finmap \nterm{Val}$
            &&
            state
        \\
            $\nterm{Config}$
            & $\ni$ &
            $\rho$
            & $\coloneqq$ &
            $\nterm{Expr} \times \nterm{State}$
            &&
            configuration
        \\
            $\nterm{Bisubst}$
            & $\ni$ &
            $\Gamma$
            & $\coloneqq$ &
            $\mathbb{X} \rightarrow \nterm{Expr} \times \nterm{Expr}$
            &&
            bisubstitution
    \end{tabular}
    \caption{Syntax of \LambdaLang}
    \label{fig:syntax}
\end{figure}
\begin{figure}[tp]
    \begin{tabular}{rcl}
            $\datalangSeq{\datalangExpr_1}{\datalangExpr_2}$
            & $\coloneqq$ &
            $\datalangLet{\datalangVar}{\datalangExpr_1}{\datalangExpr_2}$
        \\
            $\datalangPair{\datalangExpr_1}{\datalangExpr_2}$
            & $\coloneqq$ &
            $\datalangBlock{\mathrm{PAIR}}{\datalangExpr_1}{\datalangExpr_2}$
        \\
            $\datalangLet{\datalangPair{\datalangVar_1}{\datalangVar_2}}{\datalangExpr_1}{\datalangExpr_2}$
            & $\coloneqq$ &
            $\datalangLet{\datalangVarTwo}{\datalangExpr_1}{$
        \\
            &&
            $\datalangLet{\datalangVar_1}{\datalangLoad{\datalangVarTwo}{1}}{$
        \\
            &&
            $\datalangLet{\datalangVar_2}{\datalangLoad{\datalangVarTwo}{2}}{$
        \\
            &&
            $\datalangExpr_2}}}$
        \\
            $\datalangRec{\datalangPair{\datalangVar_1}{\datalangVar_2}}{\datalangExpr}$
            & $\coloneqq$ &
            $\datalangRec{\datalangVarTwo}{
            \datalangLet{\datalangPair{\datalangVar_1}{\datalangVar_2}}{\datalangVarTwo}{\datalangExpr}}$
        \\
            $\datalangNil$
            & $\coloneqq$ &
            $\datalangUnit$
        \\
            $\datalangCons{\datalangExpr_1}{\datalangExpr_2}$
            & $\coloneqq$ &
            $\datalangBlock{\mathrm{CONS}}{\datalangExpr_1}{\datalangExpr_2}$
        \\
            $\datalangMatchOneline[]{\datalangExpr_0}{\datalangExpr_1}{\datalangVar}{\mathit{\datalangVar s}}{\datalangExpr_2}$
            & $\coloneqq$ &
            $\datalangLet{\datalangVarTwo}{\datalangExpr_0}{$
        \\
            &&
            $\datalangIf{\datalangEq{\datalangVarTwo}{\datalangNil}$
        \\
            &&
            $}{\datalangExpr_1$
        \\
            &&
            $}{\datalangLet{\datalangPair{\datalangVar}{\mathit{\datalangVar s}}}{\datalangVarTwo}{\datalangExpr_2}}}$
        \\
            $\datalangHole$
            & $\coloneqq$ &
            $\datalangUnit$
    \end{tabular}
    \caption{\DataLang syntactic sugar (omitting freshness conditions)}
    \label{fig:sugar}
\end{figure}
\begin{figure}[tp]
    \begin{mathparpagebreakable}
        \inferrule*
            {}{
                \boxed{- \headStep{\datalangProg} - : \datalangConfig[] \rightarrow \datalangConfig[] \rightarrow \Prop}
            }
        \and
        \inferrule*
            {}{
                \boxed{- \step{\datalangProg} - : \datalangConfig[] \rightarrow \datalangConfig[] \rightarrow \Prop}
            }
        \\
        \inferrule*[lab=StepLet]
            {}{
                \left(
                    \datalangLet{\datalangVar}{\datalangVal}{\datalangExpr},
                    \datalangState
                \right)
                \headStep{\datalangProg}
                \left(
                    \datalangExpr{} [\datalangVar \backslash \datalangVal],
                    \datalangState
                \right)
            }
        \and
        \inferrule*[lab=StepCall]
            {
                \datalangProg{} [\datalangFn] = (\datalangRec{\datalangVar}{\datalangExpr})
            }{
                \left(
                    \datalangCall{\datalangFnptr{\datalangFn}}{\datalangVal},
                    \datalangState
                \right)
                \headStep{\datalangProg}
                \left(
                    \datalangExpr{} [\datalangVar \backslash \datalangVal],
                    \datalangState
                \right)
            }
        \\
        \inferrule*[lab=StepBlock1]
            {}{
                \left(
                    \datalangBlock{\datalangTag}{\datalangExpr_1}{\datalangExpr_2},
                    \datalangState
                \right)
                \headStep{\datalangProg}
                \left(
                    \begin{array}{l}
                        \datalangLet{\datalangVar_1}{\datalangExpr_1}{\\
                        \datalangLet{\datalangVar_2}{\datalangExpr_2}{\\
                        \datalangBlockDet{\datalangTag}{\datalangVar_1}{\datalangVar_2}}}
                    \end{array},
                    \datalangState
                \right)
            }
        \and
        \inferrule*[lab=StepBlock2]
            {}{
                \left(
                    \datalangBlock{\datalangTag}{\datalangExpr_1}{\datalangExpr_2},
                    \datalangState
                \right)
                \headStep{\datalangProg}
                \left(
                    \begin{array}{l}
                        \datalangLet{\datalangVar_2}{\datalangExpr_2}{\\
                        \datalangLet{\datalangVar_1}{\datalangExpr_1}{\\
                        \datalangBlockDet{\datalangTag}{\datalangVar_1}{\datalangVar_2}}}
                    \end{array},
                    \datalangState
                \right)
            }
        \\
        \inferrule*[lab=StepBlockDet]
            {
                \forall \datalangIdx \in \datalangIdx[], \datalangLoc + \datalangIdx \notin \dom{\datalangState}
            }{
                \left(
                    \datalangBlockDet{\datalangTag}{\datalangVal_1}{\datalangVal_2},
                    \datalangState
                \right)
                \headStep{\datalangProg}
                \left(
                    \datalangLoc,
                    \datalangState{} [\datalangLoc \mapsto \datalangTag, \datalangVal_1, \datalangVal_2]
                \right)
            }
        \and
        \inferrule*[lab=StepLoad]
            {
                \datalangState{} [\datalangLoc] = \datalangVal
            }{
                \left(
                    \datalangLoad{\datalangLoc}{\datalangIdx},
                    \datalangState
                \right)
                \headStep{\datalangProg}
                \left(
                    \datalangVal,
                    \datalangState
                \right)
            }
        \and
        \inferrule*[lab=StepStore]
            {
                \datalangLoc + \datalangIdx \in \dom{\datalangState}
            }{
                \left(
                    \datalangStore{\datalangLoc}{\datalangIdx}{\datalangVal},
                    \datalangState
                \right)
                \headStep{\datalangProg}
                \left(
                    \datalangUnit,
                    \datalangState{} [\datalangLoc + \datalangIdx \mapsto \datalangVal]
                \right)
            }
        \and
        \inferrule*[lab=StepEctx]
            {
                \left(
                    \datalangExpr,
                    \datalangState
                \right)
                \headStep{\datalangProg}
                \left(
                    \datalangExpr',
                    \datalangState'
                \right)
            }{
                \left(
                    \datalangEctx{} [\datalangExpr],
                    \datalangState
                \right)
                \step{\datalangProg}
                \left(
                    \datalangEctx{} [\datalangExpr'],
                    \datalangState'
                \right)
            }
    \end{mathparpagebreakable}
    \caption{\DataLang semantics (excerpt)}
    \label{fig:semantics}
\end{figure}
\begin{figure}[tp]
\begin{tabular}{c}
\begin{Datalang}
map |-> rec λ(f, xs) = match xs with
                       | [] -> []
                       | x :: xs -> let y = f x in y :: @map (f, xs)
\end{Datalang}
\end{tabular}
\caption{Natural implementation of \datalang|map| in \DataLang}
\label{fig:map}
\end{figure}

\subsection{Language}

The syntax of \DataLang is given in \cref{fig:syntax} and its semantics in \cref{fig:semantics}.
We also introduce syntactic sugar in \cref{fig:sugar}, including some for lists.
Going back to our motivating example, we can define the \datalang|map| function on lists as in \cref{fig:map}.

\DataLang is an untyped sequential calculus with mutable state. A \DataLang program $\datalangProg$ is a finite mapping from function names $\datalangFn \in \datalangFn[]$ to definitions $d$, which are themselves (mutually-)recursive functions whose body is written $\datalangRec \datalangVar \datalangExpr$. Functions have a single parameter for simplicity, with pairs used to pass several values.

\DataLang has booleans $b \in \{\datalangTrue, \datalangFalse\}$, an if-then-else construct, and a runtime equality test between (untyped) values $\datalangEq{\datalangExpr_1}{\datalangExpr_2}$.

\DataLang does not feature general lambda expressions -- its programs correspond to closure-converted or lambda-lifted source programs. Functions names $\datalangFn$ can be turned into values written $\datalangFnptr{\datalangFn}$, to be used directly in function calls or as parameters to higher-order functions.

To express constructors, \DataLang features mutable memory blocks with an abstract \emph{tag} ($\datalangTag \in \datalangTag[]$), and two \emph{fields} which are arbitrary values ($\datalangExpr_1, \datalangExpr_2$).
One can allocate a block with $\datalangBlock{\datalangTag}{\datalangExpr_1}{\datalangExpr_2}$, access its fields with $\datalangLoad{\datalangExpr_1}{\datalangExpr_2}$ and modify them with $\datalangStore{\datalangExpr_1}{\datalangExpr_2}{\datalangExpr_3}$.
Allocation returns a location $\datalangLoc \in \datalangLoc[]$, which may not appear in source programs.

To avoid over-simplifying, the evaluation order of subexpressions $\datalangExpr_1$ and $\datalangExpr_2$ in $\datalangBlock{\datalangTag}{\datalangExpr_1}{\datalangExpr_2}$ is unspecified as in \OCaml.
To model this in the semantics, we introduce a separate, deterministic block construction $\datalangBlockDet{\datalangTag}{\datalangExpr_1}{\datalangExpr_2}$ construction which cannot appear in source programs. A block expression $\datalangBlock{\datalangTag}{\datalangExpr_1}{\datalangExpr_2}$ first nondeterministically reduces to either $\datalangLet{\datalangVar_1}{\datalangExpr_1}{\datalangLet{\datalangVar_2}{\datalangExpr_2}{\datalangBlockDet{\datalangTag}{\datalangVar_1}{\datalangVar_2}}}$ through \RefTirName{StepBlock1} or $\datalangLet{\datalangVar_2}{\datalangExpr_2}{\datalangLet{\datalangVar_1}{\datalangExpr_1}{\datalangBlockDet{\datalangTag}{\datalangVar_1}{\datalangVar_2}}}$ through \RefTirName{StepBlock2}. $\datalangBlockDet{\datalangTag}{\datalangVal_1}{\datalangVal_2}$ performs the allocation through \RefTirName{StepBlockDet}.

The values in \DataLang are functions $\datalangFnptr{\datalangFn}$, locations $\datalangLoc$ of allocated blocks, booleans $b$, tags $t$ (taken in an arbitrary, denumerable set), the unit value $\datalangUnit$, and indices $i \in \{\datalangZero, \datalangOne, \datalangTwo\}$ inside blocks. (Our transformation never mutates tags in position $\datalangZero$, nor do \OCaml programs, but for example \Mezzo supports it.)

On top of these basic language features, \cref{fig:sugar} introduces syntactic sugar for pairs $\datalangPair {\datalangExpr_1} {\datalangExpr_2}$ as blocks with a specific tag $\mathrm{PAIR}$, for decomposing blocks in $\term{\color{datalangKeyword1}{let}}$-bindings ($\datalangLet {\datalangPair \datalangVar \datalangVarTwo} {\datalangExpr_1} {\datalangExpr_2}$) and in arguments of toplevel functions ($\datalangFn \mapsto \datalangRec {\datalangPair \datalangVar \datalangVarTwo} \datalangExpr$), and for (untyped) lists by defining the empty list $\datalangNil$ as $\datalangUnit$, and for (mutable) cons-cells as blocks with a specific tag $\mathrm{CONS}$. Pattern-matching on lists can be expressed by comparing the list with $\datalangNil$, using our block-deconstructing $\term{\color{datalangKeyword1}{let}}$ (which ignores the tag) to deconstruct cons-cells.

As a side note, we use named expression variables here but the \Coq mechanization actually adopts de Bruijn syntax better suited to define transformations involving binders.
More precisely, our formalisation relies on the \Autosubst library~\cite{autosubst-2015}.
Our definitions respect $\alpha$-equivalence on term variables $\datalangVar, \datalangVarTwo$: we implicitly assume any term variable in bound position to be chosen distinct from all other variables in context.
Function names $\datalangFn$ are not $\alpha$-renamed, as the transformation relates names in the source and target of the transformation.

\subsection{Transformation}
\label{subsec:transformation}

We now define the TMC transformation as a relation $\datalangProg_s \tmc \datalangProg_t$ between programs and their transformation. The relation is total, in the sense that any \DataLang program $\datalangProg_s$ can be related to at least one transformed program $\datalangProg_t$, but it not deterministic, for each input program it captures a (finite) set of admissible transformations, which we all prove valid. This non-determinism captures several choices that have to be done by the user through a user interface to control the transformation, or by the compiler implementation, influencing performance and evaluation order of the result. In this section, we do not describe how these choices are resolved -- there is a large design space. We present the choices we made for \OCaml compiler in \cref{sec:implementation}.

As formalized in \cref{fig:tmc}, transforming a \DataLang program $\datalangProg$ consists in:

\begin{figure}[tp]
    \[
        p_s \tmc p_t
        \coloneqq
        \exists \xi \ldotp
        \bigwedge \left[ \begin{array}{l}
                \dom{\xi} \subseteq \dom{p_s}
            \\
                \dom{p_t} = \dom{p_s} \cup \codom{\xi}
            \\
                \forall f, d_s \ldotp
                p_s [f] = d_s \implies
                \exists d_t \ldotp
                p_t [f] = d_t \wedge d_s \tmcDir{\xi} d_t
            \\
                \forall f, f_\mathit{dps}, d_s \ldotp
                p_s [f] = d_s \wedge \xi [f] = f_{\mathit{dps}} \implies
                \exists d_t \ldotp
                p_t [f_\mathit{dps}] = d_t \wedge d_s \tmcDps{\xi} d_t
        \end{array} \right.
    \]
    \caption{TMC transformation}
    \label{fig:tmc}
\end{figure}
\paragraph{1. Choosing a subset of toplevel functions to be TMC-transformed}
In \OCaml, the programmer has to annotate these.
For each such function $\datalangFn$, we also require a fresh function name $\datalangRenaming [\datalangFn]$ (that is not defined in $\datalangProg_s$) that will be defined to the \emph{destination-passing style} (DPS) version of $\datalangFn$ in the transformed program $\datalangProg_t$.

Formally, the subset is determined by the domain of the renaming function $\datalangRenaming$, which is passed as a parameter to the auxiliary transformations we describe next.

\begin{figure}[tp]
    \begin{mathparpagebreakable}
        \inferrule*
            {}{
                \boxed{- \tmcDir{\datalangRenaming} - : \datalangDef[] \rightarrow \datalangDef[] \rightarrow \Prop}
            }
        \and
        \inferrule*
            {}{
                \boxed{- \tmcDir{\datalangRenaming} - : \datalangExpr[] \rightarrow \datalangExpr[] \rightarrow \Prop}
            }
        \\
        \inferrule*[lab=DirDef]
            {
                \datalangExpr_s \tmcDir{\datalangRenaming} \datalangExpr_t
            }{
                \datalangRec{\datalangVar}{\datalangExpr_s}
                \tmcDir{\datalangRenaming}
                \datalangRec{\datalangVar}{\datalangExpr_t}
            }
        \\
        \inferrule*[lab=DirVal]
            {}{
                \datalangVal
                \tmcDir{\datalangRenaming}
                \datalangVal
            }
        \and
        \inferrule*[lab=DirVar]
            {}{
                \datalangVar
                \tmcDir{\datalangRenaming}
                \datalangVar
            }
        \and
        \inferrule*[lab=DirLet]
            {
                \datalangExpr_{s1} \tmcDir{\datalangRenaming} \datalangExpr_{t1}
            \and
                \datalangExpr_{s2} \tmcDir{\datalangRenaming} \datalangExpr_{t2}
            }{
                \datalangLet{\datalangVar}{\datalangExpr_{s1}}{\datalangExpr_{s2}}
                \tmcDir{\datalangRenaming}
                \datalangLet{\datalangVar}{\datalangExpr_{t1}}{\datalangExpr_{t2}}
            }
        \\
        \inferrule*[lab=DirCall]
            {
                \datalangExpr_{s1} \tmcDir{\datalangRenaming} \datalangExpr_{t1}
            \and
                \datalangExpr_{s2} \tmcDir{\datalangRenaming} \datalangExpr_{t2}
            }{
                \datalangCall {\datalangExpr_{s1}} {\datalangExpr_{s2}}
                \tmcDir{\datalangRenaming}
                \datalangCall {\datalangExpr_{t1}} {\datalangExpr_{t2}}
            }
        \and
        \inferrule*[lab=DirBlock]
            {
                \datalangExpr_{s1} \tmcDir{\datalangRenaming} \datalangExpr_{t1}
            \and
                \datalangExpr_{s2} \tmcDir{\datalangRenaming} \datalangExpr_{t2}
            }{
                \datalangBlock \datalangTag {\datalangExpr_{s1}} {\datalangExpr_{s2}}
                \tmcDir{\datalangRenaming}
                \datalangBlock \datalangTag {\datalangExpr_{t1}} {\datalangExpr_{t2}}
            }
        \\
        \inferrule*[lab=DirBlockDPS1]
            {
                (\datalangVar, \datalangOne, \datalangExpr_{s1}) \tmcDps{\datalangRenaming} \datalangExpr_{t1}
            \and
                \datalangExpr_{s2} \tmcDir{\datalangRenaming} \datalangExpr_{t2}
            }{
                \datalangBlock{\datalangTag}{\datalangExpr_{s1}}{\datalangExpr_{s2}}
                \tmcDir{\datalangRenaming}
                \begin{array}{l}
                    \datalangLet{\datalangVar}{\datalangBlock{\datalangTag}{\datalangHole}{\datalangExpr_{t2}}}{\\
                    \datalangSeq{\datalangExpr_{t1}}{\datalangVar}}
                \end{array}
            }
        \and
        \inferrule*[lab=DirBlockDPS2]
            {
                \datalangExpr_{s1} \tmcDir{\datalangRenaming} \datalangExpr_{t1}
            \and
                (\datalangVar, \datalangTwo, \datalangExpr_{s2}) \tmcDps{\datalangRenaming} \datalangExpr_{t2}
            }{
                \datalangBlock{\datalangTag}{\datalangExpr_{s1}}{\datalangExpr_{s2}}
                \tmcDir{\datalangRenaming}
                \begin{array}{l}
                    \datalangLet{\datalangVar}{\datalangBlock{\datalangTag}{\datalangExpr_{t1}}{\datalangHole}}{\\
                  \datalangSeq{\datalangExpr_{t2}}{\datalangVar}}
                \end{array}
            }
        \and
        \inferrule*[lab=DirBlockDPS12]
            {
                (\datalangVar, \datalangOne, \datalangExpr_{s1}) \tmcDps{\datalangRenaming} \datalangExpr_{t1}
            \and
                (\datalangVar, \datalangTwo, \datalangExpr_{s2}) \tmcDps{\datalangRenaming} \datalangExpr_{t2}
            }{
                \datalangBlock{\datalangTag}{\datalangExpr_{s1}}{\datalangExpr_{s2}}
                \tmcDir{\datalangRenaming}
                \begin{array}{l}
                    \datalangLet{\datalangVar}{\datalangBlock{\datalangTag}{\datalangHole}{\datalangHole}}{\\
                    \datalangSeq{\datalangExpr_{t1}}{
                      \datalangSeq{\datalangExpr_{t2}}{
                        \datalangVar}}}
                \end{array}
            }
    \end{mathparpagebreakable}
    \caption{Direct TMC transformation (omitting congruence rules similar to \TirName{DirCall})}
    \label{fig:tmc_dir}
\end{figure}

%%% Local Variables:
%%% mode: latex
%%% TeX-master: "../main"
%%% End:

\paragraph{2. For each function $\datalangFn$ defined in $\datalangProg$, computing its direct transform.}
We introduce in \cref{fig:tmc_dir} the relations $\datalangDef_s \tmcDir{\datalangRenaming} \datalangDef_t$ for definitions and $\datalangExpr_s \tmcDir{\datalangRenaming} \datalangDef_t$ for expressions.

$\datalangDef_s \tmcDir{\datalangRenaming} \datalangDef_t$ expresses that:
1)~$\datalangDef_t$ has the same calling convention as $\datalangDef_s$.
2)~The body of $\datalangDef_t$ is the direct transform of the body of $\datalangDef_s$.

$\datalangExpr_s \tmcDir{\datalangRenaming} \datalangDef_t$ expresses that $\datalangExpr_t$ is the direct transform of $\datalangExpr_s$.
Intuitively, this means $\datalangExpr_t$ computes the same thing as $\datalangExpr_s$.

The direct-style transform corresponds to the case where we do not have a block available to serve as a destination: this version is used in an arbitrary calling context, not necessarily under a constructor. Most rules are straightforward congruences -- we recursively transform subexpressions and preserve the term constructor. We omit the rules for loads $\datalangLoad{\datalangExpr_1}{\datalangExpr_2}$, stores $\datalangStore{\datalangExpr_1}{\datalangExpr_2}{\datalangExpr_3}$, and the deterministic blocks $\datalangBlockDet \datalangTag {\datalangExpr_1} {\datalangExpr_2}$ which are such simple congruences, just like calls $\datalangCall {\datalangExpr_1} {\datalangExpr_2}$.

The key cases are for a block construct $\datalangBlock \datalangTag {\datalangExpr_1} {\datalangExpr_2}$. We can use this block as a destination, and switch to the destination-passing-style calling convention -- these rules are the only places in the direct-style transformation where destination-passing-style is introduced. The rules (\RefTirName{DirBlock}, \RefTirName{DirBlockDPS1}, \RefTirName{DirBlockDPS2}, \RefTirName{DirBlockDPS12}, \RefTirName{DirBlockDPS21}) allow a non-deterministic choice of a subset of the block arguments to be evaluated in destination-passing style, in some order. The only difference between the \TirName{DPS12} and \TirName{DPS21} versions are the order in which the subterms are computed.

In details, the terms produced by these proceed as follows:
1)~They partially initialize a new memory block, with a hole for some of the arguments.
2)~They evaluate the DPS transformation of the corresponding subterms, in some order,
   passing the uninitialized field as destination.
3)~They return the now fully initialized block.

An implementation would typically determine which subexpression would benefit from destination-passing style, that is, contains function calls $\datalangCall {\datalangFnptr \datalangFn} \datalangExpr$ in tail position (relatively to the subexpression) that have a destination-passing variant $\datalangRenaming [\datalangFn]$.


\begin{figure}[tp]
    \begin{mathparpagebreakable}
        \inferrule*
            {}{
                \boxed{- \tmcDps{\datalangRenaming} - : \datalangDef[] \rightarrow \datalangDef[] \rightarrow \Prop}
            }
        \and
        \inferrule*
            {}{
                \boxed{- \tmcDps{\datalangRenaming} - : \datalangExpr[] \times \datalangExpr[] \times \datalangExpr[] \rightarrow \datalangExpr[] \rightarrow \Prop}
            }
        \\
        \inferrule*[lab=DPSDef]
            {
                (\datalangVar_\mathit{dst}, \datalangVar_\mathit{idx}, \datalangExpr_s) \tmcDps{\datalangRenaming} \datalangExpr_t
            }{
                \datalangRec{\datalangVar}{\datalangExpr_s}
                \tmcDps{\datalangRenaming}
                \datalangRec{\datalangPair{\datalangPair{\datalangVar_\mathit{dst}}{\datalangVar_\mathit{idx}}}{\datalangVar}}{\datalangExpr_t}
            }
        \\
        \inferrule*[lab=DPSLet]
            {
                \datalangExpr_{s1} \tmcDir{\datalangRenaming} \datalangExpr_{t1}
            \and
                (\datalangExpr_\mathit{dst}, \datalangExpr_\mathit{idx}, \datalangExpr_{s2}) \tmcDps{\datalangRenaming} \datalangExpr_{t2}
            }{
                \left(
                  {\begin{array}{l}
                    \datalangExpr_\mathit{dst},
                    \datalangExpr_\mathit{idx}, \\
                    \datalangLet{\datalangVar}{\datalangExpr_{s1}}{\datalangExpr_{s2}}
                  \end{array}}
                \right)
                \tmcDps{\datalangRenaming}
                \datalangLet{\datalangVar}{\datalangExpr_{t1}}{\datalangExpr_{t2}}
            }
        \quad
        \inferrule*[lab=DPSIf]
            {
                \datalangExpr_{s0} \tmcDir{\datalangRenaming} \datalangExpr_{t0}
            \and
                (\datalangExpr_\mathit{dst}, \datalangExpr_\mathit{idx}, \datalangExpr_{s1}) \tmcDps{\datalangRenaming} \datalangExpr_{s2}
            \and
                (\datalangExpr_\mathit{dst}, \datalangExpr_\mathit{idx}, \datalangExpr_{s2}) \tmcDps{\datalangRenaming} \datalangExpr_{t2}
            }{
                \left(
                  {\begin{array}{l}
                    \datalangExpr_\mathit{dst},
                    \datalangExpr_\mathit{idx}, \\
                    \datalangIf{\datalangExpr_{s0}}{\datalangExpr_{s1}}{\datalangExpr_{s2}}
                  \end{array}}
                \right)
                \tmcDps{\datalangRenaming}
                \datalangIf{\datalangExpr_{t0}}{\datalangExpr_{t1}}{\datalangExpr_{t2}}
            }
        \\
        \inferrule*[lab=DPSBlock1]
            {
                (\datalangVar, \datalangOne, \datalangExpr_{s1}) \tmcDps{\datalangRenaming} \datalangExpr_{t1}
            \and
                \datalangExpr_{s2} \tmcDir{\datalangRenaming} \datalangExpr_{t2}
            }{
                \left(
                  {\begin{array}{l}
                    \datalangExpr_\mathit{dst},
                    \datalangExpr_\mathit{idx}, \\
                    \datalangBlock{\datalangTag}{\datalangExpr_{s1}}{\datalangExpr_{s2}}
                  \end{array}}
                \right)
                \tmcDps{\datalangRenaming}
                \begin{array}{l}
                    \datalangLet{\datalangVar}{\datalangBlock{\datalangTag}{\datalangHole}{\datalangExpr_{t2}}}{\\
                    \datalangSeq{\datalangStore{\datalangExpr_\mathit{dst}}{\datalangExpr_\mathit{idx}}{\datalangVar}}{\datalangExpr_{t1}}}
                \end{array}
            }
        \quad
        \inferrule*[lab=DPSBlock2]
            {
                \datalangExpr_{s1} \tmcDir{\datalangRenaming} \datalangExpr_{t1}
            \and
                (\datalangVar, \datalangTwo, \datalangExpr_{s2}) \tmcDps{\datalangRenaming} \datalangExpr_{t2}
            }{
                \left(
                  {\begin{array}{l}
                    \datalangExpr_\mathit{dst},
                    \datalangExpr_\mathit{idx},
                    \\
                    \datalangBlock{\datalangTag}{\datalangExpr_{s1}}{\datalangExpr_{s2}}
                  \end{array}}
                \right)
                \tmcDps{\datalangRenaming}
                \begin{array}{l}
                    \datalangLet{\datalangVar}{\datalangBlock{\datalangTag}{\datalangExpr_{t1}}{\datalangHole}}{\\
                    \datalangSeq{\datalangStore{\datalangExpr_\mathit{dst}}{\datalangExpr_\mathit{idx}}{\datalangVar}}{\datalangExpr_{t2}}}
                \end{array}
            }
        % \and
        % \inferrule*[lab=DPSBlock21]
        %     {
        %         (\datalangVar, \datalangTwo, \datalangExpr_{s1}) \tmcDps{\datalangRenaming} \datalangExpr_{t1}
        %     \and\datalangExpr_{t1}
        %         (\datalangVar, \datalangTwo, \datalangExpr_{s2}) \tmcDps{\datalangRenaming} \datalangExpr_{t2}
        %     }{
        %         \left(
        %             \datalangExpr_\mathit{dst},
        %             \datalangExpr_\mathit{idx},
        %             \datalangBlock{\datalangTag}{\datalangExpr_{s1}}{\datalangExpr_{s2}}
        %         \right)
        %         \tmcDps{\datalangRenaming}
        %         \begin{array}{l}
        %             \datalangLet{\datalangVar}{\datalangBlock{\datalangTag}{\datalangHole}{\datalangHole}}{\\
        %             \datalangSeq{\datalangStore{\datalangExpr_\mathit{dst}}{\datalangExpr_\mathit{idx}}{\datalangVar}}{\\
        %             \datalangSeq{\datalangExpr_{t2}}{\datalangExpr_{t1}}}}
        %         \end{array}
        %     }
        % \;
        % \inferrule*[lab=DPSBlock12]
        %     {
        %         (\datalangVar, \datalangTwo, \datalangExpr_{s1}) \tmcDps{\datalangRenaming} \datalangExpr_{t1}
        %     \and\datalangExpr_{t1}
        %         (\datalangVar, \datalangTwo, \datalangExpr_{s2}) \tmcDps{\datalangRenaming} \datalangExpr_{t2}
        %     }{
        %         \left(
        %             \datalangExpr_\mathit{dst},
        %             \datalangExpr_\mathit{idx},
        %             \datalangBlock{\datalangTag}{\datalangExpr_{s1}}{\datalangExpr_{s2}}
        %         \right)
        %         \tmcDps{\datalangRenaming}
        %         \begin{array}{l}
        %             \datalangLet{\datalangVar}{\datalangBlock{\datalangTag}{\datalangHole}{\datalangHole}}{\\
        %             \datalangSeq{\datalangStore{\datalangExpr_\mathit{dst}}{\datalangExpr_\mathit{idx}}{\datalangVar}}{\\
        %             \datalangSeq{\datalangExpr_{t1}}{\datalangExpr_{t2}}}}
        %         \end{array}
        %     }
        \\
        \inferrule*[lab=DPSCall]
            {
                \datalangFn \in \dom{\datalangRenaming}
            \and
                \datalangExpr_s \tmcDir{\datalangRenaming} \datalangExpr_t
            }{
                \left(
                    \datalangExpr_\mathit{dst},
                    \datalangExpr_\mathit{idx},
                    \datalangCall{\datalangFnptr{\datalangFn}}{\datalangExpr_s}
                \right)
                \tmcDps{\datalangRenaming}
                \datalangCall{
                    \datalangFnptr{\datalangRenaming [\datalangFn]}
                }{
                    \datalangPair{\datalangPair{\datalangExpr_\mathit{dst}}{\datalangExpr_\mathit{idx}}}{\datalangExpr_t}
                }
            }
        \and
        \inferrule*[lab=DPSBase]
            {
                \datalangExpr_s \tmcDir{\datalangRenaming} \datalangExpr_t
            }{
                \left(
                    \datalangExpr_\mathit{dst},
                    \datalangExpr_\mathit{idx},
                    \datalangExpr_s
                \right)
                \tmcDps{\datalangRenaming}
                \datalangStore{\datalangExpr_\mathit{dst}}{\datalangIdx}{\datalangExpr_t}
            }
    \end{mathparpagebreakable}
    \caption{Destination-passing style TMC transformation of definitions and expressions (in full)}
    \label{fig:tmc_dps}
\end{figure}

%%% Local Variables:
%%% mode: latex
%%% TeX-master: "../main"
%%% End:

\paragraph{3. For each TMC-transformed function $\datalangFn$, choosing a destination-passing-style transform}
We introduce in \cref{fig:tmc_dps} the relations $\datalangDef_s \tmcDps{\datalangRenaming} \datalangDef_t$ for definitions and $(\datalangExpr_\mathit{dst}, \datalangExpr_\mathit{idx}, \datalangExpr_s) \tmcDps{\datalangRenaming} \datalangExpr_t$ for expressions.

$\datalangDef_s \tmcDps{\datalangRenaming} \datalangDef_t$ expresses that:
1)~The function defined in $\datalangDef_t$ has an additional parameter representing the destination where it must write its result; it is a pair of the location of a memory block $\datalangVar_\mathit{dst}$ along with the index $\datalangVar_\mathit{idx}$ of a particular field in this block.
2)~The body of $\datalangDef_t$ is a DPS transform of the body of $\datalangDef_s$ under the given destination.

$(\datalangExpr_\mathit{dst}, \datalangExpr_\mathit{idx}, \datalangExpr_s) \tmcDps{\datalangRenaming} \datalangExpr_t$ expresses that $\datalangExpr_t$ is a DPS transform of $\datalangExpr_s$ under destination $(\datalangExpr_\mathit{dst}, \datalangExpr_\mathit{idx})$.
Intuitively, this means $\datalangExpr_t$ computes the same thing as $\datalangExpr_s$ but writes it into the destination instead of returning it.
We will formalize this intuition in \cref{sec:specification}.

Note that the rule \RefTirName{DPSDef}, which relates the two judgments, uses the expression-level relation $(\mathrel{\tmcDps{\datalangRenaming}})$ with a term variable $\datalangVar_\mathit{idx}$ to represent the index, not just a constant index $\datalangOne$ or $\datalangTwo$ as in block rules. These are the only two sort of expressions used to represent offsets in the transformation.

In the direct-style relation, congruence rules apply the same direct-style transformation to all subexpressions. The congruence-like rule of the DPS relation, for example \RefTirName{DPSLet} and \RefTirName{DPSIf}, are different. They apply the DPS transformation to sub-expressions which are in tail position relative to the expression, and the direct-style transformation to all other subexpressions. (The if-then-else construct can have two different subexpressions in tail position, because at most one of them is evaluated at runtime.)

The rules \RefTirName{DPSBlock1} and \RefTirName{DPSBlock2}
correspond to the rules \RefTirName{DirBlockDPS1} and \RefTirName{DirBlockDPS2} in the direct-style transformation, but the behavior of the transformed code is different. Consider the translation of $
\left(
  \datalangExpr_\mathit{dst},
  \datalangExpr_\mathit{idx},
  \datalangBlock{\datalangTag}{\datalangExpr_{s1}}{\datalangExpr_{s2}}
\right)
$ into $
\datalangLet{\datalangVar}{\datalangBlock{\datalangTag}{\datalangExpr_{t1}}{\datalangHole}}{
  \datalangSeq{\datalangStore{\datalangExpr_\mathit{dst}}{\datalangExpr_\mathit{idx}}{\datalangVar}}{\datalangExpr_{t2}}}
$ by \RefTirName{DirBlockDPS2}. First we create a new destination $\datalangVar$, with a hole in second position. Then, instead of computing the corresponding subterm, we write this new destination $x$ into the \emph{current} destination $({\datalangExpr_\mathit{dst}}, {\datalangExpr_\mathit{idx}})$. Finally we evaluate $\datalangExpr_{t2}$, which is the DPS transform of the subexpression $\datalangExpr_{s2}$, with the destination $(\datalangVar, \datalangTwo)$. Notice that $\datalangExpr_{t2}$ is in tail position relative to the transformed expression, while it was not in tail position (but under a constructor) in the source expression. This is the key step of the TMC transformation. Both rules \RefTirName{DirBlockDPS2} and \RefTirName{DirBlockDPS12} put the second subterm in tail position, and both rules \RefTirName{DirBlockDPS1} and \RefTirName{DirBlockDPS21} put the first subterm in tail position. In the case of lists as in our running example \ocaml|y :: map f xs|, we obviously want to put the second subterm in tail position, and to use \RefTirName{DirBlockDPS2}. But consider a \ocaml|map| function on binary trees \ocaml|Node(map f left, map f right)|, the implementation must choose one subterm to put in tail position and another to keep in non-tail position.

The rule \RefTirName{DPSCall} applies only to calls $\datalangCall {\datalangFnptr \datalangFn} {\datalangExpr_s}$ to a known function $\datalangFn$, on the condition that a DPS variant has been generated for $\datalangFn$: $\datalangFn \in \dom{\datalangRenaming}$. In this case, the function call can be compiled to a call to the DPS variant $\datalangRenaming [\datalangFn]$, transferring the responsibility to write to the destination to the callee. This is the case where the DPS transform is beneficial, as this transformation may turn a non-tail-call into a tail-call -- when it occurs under a block, in a subterm that was moved to tail position. This rule is selected for \ocaml|map| in our \ocaml|y :: map f xs| example.

Finally, there is a catch-all rule \RefTirName{DPSBase} that applies whenever none of the other rules can be selected, and may be non-deterministically applied in any case. This case trivially realizes the DPS calling convention by evaluating the subterm to a result and writing this result in the desired destination. This is what happens in the base case of \ocaml|map_dps|, where the empty list \ocaml|[]| is transformed into \ocaml|dst.(idx) <- []|.

\medskip

\begin{figure}[tp]
\begin{minipage}{.40\columnwidth}
\begin{Datalang}
map |-> rec (fn, xs) =
  match xs with
  | [] ==> 
      []
  | x :: xs ==>
      let y = fn x in
      let dst = y :: () in
      @map_dps ((dst, 2), (fn, xs)) ;
      dst
\end{Datalang}
\end{minipage}
\hfill
\begin{minipage}{.52\columnwidth}
\begin{Datalang}
map_dps |-> rec ((dst, idx), (fn, xs)) =
  match xs with
  | [] ==> 
      dst.(idx) <- []
  | x :: xs ==>
      let y = fn x in
      let dst' = y :: () in
      dst.(idx) <- dst' ;
      @map_dps ((dst', 2), (fn, xs))
\end{Datalang}
\end{minipage}
\caption{Direct and DPS transforms of \datalang|map| (\DataLang)}
\label{fig:map_tmc}
\end{figure}

As an example, we can show that the two \DataLang functions defined in \cref{fig:map_tmc} are respectively the direct and DPS transforms of the original \datalang|map| function defined in \cref{fig:map}.
Thanks to the rules \RefTirName{DirBlockDPS2} and \RefTirName{DPSCall}, the non-tail recursive call is replaced by the allocation of a partially initialized block followed by a call to \datalang|map_dps|.

\Xgabriel{Do we really want to have these 12 and 21 rules? They make the presentation noticeably heavier for very small benefits. But if we don't include them, and a reviewer can think of them, they would suspect that we missed them.}

Notice that it is always possible to replace the rule \RefTirName{DirBlockDPS1} by the rule \RefTirName{DirBlockDPS21}, where the first subterm is transformed to immediately use the rule \RefTirName{Base} -- and similarly for \RefTirName{DPSBlock1}, etc.. In our example, \ocaml|y :: map f xs| is transformed into \ocaml|let dst = y :: ? in map_dps dst 2 f xs; dst|, but we could instead produce \ocaml|let dst = ? :: ? in dst.(1) <- y; map_dps dst 2 f xs; dst|, without changing the stack-consumption behavior of the program. This would however produce larger code, with slightly worse constant factors. The performance of the result of the TMC transformation is very sensitive to such constant factors because \ocaml|map| is a very fast function in the first place, so it is important to produce the simplest, best possible code in common scenarios.

\subsection{Realizing the relation as a function}

Our Coq formalization includes a function that takes an input program and outputs a related program, following the one-pass implementation approach that we introduced in the OCaml compiler (\cref{subsec:implementation}).

%%% Local Variables:
%%% mode: latex
%%% TeX-master: "main"
%%% End:


\separate{\clearpage}{}
\section{\OCaml Implementation}
\label{sec:implementation}

\subsection{Implementation history}

We first proposed adding TMC as an optional program transformation to
the OCaml compiler in
May 2015.\footnote{\nonanon{URL}{\url{https://github.com/ocaml/ocaml/pull/181}}}
The proposal was received favorably, but it never received an in-depth
review and a detailed performance evaluation and remained unmerged for
years.

We re-activated the discussion in July 2020 with a modified
implementation, and a careful performance
evaluation\footnote{\nonanon{URL}{\url{https://github.com/ocaml/ocaml/pull/9760}}}. The
new implementation put a larger focus on producing readable code,
giving more control to the user through annotations. It also removed
an optimization of the previous implementation that would specialize
the DPS version, generating a distinct definition for each block
offset. The new design was carefully reviewed by Basile Clément and
Pierre Chambart, and was finally merged in November 2021, available to
users with OCaml 4.14, released in March 2022.

The TMC transformation is that it adds extra parameters to functions
(two parameters, the block and the offset). At the time the OCaml
compiler had a limitation on some supported architectures, where it
would not optimze tail calls above a certain number of arguments
(enough that they cannot be all passed by registers), breaking the
tail-call promises of TMC on those systems. Xavier Leroy implemented
a change to the OCaml calling convention for those architectures in
May 2021,\footnote{\url{https://github.com/ocaml/ocaml/pull/10595}},
which was motivated by the TMC work.

\subsection{Examples}

Many functions that consume and produce lists are
tail-recursive-modulo-cons, in the sense that all they have a TMC
decomposition where all recursive calls are in TMC position. Notable
functions include \ocaml{map}, as already discussed, but also for
example:

\begin{Ocaml}
let[@tail_mod_cons] rec filter p = function
| [] -> []
| x :: xs -> if p x then x :: filter p xs else filter p xs

let[@tail_mod_cons] rec merge cmp l1 l2 =
  match l1, l2 with
  | [], l | l, [] -> l
  | h1 :: t1, h2 :: t2 ->
      if cmp h1 h2 <= 0
      then h1 :: merge cmp t1 l2
      else h2 :: merge cmp l1 t2
\end{Ocaml}

TMC is not useful only for lists or other ``linear'' data types, with
at most one recursive occurrence of the datatype in each
constructor.

\paragraph{A non-example} Consider a \ocaml{map} function on binary
trees:
\begin{Ocaml}
let[@tail_mod_cons] rec map f = function
| Leaf v -> Leaf (f v)
| Node(t1, t2) -> Node(map f t1, (map[@tailcall]) f t2)
\end{Ocaml}
In this function, there are two recursive calls, but only one of them
can be optimized; we used the \ocaml{[@tailcall]} attribute to direct
our implementation to optimize the call to the right child, as we will
discuss later. This is a \emph{bad} example of TMC usage in most
cases, given that
\begin{itemize}
\item If the tree is arbitrary, there is no reason that it would be
  right-leaning rather than left-leaning. Making only the right-child
  calls tail-calls does not protect us from stack overflows.
\item If the tree is known to be balanced, then in practice the depth
  is probably very small in both directions, so the TMC transformation
  is not necessary to have a well-behaved function.
\end{itemize}

\paragraph{Interesting non-linear examples} There \emph{are} interesting
examples of TMC-transformation on functions operating on tree-like
data structures, when there are natural assumptions about which child
is likely to contain a deep subtree. The OCaml compiler itself
contains a number of them; consider for example the following function
from the \ocaml{Cmm} module, one of its lower-level program
representations:
\begin{Ocaml}
let[@tail_mod_cons] rec map_tail f = function
  | Clet(id, exp, body) ->
      Clet(id, exp, map_tail f body)
  | Cifthenelse(cond, ifso, ifnot) ->
      Cifthenelse(cond, map_tail f ifso, (map_tail[@tailcall]) f ifnot)
  | Csequence(e1, e2) ->
      Csequence(e1, map_tail f e2)
  | Cswitch(e, tbl, el) ->
      Cswitch(e, tbl, Array.map (map_tail f) el)
  [...]
  | Cexit _ | Cop (Craise _, _, _) as cmm ->
      cmm
  | Cconst_int _ | Cvar _ | Ctuple _ | Cop _ as c ->
      f c
\end{Ocaml}

This function is traversing the ``tail'' context of an arbitrary
program term -- a meta-example! The \ocaml{Cifthenelse} node acts as
our binary-node constructor, we do not know which side is likely to be
larger, so TMC is not so interesting. The recursive calls for
\ocaml{Cswitch} are not in TMC position. But on the other hand the
\ocaml{Clet}, \ocaml{Csequence} cases are very beneficial to have in
TMC: while they have several recursive subtrees, they are in practice
only deeply nested in the direction that is turned into a tailcall by
the transformation. The OCaml compiler does sometimes encounter
machine-generated programs with a unusually long sequence of either
constructions, and the TMC transformation may very well avoid a stack
overflow in this case.

Another example would be
\href{https://github.com/ocaml/ocaml/pull/9636}{\#9636}, a patch to
the OCaml compiler proposed in June 2020 by Mark Shinwell, to get
a partially-tail-recursive implementation of the ``Common
Subexpression Elimination'' (CSE) pass through a manual
contiation-passing-style transform. Xavier Leroy remarked that the
existing implementation in fact fits the TMC fragment. Not all
recursive calls become tail-calls (this would require a more powerful
transformation or a longer, less readable patch), but the behavior of
TMC on the unchanged code matches the tail-call-ness proposed in the
human-written patch.

\subsection{Design choices}

\subsubsection{Resolving non-determinism}

The main question faced by an implementation is how to resolve the
inherent non-determinism. We identified fairly different kinds of
non-determinism.

\paragraph{Choices with an obviously better alternative.} Some choices
have an obviously better alternative, because they allow to strictly
improve more programs. For example, some implementations of TMC limit
themselves to calls that are immediately inside a constructor: they
would optimize the recursive call to \ocaml|f| in \ocaml|Cons(x, f y)|
but not, for example, in \ocaml|Cons(x1, Cons(x2, f y))|, or in
\ocaml|Cons (x, if p then f y else z)|.

Our implementation optimizes calls to functions under an arbitrary
nesting of tail-recursive contexts and constructor contexts, which is
the most general approach allowed by our relation. This comes at the
cost of some implementation complexity, but is otherwise an easy
choice to make when deciding on a specification.

\paragraph{Choices with a subtly better alternative.} In some cases it
is possible to put a bit more work in the choice heuristics, to
generate slightly better code, in terms of performance or
readability. For example, some natural way to simplify the
implementation would result in more subterms being transformed in
destination-passing-style, only to finally use the
\RefTirName{DPSBase} rule without any DPS function call. The generated
code is slightly less pleasant to read, and in our experience it can
be noticeably slower.

Code readability is important in our context of an on-demand program
transformation used by performance experts: those experts will often
read the intermediate representations produced by the compiler to
check that their performance assumptions hold.

\paragraph{Incomparable choices: force the user to decide} The
remaining choices are between incomparable alternatives, that could
each be better than the other depending on the specific program or
programmer intent. Consider again our binary tree example,
\ocaml|Node(map f left, map f right)|: we could make either the first
or the second argument a tail-call.

Our policy in such cases is to let the user decide, by providing
control over the transformation choices using \ocaml{[@tailcall]}
program annotations, and to force the user to decide by raising an
error in case of ambiguity:
\begin{Ocaml}
  | Cifthenelse(cond, ifso, ifnot) ->
      Cifthenelse(cond, map_tail f ifso, map_tail f ifnot)
      /[^^^^^^^^^^^^^^^^^^^^^^^^^^^^^^^^^^^^^^^^^^^^^^^^^^^^]/
/[Error]/: "[@tail_mod_cons]": this constructor application may be TMC-transformed
       in several different ways. Please disambiguate by adding an explicit
       "[@tailcall]" attribute to the call that should be made tail-recursive,
       or a "[@tailcall false]" attribute on calls that should not be
       transformed.
\end{Ocaml}

This approach would be incompatible with a view of the TMC
transformation as an implicit optimization, applied whenever
possible -- in that case one would rather have the compiler make
arbitrary choices rather than fail. We rather view TMC as a tool for
expert users to better reason about program performance. It should be
predictable and flexible (let the user express
their intent). Failing on the lack of annotations is an acceptable way
to drive user interaction.

\subsubsection{First-order implementation} A notable limitation of the
OCaml implementation is that it is first-order and non-modular in
nature: only direct calls to known functions can be converted into DPS
style, and the availability of a DPS variant is not exposed through
module abstractions.

It would be possible to allow DPS calls through external modules or
higher-order function parameters, by annotating interfaces and
function arguments with a \ocaml{[@tail_mod_cons]} annotation, and
elaborating this into a pair of function, the direct-style and the DPS
version.

\subsection{Constructor compression}

\subsection{Implementation}

\subsection{Evaluation: benchmarks}

\subsection{Evaluation: adoption}

% [@tail_mod_cons] usage on Github

% Search URL:
%   https://github.com/search?q=tail_mod_cons+lang%3Aocaml&type=code

% Results (November 13th 2023):
%   - in the OCaml stdlib
%     + Melange
%       (reused from stdlib)
%     + janestreet/base
%       (with manual unfolding)
%   - in other general utility modules
%     https://github.com/RedPRL/asai/blob/ac523674579772e5929c8d912d407b8b7db89c74/src/Utils.ml#L33
%     https://github.com/jonathan-laurent/KaTie/blob/2db0d2cf223fae003b26bdf4c82a9788b5804bc7/src/utils.ml#L95
%     https://github.com/jamsidedown/ocaml-cll/blob/394effa65b5d3f883e2de8322ed60fb7aa495bec/examples/crab_cups.ml#L61
%   - in other user code
%     at list type:
%       https://github.com/brendanzab/language-garden/blob/9cc21e2f57228556cf0b867ca2874ceb4a4250ec/compile-arith/lib/StackLang.ml#L63
%       https://github.com/abella-prover/abella/blob/ceee13822001137898ca5de9c76a315da3d1f0e3/src/extensions.ml#L421
%       https://github.com/bikallem/spring/blob/6d10fef0b21caea3f975ff13132f5994e7c37db5/lib_spring/response.ml#L99
%       https://github.com/bikallem/spring/blob/6d10fef0b21caea3f975ff13132f5994e7c37db5/lib_spring/headers.ml#L150
%       https://github.com/polytypic/io/blob/dbe4d37a98ef958178fe18bb4dae396a5537392e/src/io/io.ml#L8
%       https://github.com/avsm/eeww/blob/5b6c0617306f4c9d8b44bf07ef4d45ec9668dd0c/lib/cohttp/cohttp-eio/src/rwer.ml#L43
%       https://github.com/just-max/less-power/blob/96f8e88f72275246b6bc175e44c732aa6e08ce7f/src/common/path_util.ml#L18
%       https://github.com/dalps/ocaml-challenge/blob/8fe115023457b6b4359ef736c78ba980cd871717/3/extract/solve.ml#L4
%       https://github.com/Octachron/ocaml-changelog-analyzer/blob/d6757b9576b5070ea3cfe6d4f4f79aaa3cc6e5d4/parse/release.ml#L4
%       https://github.com/Skyb0rg007/PL-Reading-Group/blob/7002ae35ba69fb9010e5049eee88ab475bc79ec3/list-optimizations/lib/adt.ml#L44
%       https://github.com/jfeser/symetric/blob/f49a57afc90a2c1e38f1097f98bd74725c074b9b/lib/tower.ml#L75
%       https://github.com/verse-lab/sisyphus/blob/a08addae06bbccb1e6ea68bac3d7f9eeea4197be/lib/dynamic/utils.ml#L65
%       https://github.com/jaymody/ocaml-tokenizers/blob/06cbce75d2865690cb39017222efff5c284bc721/lib/utils.ml#L18
%       https://github.com/Bannerets/ocaml-kdl/blob/687c404ebeceef7fd905935c23665294133f99e6/src/lens.ml#L83
%       https://github.com/OCADml/OCADml/blob/f5dfbf55f9c19ad808daa0df4d76a55e58756e5f/lib/poly3.ml#L33
%     other type
%       https://github.com/johnridesabike/acutis/blob/4113fb516d9c5dd6f2dbfd9657f52c5ae7a5dcae/lib/matching.ml#L161
%         (on-demand stream as a record of functions)
%       https://github.com/RedPRL/ocaml-bwd/blob/fbf496b29532085b38073eaa62ba3d22ac619d5d/src/BwdNoLabels.ml#L60
%         (snoc list)
%       https://github.com/Skyb0rg007/PL-Reading-Group/blob/7002ae35ba69fb9010e5049eee88ab475bc79ec3/effects/lib/free/tseq.ml#L44
%         (difference list gadt)

%   - Misuse
%     https://github.com/chengsun/simd-sexp/commit/e714a8205c6df512d920a31a4dd2448975703c55
%       (already tail-recursive)
%     https://github.com/gridbugs/llama/blob/96d1c96c31ff136f465b5f37c981fff591bac6fd/src/midi/byte_array_parser.ml#L50
%       (needless complexity)
%     https://github.com/LimitEpsilon/lambda-interpreter/blob/ab8e81de7f896aa64eba14e23d964b9705e94161/lib/evaluate.ml#L46
%       (already tail-recursive)

\subsection{TMC and OCaml 5}

%%% Local Variables:
%%% mode: latex
%%% TeX-master: "main"
%%% End:

\separate{\clearpage}{}
\section{Specifying TMC}
\label{sec:specification}

In this section, we gradually introduce aspects of our relational separation logic, by introducing our specifications for the direct-style and destination-passing-style transformations of \cref{sec:formalization} in relational separation logic.
% In the next sections, we will unroll the proof of these specifications and extract a soundness theorem that lives outside of separation logic, depending only on the \DataLang semantics.

\subsection{Direct transformation}

Intuitively, the direct transformation $\datalangExpr_s \tmcDir{\datalangRenaming} \datalangExpr_t$ preserves the behaviours of the source expression $\datalangExpr_s$.
Basically, $\datalangExpr_s$ and $\datalangExpr_t$ compute the same thing.
Using \emph{relational Hoare logic}, an assumed extension of standard Hoare logic relating two expressions, we would write:
\[
    \iSimvHoare{
        \datalangExpr_s \tmcDir{\datalangRenaming} \datalangExpr_t
    }{
        \datalangVal_s, \datalangVal_t \ldotp
        \datalangVal_s \iSimilar \datalangVal_t
    }{
        \datalangExpr_s
    }{
        \datalangExpr_t
    }
\]

The informal meaning of this specification is that 1) $\datalangExpr_t$ refines $\datalangExpr_s$ in the sense that any behaviour (converging, diverging or stuck execution) of $\datalangExpr_t$ is also a behaviour of $\datalangExpr_s$ and 2) if $\datalangExpr_t$ converges to value $\datalangVal_t$, $\datalangExpr_s$ also converges to some value $\datalangVal_s$ that is \emph{similar} to $\datalangVal_s$.
We will formalize the notion of \emph{behaviour} in \cref{sec:simulation} and that of \emph{similarity} later in this section.
For the time being, the ready may assume similarity is just equality on values.

\subsection{DPS transformation}

The DPS transformation $(\datalangLoc, \datalangIdx, \datalangExpr_s) \tmcDps{\datalangRenaming} \datalangExpr_t$ is parameterized by a destination $(\datalangLoc, \datalangIdx)$ pointing to an uninitialized field of some block.
Intuitively, $\datalangExpr_t$ computes the same thing as $\datalangExpr_s$ but writes it into the destination instead of returning it.
Using \emph{relational separation logic}, an assumed further extension of relational Hoare logic with the concepts of separation logic, we write:
\[
    \iSimvHoare{
        (\datalangLoc, \datalangIdx, \datalangExpr_s) \tmcDps{\datalangRenaming} \datalangExpr_t \iSep
        (\datalangLoc + \datalangIdx) \iPointsto_t \datalangHole
    }{
        \datalangVal_s, \datalangUnit \ldotp
        \exists \datalangVal_t \ldotp
        (\datalangLoc + \datalangIdx) \iPointsto_t \datalangVal_t \iSep
        \datalangVal_s \iSimilar \datalangVal_t
    }{
        \datalangExpr_s
    }{
        \datalangExpr_t
    }
\]

The \emph{points-to assertion} $(\datalangLoc + \datalangIdx) \iPointsto_t -$ expresses that the destination is uniquely owned by the transformed program.
The specification requires $\datalangExpr_t$ to fill it with some value that is similar to the returned source value.
This captures destination-passing style in simple terms.

\subsection{Heap bijection}

Defining value similarity as just syntactic equality is not sufficient: corresponding source and target block allocations are not done in locksteps, so their location may differ.
For example, consider the \datalang{map} function (\cref{fig:map}) and its DPS transform (\cref{fig:map_tmc}).
In the source program, the cons cell \datalang{y :: @map (fn, xs)} is allocated after the recursive call.
In the transformed program, the corresponding block is allocated before the call.

To deal with this, we introduce a \emph{heap bijection} as in \Simuliris~\citep*{simuliris-2022}.
Corresponding source and target locations --- pointing to block fields in our semantics --- are logically and permanently registered in this bijection. 
We can then refer to a registered pair of locations though the assertion $\datalangLoc_s \iInBij \datalangLoc_t$.
This mechanism is formalized by the \RefTirName{BijInsert} rule:

\begin{mathpar}
    \inferrule*[lab=BijInsert]
        {
            \datalangLoc_s \iPointsto_s \datalangVal_s
        \and
            \datalangLoc_t \iPointsto_t \datalangVal_t
        \and
            \datalangVal_s \iSimilar \datalangVal_t
        }{
            \datalangLoc_s \iInBij \datalangLoc_t
        }
\end{mathpar}

We can formally define value similarity $\datalangVal_s \iSimilar \datalangVal_t$ in \cref{fig:isimilar}.
It coincides with equality except on blocks, for which we require all fields to be registered in the bijection.

\begin{figure}[tp]
    \centering
    \begin{mathparpagebreakable}
        \inferrule*
            {}{
                \lambdUnit \iSimilar \lambdUnit
            }
        \and
        \inferrule*
            {}{
                i \iSimilar i
            }
        \and
        \inferrule*
            {}{
                n \iSimilar n
            }
        \and
        \inferrule*
            {}{
                b \iSimilar b
            }
        \and
        \inferrule*
            {
                \forall i \in \{0,1,2\} \ldotp
                (\loc_s + i) \iInBij (\loc_t + i)
            }{
                \loc_s \iSimilar \loc_t
            }
        \and
        \inferrule*
            {
                f \in \dom{p_s}
            }{
                \lambdFunc{f} \iSimilar \lambdFunc{f}
            }
    \end{mathparpagebreakable}
    \caption{Similarity in $\iProp$}
    \label{fig:isimilar}
\end{figure}

%%% Local Variables:
%%% mode: latex
%%% TeX-master: "main"
%%% End:


\separate{\clearpage}{}
\section{Relational separation logic}
\label{sec:program_logic}

\begin{figure}[tp]
    \begin{mathparpagebreakable}
        \inferrule*[lab=RelPost]
            {
                \iPred (\datalangVal_s, \datalangVal_t)
            }{
                \iSimv[\iProt]{\iPred}{\datalangVal_s}{\datalangVal_t}
            }
        \and
        \inferrule*[lab=RelStuck]
            {
                \mathrm{strongly \mathhyphen stuck}_{\datalangProg_s} (\datalangExpr_s)
            \and
                \mathrm{strongly \mathhyphen stuck}_{\datalangProg_t} (\datalangExpr_t)
            }{
                \iSimv[\iProt]{\iPred}{\datalangExpr_s}{\datalangExpr_t}
            }
        \\
        \inferrule*[lab=RelBind]
            {
                \iSimv[\iProt]{\lambdaAbs (\datalangVal_s, \datalangVal_t) \ldotp \iSimv[\iProt]{\iPred}{\datalangEctx_s [\datalangVal_s]}{\datalangEctx_t [\datalangVal_t]}}{\datalangExpr_s}{\datalangExpr_t}
            }{
                \iSimv[\iProt]{\iPred}{\datalangEctx_s [\datalangExpr_s]}{\datalangEctx_t [\datalangExpr_t]}
            }
        \\
        \inferrule*[lab=RelSrcPure]
            {
                \datalangExpr_s \pureStep{\datalangProg_s} \datalangExpr_s'
            \and
                \iSimv[\iProt]{\iPred}{\datalangExpr_s'}{\datalangExpr_t}
            }{
                \iSimv[\iProt]{\iPred}{\datalangExpr_s}{\datalangExpr_t}
            }
        \and
        \inferrule*[lab=RelTgtPure]
            {
                \datalangExpr_t \pureStep{\datalangProg_t} \datalangExpr_t'
            \and
                \iSimv[\iProt]{\iPred}{\datalangExpr_s}{\datalangExpr_t'}
            }{
                \iSimv[\iProt]{\iPred}{\datalangExpr_s}{\datalangExpr_t}
            }
        \\
        \inferrule*[lab=RelSrcBlock1]
            {
                \iSimv[\iProt]{\iPred}{
                    \begin{array}{l}
                        \datalangLet{\datalangVar_1}{\datalangExpr_{s1}}{\\
                        \datalangLet{\datalangVar_2}{\datalangExpr_{s2}}{\\
                        \datalangBlockDet{\datalangTag}{\datalangVar_1}{\datalangVar_2}}}
                    \end{array}
                }{\datalangExpr_t}
            }{
                \iSimv[\iProt]{\iPred}{\datalangBlock{\datalangTag}{\datalangExpr_{s1}}{\datalangExpr_{s2}}}{\datalangExpr_t}
            }
        \and
        \inferrule*[lab=RelSrcBlock2]
            {
                \iSimv[\iProt]{\iPred}{
                    \begin{array}{l}
                        \datalangLet{\datalangVar_2}{\datalangExpr_{s2}}{\\
                        \datalangLet{\datalangVar_1}{\datalangExpr_{s1}}{\\
                        \datalangBlockDet{\datalangTag}{\datalangVar_1}{\datalangVar_2}}}
                    \end{array}
                }{\datalangExpr_t}
            }{
                \iSimv[\iProt]{\iPred}{\datalangBlock{\datalangTag}{\datalangExpr_{s1}}{\datalangExpr_{s2}}}{\datalangExpr_t}
            }
        \and
        \inferrule*[lab=RelTgtBlock]
            {
                \iSimv[\iProt]{\iPred}{\datalangExpr_s}{
                    \begin{array}{l}
                        \datalangLet{\datalangVar_1}{\datalangExpr_{t1}}{\\
                        \datalangLet{\datalangVar_2}{\datalangExpr_{t2}}{\\
                        \datalangBlockDet{\datalangTag}{\datalangVar_1}{\datalangVar_2}}}
                    \end{array}
                }
            \and
                \iSimv[\iProt]{\iPred}{\datalangExpr_s}{
                    \begin{array}{l}
                        \datalangLet{\datalangVar_2}{\datalangExpr_{t2}}{\\
                        \datalangLet{\datalangVar_1}{\datalangExpr_{t1}}{\\
                        \datalangBlockDet{\datalangTag}{\datalangVar_1}{\datalangVar_2}}}
                    \end{array}
                }
            }{
                \iSimv[\iProt]{\iPred}{\datalangExpr_s}{\datalangBlock{\datalangTag}{\datalangExpr_{t1}}{\datalangExpr_{t2}}}
            }
        \\
        \inferrule*[lab=RelSrcBlockDet]
            {
                \forall \datalangLoc_s \ldotp
                \datalangLoc_s \iPointsto_s (\datalangTag, \datalangVal_{s1}, \datalangVal_{s2}) \iSepImp
                \iSimv[\iProt]{\iPred}{\datalangLoc_s}{\datalangExpr_t}
            }{
                \iSimv[\iProt]{\iPred}{\datalangBlockDet{\datalangTag}{\datalangVal_{s1}}{\datalangVal_{s2}}}{\datalangExpr_t}
            }
        \and
        \inferrule*[lab=RelTgtBlockDet]
            {
                \forall \datalangLoc_t \ldotp
                \datalangLoc_t \iPointsto_t (\datalangTag, \datalangVal_{t1}, \datalangVal_{t2}) \iSepImp
                \iSimv[\iProt]{\iPred}{\datalangExpr_s}{\datalangLoc_t}
            }{
                \iSimv[\iProt]{\iPred}{\datalangExpr_s}{\datalangBlockDet{\datalangTag}{\datalangVal_{t1}}{\datalangVal_{t2}}}
            }
        \\
        \inferrule*[lab=RelSrcLoad]
            {
                (\datalangLoc_s + \datalangIdx) \iPointsto_s \datalangVal_s
            \\\\
                (\datalangLoc_s + \datalangIdx) \iPointsto_s \datalangVal_s \iSepImp
                \iSimv[\iProt]{\iPred}{\datalangVal_s}{\datalangExpr_t}
            }{
                \iSimv[\iProt]{\iPred}{\datalangLoad{\datalangLoc_s}{\datalangIdx}}{\datalangExpr_t}
            }
        \and
        \inferrule*[lab=RelSrcStore]
            {
                (\datalangLoc_s + \datalangIdx) \iPointsto_s \datalangVal_s
            \\\\
                (\datalangLoc_s + \datalangIdx) \iPointsto_s \datalangVal_s' \iSepImp
                \iSimv[\iProt]{\iPred}{\datalangUnit}{\datalangExpr_t}
            }{
                \iSimv[\iProt]{\iPred}{\datalangStore{\datalangLoc_s}{\datalangIdx}{\datalangVal_s'}}{\datalangExpr_t}
            }
        \\
        \inferrule*[lab=RelTgtLoad]
            {
                (\datalangLoc_t + \datalangIdx) \iPointsto_t \datalangVal_t
            \\\\
                (\datalangLoc_s + \datalangIdx) \iPointsto_t \datalangVal_t \iSepImp
                \iSimv[\iProt]{\iPred}{\datalangExpr_s}{\datalangVal_t}
            }{
                \iSimv[\iProt]{\iPred}{\datalangExpr_s}{\datalangLoad{\datalangLoc_t}{\datalangIdx}}
            }
        \and
        \inferrule*[lab=RelTgtStore]
            {
                (\datalangLoc_t + \datalangIdx) \iPointsto_t \datalangVal_t
                \\\\
                (\datalangLoc_t + \datalangIdx) \iPointsto_t \datalangVal_t' \iSepImp
                \iSimv[\iProt]{\iPred}{\datalangExpr_s}{\datalangUnit}
            }{
                \iSimv[\iProt]{\iPred}{\datalangExpr_s}{\datalangStore{\datalangLoc_t}{\datalangIdx}{\datalangVal_t'}}
            }
%        \\
%        \inferrule*[lab=RelBijInsert]
%            {
%                \datalangLoc_s \iPointsto_s \datalangVal_s
%            \and
%                \datalangLoc_t \iPointsto_t \datalangVal_t
%            \and
%                \datalangVal_s \iSimilar \datalangVal_t
%            \and
%                \datalangLoc_s \iInBij \datalangLoc_t \iSepImp
%                \iSimv[\iProt]{\iPred}{\datalangExpr_s}{\datalangExpr_t}
%            }{
%                \iSimv[\iProt]{\iPred}{\datalangExpr_s}{\datalangExpr_t}
%            }
        \\
        \inferrule*[lab=RelLoad]
            {
                \datalangLoc_s \iSimilar \datalangLoc_t
            \and
                \forall \datalangVal_s, \datalangVal_t \ldotp
                \datalangVal_s \iSimilar \datalangVal_t
                \iSepImp
                \iPred (\datalangVal_s, \datalangVal_t)
            }{
                \iSimv[\iProt]{\iPred}{\datalangLoad{\datalangLoc_s}{\datalangIdx}}{\datalangLoad{\datalangLoc_t}{\datalangIdx}}
            }
        \and
        \inferrule*[lab=RelStore]
            {
                \datalangLoc_s \iSimilar \datalangLoc_t
            \and
                \datalangVal_s \iSimilar \datalangVal_t
            \and
                \iPred (\datalangUnit, \datalangUnit)
            }{
                \iSimv[\iProt]{\iPred}{\datalangStore{\datalangLoc_s}{\datalangIdx}{\datalangVal_s}}{\datalangStore{\datalangLoc_t}{\datalangIdx}{\datalangVal_t}}
            }
        \\
        \inferrule*[lab=RelApplyProtocol]
            {
              \iProt (\iPredTwo, \datalangExpr_s, \datalangExpr_t)
            \and
               \forall \datalangExpr_s', \datalangExpr_t' \ldotp
               \iPredTwo (\datalangExpr_s', \datalangExpr_t') \iSepImp
               \iSimv[\iProt]{\iPred}{\datalangExpr_s'}{\datalangExpr_t'}
            }{
                \iSimv[\iProt]{\iPred}{\datalangExpr_s}{\datalangExpr_t}
            }
    \end{mathparpagebreakable}
    \caption{Reasoning rules for simulation (excerpt, omitting freshness conditions)}
    \label{fig:program_logic}
\end{figure}

In this section, we describe the inference rules of our relational program logic,
presented in \cref{fig:program_logic}.
% In the next section, we use this logic to prove the specifications.
% In section \cref{sec:simulation}, we show how it is actually defined and what notion of soundness supports it.
We omitted some standard rules for brevity.\Xgabriel{Which ones?}
Compared to the specifications of \cref{sec:specification}, we introduced an additional protocol parameter $\iProt$.
We explain it together with the \RefTirName{RelProtocol} rule in \cref{subsec:protocols}.

\Xgabriel{There are two judgments lying around, one with a precondition and one without. I would expect here some explanation about how those two judgments relate. TODO for Clément.}

\subsection{Language-independent rules}
The following rules are independent of \DataLang and could be reused as is in further works.

\RefTirName{RelPost} states that two expressions are related when they are in the relational postcondition.

\RefTirName{RelStuck} relates \emph{strongly stuck} expressions.
An expression is strongly stuck when it is stuck for any heap state.

\RefTirName{RelBind} is a standard bind rule sequencing computations on both sides.

\RefTirName{RelSrcPure} and \RefTirName{RelTgtPure} let us take pure reduction steps in either the source or target.
Pure steps (omitted for brevity) are the reduction steps that are deterministic and do not depend on the state.

\subsection{Language-specific rules: non-determinism}
$\iSimv[\iProt]{\iPred}{\datalangExpr_s}{\datalangExpr_t}$ asserts that $\datalangExpr_t$ refines $\datalangExpr_s$: any behavior of $\datalangExpr_t$ is also a behavior of $\datalangExpr_s$.
Consequently, non-determinism is treated differently in the source and target: we treat non-determinism as \emph{angelic} in source reductions and \emph{demonic} in target reductions.

Our operational semantics uses non-determinism in the reduction of constructors: $\datalangBlock{\datalangTag}{\datalangExpr_1}{\datalangExpr_2}$ reduces to $\datalangBlockDet{\datalangTag}{\datalangVar_1}{\datalangVar_2}$, where $\datalangVar_1$ and $\datalangVar_2$ are bound to $\datalangExpr_1$ and $\datalangExpr_2$ in some non-deterministic order.
In the program logic, the user may \emph{choose} an order for the source reduction, by using one of the rules \RefTirName{RelSrcBlock1} or \RefTirName{RelSrcBlock2}, and it has to prove that the expressions are related against \emph{any} target order, by proving the two premises of the rule \RefTirName{RelTgtBlock}.

\subsection{Language-specific rules: private locations.}
We can reason on points-to assertions --- that we interpret as locations private to the source or target --- in a standard way.
From a deterministic constructor $\datalangBlockDet{\datalangTag}{\datalangVal_1}{\datalangVal_2}$, we can apply \RefTirName{RelSrcBlockDet} or \RefTirName{RelTgtBlockDet}, yielding a points-to assertion for the allocated block.
The rules \RefTirName{RelSrcLoad} and \RefTirName{RelTgtLoad} let us load the pointed value while \RefTirName{RelSrcStore} and \RefTirName{RelTgtStore} let us update it with a new value.

\subsection{Language-specific rules: locations in the bijection.}
Corresponding source and target locations registered in the bijection through \RefTirName{BijInsert} have given up their respective points-to assertions but can still be accessed using the rules \RefTirName{RelLoad} and \RefTirName{RelStore}.

\RefTirName{RelLoad} states that simultaneously loading from two corresponding blocks yields similar values.

\RefTirName{RelStore} let us simultaneously store similar values into the same field of two corresponding blocks.

Together, these two rules enforce the bijection invariant: the contents of corresponding blocks are always similar.

\section{Abstract protocols} \label{sec:protocols} \label{subsec:protocols}

\begin{figure}[tp]
    \begin{tabular}{rcl}
            $\iProt_\mathrm{dir} (\iPredTwo, \datalangExpr_s, \datalangExpr_t)$
            & $\coloneqq$ &
            $\exists \datalangFn, \datalangVal_s, \datalangVal_t \ldotp$
        \\
            &&
            $\datalangFn \in \dom{\datalangProg_s} \iSep
            \datalangExpr_s = \datalangCall{\datalangFnptr{\datalangFn}}{\datalangVal_s} \iSep
            \datalangExpr_t = \datalangCall{\datalangFnptr{\datalangFn}}{\datalangVal_t} \iSep {}$
        \\
            &&
            $\datalangVal_s \iSimilar \datalangVal_t \iSep {}$
        \\
            &&
            $\forall \datalangValTwo_s, \datalangValTwo_t \ldotp
            \datalangValTwo_s \iSimilar \datalangValTwo_t \iWand
            \iPredTwo (\datalangValTwo_s, \datalangValTwo_t)$
        \\
            $\iProt_\mathrm{DPS} (\iPredTwo, \datalangExpr_s, \datalangExpr_t)$
            & $\coloneqq$ &
            $\exists \datalangFn, \datalangFn_\mathit{dps}, \datalangVal_s, \datalangLoc_1, \datalangLoc_2, \datalangLoc, \datalangIdx, \datalangVal_t \ldotp$
        \\
            &&
            $\datalangFn \in \dom{\datalangProg_s} \iSep
            \datalangRenaming [\datalangFn] = \datalangFn_\mathit{dps} \iSep
            \datalangExpr_s = \datalangCall{\datalangFnptr{\datalangFn}}{\datalangVal_s} \iSep
            \datalangExpr_t = \datalangCall{\datalangFnptr{\datalangFn_\mathit{dps}}}{\datalangLoc_1} \iSep {}$
        \\
            &&
            $(\datalangLoc_1 + 1) \iPointsto_t (\datalangLoc_2, \datalangVal_t) \iSep
            (\datalangLoc_2 + 1) \iPointsto_t (\datalangLoc, \datalangIdx) \iSep
            (\datalangLoc + \datalangIdx) \iPointsto \datalangHole \iSep
            \datalangVal_s \iSimilar \datalangVal_t$
        \\
            &&
            $\forall \datalangValTwo_s, \datalangValTwo_t \ldotp
            (\datalangLoc + \datalangIdx) \iPointsto \datalangValTwo_t \iSep
            \datalangValTwo_s \iSimilar \datalangValTwo_t \iWand
            \iPredTwo (\datalangValTwo_s, \datalangUnit)$
        \\
            $\iProt_\mathrm{TMC}$
            & $\coloneqq$ &
            $\iProt_\mathrm{dir} \sqcup \iProt_\mathrm{DPS}
            =
            \lambdaAbs (\iPredTwo, \datalangExpr_s, \datalangExpr_t) \ldotp \iProt_\mathrm{dir} (\iPredTwo, \datalangExpr_s, \datalangExpr_t) \iOr \iProt_\mathrm{DPS} (\iPredTwo, \datalangExpr_s, \datalangExpr_t)$
    \end{tabular}
    \caption{TMC protocol ($\iProt_\mathrm{TMC}$)}
    \label{fig:protocol}
\end{figure}

In \cref{sec:simulation}, we explain how our relation is defined coinductively and the first step of the proof essentially amounts to coinduction.
To internalize the coinduction hypothesis into the program logic, we introduce an additional parameter $\iProt$, a \emph{protocol}~\citep*{protocols-2021}, which are general proof-state transformers of type
\begin{mathline}
(\datalangExpr[] \to \datalangExpr[] \to \iProp) \to \datalangExpr[] \to \datalangExpr[] \to \iProp
\end{mathline}

Protocols are used in the logic via the $\RefTirName{RelProtocol}$ rule.
A pair of expressions $\datalangExpr_s$ and $\datalangExpr_t$ is supported by the protocol when it relates them to a postcondition $\iPredTwo$, capturing the possible results of an abstract/axiomatic transition from $\datalangExpr_s$ and $\datalangExpr_t$.
To conclude that $\datalangExpr_s$ and $\datalangExpr_t$ are related, one must prove that any two $\datalangExpr_s'$ and $\datalangExpr_t'$ accepted by this postcondition $\iPredTwo$ remain related.

\subsection{TMC protocols}

In our correctness proof for the TMC transformation, we use a sepecific protocol $\iProt_\mathrm{TMC}$ defined in \cref{fig:protocol} by combining two sub-protocols $\iProt_\mathrm{dir}$ and $\iProt_\mathrm{DPS}$ for the direct-style and DPS-style functions. Our coinduction hypothesis assumes toplevel function calls to be compatible with the direct and DPS specifications that we want to prove, and allows to reason about recursive calls to those functions inside the function bodies we are trying to relate.

$\iProt_\mathrm{dir}$ specifies the \emph{direct calling convention} induced by the direct transformation.
It requires $\datalangExpr_s$ and $\datalangExpr_t$ to be function calls to the same function with similar arguments.
To apply it, users can choose any postcondition implied by value similarity.
This rule is equivalent to the \TirName{Sim-Call} rule of \Simuliris, and most useful protocols are formed by combining $\iProp_\mathrm{dir}$ with other, more specialized protocols.

$\iProt_\mathrm{DPS}$ specifies the \emph{DPS calling convention} induced by the DPS transformation.
It requires $\datalangExpr_s$ to be a function call to a TMC-transformed function $\datalangFn$ and $\datalangExpr_t$ to be a function call to the DPS transform of $\datalangFn$.
As in the DPS specification, ownership of the destination must be passed to the protocol.
To apply it, users can choose any postcondition implied by the postcondition of the DPS specification, including the recovered ownership of the modified destination.

In the instantiated program logic, the application of the $\iProt_\mathrm{TMC}$ protocol is equivalent to the following rules:
\begin{mathpar}
    \inferrule*[lab=RelDir]
        {
            \datalangFn \in \dom{\datalangProg_s}
        \\\\
            \datalangVal_s \iSimilar \datalangVal_t
        \\\\
            \forall \datalangValTwo_s, \datalangValTwo_t \ldotp
            \datalangValTwo_s \iSimilar \datalangValTwo_t \iWand
            \iPred (\datalangValTwo_s, \datalangValTwo_t)
        }{
            \iSimv{
                \iPred
            }{
                \datalangCall{\datalangFnptr{\datalangFn}}{\datalangVal_s}
            }{
                \datalangCall{\datalangFnptr{\datalangFn}}{\datalangVal_t}
            }
        }
    \and
    \inferrule*[lab=RelDPS]
        {
            \datalangRenaming [\datalangFn] = \datalangFn_\mathit{dps}
        \\\\
            (\datalangLoc_1 + 1) \iPointsto_t (\datalangLoc_2, \datalangVal_t)
        \\\\
            (\datalangLoc_2 + 1) \iPointsto_t (\datalangLoc, \datalangIdx)
        \\\\
            (\datalangLoc + \datalangIdx) \iPointsto_t \datalangHole
        \\\\
            \datalangVal_s \iSimilar \datalangVal_t
        \\\\
            \forall \datalangValTwo_s, \datalangValTwo_t \ldotp
            (\datalangLoc + \datalangIdx) \iPointsto_t \datalangValTwo_t \iWand
            \datalangValTwo_s \iSimilar \datalangValTwo_t \iWand
            \iPred (\datalangValTwo_s, \datalangUnit)
        }{
            \iSimv{
                \iPred
            }{
                \datalangCall{\datalangFnptr{\datalangFn}}{\datalangVal_s}
            }{
                \datalangCall{\datalangFnptr{\datalangFn_\mathit{dps}}}{\datalangLoc_1}
            }
        }
\end{mathpar}

\subsection{Other examples of protocols}

Our program logic can be instantiated with other protocols to reason other program transformations.
To demonstrate this generality, we have also verified an inlining and an accumulator-passing-style (APS) transformation --- both included in our mechanization.

\paragraph{Inlining:} Here, the relation $\datalangExpr_s \rightsquigarrow \datalangExpr_t$ allows $\datalangExpr_t$ to recursively inline functions in $\datalangExpr_s$.
As with TMC, it captures all possible inlining strategies.

This relation can be proved correct by using a fairly simple protocol (combined with $\iProt_\mathrm{dir}$) relating a source function and its body:

\begin{tabular}{rcl}
        $\iProt_\mathrm{inline} (\iPredTwo, \datalangExpr_s, \datalangExpr_t)$
        & $\coloneqq$ &
        $\exists \datalangFn, \datalangVar, \datalangExpr_s', \datalangExpr_t', \datalangVal_s, \datalangVal_t \ldotp$
    \\
        &&
        $
        \datalangExpr_s = \datalangCall{\datalangFnptr{\datalangFn}}{\datalangVal_s}
        \ \iSep\ %
        \datalangVal_s \iSimilar \datalangVal_t
        \ \iSep {}$
    \\
        &&
        $\datalangProg_s [\datalangFn] = (\datalangRec{\datalangVar}{\datalangExpr_s'})
        \ \iSep\ %
        \datalangExpr_s' \rightsquigarrow \datalangExpr_t'
        \ \iSep\ %
        \datalangExpr_t = (\datalangLet{\datalangVar}{\datalangVal_t}{\datalangExpr_t'})
        \ \iSep {}$
    \\
        &&
        $\forall \datalangValTwo_s, \datalangValTwo_t \ldotp
        \datalangValTwo_s \iSimilar \datalangValTwo_t \iWand
        \iPredTwo (\datalangValTwo_s, \datalangValTwo_t)$
\end{tabular}
\medskip

\paragraph{Accumulator-passing style}: The APS transformation is a variant of the TMC transformation where the contexts that are made tail-recursive are applications of associative arithmetic operators, typically of the form $(\datalangExpr + \datalangCtxHole)$ or $\datalangExpr_1 + (\datalangExpr_2 \times \datalangCtxHole)$. (See the discussion in \citet*{tmc-koka-2023}.)

We defined an APS transformation, after extending \DataLang with integers and arithmetic operations. We verify it with a protocol similar to $\iProt_\mathrm{DPS}$ that allows calling the APS transform of a source function with an integer accumulator:

\medskip
\begin{tabular}{rcl}
        $\iProt_\mathrm{APS} (\iPredTwo, \datalangExpr_s, \datalangExpr_t)$
        & $\coloneqq$ &
        $\exists \datalangFn, \datalangFn_\mathit{aps}, \datalangVal_s, \datalangVal_\mathit{acc}, \datalangVal_t \ldotp$
    \\
        &&
        $\datalangFn \in \dom{\datalangProg_s}
        \ \iSep\ %
        \datalangRenaming [\datalangFn] = \datalangFn_\mathit{aps}
        \iSep {}$
    \\
        &&
        $\datalangVal_s \iSimilar \datalangVal_t
        \ \iSep\ %
        \datalangExpr_s = \datalangCall{\datalangFnptr{\datalangFn}}{\datalangVal_s}
        \ \iSep\ %
        \datalangExpr_t = \datalangCall{\datalangFnptr{\datalangFn_\mathit{aps}}}{\datalangPair{\datalangVal_\mathit{acc}}{\datalangVal_t}}
        \iSep {}$
    \\
        &&
        $\forall \datalangVal_s', \datalangExpr_t' \ldotp$
    \\
        &&
        $\bm{\mathrm{match}}\ \datalangVal_s'\ \bm{\mathrm{with}}\ \datalangNat \Rightarrow \datalangExpr_t' = \datalangVal_\mathit{acc} + \datalangNat \mid \_ \Rightarrow \mathrm{strongly \mathhyphen stuck}_{\datalangProg_t} (\datalangExpr_t')\ \bm{\mathrm{end}} \iWand$
    \\
        &&
        $\iPredTwo (\datalangVal_s', \datalangExpr_t')$
\end{tabular}
\medskip

\Xgabriel{TODO Clément: read this.}
One subtlety is that our \DataLang language is untyped, so arithmetic operations (here addition) may get stuck on non-integer values. If the function call $\datalangCall {\datalangFnptr \datalangFn} {\datalangVal_s}$ in the source program returns a non-integer value, then the outer context $\datalangVar_\mathit{acc} + \datalangCtxHole$ gets stuck. But in the transformed program, this failure happens inside the body of the APS-transformed function $\datalangFnptr {\datalangFn_\mathit{aps}}$. To represent this failure case in our protocol, the postcondition $\iPredTwo$ relates non-integer source return value $\datalangVal_s'$ with any strongly stuck expression $\datalangExpr_t'$ in the target. This relies on the generality of our protocols being predicate transformers on expressions, not just values.

%%% Local Variables:
%%% mode: latex
%%% TeX-master: "main"
%%% End:


\separate{\clearpage}{}
\section{Abstract protocols} \label{sec:protocols} \label{subsec:protocols}

\begin{figure}[tp]
    \begin{tabular}{rcl}
            $\iProt_\mathrm{dir} (\iPredTwo, \datalangExpr_s, \datalangExpr_t)$
            & $\coloneqq$ &
            $\exists \datalangFn, \datalangVal_s, \datalangVal_t \ldotp$
        \\
            &&
            $\datalangFn \in \dom{\datalangProg_s} \iSep
            \datalangExpr_s = \datalangCall{\datalangFnptr{\datalangFn}}{\datalangVal_s} \iSep
            \datalangExpr_t = \datalangCall{\datalangFnptr{\datalangFn}}{\datalangVal_t} \iSep {}$
        \\
            &&
            $\datalangVal_s \iSimilar \datalangVal_t \iSep {}$
        \\
            &&
            $\forall \datalangValTwo_s, \datalangValTwo_t \ldotp
            \datalangValTwo_s \iSimilar \datalangValTwo_t \iWand
            \iPredTwo (\datalangValTwo_s, \datalangValTwo_t)$
        \\
            $\iProt_\mathrm{DPS} (\iPredTwo, \datalangExpr_s, \datalangExpr_t)$
            & $\coloneqq$ &
            $\exists \datalangFn, \datalangFn_\mathit{dps}, \datalangVal_s, \datalangLoc_1, \datalangLoc_2, \datalangLoc, \datalangIdx, \datalangVal_t \ldotp$
        \\
            &&
            $\datalangFn \in \dom{\datalangProg_s} \iSep
            \datalangRenaming [\datalangFn] = \datalangFn_\mathit{dps} \iSep
            \datalangExpr_s = \datalangCall{\datalangFnptr{\datalangFn}}{\datalangVal_s} \iSep
            \datalangExpr_t = \datalangCall{\datalangFnptr{\datalangFn_\mathit{dps}}}{\datalangLoc_1} \iSep {}$
        \\
            &&
            $(\datalangLoc_1 + 1) \iPointsto_t (\datalangLoc_2, \datalangVal_t) \iSep
            (\datalangLoc_2 + 1) \iPointsto_t (\datalangLoc, \datalangIdx) \iSep
            (\datalangLoc + \datalangIdx) \iPointsto \datalangHole \iSep
            \datalangVal_s \iSimilar \datalangVal_t$
        \\
            &&
            $\forall \datalangValTwo_s, \datalangValTwo_t \ldotp
            (\datalangLoc + \datalangIdx) \iPointsto \datalangValTwo_t \iSep
            \datalangValTwo_s \iSimilar \datalangValTwo_t \iWand
            \iPredTwo (\datalangValTwo_s, \datalangUnit)$
        \\
            $\iProt_\mathrm{TMC}$
            & $\coloneqq$ &
            $\iProt_\mathrm{dir} \sqcup \iProt_\mathrm{DPS}
            =
            \lambdaAbs (\iPredTwo, \datalangExpr_s, \datalangExpr_t) \ldotp \iProt_\mathrm{dir} (\iPredTwo, \datalangExpr_s, \datalangExpr_t) \iOr \iProt_\mathrm{DPS} (\iPredTwo, \datalangExpr_s, \datalangExpr_t)$
    \end{tabular}
    \caption{TMC protocol ($\iProt_\mathrm{TMC}$)}
    \label{fig:protocol}
\end{figure}

In \cref{sec:simulation}, we explain how our relation is defined coinductively and the first step of the proof essentially amounts to coinduction.
To internalize the coinduction hypothesis into the program logic, we introduce an additional parameter $\iProt$, a \emph{protocol}~\citep*{protocols-2021}, which is a general proof-state transformer of type
\begin{mathline}
(\datalangExpr[] \to \datalangExpr[] \to \iProp) \to \datalangExpr[] \to \datalangExpr[] \to \iProp
\end{mathline}

Protocols are used in the logic via the $\RefTirName{RelProtocol}$ rule.
A pair of expressions $\datalangExpr_s$ and $\datalangExpr_t$ is supported by the protocol when it relates them to a postcondition $\iPredTwo$, capturing the possible results of an abstract/axiomatic transition from $\datalangExpr_s$ and $\datalangExpr_t$.
To conclude that $\datalangExpr_s$ and $\datalangExpr_t$ are related, one must prove that any two $\datalangExpr_s'$ and $\datalangExpr_t'$ accepted by this postcondition $\iPredTwo$ remain related.

\subsection{TMC protocols}

In our correctness proof for the TMC transformation, we use a specific protocol $\iProt_\mathrm{TMC}$ defined in \cref{fig:protocol} by combining two sub-protocols $\iProt_\mathrm{dir}$ and $\iProt_\mathrm{DPS}$ for the direct-style and DPS-style functions. Our coinduction hypothesis assumes toplevel function calls to be compatible with the direct and DPS specifications that we want to prove, and allows to reason about recursive calls to those functions inside the function bodies we are trying to relate.

$\iProt_\mathrm{dir}$ specifies the \emph{direct calling convention} induced by the direct transformation.
It requires $\datalangExpr_s$ and $\datalangExpr_t$ to be function calls to the same function with similar arguments.
To apply it, users can choose any postcondition implied by value similarity.
This rule is equivalent to the \TirName{Sim-Call} rule of \Simuliris.
Most useful protocols are formed by combining $\iProt_\mathrm{dir}$ with other, more specialized protocols.

$\iProt_\mathrm{DPS}$ specifies the \emph{DPS calling convention} induced by the DPS transformation.
It requires $\datalangExpr_s$ to be a function call to a TMC-transformed function $\datalangFn$ and $\datalangExpr_t$ to be a function call to the DPS transform of $\datalangFn$.
As in the DPS specification, ownership of the destination must be passed to the protocol.
To apply it, users can choose any postcondition implied by the postcondition of the DPS specification, including the recovered ownership of the modified destination.

% In the instantiated program logic, the application of the $\iProt_\mathrm{TMC}$ protocol is equivalent to the following rules:
% \begin{mathpar}
%     \inferrule*[lab=RelDir]
%         {
%             \datalangFn \in \dom{\datalangProg_s}
%         \\\\
%             \datalangVal_s \iSimilar \datalangVal_t
%         \\\\
%             \forall \datalangValTwo_s, \datalangValTwo_t \ldotp
%             \datalangValTwo_s \iSimilar \datalangValTwo_t \iWand
%             \iPred (\datalangValTwo_s, \datalangValTwo_t)
%         }{
%             \iSimv{
%                 \iPred
%             }{
%                 \datalangCall{\datalangFnptr{\datalangFn}}{\datalangVal_s}
%             }{
%                 \datalangCall{\datalangFnptr{\datalangFn}}{\datalangVal_t}
%             }
%         }
%     \and
%     \inferrule*[lab=RelDPS]
%         {
%             \datalangRenaming [\datalangFn] = \datalangFn_\mathit{dps}
%         \\\\
%             (\datalangLoc_1 + 1) \iPointsto_t (\datalangLoc_2, \datalangVal_t)
%         \\\\
%             (\datalangLoc_2 + 1) \iPointsto_t (\datalangLoc, \datalangIdx)
%         \\\\
%             (\datalangLoc + \datalangIdx) \iPointsto_t \datalangHole
%         \\\\
%             \datalangVal_s \iSimilar \datalangVal_t
%         \\\\
%             \forall \datalangValTwo_s, \datalangValTwo_t \ldotp
%             (\datalangLoc + \datalangIdx) \iPointsto_t \datalangValTwo_t \iWand
%             \datalangValTwo_s \iSimilar \datalangValTwo_t \iWand
%             \iPred (\datalangValTwo_s, \datalangUnit)
%         }{
%             \iSimv{
%                 \iPred
%             }{
%                 \datalangCall{\datalangFnptr{\datalangFn}}{\datalangVal_s}
%             }{
%                 \datalangCall{\datalangFnptr{\datalangFn_\mathit{dps}}}{\datalangLoc_1}
%             }
%         }
% \end{mathpar}

\subsection{Other examples of protocols}

Our program logic can be instantiated with other protocols to reason other program transformations.
To demonstrate this generality, we have also verified an inlining and an accumulator-passing-style (APS) transformation --- both included in our mechanization.

\paragraph{Inlining:} Here, the relation $\datalangExpr_s \rightsquigarrow \datalangExpr_t$ allows $\datalangExpr_t$ to recursively inline functions in $\datalangExpr_s$.
As with TMC, it captures all possible inlining strategies.
%
This relation can be proved correct by using a fairly simple protocol (combined with $\iProt_\mathrm{dir}$) relating a source function and its body:

\begin{tabular}{rcl}
        $\iProt_\mathrm{inline} (\iPredTwo, \datalangExpr_s, \datalangExpr_t)$
        & $\coloneqq$ &
        $\exists \datalangFn, \datalangVar, \datalangExpr_s', \datalangExpr_t', \datalangVal_s, \datalangVal_t \ldotp$
    \\
        &&
        $
        \datalangExpr_s = \datalangCall{\datalangFnptr{\datalangFn}}{\datalangVal_s}
        \ \iSep\ %
        \datalangVal_s \iSimilar \datalangVal_t
        \ \iSep {}$
    \\
        &&
        $\datalangProg_s [\datalangFn] = (\datalangRec{\datalangVar}{\datalangExpr_s'})
        \ \iSep\ %
        \datalangExpr_s' \rightsquigarrow \datalangExpr_t'
        \ \iSep\ %
        \datalangExpr_t = (\datalangLet{\datalangVar}{\datalangVal_t}{\datalangExpr_t'})
        \ \iSep {}$
    \\
        &&
        $\forall \datalangValTwo_s, \datalangValTwo_t \ldotp
        \datalangValTwo_s \iSimilar \datalangValTwo_t \iWand
        \iPredTwo (\datalangValTwo_s, \datalangValTwo_t)$
\end{tabular}
\medskip

\paragraph{Accumulator-passing style}: The APS transformation is a variant of the TMC transformation where the contexts that are made tail-recursive are applications of associative arithmetic operators, typically of the form $(\datalangExpr + \datalangCtxHole)$ or $\datalangExpr_1 + (\datalangExpr_2 \times \datalangCtxHole)$. (See the discussion by \citet*{tmc-koka-2023}.)

We define an APS transformation, after extending \DataLang with integers and arithmetic operations. We verify it with a protocol similar to $\iProt_\mathrm{DPS}$ that allows calling the APS transform of a source function with an integer accumulator: if $\datalangCall{\datalangFn}{\datalangVal_s}$ returns $n$, then $\datalangCall{\datalangFn_\mathit{aps}}{\datalangPair{\datalangVal_{\mathit{acc}}}{\datalangVal_t}}$ returns $\datalangVal_\mathit{acc} + n$.

\medskip
\begin{tabular}{rcl}
        $\iProt_\mathrm{APS} (\iPredTwo, \datalangExpr_s, \datalangExpr_t)$
        & $\coloneqq$ &
        $\exists \datalangFn, \datalangFn_\mathit{aps}, \datalangVal_s, \datalangVal_\mathit{acc}, \datalangVal_t \ldotp$
    \\
        &&
        $\datalangFn \in \dom{\datalangProg_s}
        \ \iSep\ %
        \datalangRenaming [\datalangFn] = \datalangFn_\mathit{aps}
        \iSep {}$
    \\
        &&
        $\datalangVal_s \iSimilar \datalangVal_t
        \ \iSep\ %
        \datalangExpr_s = \datalangCall{\datalangFnptr{\datalangFn}}{\datalangVal_s}
        \ \iSep\ %
        \datalangExpr_t = \datalangCall{\datalangFnptr{\datalangFn_\mathit{aps}}}{\datalangPair{\datalangVal_\mathit{acc}}{\datalangVal_t}}
        \iSep {}$
    \\
        &&
        $\forall \datalangVal_s', \datalangExpr_t' \ldotp$
    \\
        &&
        $\bm{\mathrm{match}}\ \datalangVal_s'\ \bm{\mathrm{with}}\ \datalangNat \Rightarrow \datalangExpr_t' = \datalangVal_\mathit{acc} + \datalangNat \mid \_ \Rightarrow \mathrm{strongly \mathhyphen stuck}_{\datalangProg_t} (\datalangExpr_t')\ \bm{\mathrm{end}} \iWand$
    \\
        &&
        $\iPredTwo (\datalangVal_s', \datalangExpr_t')$
\end{tabular}
\medskip
\Xfrancois{Explain why this guarantee (``$f_\mathit{aps}$ will get stuck'') is needed. Do we want to prove that if the source crashes then the target crashes?}

One subtlety is that our \DataLang language is untyped, so arithmetic operations (here addition) may get stuck on non-integer values. If the function call $\datalangCall {\datalangFnptr \datalangFn} {\datalangVal_s}$ in the source program returns a non-integer value, then the outer context $\datalangVal_\mathit{acc} + \datalangCtxHole$ gets stuck. But\Xfrancois{not easy to follow your reasoning here} in the transformed program, this failure happens inside the body of the APS-transformed function $\datalangFnptr {\datalangFn_\mathit{aps}}$. To represent this failure case in our protocol, the postcondition $\iPredTwo$ relates a non-integer source return value $\datalangVal_s'$ with any strongly stuck expression $\datalangExpr_t'$ in the target. This relies on the generality of our protocols being predicate transformers on expressions, not just values.

%%% Local Variables:
%%% mode: latex
%%% TeX-master: "main"
%%% End:


\separate{\clearpage}{}
\section{Proof of the specification}
\label{sec:proof}

In this section, we prove the specifications of \cref{sec:specification}.

As mentioned in \cref{sec:protocols}, we instantiate our program logic with a specific protocol $\iProt_\mathrm{TMC}$ defined in \cref{fig:protocol}. We define a shorthand notation for this instantiation:
\begin{mathline}
    \iSimv{
        \iPred
    }{
        \datalangExpr_s
    }{
        \datalangExpr_t
    }
    \coloneqq
    \iSimv[
        \iProt_\mathrm{TMC}
    ]{
        \iPred
    }{
        \datalangExpr_s
    }{
        \datalangExpr_t
    }
\end{mathline}

So far, we worked with \emph{closed} expressions, that have no free variables.\Xfrancois{This was never said.}
We need to generalize the specifications to \emph{open} expressions that may have free variables, as is standard.
To do so, we introduce a \emph{runtime relation} $\iRsimv{\iPred}{\datalangExpr_s}{\datalangExpr_t}$ in \cref{fig:rsim}.
It requires $\datalangBisubst_s (\datalangExpr_s)$ and $\datalangBisubst_t (\datalangExpr_t)$ to be related for any \emph{well-formed closing bisubstitution} $\datalangBisubst \in \datalangVar[] \rightarrow \datalangVal[] \times \datalangVal[]$.
In practice, $\datalangBisubst$ contains $\datalangLetKeyword$-bound variables that have been $\beta$-reduced, and their substitute source and target values.

In addition, we will only consider \emph{valid source expressions} --- denoted by $\wf{\datalangExpr_s}$ ---, \ie those that do not involve any location, deterministic block expressions, or undefined source function.

\begin{lemma}[Specification of direct transformation] \label{thm:dir}
    \[
        \iRsimvHoare{
            \wf{\datalangExpr_s} \iSep
            \datalangExpr_s \tmcDir{\datalangRenaming} \datalangExpr_t
        }{
            \datalangVal_s, \datalangVal_t \ldotp
            \datalangVal_s \iSimilar \datalangVal_t
        }{
            \datalangExpr_s
        }{
            \datalangExpr_t
        }
    \]
\end{lemma}
\Xfrancois{Here and below the pure assumptions should be put outside, with an implication, they are ``static'' (compile-time) assumptions.}

\begin{lemma}[Specification of DPS transformation] \label{thm:dps}
    \[
        \iRsimvHoare{
            \wf{\datalangExpr_s} \iSep
            (\datalangLoc, \datalangIdx, \datalangExpr_s) \tmcDps{\datalangRenaming} \datalangExpr_t \iSep
            (\datalangLoc + \datalangIdx) \iPointsto_t \datalangHole
        }{
            \datalangVal_s, \datalangUnit \ldotp
            \exists \datalangVal_t \ldotp
            (\datalangLoc + \datalangIdx) \iPointsto_t \datalangVal_t \iSep
            \datalangVal_s \iSimilar \datalangVal_t
        }{
            \datalangExpr_s
        }{
            \datalangExpr_t
        }
    \]
\end{lemma}
\Xfrancois{TODO point back to the beginning of the paper for an informal explanation of these statements.}

Both proofs proceed by induction over $\datalangExpr_s$ and mutual induction over $\datalangExpr_s \tmcDir{\datalangRenaming} \datalangExpr_t$ and $(\datalangLoc, \datalangIdx, \datalangExpr_s) \tmcDps{\datalangRenaming} \datalangExpr_t$. In each case we then apply the relevant rules of the program logic. % -- we refer to our mechanization for the details.
    % We only sketch the cases shown in \cref{fig:tmc_dir} and \cref{fig:tmc_dps} and refer to our mechanization for the complete proof.
    % \begin{itemize}[align=left, leftmargin=*]
    %     \item[\RefTirName{DirVal}:] Follows from $\wf{\datalangVal}$.
    %     \item[\RefTirName{DirVar}:] Follows from $\wf{\datalangBisubst}$.
    %     \item[\RefTirName{DirLet}:] Follows from IH for $\datalangExpr_{s1}$ and IH for $\datalangExpr_{s2}$.
    %     \item[\RefTirName{DirBlockDPS1}:] Follows from \RefTirName{RelSrcBlock1}, \RefTirName{RelTgtBlock}, IH for $\datalangExpr_{s1}$, \RefTirName{RelTgtBlockDet}, IH for $\datalangExpr_{s2}$, \RefTirName{RelSrcBlockDet} and \RefTirName{BijInsert}.
    %     \item[\RefTirName{DirBlockDPS2}:] Similar to \RefTirName{DirBlockDPS1}.
    %     \item[\RefTirName{DPSBase}:] Follows from IH for $\datalangExpr_s \tmcDir{\datalangRenaming} \datalangExpr_t$ and \RefTirName{RelTgtStore}.
    %     \item[\RefTirName{DPSLet}:] Follows from IH for $\datalangExpr_{s1}$ and IH for $\datalangExpr_{s2}$.
    %     \item[\RefTirName{DPSCall}:] Follows from \RefTirName{SimTgtBlock}, \RefTirName{SimTgtBlockDet}, IH for $\datalangExpr_s$ and \RefTirName{RelDPS}.
    %     \item[\RefTirName{DPSIf}:] Follows from IH for $\datalangExpr_{s0}$, IH for $\datalangExpr_{s1}$ and IH for $\datalangExpr_{s2}$.
    %     \item[\RefTirName{DPSBlock1}:] Follows from \RefTirName{RelSrcBlock1}, \RefTirName{RelTgtBlock}, IH for $\datalangExpr_{s1}$, \RefTirName{RelTgtBlockDet}, \RefTirName{RelTgtStore} IH for $\datalangExpr_{s2}$, \RefTirName{RelSrcBlockDet} and \RefTirName{BijInsert}.
    %     \item[\RefTirName{DPSBlock2}:] Similar to \RefTirName{DPSBlock1}. \qedhere
    % \end{itemize}

\begin{figure}[tp]
    \begin{align*}
    		\wf{\datalangBisubst}
    		&\coloneqq
    		\forall \datalangVar \ldotp
    		\exists \datalangVal_s, \datalangVal_t \ldotp
    		\datalangBisubst (\datalangVar) = (\datalangVal_s, \datalangVal_t) \iSep
    		\datalangVal_s \iSimilar \datalangVal_t
    	\\
    		\iRsimv[\iProt]{\iPred}{\datalangExpr_s}{\datalangExpr_t}
    		&\coloneqq
    		\forall \datalangBisubst \ldotp
    		\wf{\datalangBisubst} \iWand
    		\iSimv[\iProt]{\iPred}{\datalangBisubst_s (\datalangExpr_s)}{\datalangBisubst_t (\datalangExpr_t)}
    	\\
    	   \iRsimvHoare[\iProt]{P}{\iPred}{\datalangExpr_s}{\datalangExpr_t}
    	   &\coloneqq
    	   \iPersistent \left( P \iWand \iRsimv[\iProt]{\iPred}{\datalangExpr_s}{\datalangExpr_t} \right)
    \end{align*}
    \caption{Logical relation on open expressions}
    \label{fig:rsim}
\end{figure}

%%% Local Variables:
%%% mode: latex
%%% TeX-master: "main"
%%% End:


\separate{\clearpage}{}
\section{Simulation}

Simuliris~\citep*{TODO-simuliris} suggests a definition of simulation relations in Iris. As we explained in \cref{sec:howto-relation}, it carefully avoids using the ``later'' modality by using Coq's native notion of coinduction. Simuliris has several desirable properties for us: it can be defined in the un-modified Iris base logic (unlike Transfite Iris~\citep*{transfinite-iris}), it is defined in a way that makes it easy to specialize to other programming languages, and it supports showing the preservation of termination. On the other hand, the original definition is complex as it supports concurrency and notions of fairness. We reused the definition of Simuliris, simplified for the sequential setting. We then extended the resulting definition to support our treatment of stuck states as failures, and generalize the notion of external calls to abstract protocols $\iProt$. Most of the ideas below come from the Simuliris work directly, we will explicitly point out the parts that differ.

\paragraph{Relating states} The \emph{state interpertation} $I(\datalangState_s, \datalangState_t)$ relates source and target states. It extends the unary notion of state intereptation predicate $I(\datalangState)$. Recall that in separation logic, formulas contain both pure propositions and ressources, which include in particular logical points-to assertions $\datalangLoc \iPointsto v$. $I(\datalangState)$ enforces the consistency between those logical assertions and the physical state $\datalangState$.

In the relational setting, the state interpretation becomes a relation $I(\datalangState_s, \datalangState_t)$ tracking points-to predicates on either states $\datalangLoc_s \iPointsto_s \datalangVal_s$ and $\datalangLoc_t \iPointsto_t \datalangVal_t$. It is additionally in charge of maintaing the heap bijection mentioned in \cref{TODO} by tracking predicates $\datalangLoc_s \iInBij \datalangLoc_t$. This assertion is persistent: one cannot remove locations from the bijection.

\paragraph{Relating values} TODO relation between values.

\paragraph{Relating expressions} The definition of the expression relation $\iSim[\iProt]{\iPred}{\datalangExpr_s}{\datalangExpr_t}$ is shown in \cref{fig:sim}; it reads as ``$\datalangExpr_s$ simulates $\datalangExpr_t$ under the postcondition $\iPred$ and protocol $\iProt$''. It expresses that the source $\datalangExpr_s$ can simulate any computation of the target $\datalangExpr_t$ until they both get stuck, both diverge, or reach two expressions in the relational post-condition $\iPred$. The parameter $\iProt$ is the relational protocol that must be followed by function calls in $\datalangExpr_s$ and $\datalangExpr_t$ -- generalizing the Simuliris rule for external calls.

\paragraph{Mixed induction and coinduction}
The definition of the relation itself starts with a double fixpoint, a greatest fixpoint $\nuAbs sim .$ (coinduction) of smallest fixpoint $\muAbs {sim \mathhyphen inner} .$ (induction) of a relation $\mathrm{sim \mathhyphen body}_\iProt$. Inside the relation definition, the cases that use the inductive $sim \mathhyphen inner$ in their recursive occurrence can only be repeated finitely many times, while the cases that use the coinductive $sim$ in their recursive occurrence can be repeated infinitely, but can only do so under a form of guardedness condition.

\paragraph{Simulation clauses} The relation between $\datalangExpr_s$ and $\datalangExpr_t$, at a given post-condition $\iPred$, must be parametric over all physical states in the state interpretation relation $I(\datalangState_s, \datalangState_t)$. It is established by one of the following clauses:

\begin{enumerate}
\item[\circled{1}] Halting clause: $\datalangExpr_s$, $\datalangExpr_t$ can stop if they are already in the post-condition $\iPred$ -- and the states are still related.
\item[\circled{2}]
\item[\circled{3}]
\item[\circled{4}]
\item[\circled{5}]
\end{enumerate}

% sim-body:
% cas 1, 2, 3, 4: on raconte
% "ouverte" comprend un autre cas.
% cas 5: les appels externes, protocoles (on raconte)

\begin{figure}[tp]
    \begin{align*}
    		\mathrm{sim \mathhyphen body}_\iProt
    		&\coloneqq
    		\begin{array}{l}
    				\lambdaAbs sim \ldotp
    				\lambdaAbs sim \mathhyphen inner \ldotp
    				\lambdaAbs {(\iPred, \datalangExpr_s, \datalangExpr_t)} \ldotp
    				\forall \datalangState_s, \datalangState_s \ldotp
    				\mathrm{I} (\datalangState_s, \datalangState_t)
    				\iSepImp \iBupd
    			\\
    				\bigvee \left[ \begin{array}{ll}
    							\circled{1}
    						&
    							\mathrm{I} (\datalangState_s, \datalangState_t) \iSep
    							\iPred (\datalangExpr_s, \datalangExpr_t)
    					\\
    					        \circled{2}
                            &
                                \mathrm{I} (\datalangState_s, \datalangState_t) \iSep
    							\mathrm{strongly \mathhyphen stuck}_{\datalangProg_s} (\datalangExpr_s) \iSep
    							\mathrm{strongly \mathhyphen stuck}_{\datalangProg_t} (\datalangExpr_s)
    					\\
    							\circled{3}
    						&
    							\exists \datalangExpr_s', \datalangState_s' \ldotp
    							(\datalangExpr_s, \datalangState_s) \step{\datalangProg_s}^+ (\datalangExpr_s', \datalangState_s') \iSep
    							\mathrm{I} (\datalangState_s', \datalangState_t) \iSep
    							sim \mathhyphen inner (\iPred, \datalangExpr_s', \datalangExpr_t)
    					\\
    							\circled{4}
    						&
								\mathrm{reducible}_{\datalangProg_t} (\datalangExpr_t, \datalangState_t) \iSep
								\forall \datalangExpr_t', \datalangState_t' \ldotp
								(\datalangExpr_t, \datalangState_t) \step{\datalangProg_t} (\datalangExpr_t', \datalangState_t')
								\iSepImp \iBupd
						\\
                            &
								\bigvee \left[ \begin{array}{ll}
											\circled{A}
										&
											\mathrm{I} (\datalangState_s, \datalangState_t') \iSep
											sim \mathhyphen inner (\iPred, \datalangExpr_s, \datalangExpr_t')
									\\
											\circled{B}
										&
											\exists \datalangExpr_s', \datalangState_s' \ldotp
											(\datalangExpr_s, \datalangState_s) \step{\datalangProg_s}^+ (\datalangExpr_s', \datalangState_s') \iSep
											\mathrm{I} (\datalangState_s', \datalangState_t') \iSep
											sim (\iPred, \datalangExpr_s', \datalangExpr_t')
								\end{array} \right.
    					\\
    							\circled{5}
    						&
    							\exists \datalangEctx_s, \datalangExpr_s', \datalangEctx_t, \datalangExpr_t', \iPredTwo \ldotp
    					\\
    					    &
    					        \datalangExpr_s = \datalangEctx_s [\datalangExpr_s'] \iSep
    							\datalangExpr_t = \datalangEctx_t [\datalangExpr_t'] \iSep
    						    \iProt (\iPredTwo, \datalangExpr_s', \datalangExpr_t') \iSep
    						    \mathrm{I} (\datalangState_s, \datalangState_t) \iSep {}
    					\\
                            &
								\forall \datalangExpr_s'', \datalangExpr_t'' \ldotp
								\iPredTwo (\datalangExpr_s'', \datalangExpr_t'') \iSepImp
								sim (\iPred, \datalangEctx_s [\datalangExpr_s''], \datalangEctx_t [\datalangExpr_t''])
    				\end{array} \right.
    		\end{array}
    	\\
    	    \mathrm{sim \mathhyphen inner}_\iProt
    	    &\coloneqq
    	    \lambdaAbs sim \ldotp
    	    \muAbs sim \mathhyphen inner \ldotp
    	    \mathrm{sim \mathhyphen body}_\iProt (sim, sim \mathhyphen inner)
    	\\
    		\mathrm{sim}_\iProt
    		&\coloneqq
    		\nuAbs sim \ldotp
    		\mathrm{sim \mathhyphen inner}_\iProt (sim)
    	\\
    		\iSim[\iProt]{\iPred}{\datalangExpr_s}{\datalangExpr_t}
    		&\coloneqq
    		\mathrm{sim}_\iProt (\iPred, \datalangExpr_s, \datalangExpr_t)
    	\\
    	   \iSimv[\iProt]{\iPred}{\datalangExpr_s}{\datalangExpr_t}
    	   &\coloneqq
    	   \iSim[\iProt]{\lambdaAbs (\datalangExpr_s', \datalangExpr_t') \ldotp \exists \datalangVal_s, \datalangVal_t \ldotp \datalangExpr_s' = \datalangVal_s \iSep \datalangExpr_t' = \datalangVal_t \iSep \iPred (\datalangVal_s, \datalangVal_t)}{\datalangExpr_s}{\datalangExpr_t}
    \end{align*}
    \caption{Simulation with protocol}
    \label{fig:sim}
\end{figure}

\separate{\clearpage}{}
\section{Related Work}

\subsection{TMC Support in Compilers}

Tail-recursion modulo cons was well-known in the Lisp community as
early as the 1970s. For example the REMREC system~\citep*{risch-73}
automatically transforms recursive functions into loops, and
supports modulo-cons tail recursion. It also supports tail-recursion
modulo associative arithmetic operators, which is outside the scope
of our work, but supported by the GCC compiler for example. The TMC
fragment is precisely described (in prose) by \citet*{friedman-wise-75}.

In the Prolog community it is a common pattern to implement destination-passing style through unification variables; in particular ``difference lists'' are a common representation of lists with a final hole.
Unification variables are first-class values: in particular they can be passed as function arguments, providing expressive, first-class support for destination-passing style in the source language.
For example, we do not support optimizing tail contexts of the form \ocaml{List.append li _?}, only direct constructor applications; this can be expressed in Prolog as just the difference list \ocaml{(List.append li X, X)} for a fresh destination variable \ocaml{X}.
But this expressiveness comes at a performance cost, and there is no static checking that the data is fully initialized at the end of computation.

\citet*{tmc-scala-2013} implement Ozma, an experimental backend for
the Scala programming language that targets the Oz virtual machine.
Oz~\citep*{oz-1994,oz-1995} is a language that integrates features
from logic- and functional-programming style, and in particular offers
pervasive ``dataflow values'', a generalization of Prolog unification
variables. Ozma brings to Scala an idiom of Oz, where Prolog-style
difference lists are used to represent potentially-infinite streams
that model synchronous concurrent agents. These lists must be
processed in constant stack space, so Ozma introduces
a tail-modulo-cons transformation in Prolog fashion: all data
constructor arguments (and some explicitly-annotated
function arguments) are transformed into dataflow values. The
motivation is expressivity, not performance, and the Ozma backend is
not competitive with the Scala JVM backend. Besides the obvious
engineering-effort differences, the pervasive use of dataflow values
may incur high constant overhead, making this approach unsuitable to
bring TMC to performance-conscious Scala users.

Independently of our work, Koka has implemented TMC starting in August
2020\footnote{\url{https://github.com/koka-lang/koka/commit/f6a343d31f486ea5edd44798dca7bca52d7b450c}}~\citep*{tmc-koka-2023}.
An interesting problem they had to solve, which does not occur in \OCaml,
is how to support TMC in presence of non-linear continuations. Our
correctness argument for TMC relies on the fact that the destination
is uniquely owned, and written exactly once; this property may not
hold in programs that use multishot continuations (\ocaml|call/cc|,
\ocaml|let/cc|, \ocaml|delim/cc|) or multishot effect handlers. The standard Koka runtime uses its
reference-counting machinery to determine that a destination is not
uniquely-owned anymore, and stores extra metadata in
partially-initialized blocks to be able to copy them on-demand in this
case. Its JavaScript back-end instead reverts to a CPS transformation
when non-linear control flow is detected.

\subsection{Reasoning About Destination-Passing-Style}

In general, if we think of non-tail recursive functions as having an ``evaluation context'' representing the continuation of the recursive call, then the techniques to turn classes of calls into tail-calls correspond to different reified representations of non-tail contexts, equipped with specific (efficient) implementations of context composition and hole-plugging.
TMC comes from representing data-construction contexts as the partial data itself, with hole-plugging by mutation.
Associative-operator transformations represent the context \ocaml|1 + (4 + _?)| as the number \ocaml|5| directly.
(Sometimes it suffices to keep around an abstraction of the context; see John Clements' work stack-based security.)

\citet*{minamide-98} gives a ``functional'' interface to destination-passing-style programs, by presenting a partial data-constructor composition \ocaml{Foo(x,Bar(_?))} as a use-once, linear-typed function \ocaml{linfun h -> Foo(x,Bar(h))}.
Those special linear functions remain implemented as partial data, but they expose a referentially-transparent interface to the programmer, restricted by a linear type discipline.
This is a beautiful way to represent destination-passing style, orthogonal to our work: users of Minamide's system would still have to write the transformed version by hand, and we could implement a transformation into destination-passing style expressed in his system.
\citet*{sobel-friedman-98}, inspired by Minamide's work, derive tree-traversal functions with a concrete representation of their continuations that implement pointer inversion to remain tail-recursive.
Mezzo~\citep*{mezzo-2016} supports a more general-purpose type system based on separation logic, which can directly express uniquely-owned partially-initialized data, and its transformation into immutable, duplicable results.
(See the \href{https://protz.github.io/mezzo/code_samples/list.mz.html}{List} module of the Mezzo standard library, and in particular \ocaml{cell}, \ocaml{freeze} and \ocaml{append} in destination-passing-style).

\subsection{Correctness Proof for TMC}

\citet*{tmc-koka-2023} provide a pen-and-paper correctness argument for TMC, or in fact a family of approaches based on optimized representations of classes of non-tail contexts, in the style of program calculation.
The clarity of their exposition is remarkable.

The implementation they prove correct, which corresponds to the approach described in \citet*{minamide-98}, results in a slightly less efficient generation where recursive calls to \ocaml{map_dps} are passed \emph{both} the start of the list and the destination to be written at its end.
The start of the list remains constant over all recursive calls, so our DPS version does not propagate it.

We were inspired by the generality of their presentation and verified that our proof technique can also be applied to some other TMC variants, by extending our mechanized development with a correctness proof for an
accumulator-passing-style transformation.

Finally, our correctness results are more precise and slightly stronger: they work in a well-typed setting where programs do not fail, whereas we use untyped terms and show preservation of failure; their proofs assume a deterministic, non-effectful language, whereas we use a more general non-deterministic, effectful language; finally, they have a very simple definition of program equivalence (reducing to the exact same value) that works well for semi-formal reasoning, but is unsuitable to scale the argument to other programming languages, whereas we use a standard notion of behavioral refinement that can scale to less idealized settings.
%
We get this extra generality mostly as a direct result of our methodology (relational program logic backed by a simulation); but showing preservation of failure requires some care when handling arithmetic operators in the accumulator-passing-style variant.

\subsection{Relational Reasoning in Separation Logic}

Defining a program logic to capture unary program properties is a typical usage of \Iris; relational properties are rarer. \citet*{tassarotti-2017} use (a linear variant of) \Iris to prove the correctness of a program transformation that implements communication channels using shared references. See the related work of ReLoC Reloaded~\citep*{reloc-2021}.
\Xfrancois{+ thèse de Friis Vindum.}
\Xfrancois{ReLoC a été utilisé surtout pour vérifier des algos concurrents.}

A relational program logic can be justified in \Iris by interpreting it as a unary relation on the target program, typically involving the \texttt{wp} predicate of the base language. This approach is inspired by CaReSL~\citep*{caresl-2013}. We follow a more direct, traditional approach of interpreting the program logic as a (binary) simulation relation (defined in the \Iris meta-logic) which is shown (adequacy) to imply a refinement between the program behaviors (denotations).

It is in fact surprisingly difficult to define simulations in \Iris, if we expect them to be adequate (to correspond to the usual notion of simulation outside the \Iris world). This is due to meta-theoretical difficulties around the ``later'' modality which led to the Transfinite \Iris variant~\citep*{transfinite-iris-2021}. We started by defining simulations in Transfinite \Iris, but later moved to the \Simuliris approach~\citep*{simuliris-2022}, where simulations are defined in standard \Iris without using the ``later'' modality, using coinduction instead (via its impredicative encoding).

As a minor technical point of comparison to \Simuliris, our definition of behaviors (denotations) includes non-termination, successfully evaluating to a value, but also failing with an error, and refinement preserves all three kind of behaviors. We do not model undefined behaviors.

We believe that our approach (relational program logics justified by a simulation) is showing promises for compiler verification. Verification of CompCert\Xgabriel{TODO citation} passes typically prove a forward simulation result, which is strengthened into a backward simulation thanks to a determinism assumption. We get the desired backward simulation directly, with compositional proofs.

\subsection{Protocols}

The function-call rule of \Simuliris only relates calls to the same function, so it is unsuitable for program transformations that also transform function definitions. We parameterize our program logic and notion of simulation on a \emph{protocol} $\iProt$, an arbitrary predicate transformer injected into the relation. This approach is reminiscent of the \emph{axiomatic semantics}~\citep*{axiomatic-semantics-2014} proposed to reason about foreign function calls. In the \Iris community, we were directly inspired by \emph{protocols} \citet*{protocols-2021}, and this approach was also reused recently, in a unary setting, by \citet*{melocoton-2023}. Our notion of protocol is slightly more general than in those two works, as it can relate arbitrary expressions in evaluation position (not just function calls), and ``return'' after an axiomatic transition with arbitrary expressions (not just values). We use this extra generality to reason about accumulator-passing-style transformation in presence of ill-typed programs -- see \cref{subsec:protocols}.

%%% Local Variables:
%%% mode: latex
%%% TeX-master: "main"
%%% End:


% \separate{\clearpage}{}
% \section{Conclusion and future work}

concurrency

effect handlers

compression of constructors

prove other program transformation using this framework (CPS)

% From the Author Instruction:
%
% > POPL-specific: The sections at the end of the paper should be ordered as follows:
% > first appendices, then acknowledgments, and last references.
% > The acknowledgments and references will not count towards the page limit.
%
% by email, Conference Publishing in fact clarifies that there cannot be appendices,
% so our appendices will have to be published in a separate "full" version
% of the paper, and it is more convenient to include them at the end.

%% Acknowledgments
\begin{acks}
  This work was initiated by Frédéric Bour who implemented TMC in an
  experimental variant of the OCaml compiler and, in 2015, submitted his work
  for inclusion. At the time, the proposal remained stalled due to lack
  of review and integration effort among maintainers. The broad lines
  of the final implementation, in term of generated code, were already
  in Frédéric Bour's initial version, as well as the idea to have an
  opt-in transformation controlled by an attribute
  (\cref{subsec:optin}). This experiment also generated
  performance data that motivated us to push further. (One notable
  technical difference is that instead of letting a destination-passing-style
  function take two parameters \ocaml|dst| and \ocaml|ofs|, Bour
  would generate several versions of the function,
  where the parameter \ocaml|ofs| was specialized to
  a constant. He found that this did not noticeably improve
  performance, and changed back to a parameter to simplify the
  implementation and reduce code size.)

  In 2020, Gabriel Scherer restarted Frédéric Bour's effort with
  a review followed by a partial re-implementation of the
  transformation that introduced the applicative style discussed in
  \cref{subsec:applicative-impl} as well as much of the current
  attribute-based user interface to control the transformation
  (\cref{subsec:user-control}). Basile Clément in turn reviewed
  Scherer's version, and introduced constructor compression
  (\cref{subsec:constructor-compression}). Xavier Leroy implemented
  a change to the \OCaml calling convention to remove parameter-number
  restrictions on tail calls on some architectures
  (\cref{subsec:PR-history}). The work was finally reviewed by Pierre
  Chambart and merged in the upstream \OCaml compiler in November
  2021.

  Clément Allain started working on a mechanized soundness proof for
  the TMC transformation in Iris in Summer 2022, as a master's
  internship supervised by François Pottier. They discovered that the
  question of defining simulations in Iris is surprisingly
  interesting, and that correctness proofs of transformations of
  general recursive functions require coinductive reasoning. Clément
  Allain wrote the bulk of the correctness proof at this point, and
  finished the mechanization work over 2023.

  Finally, the present research article itself was written by Clément
  Allain and Gabriel Scherer in Summer 2023 and Spring 2024, with
  excellent review comments from François Pottier, the POPL'25
  anonymous reviewers, and Anton Lorenzen.
\end{acks}

%%% Local Variables:
%%% mode: latex
%%% TeX-master: "main"
%%% End:


%% Bibliography/References

\clearpage
\bibliography{tmc}

%% Appendices

\clearpage
\appendix

\begin{version}{\FALSE}
\section{Extended republication}
\label{app:extended-republication}

A preliminary, work-in-progress version of this work was published at
a national conference -- all talks given in a non-English
language. This makes the present submission to a PACMPL-published
conference a partial republication.

This is unusual, so we decided to write this appendix section to
explain this choice. Our explanation is carefully worded to maintain
anonymity. (Note: there is overlap between the authors of both
presentations, but they are not equal sets, so guessing which previous
version we are talking about would be only partially de-anonimizing.)

\subsection{The letter of the law.}

We believe that partial republication fully satisfies the
\href{http://www.sigplan.org/Resources/Policies/Republication/}{SIGPLAN
  Republication Policy}, which we quote below in full
(with our emphasis):

\begin{quote}
  \textbf{Prior Publication of Closely Related Work.}  If a closely
  related paper was previously accepted at a conference or a workshop
  with proceedings of any kind, the original paper must be cited, its
  relationship to the current paper explained, and the program chair
  must be explicitly informed. The second paper will be judged on the
  additional value of its publication in the new venue. Such value
  might come from the \textbf{wider audience for the new venue} or
  from subjecting the work to a \textbf{higher standard of
    review}. Program committees should consider factors such as the
  following in deciding whether to publish the second paper:
    \begin{itemize}
    \item The call for papers for the first venue clearly states that publication in the venue is not intended to preclude later publication.
    \item The original proceedings are not easily accessible to the SIGPLAN community.
    \item The first venue \textbf{targeted a geographically limited audience}.
    \end{itemize}
\end{quote}

The present submission will be subject to higher standards
(the acceptance rate of the national venue is in the
80-100\% range). If accepted, it would reach a wider audience. The
past submission targeted a geographically limited audience. (If we
remember correctly, everyone attending the conference that year was
working in the same country it was located in.)

\subsection{The spirit of the law.}

We believe that a partial republication mirrors the established
practice of publishing a first version of a paper in conference
proceedings, and then an extended version in a journal. Here the
situation is different but similar, we published a work-in-progress
presentation of our work in a minor/national conference, and we are
submitting a full presentation of the same work
(and additional contributions) to a major/international
conference-journal hybrid.

Common/accepted conference-to-journal republication typically expect
``enhanced versions'' with, according to the Journal of Functional
Programming guidelines for example, ``should contain additional
discussion, examples, or proofs''. We believe that our new
presentation of our work is well above this bar, as it includes major
new scientific contributions.

\subsection{Comparing the two versions.}

The work-in-progress paper was written in October 2020, after we
implemented TMC for OCaml, but before it was integrated into the OCaml
compiler. It focuses on the implementation in OCaml, and does not
contain any correctness argument for the program transformation --
a major missing piece for a high-quality publication in an
international conference.

The majority of the presentation in the present paper is new. The
following parts are reused from the previous paper --
integrated by the same author who originally wrote them:
\begin{itemize}
\item Two thirds of the introduction (the first two pages).
\item \cref{subsec:ocaml-examples} on examples of TRMC functions in
  OCaml (1.5 pages).
\end{itemize}
We thus estimate the reused content at about 3.5 pages, the rest of
the content was written from scratch for the present submission.

The major scientific difference between the two papers is the
correctness proof in Iris, which is a significant contribution of its
own. To our knowledge, it is the first mechanized proof of correctness
for the tail-modulo-constructor program transformation, and the first
verification of a program transformation (for all inputs) on top of
\Simuliris. This constitutes the majority of the content of the
current presentation.

More minor differences include an analysis of the adoption of TMC in
OCaml libraries post-upstreaming, and a comparison with the Koka
work~\citep*{tmc-koka-2023} which was developed simultaneously with
ours.

\subsection{Motivating our choice.}

An alternative that we of course considered would be to present only
the correctness proof for TMC in the present paper, without reusing
any contributions and presentation from the work-in-progress,
implementation-focused paper.

This approach would however result in a ``small'' publication about
the (non-trivial) implementation work in the OCaml compiler, and
a more prestigious international submission for the correctness proof
alone. It would deprive from symbolic/academic reward the contributors
who lead the implementation and upstreaming work, without which none
of the present work would exist in the first place. (This approach
would also weaken national conferences, by making researchers think
twice about submitting implementation-focused work in progress.)

This is in fact a common dynamic in programming-language research that
combines delicate implementation work with delicate meta-theory
work. The latter is more easily identified as challenging research,
easier to evaluate for quality and novelty, and tends to gather more
scientific credit. The implementation work also typically comes
before, and it is natural to do the verification work separately, with
a different (sometimes entirely disjoint) group of contributors. We
are trying to avoid this dynamic by grouping the two sorts of
contributions together in the same publication.

%%% Local Variables:
%%% mode: latex
%%% TeX-master: "main"
%%% End:

\end{version}

\section{More on TMC in \OCaml}
\label{app:more-ocaml}

\subsection{Alternatives}
\label{subsec:alternative-impls}

We mention other approaches than TMC to solve the problem of overflowing the call stack, and explain why they were less suitable for \OCaml. We also discuss a significant implementation change between \OCaml 4 and \OCaml 5.

\subsubsection{Doing a CPS transformation}

Instead of a program transformation in \emph{destination-passing} style, we could perform a more general program transformation that can make more functions tail-recursive, for example a generic \emph{continuation-passing} style (CPS) transformation.

We have three arguments for implementing the TMC transformation:
\begin{itemize}
\item The TMC transformation generates more efficient code, using mutation instead of function calls. On the \OCaml runtime, the difference is a large constant factor.\footnote{On a toy benchmark with large-sized lists, the CPS version is 100\% slower and has 130\% more allocations than the non-tail-recursive version.}

\item The CPS transformation can be expressed at the source level, and can be made reasonably nice-looking using some monadic-binding syntactic sugar. TMC can only be done by the compiler, or using safety-breaking features.
\end{itemize}

\subsubsection{Lazy data}

Lazy (call-by-need) languages will also often avoid running into stack overflows: as soon as a lazy data structure is returned, which is the default, functions such as \ocaml{map} will return immediately, with recursive calls frozen in a lazy thunk, waiting to be evaluated on-demand as the user traverses the result structure.
Users still need to worry about tail-recursion for their strict functions; strict functions are often preferred when writing efficient code.

\subsubsection{Unlimiting the system stack}

Some operating systems can provide an unlimited system stack; such as \texttt{ulimit -s unlimited} on Linux systems -- the system stack is then resized on-demand.
Frustratingly, unlimited stacks are not available on all systems, and not the default on any system in wide use.
Convincing all users to setup their system in a non-standard way would be \emph{much} harder than performing a program transformation or accepting the CPS overhead for some programs.

\subsubsection{Using another stack}

Using the native system stack is a choice of the \OCaml~4 implementation.
Some other implementations of functional languages, such as SML/NJ, use a different stack (the \OCaml bytecode interpreter does this), or directly allocate stack frames on their GC-managed heap.
This approach can make ``stack overflow'' go away completely, and it also makes it very simple to implement stack-capture control operators, such as continuations, or other stack operations such as continuation marks.

On the other hand, using the native stack brings compatibility benefits (coherent stack traces for mixed \OCaml+C programs), and seems to improve the performance of function calls (on benchmarks that are only testing function calls and return, such as Ackermann or the naive Fibonacci, \OCaml can be 4x, 5x faster than SML/NJ.)

\paragraph{\OCaml~5}

\OCaml~5.0.0 was released in December 2022, about a year after we landed our TMC implementation in the \OCaml~4 compiler.
\OCaml~5 uses a different runtime to support multicore programming, and a different calling convention to support algebraic effects.
In particular, it only uses the system stack for C calls, and a ``cactus stack'' for \OCaml calls.

\OCaml~5 manages its own stack, but the implementors still decided to have a stack limit.
It is sensibly higher by default (1Gb at the time of writing) than the system stack (8Mb), so many ``small'' can now use non-tail-recursive functions without fears of stack overflows, but overflows remain possible and likely in practice on large inputs.
The reason to keep a (user-settable) limit for the \OCaml call stack is usability in the face of buggy programs.
When programmers write recursive functions, they occasionally go through incorrect versions with a faulty base case that fail to terminate.
In this case, we want the system to fail quickly with an error, instead of remaining silent for a few minutes, crashing once all the machine memory has been consumed.

We were curious about whether users would still see a need for TMC in \OCaml~5, or whether they would stop using the feature in practice.
The feedback we got from expert users is that stack overflows is still an issue they worry about.
We observe that TMC adoption in \OCaml code bases keep growing after the transition to \OCaml~5.

\subsection{Design Choices}
\label{subsec:design-choices}

\subsubsection{Opt-in Rather than Opt-out}\label{subsec:optin}
We made our transformation \emph{opt-in}, the user has to explicit use
the \ocaml{[@tail_mod_cons]}, rather than \emph{opt-out}, where the
compiler would automatically detect program in the TMC fragment to
transform them. There are at least three reasons for this choice, in
increasing order of importance. First, the transformation duplicates
code, as it generates a direct- and a destination-passing-style
version of the function; in particular we could suffer from
a code-size blow-up on nested transformable functions. Second, the
transformation may change the evaluation order of function arguments,
in a way that conforms to the OCaml specification but may break
programs in practice (bad!), so we would have to restrict it to only
transform calls with syntactically-pure arguments. Third and most
importantly, our implementation may have bugs, which may silently
introduce miscompilations in user code. There are millions of lines of
OCaml code out there, and we have no idea which portion would be
transformed. Doing this implicitly would risk running into
compile-time, runtime or correctness issues that we have not foreseen,
it would have been much harder to convince OCaml maintainers to
include this feature if it was opt-out.

\subsubsection{Resolving Non-determinism}

The main question faced by an implementation is how to resolve the
inherent non-determinism in the rewrite relation -- how to decide
between several possible results of the transformation. We identified
fairly different kinds of non-determinism.

\paragraph{Choices with an obviously better alternative.}
Some choices offer an alternative that is obviously better than the others, because it allows to strictly improve more programs. For example, some implementations of TMC limit themselves to calls that are immediately inside a constructor: they would optimize the recursive call to \ocaml|f| in \ocaml|Cons(x, f y)| but not, for example, in \ocaml|Cons(x1, Cons(x2, f y))|, or in \ocaml|Cons (x, if p then f y else z)|.

In the terms of \cref{subsec:specification}, those implementations only allow to optimize calling contexts of the form $\datalangTailCtx{[\datalangConsFrame]}$ or $\datalangTailCtx{[\datalangConsCtx]}$. Our implementation supports the more general situation $\datalangTMCCtx = (\datalangTailFrame | \datalangConsFrame)^{*}$. Both approaches are included in our non-deterministic relation -- we prove them both correct -- but optimizing more calls is obviously better.

\paragraph{Choices with a subtly better alternative.} In some cases it
is possible to put a bit more work in the choice heuristics, to
generate slightly better code, in terms of performance or
readability. For example, some natural way to simplify the
implementation would result in more subterms being transformed in
destination-passing-style, only to finally use the
\RefTirName{DPSBase} rule without any DPS function call. The generated
code is slightly less pleasant to read, and in our experience it can
be noticeably slower. (We detail such a situation in \cref{subsec:constructor-compression}.)

Code readability is important in our context of an on-demand program
transformation used by performance experts: those experts will often
read the intermediate representations produced by the compiler to
check that their performance assumptions hold.

\paragraph{Incomparable choices: force the user to decide} \label{subsec:user-control}
The remaining choices are between incomparable alternatives, that could
each be better than the other depending on the specific program or
programmer intent. Consider again our binary tree example,
\ocaml|Node(map f left, map f right)|: we could make either the first
or the second argument a tail-call.

Our policy in such cases is to let the user decide, by providing
control over the transformation choices using \ocaml{[@tailcall]}
program annotations, and to force the user to decide by raising an
error in case of ambiguity:
\begin{Ocaml}
  | Cifthenelse(cond, ifso, ifnot) ->
      Cifthenelse(cond, map_tail f ifso, map_tail f ifnot)
      /[^^^^^^^^^^^^^^^^^^^^^^^^^^^^^^^^^^^^^^^^^^^^^^^^^^^^]/
/[Error]/: "[@tail_mod_cons]": this constructor application may be TMC-transformed
       in several different ways. Please disambiguate by adding an explicit
       "[@tailcall]" attribute to the call that should be made tail-recursive,
       or a "[@tailcall false]" attribute on calls that should not be
       transformed.
\end{Ocaml}

This approach is incompatible with a view of the TMC
transformation as an implicit optimization, applied whenever
possible -- in that case one would rather have the compiler make
arbitrary choices rather than fail. We rather view TMC as a tool for
expert users to better reason about program performance. It should be
predictable and flexible (let the user express
their intent). Failing due to the lack of annotations is an acceptable way
to drive user interaction.

\subsubsection{First-order Implementation}\label{subsubsec:first-order}
A notable limitation of the \OCaml implementation is that it is first-order and non-modular in nature: only direct calls to known functions can be converted into DPS style, and the availability of a DPS variant is not exposed through module abstractions.
It would be possible to allow DPS calls through external modules or higher-order function parameters, by annotating interfaces and function arguments with a \ocaml{[@tail_mod_cons]} annotation, and elaborating them into a pair of functions, the direct-style and the DPS version.

\subsection{Implementation history}
\label{subsec:PR-history}

We first proposed adding TMC as an optional program transformation to
the \OCaml compiler in
May 2015.\footnote{\nonanon{URL}{\url{https://github.com/ocaml/ocaml/pull/181}}}
The proposal was received favorably, but it never received an in-depth
review and a detailed performance evaluation and remained unmerged for
years.

We restarted the implementation in 2020, resulting in a new pull July 2020 request with a modified implementation, and a careful performance
evaluation\footnote{\nonanon{URL}{\url{https://github.com/ocaml/ocaml/pull/9760}}}. The
new implementation put a larger focus on producing readable code,
giving more control to the user through annotations. It also removed
an optimization of the previous implementation that would specialize
the DPS version, generating a distinct definition for each block
offset. The new design was carefully reviewed by Basile Clément and
Pierre Chambart, and was finally merged in November 2021, available to
users with \OCaml~4.14, released in March 2022.

The TMC transformation is that it adds extra parameters to functions
(two parameters, the block and the offset). At the time the \OCaml
compiler had a limitation on some supported architectures, where it
would not optimize tail calls above a certain number of arguments
(enough that they cannot be all passed by registers), breaking the
tail-call promises of TMC on those systems. Xavier Leroy implemented
a change to the \OCaml calling convention for those architectures in
May 2021,\footnote{\url{https://github.com/ocaml/ocaml/pull/10595}},
which was motivated by the TMC work.

\subsection{Implementation}
\label{subsec:implementation}

Implementing the TMC transform is not an obvious top-down or bottom-up traversal, because it relies on information flowing in two opposite directions: the transformation is \emph{possible} for subterms that are in tail-or-constructor position relative to the root of the function (top-down information), and it is \emph{desirable} for subterms whose leaves contain calls to TMC-transformed functions (bottom-up information). A naive approach is to perform a recursive traversal that tracks the top-down context and, on each subterm, recursively traverses the whole subterm to check desirability; this has quadratic time complexity, which is best avoided in production compilers.

Instead of trying to transform each subterm in a single pass, our implementation computes for each subterm a ``choice'', a summary of all the bottom-up information that is relevant to choose how to transform this subterm -- once we have the top-down information available. This includes (1) the direct-style transform of the subterm, (2) the DPS-style transform of the subterm, (3) metadata tracking whether the transformation is beneficial (if a TMC-function call was found in tail-modulo-cons position), the list of TMC calls it contains (for error diagnostics), and whether some were explicitly requested by the user (for disambiguation). Computing transformed terms is not compositional; computing choices is compositional. We compute a choice for the whole function body (in one pass), from which we can directly extract the direct-style and DPS-style transformations of the function.

\paragraph{An applicative functor}
\label{subsec:applicative-impl}

The computation of those choices described in \cref{subsec:implementation} can be expressed elegantly. Instead of a monomorphic type \ocaml|Choice.t| that represents a choice of how to transform a term, we use a polymorphic type \ocaml|'a Choice.t| that represents how to transform zero, one or several subterms that occur in the source; for example transforming two subterms in the same context relies on a choice of type \ocaml|(lambda * lambda) Choice.t|, where \ocaml|lambda| is the type of terms in the intermediate representation we are working on. This \ocaml|'a Choice.t| type can be equipped with an applicative functor interface:

\begin{minipage}{0.4\linewidth}
\begin{Ocaml}
module Choice : sig
  type 'a t = {
    dps : 'a Dps.t;
    direct : unit -> 'a;
    tmc_calls : tmc_call_info list;
    benefits_from_dps : bool;
    explicit_tailcall_request : bool;
  }





end
\end{Ocaml}
\end{minipage}
\hfill
\begin{minipage}{0.5\linewidth}
\begin{Ocaml}
  (* construct a choice from an arbitrary term *)
  val lambda : lambda -> lambda t

  (* applicative functor interface *)
  val unit : unit t
  val map : ('a -> 'b) -> 'a t -> 'b t
  val pair : 'a t * 'b t -> ('a * 'b) t

  (* extract the direct-style and DPS transforms *)
  val direct : lambda t -> lambda
  val dps :
    lambda t -> tail:bool -> dst:offset dst ->
    lambda
\end{Ocaml}
\end{minipage}

With this interface, the easy cases of the transformation can be expressed compactly:
\begin{Ocaml}
  let rec choice ctx ~tail t =
    match t with
    [...]
    | Lifthenelse (l1, l2, l3) ->
        let l1 = traverse ctx l1 in
        let+ l2 = choice ctx ~tail l2
        and+ l3 = choice ctx ~tail l3
        in Lifthenelse (l1, l2, l3)
    [...]
\end{Ocaml}
The binding operators \ocaml|let+|, \ocaml|and+| are standard syntax for applicative-like structures in \OCaml: \ocaml|let+| desugars to \ocaml|Choice.map| and \ocaml|and+| desugars to \ocaml|Choice.pair|. \ocaml|traverse| performs a direct-style translation, and \ocaml|traverse_letrec| may also transform \ocaml|let|-bound functions marked with \ocaml|[@tail_mod_constr]| and add them to the local transformation environment.

\subsection{Evaluation: adoption}
\label{subsec:adoption}

% [@tail_mod_cons] usage on Github

% Search URL:
%   https://github.com/search?q=tail_mod_cons+lang%3Aocaml&type=code

% Results (November 13th 2023):
%   - in the OCaml stdlib
%     + Melange
%       (reused from stdlib)
%     + janestreet/base
%       (with manual unfolding)
%   - in other general utility modules
%     https://github.com/RedPRL/asai/blob/ac523674579772e5929c8d912d407b8b7db89c74/src/Utils.ml#L33
%     https://github.com/jonathan-laurent/KaTie/blob/2db0d2cf223fae003b26bdf4c82a9788b5804bc7/src/utils.ml#L95
%     https://github.com/jamsidedown/ocaml-cll/blob/394effa65b5d3f883e2de8322ed60fb7aa495bec/examples/crab_cups.ml#L61
%   - in other user code
%     at list type:
%       https://github.com/brendanzab/language-garden/blob/9cc21e2f57228556cf0b867ca2874ceb4a4250ec/compile-arith/lib/StackLang.ml#L63
%       https://github.com/abella-prover/abella/blob/ceee13822001137898ca5de9c76a315da3d1f0e3/src/extensions.ml#L421
%       https://github.com/bikallem/spring/blob/6d10fef0b21caea3f975ff13132f5994e7c37db5/lib_spring/response.ml#L99
%       https://github.com/bikallem/spring/blob/6d10fef0b21caea3f975ff13132f5994e7c37db5/lib_spring/headers.ml#L150
%       https://github.com/polytypic/io/blob/dbe4d37a98ef958178fe18bb4dae396a5537392e/src/io/io.ml#L8
%       https://github.com/avsm/eeww/blob/5b6c0617306f4c9d8b44bf07ef4d45ec9668dd0c/lib/cohttp/cohttp-eio/src/rwer.ml#L43
%       https://github.com/just-max/less-power/blob/96f8e88f72275246b6bc175e44c732aa6e08ce7f/src/common/path_util.ml#L18
%       https://github.com/dalps/ocaml-challenge/blob/8fe115023457b6b4359ef736c78ba980cd871717/3/extract/solve.ml#L4
%       https://github.com/Octachron/ocaml-changelog-analyzer/blob/d6757b9576b5070ea3cfe6d4f4f79aaa3cc6e5d4/parse/release.ml#L4
%       https://github.com/Skyb0rg007/PL-Reading-Group/blob/7002ae35ba69fb9010e5049eee88ab475bc79ec3/list-optimizations/lib/adt.ml#L44
%       https://github.com/jfeser/symetric/blob/f49a57afc90a2c1e38f1097f98bd74725c074b9b/lib/tower.ml#L75
%       https://github.com/verse-lab/sisyphus/blob/a08addae06bbccb1e6ea68bac3d7f9eeea4197be/lib/dynamic/utils.ml#L65
%       https://github.com/jaymody/ocaml-tokenizers/blob/06cbce75d2865690cb39017222efff5c284bc721/lib/utils.ml#L18
%       https://github.com/Bannerets/ocaml-kdl/blob/687c404ebeceef7fd905935c23665294133f99e6/src/lens.ml#L83
%       https://github.com/OCADml/OCADml/blob/f5dfbf55f9c19ad808daa0df4d76a55e58756e5f/lib/poly3.ml#L33
%     other type
%       https://github.com/johnridesabike/acutis/blob/4113fb516d9c5dd6f2dbfd9657f52c5ae7a5dcae/lib/matching.ml#L161
%         (on-demand stream as a record of functions)
%       https://github.com/RedPRL/ocaml-bwd/blob/fbf496b29532085b38073eaa62ba3d22ac619d5d/src/BwdNoLabels.ml#L60
%         (snoc list)
%       https://github.com/Skyb0rg007/PL-Reading-Group/blob/7002ae35ba69fb9010e5049eee88ab475bc79ec3/effects/lib/free/tseq.ml#L44
%         (difference list gadt)

%   - Misuse
%     https://github.com/chengsun/simd-sexp/commit/e714a8205c6df512d920a31a4dd2448975703c55
%       (already tail-recursive)
%     https://github.com/gridbugs/llama/blob/96d1c96c31ff136f465b5f37c981fff591bac6fd/src/midi/byte_array_parser.ml#L50
%       (needless complexity)
%     https://github.com/LimitEpsilon/lambda-interpreter/blob/ab8e81de7f896aa64eba14e23d964b9705e94161/lib/evaluate.ml#L46
%       (already tail-recursive)

The TMC transformation was first made available to users in \OCaml~4.14.0, released in March 2022.
At the time there were no users of the feature.
Notably, the \OCaml standard library had intentionally not been modified to start using it: before the release, we did not want the stdlib code base to depend on a not-available-yet feature, and after the release we decided not to push for adoption (we may be biased about the benefits/importance of TMC), and let other contributors propose its use, after doing their own evaluation of performance impact and code-clarity benefits.

Over time, \ocaml|[@tail_mod_cons]| was adopted in a few places in the \OCaml standard library, either to make some functions tail-recursive that previously were not, or simplify some complex tail-recursive implementations. As of spring 2024, the feature is now used in \ocaml|Hashtbl.find_all| and in \ocaml|List| module (\ocaml|init, map, mapi, map2, find_all, filteri, filter_map, take, take_while, of_seq|). \ocaml|init| and the \ocaml|map*| functions, which are the most heavily used, were unrolled once -- the minimal amount observed to preserve the non-tailrec performance for small lists. Other functions are written in the simplest possible way. (All \texttt{stdlib} uses so far build lists rather than another datatype.)

We performed a systematic search for \ocaml|[@tail_mod_cons]| usage on November 2023, and we found that it was also used
\begin{itemize}
\item in a few utility modules (general-purpose functions designed to extend or replace the standard library), notably in Jane Street's \texttt{base} library.
\item in the middle of domain-specific user code, to build lists, in 14 different projects
\item in the middle of user code, to build other types than lists, in three projects:
  \begin{itemize}
  \item in \texttt{RedPRL}\footnote{\url{https://github.com/RedPRL/ocaml-bwd/blob/fbf496b29532085b38073eaa62ba3d22ac619d5d/src/BwdNoLabels.ml\#L60}},
    it is used to build snoc lists
  \item in a project named \texttt{PL-reading-group}\footnote{\url{https://github.com/Skyb0rg007/PL-Reading-Group/blob/7002ae35ba69fb9010e5049eee88ab475bc79ec3/effects/lib/free/tseq.ml\#L44}},
    it is used to build a GADT of difference lists
  \item in \texttt{acutis}\footnote{\url{https://github.com/johnridesabike/acutis/blob/4113fb516d9c5dd6f2dbfd9657f52c5ae7a5dcae/lib/matching.ml\#L161}},
    in a group of mutually-recursive functions that builds a complex AST structure used options and nested records
  \end{itemize}
\end{itemize}

We also found three cases where the annotation is (in our opinion) misused: in two cases, the function is already tail-recursive, so there is no need for the annotation, and in a third case\footnote{\url{https://github.com/gridbugs/llama/blob/96d1c96c31ff136f465b5f37c981fff591bac6fd/src/midi/byte_array_parser.ml\#L50}} we consider that the use is gratuitous and can be replaced by already-available standard library functions.

%%% Local Variables:
%%% mode: latex
%%% TeX-master: "main"
%%% End:


\section{Benchmark results on \OCaml~4}

The benchmark results for \OCaml~4 are given in \cref{fig:bench4}. The
results are very close to \cref{fig:bench5}, and in particular the
qualitative analysis of the results is mostly unchanged. (We used
\texttt{ulimit -s unlimited} to disable the system stack limit, to run
the non-tail-recursive version on large lists.)

\begin{figure}[tp]
\def\svgscale{0.8}
\graphicspath{{plots/}}
\input{plots/plot.4.pdf_tex}
\caption{\ocaml|List.map| benchmark on \OCaml~4.14}
\label{fig:bench4}
\end{figure}

%%% Local Variables:
%%% mode: latex
%%% TeX-master: "main"
%%% End:


\end{document}
