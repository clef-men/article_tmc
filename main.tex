\input{\jobname.cfg}
\documentclass[acmsmall,screen]{acmart} % TODO: remove [review,draft] in the final version
\settopmatter{printfolios=true,printccs=true,printacmref=true}
% \renewcommand\footnotetextcopyrightpermission[1]{} % removes footnote with conference information in first column

% ---------------------------------------------------------

\usepackage[T1]{fontenc}
\usepackage[utf8]{inputenc}
\usepackage[scaled=0.8]{beramono}

% ---------------------------------------------------------

\usepackage{boolean}
\usepackage{reversion}

\newcommand{\Anonymous}{\FALSE}

\newcommand{\nonanon}[2]{\Anonymous{#1 removed for anonymity.}{#2}}

% ---------------------------------------------------------

\usepackage{mybiblio}

% ---------------------------------------------------------

\usepackage{enumitem}

% ---------------------------------------------------------

\usepackage{mathtools}

\let\oldexists\exists
\let\exists\relax\DeclareMathOperator{\exists}{\oldexists}
\let\oldforall\forall
\let\forall\relax\DeclareMathOperator{\forall}{\oldforall}

\DeclareMathOperator{\lambdaAbs}{\lambda}
\DeclareMathOperator{\muAbs}{\mu}
\DeclareMathOperator{\nuAbs}{\nu}

% ---------------------------------------------------------

\usepackage{amsthm}

\newenvironment{myproof}[1][\proofname]{%
  \begin{proof}[#1]$ $\par\nobreak\ignorespaces
}{%
  \end{proof}
}

% ---------------------------------------------------------

\usepackage{bm}

% ---------------------------------------------------------

\usepackage{mathpartir}
\let\TirName\textsc
\renewcommand{\TirNameStyle}[1]{\hypertarget{#1}{\TirName{#1}}}
\newcommand{\RefTirName}[1]{\hyperlink{#1}{\TirName{#1}}\xspace}

\newenvironment{mathline}
   {\abovedisplayskip 0.2em
    \belowdisplayskip 0.2em
    \begin{mathpar}}
   {\end{mathpar}}

% ---------------------------------------------------------

\usepackage[capitalize,noabbrev,nameinlink]{cleveref}

% ---------------------------------------------------------
\usepackage{mycomments}
% % uncomment the following command to hide comments
\UNXXX

% ---------------------------------------------------------

\usepackage{macros}

\usepackage{listings/ocaml}
\usepackage{listings/datalang}

% ---------------------------------------------------------

%% Copyright information
%% Supplied to authors (based on authors' rights management selection;
%% see authors.acm.org) by publisher for camera-ready submission;
%% use 'none' for review submission.

\setcopyright{rightsretained}
\acmDOI{10.1145/3704915}
\acmJournal{PACMPL}
\acmYear{2025}
\acmVolume{9}
\acmNumber{POPL}
\acmArticle{79}
\acmMonth{1}
\received{2024-07-11}
\received[accepted]{2024-11-07}

% ---------------------------------------------------------

% Reduce noise in .log file.

% https://tex.stackexchange.com/questions/655563/what-is-the-meaning-of-no-math-alphabet-change-to-frozen-version-normal-on-inpu

\makeatletter
\def\@font@info#1{}
\makeatother

% ---------------------------------------------------------
% ---------------------------------------------------------

\begin{document}

% ---------------------------------------------------------

\title{Tail Modulo Cons, \OCaml, and Relational Separation Logic}

\author{Clément Allain}
\orcid{0009-0005-2972-5181}
\affiliation{%
  \institution{Inria}
  \city{Paris}
  \country{France}
}
\email{clement.allain@inria.fr}

\author{Frédéric Bour}
\orcid{0009-0007-5268-5784}
\affiliation{%
  \institution{Tarides}
  \city{Paris}
  \country{France}
}
\email{frederic.bour@lakaban.net}

\author{Basile Clément}
\orcid{0000-0002-9126-0937}
\affiliation{%
  \institution{OCamlPro}
  \city{Paris}
  \country{France}
}
\email{bc@ocamlpro.com}

\author{François Pottier}
\orcid{0000-0002-4069-1235}
\affiliation{%
  \institution{Inria}
  \city{Paris}
  \country{France}
}
\email{francois.pottier@inria.fr}

\author{Gabriel Scherer}
\orcid{0000-0003-1758-3938}
\affiliation{%
  \institution{Inria}
  \city{Paris}
  \country{France}
}
\affiliation{%
  \institution{IRIF, Université Paris Cité}
  \city{Paris}
  \country{France}
}
\email{gabriel.scherer@inria.fr}

\begin{CCSXML}
<ccs2012>
   <concept>
       <concept_id>10011007.10011006.10011041</concept_id>
       <concept_desc>Software and its engineering~Compilers</concept_desc>
       <concept_significance>500</concept_significance>
       </concept>
   <concept>
       <concept_id>10011007.10011006.10011008.10011024.10011033</concept_id>
       <concept_desc>Software and its engineering~Recursion</concept_desc>
       <concept_significance>500</concept_significance>
       </concept>
   <concept>
       <concept_id>10003752.10003790.10011742</concept_id>
       <concept_desc>Theory of computation~Separation logic</concept_desc>
       <concept_significance>500</concept_significance>
       </concept>
   <concept>
       <concept_id>10003752.10010124.10010138.10010142</concept_id>
       <concept_desc>Theory of computation~Program verification</concept_desc>
       <concept_significance>500</concept_significance>
       </concept>
 </ccs2012>
\end{CCSXML}

\ccsdesc[500]{Software and its engineering~Compilers}
\ccsdesc[500]{Software and its engineering~Recursion}
\ccsdesc[500]{Theory of computation~Separation logic}
\ccsdesc[500]{Theory of computation~Program verification}

\begin{abstract}
    Common implementations of functional programming languages incentivize tail-recursive function definitions, as opposed to general recursive functions that consume stack space and may not scale to large inputs.
%
This distinction occasionally requires writing functions in a tail-recursive style that may be more complex and slower than the natural, non-tail-recursive definition.

This work describes our implementation of the \emph{tail modulo constructor} (TMC) transformation in the OCaml compiler, a compilation strategy that provides stack-efficiency for a larger class of functions -- tail-recursive \emph{modulo constructors} -- which include in particular the natural definition of \ocaml{List.map} and many similar recursive data-constructing functions.

We prove the correctness of this program transformation in a simplified setting -- a small untyped calculus -- that captures the salient aspects of the OCaml implementation. Our proof is mechanized in the Coq proof assistant, using the Iris base logic. An independent contribution of our work is a extension of the Simuliris approach to define simulation relations that support different calling conventions. To our knowledge, this is the first use of a simulation relation defined in Iris to prove the correctness of a compiler transformation.

\end{abstract}

\titlenote{
\Appendices{
\textbf{Appendices included}: This is a long version of the present paper, with appendices.
}{
\textbf{Appendices missing}: A long version of this paper, with appendices, is available online.
}
}

\maketitle

\keywords{compilation, separation logic, program verification}

\begin{CCSXML}
<ccs2012>
<concept>
<concept_id>10003752.10003790.10011742</concept_id>
<concept_desc>Theory of computation~Separation logic</concept_desc>
<concept_significance>500</concept_significance>
</concept>
<concept>
<concept_id>10003752.10010124.10010138.10010142</concept_id>
<concept_desc>Theory of computation~Program verification</concept_desc>
<concept_significance>500</concept_significance>
</concept>
</ccs2012>
\end{CCSXML}

\ccsdesc[500]{Theory of computation~Separation logic}
\ccsdesc[500]{Theory of computation~Program verification}

% ---------------------------------------------------------

\newcommand{\separate}{\FALSE}

\section{Introduction}

describe TMC

OCaml compiler

Daan Leijen

annoncer éléments décisifs (specs dir/dps) ?

contributions :
generalization of Simuliris simulation (more calling conventions)
mechanized proof of soundness in separation logic
intuitive specification (direct and DPS calling conventions)
global optimization

\begin{figure}[tp]
\begin{OCaml}
let rec map f xs =
  match xs with
  | [] ->
      []
  | x :: xs ->
      let y = f x in
      y :: map f xs
\end{OCaml}
\caption{\ocaml|map| in direct style (\OCamlLang)}
\label{fig:map}
\end{figure}
\begin{figure}[tp]
\begin{OCaml}
let rec rev_map f ys xs =
  match xs with
  | [] ->
      ys
  | x :: xs ->
      let y = f x in
      rev_map f (y :: ys) xs
      
let map f xs =
  let ys = rev_map f [] xs in
  rev ys
\end{OCaml}
\caption{\ocaml|map| in accumulator-passing style (\OCamlLang)}
\label{fig:rev_map}
\end{figure}
\begin{figure}[tp]
\begin{OCaml}
(* [dst] points to a partially initialized list cell. *)
let rec map_dps dst f xs =
  match xs with
  | [] ->
      (* Write our result, the empty list, to [dst]. *)
      set_field dst 1 []
  | x :: xs ->
      let y = f x in
      (* Allocate a new partially initialized list cell. *)
      let dst' = y :: [] in
      (* Write our result, [dst'], to [dst]. *)
      set_field dst 1 dst' ;
      (* Continue with [dst'] as new destination. *)
      map_dps dst' f xs

let map f xs =
  match xs with
  | [] ->
      []
  | x :: xs ->
      let y = f x in
      let dst = y :: [] in
      map_dps dst f xs ;
      dst
\end{OCaml}
\caption{\ocaml|map| in destination-passing style (\OCamlLang)}
\label{fig:map_dps}
\end{figure}


\separate{\clearpage}{}
\section{TMC on an idealized language}
\label{sec:formalization}

In this section, we formalize the ``tail modulo cons'' (TMC) transformation in an idealized language, \DataLang, that is expressive enough to account for the main aspects of TMC but does not support all features of \OCaml.
Our proof of correctness covers this idealized fragment.
We intentionally keep the presentation very close to our Coq development, which can be referred to for full details.

\begin{figure}[tp]
    \begin{tabular}{lclcl}
            $\datalangIdx[]$
            & $\ni$ &
            $\datalangIdx$
            & $\Coloneqq$ &
            $\datalangZero \mid \datalangOne \mid \datalangTwo$
        \\
            $\datalangTag[]$
            & $\ni$ &
            $\datalangTag$
        \\
            $\datalangBool[]$
            & $\ni$ &
            $\datalangBool$
    	\\
    		$\datalangLoc[]$
    		& $\ni$ &
    		$\datalangLoc$
        \\
            $\datalangFn[]$
            & $\ni$ &
            $\datalangFn$
        \\
            $\datalangVar[]$
            & $\ni$ &
            $\datalangVar, \datalangVarTwo$
    	\\
            $\datalangVal[]$
            & $\ni$ &
            $\datalangVal$
            & $\Coloneqq$ &
            $\datalangUnit \mid \datalangIdx \mid \datalangTag \mid \datalangBool \mid \datalangLoc \mid \datalangFnptr{\datalangFn}$
        \\
            $\datalangExpr[]$
            & $\ni$ &
            $\datalangExpr$
            & $\Coloneqq$ &
            $\datalangVal$
        \\
            &&
            & | &
            $\datalangVar \mid \datalangLet{\datalangVar}{\datalangExpr_1}{\datalangExpr_2} \mid \datalangCall{\datalangExpr_1}{\datalangExpr_2}$
        \\
            &&
            & | &
            $\datalangEq{\datalangExpr_1}{\datalangExpr_2}$
        \\
            &&
            & | &
            $\datalangIf{\datalangExpr_0}{\datalangExpr_1}{\datalangExpr_2}$
        \\
            &&
            & | &
            $\datalangBlock{\datalangTag}{\datalangExpr_1}{\datalangExpr_2} \mid \datalangBlockDet{\datalangTag}{\datalangExpr_1}{\datalangExpr_2}$
        \\
            &&
            & | &
            $\datalangLoad{\datalangExpr_1}{\datalangExpr_2} \mid \datalangStore{\datalangExpr_1}{\datalangExpr_2}{\datalangExpr_3}$
        \\
            $\datalangEctx[]$
            & $\ni$ &
            $\datalangEctx$
            & $\Coloneqq$ &
            $\Box$
        \\
            &&
            & | &
            $\datalangLet{\datalangVar}{\datalangEctx}{\datalangExpr_2} \mid \datalangCall{\datalangExpr_1}{\datalangEctx} \mid \datalangCall{\datalangEctx}{\datalangVal_2}$
        \\
            &&
            & | &
            $\datalangEq{\datalangExpr_1}{\datalangEctx} \mid \datalangEq{\datalangEctx}{\datalangVal_2}$
        \\
            &&
            & | &
            $\datalangIf{\datalangEctx}{\datalangExpr_1}{\datalangExpr_2}$
        \\
            &&
            & | &
            $\datalangLoad{\datalangExpr_1}{\datalangEctx} \mid \datalangLoad{\datalangEctx}{\datalangVal_2}$
        \\
            &&
            & | &
            $\datalangStore{\datalangExpr_1}{\datalangExpr_2}{K} \mid \datalangStore{\datalangExpr_1}{\datalangEctx}{\datalangVal_3} \mid \datalangStore{\datalangEctx}{\datalangVal_2}{\datalangVal_3}$
        \\
            $\datalangDef[]$
            & $\ni$ &
            $\datalangDef$
            & $\Coloneqq$ &
            $\datalangRec{\datalangVar}{\datalangExpr}$
        \\
            $\datalangProg[]$
            & $\ni$ &
            $\datalangProg$
            & $\coloneqq$ &
            $\datalangFn[] \finmap \datalangDef[]$
        \\
            $\datalangState[]$
            & $\ni$ &
            $\datalangState$    
            & $\coloneqq$ &
            $\datalangLoc[] \finmap \datalangVal[]$
        \\
            $\datalangConfig[]$
            & $\ni$ &
            $\datalangConfig$
            & $\coloneqq$ &
            $\datalangExpr[] \times \datalangState[]$
    \end{tabular}
    \caption{\DataLang syntax}
    \label{fig:syntax}
\end{figure}
\begin{figure}[tp]
    \begin{tabular}{rcl}
            $\datalangSeq{\datalangExpr_1}{\datalangExpr_2}$
            & $\coloneqq$ &
            $\datalangLet{\datalangVar}{\datalangExpr_1}{\datalangExpr_2}$
        \\
            $\datalangPair{\datalangExpr_1}{\datalangExpr_2}$
            & $\coloneqq$ &
            $\datalangBlock{\mathrm{PAIR}}{\datalangExpr_1}{\datalangExpr_2}$
        \\
            $\datalangLet{\datalangPair{\datalangVar_1}{\datalangVar_2}}{\datalangExpr_1}{\datalangExpr_2}$
            & $\coloneqq$ &
            $\datalangLet{\datalangVarTwo}{\datalangExpr_1}{$
        \\
            &&
            $\datalangLet{\datalangVar_1}{\datalangLoad{\datalangVarTwo}{1}}{$
        \\
            &&
            $\datalangLet{\datalangVar_2}{\datalangLoad{\datalangVarTwo}{2}}{$
        \\
            &&
            $\datalangExpr_2}}}$
    \end{tabular}~%
    \begin{tabular}{rcl}
            $\datalangRec{\datalangPair{\datalangVar_1}{\datalangVar_2}}{\datalangExpr}$
            & $\coloneqq$ &
            $\datalangRec{\datalangVarTwo}{
            \datalangLet{\datalangPair{\datalangVar_1}{\datalangVar_2}}{\datalangVarTwo}{\datalangExpr}}$
        \\
            $\datalangNil$
            & $\coloneqq$ &
            $\datalangUnit$
        \\
            $\datalangCons{\datalangExpr_1}{\datalangExpr_2}$
            & $\coloneqq$ &
            $\datalangBlock{\mathrm{CONS}}{\datalangExpr_1}{\datalangExpr_2}$
    \end{tabular}

    \begin{tabular}{rcl}
            $\datalangMatchOneline[]{\datalangExpr_0}{\datalangExpr_1}{\datalangVar}{\mathit{\datalangVar s}}{\datalangExpr_2}$
            & $\coloneqq$ &
            $\datalangLet{\datalangVarTwo}{\datalangExpr_0}{$
        \\
            &&
            $\datalangIf{\datalangEq{\datalangVarTwo}{\datalangNil}$
        \\
            &&
            $}{\datalangExpr_1$
        \\
            &&
            $}{\datalangLet{\datalangPair{\datalangVar}{\mathit{\datalangVar s}}}{\datalangVarTwo}{\datalangExpr_2}}}$
        \\
            $\datalangHole$
            & $\coloneqq$ &
            $\datalangUnit$
    \end{tabular}
    \caption{\DataLang syntactic sugar}
    \label{fig:sugar}
\end{figure}
\begin{figure}[tp]
    \begin{mathparpagebreakable}
        \inferrule*
            {}{
                \boxed{- \headStep{\datalangProg} - : \datalangConfig[] \rightarrow \datalangConfig[] \rightarrow \Prop}
            }
        \and
        \inferrule*
            {}{
                \boxed{- \step{\datalangProg} - : \datalangConfig[] \rightarrow \datalangConfig[] \rightarrow \Prop}
            }
        \\
        \inferrule*[lab=StepLet]
            {}{
                \left(
                    \datalangLet{\datalangVar}{\datalangVal}{\datalangExpr},
                    \datalangState
                \right)
                \headStep{\datalangProg}
                \left(
                    \datalangExpr{} [\datalangVar \backslash \datalangVal],
                    \datalangState
                \right)
            }
        \and
        \inferrule*[lab=StepCall]
            {
                \datalangProg{} [\datalangFn] = (\datalangRec{\datalangVar}{\datalangExpr})
            }{
                \left(
                    \datalangCall{\datalangFnptr{\datalangFn}}{\datalangVal},
                    \datalangState
                \right)
                \headStep{\datalangProg}
                \left(
                    \datalangExpr{} [\datalangVar \backslash \datalangVal],
                    \datalangState
                \right)
            }
        \\
        \inferrule*[lab=StepBlock1]
            {}{
                \left(
                    \datalangBlock{\datalangTag}{\datalangExpr_1}{\datalangExpr_2},
                    \datalangState
                \right)
                \headStep{\datalangProg}
                \left(
                    \begin{array}{l}
                        \datalangLet{\datalangVar_1}{\datalangExpr_1}{\\
                        \datalangLet{\datalangVar_2}{\datalangExpr_2}{\\
                        \datalangBlockDet{\datalangTag}{\datalangVar_1}{\datalangVar_2}}}
                    \end{array},
                    \datalangState
                \right)
            }
        \and
        \inferrule*[lab=StepBlock2]
            {}{
                \left(
                    \datalangBlock{\datalangTag}{\datalangExpr_1}{\datalangExpr_2},
                    \datalangState
                \right)
                \headStep{\datalangProg}
                \left(
                    \begin{array}{l}
                        \datalangLet{\datalangVar_2}{\datalangExpr_2}{\\
                        \datalangLet{\datalangVar_1}{\datalangExpr_1}{\\
                        \datalangBlockDet{\datalangTag}{\datalangVar_1}{\datalangVar_2}}}
                    \end{array},
                    \datalangState
                \right)
            }
        \\
        \inferrule*[lab=StepBlockDet]
            {
                \forall \datalangIdx \in \datalangIdx[], \datalangLoc + \datalangIdx \notin \dom{\datalangState}
            }{
                \left(
                    \datalangBlockDet{\datalangTag}{\datalangVal_1}{\datalangVal_2},
                    \datalangState
                \right)
                \headStep{\datalangProg}
                \left(
                    \datalangLoc,
                    \datalangState{} [\datalangLoc \mapsto \datalangTag, \datalangVal_1, \datalangVal_2]
                \right)
            }
        \and
        \inferrule*[lab=StepLoad]
            {
                \datalangState{} [\datalangLoc + \datalangIdx] = \datalangVal
            }{
                \left(
                    \datalangLoad{\datalangLoc}{\datalangIdx},
                    \datalangState
                \right)
                \headStep{\datalangProg}
                \left(
                    \datalangVal,
                    \datalangState
                \right)
            }
        \and
        \inferrule*[lab=StepStore]
            {
                \datalangLoc + \datalangIdx \in \dom{\datalangState}
            }{
                \left(
                    \datalangStore{\datalangLoc}{\datalangIdx}{\datalangVal},
                    \datalangState
                \right)
                \headStep{\datalangProg}
                \left(
                    \datalangUnit,
                    \datalangState{} [\datalangLoc + \datalangIdx \mapsto \datalangVal]
                \right)
            }
        \and
        \inferrule*[lab=StepEctx]
            {
                \left(
                    \datalangExpr,
                    \datalangState
                \right)
                \headStep{\datalangProg}
                \left(
                    \datalangExpr',
                    \datalangState'
                \right)
            }{
                \left(
                    \datalangEctx{} [\datalangExpr],
                    \datalangState
                \right)
                \step{\datalangProg}
                \left(
                    \datalangEctx{} [\datalangExpr'],
                    \datalangState'
                \right)
            }
    \end{mathparpagebreakable}
    \caption{\DataLang semantics (excerpt)}
    \label{fig:semantics}
\end{figure}
\begin{figure}[tp]
\begin{tabular}{c}
\begin{Datalang}
map |-> rec (fn, xs) =
  match xs with
  | [] ->
      []
  | x :: xs ->
      let y = fn x in
      y :: @map (fn, xs)
\end{Datalang}
\end{tabular}
\caption{Natural implementation of \datalang|map| in \DataLang}
\label{fig:map}
\end{figure}

\subsection{Language}

The syntax of \DataLang is given in \cref{fig:syntax} and its semantics in \cref{fig:semantics}.
We also introduce syntactic sugar in \cref{fig:sugar}, in particular shallow pattern-matching on lists.
Going back to our motivating example, we can define the \datalang|map| function on lists as in \cref{fig:map}.

\DataLang is an untyped sequential calculus with mutable state. A \DataLang program $\datalangProg$ is a finite mapping from function names $\datalangFn \in \datalangFn[]$ to mutually-recursive definitions $d$, which are themselves functions whose body is written $\datalangRec \datalangVar \datalangExpr$. Functions have a single parameter for simplicity, with pairs used to pass several values.

\DataLang has Booleans $b \in \{\datalangTrue, \datalangFalse\}$, an if-then-else construct, and a runtime equality test\footnote{We check physical equality on locations / pointers, and primitive equality between primitive types, similarly to the \texttt{eqv?} predicate of Scheme. Primitive values of distinct types, for example integers and booleans, are always considered different.} between (untyped) values $\datalangEq{\datalangExpr_1}{\datalangExpr_2}$.

\DataLang is first-order in the same sense that C is (with function pointers): it does not feature general lambda expressions, its programs correspond to closure-converted or lambda-lifted source programs.\footnote{
The usual definition of TMC that we implement and formalize is essentially first-order. See \cref{subsubsec:first-order}.}
Functions names $\datalangFn$ can be turned into values written $\datalangFnptr{\datalangFn}$, to be used directly in function calls or as parameters to higher-order functions.

To express constructors, \DataLang features mutable memory blocks with an abstract \emph{tag} ($\datalangTag \in \datalangTag[]$), and two \emph{fields} which are arbitrary values ($\datalangExpr_1, \datalangExpr_2$).
One can allocate a block with $\datalangBlock{\datalangTag}{\datalangExpr_1}{\datalangExpr_2}$, access its fields with $\datalangLoad{\datalangExpr_1}{\datalangExpr_2}$ and modify them with $\datalangStore{\datalangExpr_1}{\datalangExpr_2}{\datalangExpr_3}$.
Allocation returns a location $\datalangLoc \in \datalangLoc[]$, which may not appear in source programs.

The evaluation order of subexpressions $\datalangExpr_1$ and $\datalangExpr_2$ in $\datalangBlock{\datalangTag}{\datalangExpr_1}{\datalangExpr_2}$ is unspecified as in \OCaml.
This is crucial to allow the behavior-preserving optimization of more programs, as the TMC transformation may affect the evaluation order of subterms of data constructors.
To model this in the semantics, we introduce a separate, deterministic block construction $\datalangBlockDet{\datalangTag}{\datalangExpr_1}{\datalangExpr_2}$ which cannot appear in source programs. A block expression $\datalangBlock{\datalangTag}{\datalangExpr_1}{\datalangExpr_2}$ first nondeterministically reduces to either $\datalangLet{\datalangVar_1}{\datalangExpr_1}{\datalangLet{\datalangVar_2}{\datalangExpr_2}{\datalangBlockDet{\datalangTag}{\datalangVar_1}{\datalangVar_2}}}$ through \RefTirName{StepBlock1} or $\datalangLet{\datalangVar_2}{\datalangExpr_2}{\datalangLet{\datalangVar_1}{\datalangExpr_1}{\datalangBlockDet{\datalangTag}{\datalangVar_1}{\datalangVar_2}}}$ through \RefTirName{StepBlock2}. $\datalangBlockDet{\datalangTag}{\datalangVal_1}{\datalangVal_2}$ performs the allocation through \RefTirName{StepBlockDet}.

The values in \DataLang are functions $\datalangFnptr{\datalangFn}$, locations $\datalangLoc$ of allocated blocks, Booleans $b$, tags $t$ (taken in an arbitrary, denumerable set), the unit value $\datalangUnit$, and indices $i \in \{\datalangZero, \datalangOne, \datalangTwo\}$ inside blocks. (Our transformation never mutates tags in position $\datalangZero$, nor do \OCaml programs, but for example \Mezzo supports it.)

On top of these basic language features, \cref{fig:sugar} introduces syntactic sugar for pairs $\datalangPair {\datalangExpr_1} {\datalangExpr_2}$ as blocks with a specific tag $\mathrm{PAIR}$, for decomposing blocks in $\term{\color{datalangKeyword1}{let}}$-bindings ($\datalangLet {\datalangPair \datalangVar \datalangVarTwo} {\datalangExpr_1} {\datalangExpr_2}$) and in arguments of toplevel functions ($\datalangFn \mapsto \datalangRec {\datalangPair \datalangVar \datalangVarTwo} \datalangExpr$), and for (untyped) lists by defining the empty list $\datalangNil$ as $\datalangUnit$, and for (mutable) cons-cells as blocks with a specific tag $\mathrm{CONS}$. Pattern-matching on lists can be expressed by comparing the list with $\datalangNil$, using our block-deconstructing $\term{\color{datalangKeyword1}{let}}$ (which ignores the tag) to deconstruct cons-cells.

As a side note, we use named expression variables here but the \Coq mechanization actually adopts de Bruijn syntax, which is better suited to define transformations involving binders.
More precisely, our formalization relies on the \Autosubst library~\cite{autosubst-2015}.
Our definitions respect $\alpha$-equivalence on term variables $\datalangVar, \datalangVarTwo$: we implicitly assume any term variable in bound position to be chosen distinct from all other variables in context.
Function names $\datalangFn$ are not $\alpha$-renamed, as the transformation relates names in the source and target of the transformation.

\subsection{Transformation}
\label{subsec:transformation}

We now define the TMC transformation as a relation $\datalangProg_s \tmc \datalangProg_t$ between programs and their transformation.
The relation is total, in the sense that any \DataLang program $\datalangProg_s$ can be related to at least one transformed program $\datalangProg_t$.
It is not deterministic: for each input program it captures a (finite) set of admissible transformations, which we all prove valid.
This non-determinism captures several choices that have to be done by the user through a user interface to control the transformation, or by the compiler implementation, influencing performance and evaluation order of the result.
In this section, we do not describe how these choices are resolved -- there is a large design space.
We present the choices we made for \OCaml compiler in \cref{sec:implementation}.

As formalized in \cref{fig:tmc}, transforming a \DataLang program $\datalangProg$ consists in:

\begin{figure}[tp]
    \[
        \datalangProg_s \tmc \datalangProg_t
        \coloneqq
        \exists \datalangRenaming \ldotp
        \bigwedge \left[ \begin{array}{l}
                \dom{\datalangRenaming} \subseteq \dom{\datalangProg_s}
            \\
                \dom{\datalangProg_t} = \dom{\datalangProg_s} \cup \codom{\datalangRenaming}
            \\
                \forall \datalangFn, \datalangDef_s \ldotp
                \datalangProg_s [\datalangFn] = \datalangDef_s \implies
                \exists \datalangDef_t \ldotp
                \datalangProg_t [\datalangFn] = \datalangDef_t \wedge \datalangDef_s \tmcDir{\datalangRenaming} \datalangDef_t
            \\
                \forall \datalangFn, \datalangFn_\mathit{dps}, \datalangDef_s \ldotp
                \datalangProg_s [\datalangFn] = \datalangDef_s \wedge \datalangRenaming [\datalangFn] = \datalangFn_{\mathit{dps}} \implies
                \exists \datalangDef_t \ldotp
                \datalangProg_t [\datalangFn_\mathit{dps}] = \datalangDef_t \wedge \datalangDef_s \tmcDps{\datalangRenaming} \datalangDef_t
        \end{array} \right.
    \]
    \caption{TMC transformation}
    \label{fig:tmc}
\end{figure}
\paragraph{1. Choosing a subset of toplevel functions to be TMC-transformed}
In \OCaml, the programmer has to annotate these.
For each such function $\datalangFn$, we also require a fresh function name $\datalangRenaming [\datalangFn]$ (that is not defined in $\datalangProg_s$) that will be the \emph{destination-passing style} (DPS) version of $\datalangFn$ in the transformed program $\datalangProg_t$.

Formally, the subset is determined by the domain of the renaming function $\datalangRenaming$, which is passed as a parameter to the auxiliary transformations that we describe next.

\begin{figure}[tp]
    \begin{mathparpagebreakable}
        \inferrule*
            {}{
                \boxed{- \tmcDir{\datalangRenaming} - : \datalangExpr[] \rightarrow \datalangExpr[] \rightarrow \Prop}
            }
        \and
        \inferrule*
            {}{
                \boxed{- \tmcDir{\datalangRenaming} - : \datalangDef[] \rightarrow \datalangDef[] \rightarrow \Prop}
            }
        \\
        \inferrule*[lab=DirVal]
            {}{
                \datalangVal
                \tmcDir{\datalangRenaming}
                \datalangVal
            }
        \and
        \inferrule*[lab=DirVar]
            {}{
                \datalangVar
                \tmcDir{\datalangRenaming}
                \datalangVar
            }
        \and
        \inferrule*[lab=DirLet]
            {
                \datalangExpr_{s1} \tmcDir{\datalangRenaming} \datalangExpr_{t1}
            \and
                \datalangExpr_{s2} \tmcDir{\datalangRenaming} \datalangExpr_{t2}
            }{
                \datalangLet{\datalangVar}{\datalangExpr_{s1}}{\datalangExpr_{s2}}
                \tmcDir{\datalangRenaming}
                \datalangLet{\datalangVar}{\datalangExpr_{t1}}{\datalangExpr_{t2}}
            }
        \\
        \inferrule*[lab=DirBlockDPS1]
            {
                \datalangExpr_{s1} \tmcDir{\datalangRenaming} \datalangExpr_{t1}
            \and
                (\datalangVar, \datalangTwo, \datalangExpr_{s2}) \tmcDps{\datalangRenaming} \datalangExpr_{t2}
            }{
                \datalangBlock{\datalangTag}{\datalangExpr_{s1}}{\datalangExpr_{s2}}
                \tmcDir{\datalangRenaming}
                \begin{array}{l}
                    \datalangLet{\datalangVar}{\datalangBlock{\datalangTag}{\datalangExpr_{t1}}{\datalangHole}}{\\
                    \datalangSeq{\datalangExpr_{t2}}{\datalangVar}}
                \end{array}
            }
        \and
        \inferrule*[lab=DirBlockDPS2]
            {
                \datalangExpr_{s2} \tmcDir{\datalangRenaming} \datalangExpr_{t2}
            \and
                (\datalangVar, \datalangOne, \datalangExpr_{s1}) \tmcDps{\datalangRenaming} \datalangExpr_{t1}
            }{
                \datalangBlock{\datalangTag}{\datalangExpr_{s1}}{\datalangExpr_{s2}}
                \tmcDir{\datalangRenaming}
                \begin{array}{l}
                    \datalangLet{\datalangVar}{\datalangBlock{\datalangTag}{\datalangHole}{\datalangExpr_{t2}}}{\\
                    \datalangSeq{\datalangExpr_{t1}}{\datalangVar}}
                \end{array}
            }
        \\
        \inferrule*[lab=DirDef]
            {
                \datalangExpr_s \tmcDir{\datalangRenaming} \datalangExpr_t
            }{
                \datalangRec{\datalangVar}{\datalangExpr_s}
                \tmcDir{\datalangRenaming}
                \datalangRec{\datalangVar}{\datalangExpr_t}
            }
    \end{mathparpagebreakable}
    \caption{Direct TMC transformation (omitting congruence rules similar to \TirName{DirLet} and freshness conditions)}
    \label{fig:tmc_dir}
\end{figure}
\paragraph{2. For each function $\datalangFn$ defined in $\datalangProg$, computing its direct transform.}
We introduce in \cref{fig:tmc_dir} the relations $\datalangDef_s \tmcDir{\datalangRenaming} \datalangDef_t$ for definitions and $\datalangExpr_s \tmcDir{\datalangRenaming} \datalangExpr_t$ for expressions.

$\datalangDef_s \tmcDir{\datalangRenaming} \datalangDef_t$ expresses that:
1)~$\datalangDef_t$ has the same calling convention as $\datalangDef_s$.
2)~The body of $\datalangDef_t$ is the direct transform of the body of $\datalangDef_s$.

$\datalangExpr_s \tmcDir{\datalangRenaming} \datalangExpr_t$ expresses that $\datalangExpr_t$ is the direct transform of $\datalangExpr_s$.
Intuitively, this means $\datalangExpr_t$ computes the same thing as $\datalangExpr_s$.

The direct-style transform corresponds to the case where we do not have a block that can serve as a destination: this version is used in an arbitrary calling context, not necessarily under a constructor. Most rules are straightforward congruences -- we recursively transform subexpressions and preserve the term constructor. We omit the rules for loads $\datalangLoad{\datalangExpr_1}{\datalangExpr_2}$, stores $\datalangStore{\datalangExpr_1}{\datalangExpr_2}{\datalangExpr_3}$, and the deterministic blocks $\datalangBlockDet \datalangTag {\datalangExpr_1} {\datalangExpr_2}$ which are such simple congruences, just like calls $\datalangCall {\datalangExpr_1} {\datalangExpr_2}$.

The key cases are for a block construct $\datalangBlock \datalangTag {\datalangExpr_1} {\datalangExpr_2}$. We can use this block as a destination, and switch to the destination-passing-style calling convention -- these rules are a source of non-determinism, and the only places in the direct-style transformation where destination-passing-style is introduced. \RefTirName{DirBlock} is a simple congruence rule that keeps both arguments in direct style. The rules (\RefTirName{DirBlockDPS1}, \RefTirName{DirBlockDPS2}) choose a block argument to be evaluated in destination-passing style. (It is also possible to transform both arguments in DPS style, and we include extra rules for this in our formalization. \Xgabriel{TODO: measure the performance of this tweak for \ocaml{map} on binary trees.})

In details, the terms produced by these rules proceed as follows:
1)~They partially initialize a new memory block, with a hole for one of their arguments.
2)~They evaluate the DPS transformation of the corresponding argument,
   passing the uninitialized field as destination.
3)~They return the now fully initialized block.

An implementation would typically determine which subexpression would benefit from destination-passing style, that is, contains function calls $\datalangCall {\datalangFnptr \datalangFn} \datalangExpr$ in tail position (relatively to the subexpression) that have a destination-passing variant $\datalangRenaming [\datalangFn]$.

\begin{figure}[tp]
    \begin{mathparpagebreakable}
        \inferrule*
            {}{
                \boxed{- \tmcDps{\xi} - : \nterm{Expr} \times \nterm{Expr} \times \nterm{Expr} \rightarrow \nterm{Expr} \rightarrow \Prop}
            }
        \and
        \inferrule*
            {}{
                \boxed{- \tmcDps{\xi} - : \nterm{Def} \rightarrow \nterm{Def} \rightarrow \Prop}
            }
        \\
        \inferrule*[lab=DPSBase]
            {
                e_s \tmcDir{\xi} e_t
            }{
                \left(
                    \mathit{dst},
                    \mathit{idx},
                    e_s
                \right)
                \tmcDps{\xi}
                \lambdStore{\mathit{dst}}{i}{e_t}
            }
        \and
        \inferrule*[lab=DPSLet]
            {
                e_{s1} \tmcDir{\xi} e_{t1}
            \and
                (\mathit{dst}, \mathit{idx}, e_{s2}) \tmcDps{\xi} e_{t2}
            }{
                \left(
                    \mathit{dst},
                    \mathit{idx},
                    \lambdLet{x}{e_{s1}}{e_{s2}}
                \right)
                \tmcDps{\xi}
                \lambdLet{x}{e_{t1}}{e_{t2}}
            }
        \and
        \inferrule*[lab=DPSCall]
            {
                f \in \dom{\xi}
            \and
                e_s \tmcDir{\xi} e_t
            }{
                \left(
                    \mathit{dst},
                    \mathit{idx},
                    \lambdCall{\lambdFunc{f}}{e_s}
                \right)
                \tmcDps{\xi}
                \lambdCall{\lambdFunc{\xi [f]}}{e_t}
            }
        \and
        \inferrule*[lab=DPSIf]
            {
                e_{s0} \tmcDir{\xi} e_{t0}
            \and
                (\mathit{dst}, \mathit{idx}, e_{s1}) \tmcDps{\xi} e_{s2}
            \and
                (\mathit{dst}, \mathit{idx}, e_{s2}) \tmcDps{\xi} e_{t2}
            }{
                \left(
                    \mathit{dst},
                    \mathit{idx},
                    \lambdIf{e_{s0}}{e_{s1}}{e_{s2}}
                \right)
                \tmcDps{\xi}
                \lambdIf{e_{t0}}{e_{t1}}{e_{t2}}
            }
        \\
        \inferrule*[lab=DPSConstr1]
            {
                e_{s1} \tmcDir{\xi} e_{t1}
            \and
                (\mathit{dst}, \lambdTwo, e_{s2}) \tmcDir{\xi} e_{t2}
            }{
                \left(
                    \mathit{dst},
                    \mathit{idx},
                    \lambdConstr{c}{e_{s1}}{e_{s2}}
                \right)
                \tmcDps{\xi}
                \begin{array}{l}
                    \lambdLet{\mathit{dst'}}{\lambdConstr{c}{e_{t1}}{\lambdHole}}{\\
                    \lambdSeq{\lambdStore{\mathit{dst}}{\mathit{idx}}{\mathit{dst'}}}{e_{t2}}
                    }
                \end{array}
            }
        \and
        \inferrule*[lab=DPSConstr2]
            {
                e_{s2} \tmcDir{\xi} e_{t2}
            \and
                (\mathit{dst}, \lambdOne, e_{s1}) \tmcDir{\xi} e_{t1}
            }{
                \left(
                    \mathit{dst},
                    \mathit{idx},
                    \lambdConstr{c}{e_{s1}}{e_{s2}}
                \right)
                \tmcDps{\xi}
                \begin{array}{l}
                    \lambdLet{\mathit{dst'}}{\lambdConstr{c}{\lambdHole}{e_{t2}}}{\\
                    \lambdSeq{\lambdStore{\mathit{dst}}{\mathit{idx}}{\mathit{dst'}}}{e_{t1}}
                    }
                \end{array}
            }
        \\
        \inferrule*[lab=DPSDef]
            {
                (\mathit{dst}, \mathit{idx}, e_s) \tmcDps{\xi} e_t
            }{
                \lambdDef{x}{e_s}
                \tmcDps{\xi}
                {\begin{array}{l}
                    \lambdDef{y}{\\ \ \begin{array}{l}
                        \lambdLet{z}{\lambdLoad{y}{\lambdOne}}{\\
                        \lambdLet{\mathit{dst}}{\lambdLoad{z}{\lambdOne}}{\\
                        \lambdLet{\mathit{idx}}{\lambdLoad{z}{\lambdTwo}}{\\
                        \lambdLet{x}{\lambdLoad{y}{\lambdTwo}}{\\
                        e_t
                        }}}}
                    \end{array}}
                \end{array}}
            }
    \end{mathparpagebreakable}
    \caption{Destination-passing style TMC transformation (omitting freshness conditions)}
    \label{fig:tmc_dps}
\end{figure}
\paragraph{3. For each TMC-transformed function $\datalangFn$, choosing a destination-passing-style transform}
We introduce in \cref{fig:tmc_dps} the relations $\datalangDef_s \tmcDps{\datalangRenaming} \datalangDef_t$ for definitions and $(\datalangExpr_\mathit{dst}, \datalangExpr_\mathit{idx}, \datalangExpr_s) \tmcDps{\datalangRenaming} \datalangExpr_t$ for expressions.

$\datalangDef_s \tmcDps{\datalangRenaming} \datalangDef_t$ expresses that:
1)~The function defined in $\datalangDef_t$ has an additional parameter representing the destination where it must write its result. This parameter is a pair of the location of a memory block $\datalangVar_\mathit{dst}$ along with the index $\datalangVar_\mathit{idx}$ of a particular field in this block.
2)~The body of $\datalangDef_t$ is a DPS transform of the body of $\datalangDef_s$ under the given destination.

$(\datalangExpr_\mathit{dst}, \datalangExpr_\mathit{idx}, \datalangExpr_s) \tmcDps{\datalangRenaming} \datalangExpr_t$ expresses that $\datalangExpr_t$ is a DPS transform of $\datalangExpr_s$ under destination $(\datalangExpr_\mathit{dst}, \datalangExpr_\mathit{idx})$.
Intuitively, this means $\datalangExpr_t$ computes the same thing as $\datalangExpr_s$ but writes it into the destination instead of returning it.
We will formalize this intuition in \cref{sec:specification}.

Note that the rule \RefTirName{DPSDef}, which relates the two judgments, uses the expression-level relation $(\mathrel{\tmcDps{\datalangRenaming}})$ with a term variable $\datalangVar_\mathit{idx}$ to represent the index, not just a constant index $\datalangOne$ or $\datalangTwo$ as in block rules. These are the only two sort of expressions used to represent offsets in the transformation.

In the direct-style relation, congruence rules apply the same direct-style transformation to all subexpressions. The congruence-like rule of the DPS relation, for example \RefTirName{DPSLet} and \RefTirName{DPSIf}, are different. They apply the DPS transformation to sub-expressions which are in tail position relative to the expression, and the direct-style transformation to all other subexpressions. The if-then-else construct has two different subexpressions in tail position, at most one of them is evaluated at runtime.

The rules \RefTirName{DPSBlock1} and \RefTirName{DPSBlock2}
correspond to the rules \RefTirName{DirBlockDPS1} and \RefTirName{DirBlockDPS2} in the direct-style transformation, but the transformed code is different. Consider the translation of $
\left(
  \datalangExpr_\mathit{dst},
  \datalangExpr_\mathit{idx},
  \datalangBlock{\datalangTag}{\datalangExpr_{s1}}{\datalangExpr_{s2}}
\right)
$ into $
\datalangLet{\datalangVar}{\datalangBlock{\datalangTag}{\datalangExpr_{t1}}{\datalangHole}}{
  \datalangSeq{\datalangStore{\datalangExpr_\mathit{dst}}{\datalangExpr_\mathit{idx}}{\datalangVar}}{\datalangExpr_{t2}}}
$ by \RefTirName{DirBlockDPS2}. First we create a new destination $\datalangVar$, with a hole in second position. Then, instead of computing the corresponding subterm, we write this new destination $x$ into the \emph{current} destination $({\datalangExpr_\mathit{dst}}, {\datalangExpr_\mathit{idx}})$. Finally we evaluate $\datalangExpr_{t2}$, which is the DPS transform of the subexpression $\datalangExpr_{s2}$, with the destination $(\datalangVar, \datalangTwo)$. Notice that $\datalangExpr_{t2}$ is in tail position relative to the transformed expression, while it was not in tail position in the source expression. This is the key step of the TMC transformation, that turns non-tail calls into tail calls. Rule \RefTirName{DirBlockDPS2} puts the second subterm in tail position, and \RefTirName{DirBlockDPS1} puts the first subterm in tail position. It is not always obvious which rule should be applied. In the case of lists as in our running example \ocaml|y :: map f xs|, we want the second subterm in tail position, so the transformation only uses \RefTirName{DirBlockDPS2}. But consider a \ocaml|map| function on binary trees \ocaml|Node(map f left, map f right)|: the implementation must choose one subterm to put in tail position and another to keep in non-tail position.

The rule \RefTirName{DPSCall} applies only to calls $\datalangCall {\datalangFnptr \datalangFn} {\datalangExpr_s}$ to a known function $\datalangFn$, on the condition that a DPS variant has been generated for $\datalangFn$: $\datalangFn \in \dom{\datalangRenaming}$. In this case, the function call can be compiled to a call to the DPS variant $\datalangRenaming [\datalangFn]$, transferring to the callee the responsibility to write to the destination. This is the case where the DPS transform is beneficial, as this transformation may turn a non-tail-call into a tail-call -- when it occurs under a block, in a subterm that was moved to tail position. This rule is selected for \ocaml|map| in our \ocaml|y :: map f xs| example.

Finally, there is a catch-all rule \RefTirName{DPSBase} that applies in any case, in particular whenever none of the other rules can be selected. This case trivially realizes the DPS calling convention by evaluating the subterm to a result and writing this result in the desired destination. This is what happens in the base case of \ocaml|map_dps|, where the empty list \ocaml|[]| is transformed into \ocaml|dst.(idx) <- []|.

% \begin{figure}[tp]
\begin{minipage}{.40\columnwidth}
\begin{Datalang}
map |-> rec (fn, xs) =
  match xs with
  | [] ->
      []
  | x :: xs ->
      let y = fn x in
      let dst = y :: ? in
      @map_dps ((dst, 2), fn, xs) ;
      dst
\end{Datalang}
\end{minipage}
\hfill
\begin{minipage}{.52\columnwidth}
\begin{Datalang}
map_dps |-> rec ((dst, idx), fn, xs) =
  match xs with
  | [] ->
      dst.(idx) <- []
  | x :: xs ->
      let y = fn x in
      let dst' = y :: ? in
      dst.(idx) <- dst' ;
      @map_dps ((dst', 2), fn, xs)
\end{Datalang}
\end{minipage}
\caption{Direct and DPS transforms of \datalang|map| (\cref{fig:map})}
\label{fig:map_tmc}
\end{figure}

% As an example, we can show that the two \DataLang functions defined in \cref{fig:map_tmc} are respectively the direct and DPS transforms of the original \datalang|map| function defined in \cref{fig:map}.
% Thanks to the rules \RefTirName{DirBlockDPS2} and \RefTirName{DPSCall}, the non-tail recursive call is replaced by the allocation of a partially initialized block followed by a call to \datalang|map_dps|.

\subsection{Realizing the relation as a function}

Our Coq formalization includes a function that takes an input program and outputs a related program, following the one-pass implementation approach that we introduced in the \OCaml compiler (\cref{subsec:implementation}).

%%% Local Variables:
%%% mode: latex
%%% TeX-master: "main"
%%% End:


\separate{\clearpage}{}
\section{\OCaml Implementation}
\label{sec:implementation}

For reasons of space, we moved some of the content in this section to \cref{app:more-ocaml}:
%
\cref{subsec:alternative-impls}
  discusses alternative language implementation techniques that do not require TMC.
%
\cref{subsec:PR-history} provides a summary of the history of our implementation (started in 2015, restarted in 2020, merged in 2021).
%
\cref{subsec:applicative-impl} exhibits an applicative functor structure that makes our implementation nicely compact and readable.
%
\cref{subsec:adoption} surveys the adoption of the TMC transformation in the standard library,
  and in third-party \OCaml code bases, that happened since the feature was released in 2022.

\subsection{Examples}
\label{subsec:ocaml-examples}

% Many functions that consume and produce lists are
% tail-recursive-modulo-cons, in the sense that they admit a TMC
% transformation where all recursive calls become tail calls. Notable
% functions include \ocaml{map}, as already discussed, but also for
% example:

\begin{minipage}{0.47\linewidth}
\begin{Ocaml}
let[@tail_mod_cons] rec filter p =
  function
  | [] -> []
  | x :: xs ->
    if p x
    then x :: filter p xs
    else filter p xs
\end{Ocaml}
\end{minipage}
\hfill
\begin{minipage}{0.53\linewidth}
\begin{Ocaml}
let[@tail_mod_cons] rec merge cmp l1 l2 =
  match l1, l2 with
  | [], l | l, [] -> l
  | h1 :: t1, h2 :: t2 ->
      if cmp h1 h2 <= 0
      then h1 :: merge cmp t1 l2
      else h2 :: merge cmp l1 t2
\end{Ocaml}
\end{minipage}

TMC is not useful only for lists or other ``linear'' data types, with
at most one recursive occurrence of the datatype in each
constructor. An example follows.

\paragraph{A non-example} Consider a \ocaml{map} function on binary
trees:
\begin{Ocaml}
let[@tail_mod_cons] rec map f = function
| Leaf v -> Leaf (f v)
| Node(t1, t2) -> Node(map f t1, (map[@tailcall]) f t2)
\end{Ocaml}
In this function, there are two recursive calls, but only one of them
can be optimized; we used the \ocaml{[@tailcall]} attribute to direct
our implementation to optimize the call to the right child, as we will
discuss later. This is a \emph{bad} example of TMC usage in most
cases, given that
\begin{itemize}
\item If the tree is arbitrary, there is no reason that it would be
  right-leaning rather than left-leaning. Making only the right-child
  calls tail-calls does not protect us from stack overflows.
\item If the tree is known to be balanced, then in practice the depth
  is probably very small in both directions, so the TMC transformation
  is not necessary to have a well-behaved function.
\end{itemize}

\paragraph{Interesting non-linear examples} There \emph{are} interesting
examples of TMC-transformation on functions operating on tree-like
data structures, when there are natural assumptions about which child
is likely to contain a deep subtree. The \OCaml compiler itself
contains a number of them; consider for example the following function
from the \ocaml{Cmm} module, one of its lower-level program
representations:

\begin{Ocaml}
let[@tail_mod_cons] rec map_tail f = function
  | Clet(id, exp, body) ->
      Clet(id, exp, map_tail f body)
  | Cifthenelse(cond, ifso, ifnot) ->
      Cifthenelse(cond, map_tail f ifso, (map_tail[@tailcall]) f ifnot)
  | Csequence(e1, e2) ->
      Csequence(e1, map_tail f e2)
  | Cswitch(e, tbl, el) ->
      Cswitch(e, tbl, Array.map (map_tail f) el)
  [...]
\end{Ocaml}

This function is traversing the ``tail'' context of an arbitrary program term -- a meta-example!
The \ocaml{Cifthenelse} node acts as our binary-node constructor.
We do not know which side is likely to be larger, so TMC is not so interesting.
The recursive calls for \ocaml{Cswitch} are not in TMC position.
But on the other hand the \ocaml{Clet}, \ocaml{Csequence} do benefit from the TMC transformation: while they have several recursive subtrees, they are in practice only deeply nested in the direction that is turned into a tailcall by the transformation.
The \OCaml compiler does sometimes encounter machine-generated programs with a unusually long sequence of either constructions, and the TMC transformation may very well avoid a stack overflow in this case.

Another example would be \href{https://github.com/ocaml/ocaml/pull/9636}{\#9636}, a patch to the \OCaml compiler proposed in June 2020 by Mark Shinwell, to get a partially-tail-recursive implementation of the ``Common Subexpression Elimination'' (CSE) pass through a manual continuation-passing-style transform.
Xavier Leroy remarked that the existing implementation in fact fits the TMC fragment. Not all recursive calls become tail-calls (this would require a more powerful transformation or a longer, less readable patch), but the behavior of TMC on the unchanged code matches the tail-call-ness proposed in the human-written patch.

\subsection{Specifying Which Calls are in TMC Position} \label{subsec:specification}
To reason about the stack usage of their programs, users must understand which calls are in tail-modulo-cons position.
Informally, they are the calls placed under any composition of either tail-recursive or constructor contexts.

We can in fact give a simple formal description of this intuition, here for \DataLang in \cref{fig:contexts}.
A tail frame $\datalangTailFrame$ is a single term-former with holes in tail-position.
A constructor frame $\datalangConsFrame$ is a single constructor term-former (we omit deterministic blocks, which do not occur in the source).
A tail context $\datalangTailCtx$ is an arbitrary composition of tail-frame, and a TMC context $\datalangTMCCtx$ is an arbitrary composition of tail frames and constructor frames.

If a source function can be decomposed in a TMC context $\datalangTMCCtx$ with source expressions in its holes, some of which are calls to TMC-transformed functions, then our relation admits a DPS transformation where all those function calls are tail-calls, and this transformation is reachable in our \OCaml implementation, possibly by adding some annotations.

Note in particular that we do not optimize calls to the same function we are defining, direct calls to arbitrary other functions can be transformed, if those functions have been annotated to be TMC-transformed.
This is analogous to how most functional languages support arbitrary \emph{tail calls} and not just tail self-recursion.
We seamlessly support mutually recursive functions, DPS calls into locally-bound functions, etc.

\begin{figure}[tp]
    \begin{tabular}{lclcl}
            $\datalangTailFrame[]$
            & $\ni$ &
            $\datalangTailFrame$
            & $\Coloneqq$ &
            $\datalangLet {\datalangVar} {\datalangExpr} \datalangCtxHole \mid \datalangIf \datalangExpr \datalangCtxHole \datalangCtxHole$
\\
            $\datalangConsFrame[]$
            & $\ni$ &
            $\datalangConsFrame$
            & $\Coloneqq$ &
            $\datalangBlock \datalangTag \datalangExpr \datalangCtxHole
            \mid \datalangBlock \datalangTag \datalangCtxHole \datalangExpr$
\\
            $\datalangTMCFrame[]$
            & $\ni$ &
            $\datalangTMCFrame$
            & $\Coloneqq$ &
            $\datalangTailFrame
             \mid
             \datalangConsFrame$
\\
            $\datalangTailCtx[]$
            & $\ni$ &
            $\datalangTailCtx$
            & $\Coloneqq$ &
            $\datalangCtxHole
             \mid
             \datalangTailFrame
             \mid
             \datalangTailCtx{[\datalangTailCtx]}$
\\
            $\datalangTMCCtx[]$
            & $\ni$ &
            $\datalangTMCCtx$
            & $\Coloneqq$ &
            $\datalangCtxHole
             \mid
             \datalangTMCFrame
             \mid
             \datalangTMCCtx{[\datalangTMCCtx]}$
    \end{tabular}
    \caption{\DataLang contexts for optimizable calls}
    \label{fig:contexts}
\end{figure}

\subsection{Design Choices}

\subsubsection{Opt-in Rather than Opt-out}\label{subsec:optin}
We made our transformation \emph{opt-in}, the user has to explicit use
the \ocaml{[@tail_mod_cons]}, rather than \emph{opt-out}, where the
compiler would automatically detect program in the TMC fragment to
transform them. There are at least three reasons for this choice, in
increasing order of importance. First, the transformation duplicates
code, as it generates a direct- and a destination-passing-style
version of the function; in particular we could suffer from
a code-size blow-up on nested transformable functions. Second, the
transformation may change the evaluation order of function arguments,
in a way that conforms to the OCaml specification but may break
programs in practice (bad!), so we would have to restrict it to only
transform calls with syntactically-pure arguments. Third and most
importantly, our implementation may have bugs, which may silently
introduce miscompilations in user code. There are millions of lines of
OCaml code out there, and we have no idea which portion would be
transformed. Doing this implicitly would risk running into
compile-time, runtime or correctness issues that we have not foreseen,
it would have been much harder to convince OCaml maintainers to
include this feature if it was opt-out.

\subsubsection{Resolving Non-determinism}

The main question faced by an implementation is how to resolve the
inherent non-determinism in the rewrite relation -- how to decide
between several possible results of the transformation. We identified
fairly different kinds of non-determinism.

\paragraph{Choices with an obviously better alternative.}
Some choices offer an alternative that is obviously better than the others, because it allows to strictly improve more programs. For example, some implementations of TMC limit themselves to calls that are immediately inside a constructor: they would optimize the recursive call to \ocaml|f| in \ocaml|Cons(x, f y)| but not, for example, in \ocaml|Cons(x1, Cons(x2, f y))|, or in \ocaml|Cons (x, if p then f y else z)|.

In the terms of \cref{subsec:specification}, those implementations only allow to optimize calling contexts of the form $\datalangTailCtx{[\datalangConsFrame]}$ or $\datalangTailCtx{[\datalangConsCtx]}$. Our implementation supports the more general situation $\datalangTMCCtx = (\datalangTailFrame | \datalangConsFrame)^{*}$. Both approaches are included in our non-deterministic relation -- we prove them both correct -- but optimizing more calls is obviously better.

\paragraph{Choices with a subtly better alternative.} In some cases it
is possible to put a bit more work in the choice heuristics, to
generate slightly better code, in terms of performance or
readability. For example, some natural way to simplify the
implementation would result in more subterms being transformed in
destination-passing-style, only to finally use the
\RefTirName{DPSBase} rule without any DPS function call. The generated
code is slightly less pleasant to read, and in our experience it can
be noticeably slower. (We detail such a situation in \cref{subsec:constructor-compression}.)

Code readability is important in our context of an on-demand program
transformation used by performance experts: those experts will often
read the intermediate representations produced by the compiler to
check that their performance assumptions hold.

\paragraph{Incomparable choices: force the user to decide} \label{subsec:user-control}
The remaining choices are between incomparable alternatives, that could
each be better than the other depending on the specific program or
programmer intent. Consider again our binary tree example,
\ocaml|Node(map f left, map f right)|: we could make either the first
or the second argument a tail-call.

Our policy in such cases is to let the user decide, by providing
control over the transformation choices using \ocaml{[@tailcall]}
program annotations, and to force the user to decide by raising an
error in case of ambiguity:
\begin{Ocaml}
  | Cifthenelse(cond, ifso, ifnot) ->
      Cifthenelse(cond, map_tail f ifso, map_tail f ifnot)
      /[^^^^^^^^^^^^^^^^^^^^^^^^^^^^^^^^^^^^^^^^^^^^^^^^^^^^]/
/[Error]/: "[@tail_mod_cons]": this constructor application may be TMC-transformed
       in several different ways. Please disambiguate by adding an explicit
       "[@tailcall]" attribute to the call that should be made tail-recursive,
       or a "[@tailcall false]" attribute on calls that should not be
       transformed.
\end{Ocaml}

This approach is incompatible with a view of the TMC
transformation as an implicit optimization, applied whenever
possible -- in that case one would rather have the compiler make
arbitrary choices rather than fail. We rather view TMC as a tool for
expert users to better reason about program performance. It should be
predictable and flexible (let the user express
their intent). Failing due to the lack of annotations is an acceptable way
to drive user interaction.

\subsubsection{First-order Implementation}\label{subsubsec:first-order}
A notable limitation of the \OCaml implementation is that it is first-order and non-modular in nature: only direct calls to known functions can be converted into DPS style, and the availability of a DPS variant is not exposed through module abstractions.
It would be possible to allow DPS calls through external modules or higher-order function parameters, by annotating interfaces and function arguments with a \ocaml{[@tail_mod_cons]} annotation, and elaborating them into a pair of functions, the direct-style and the DPS version.

\subsection{Constructor Compression} \label{subsec:constructor-compression} The translation as we described it formally in \cref{subsec:transformation} generates unpleasant code when many constructors are nested before the recursive call. For example, consider this strange function duplicating each element of a list:
\begin{Ocaml}
let rec dup = function [] -> [] | x :: xs -> x :: x :: dup xs
\end{Ocaml}

Such nested constructors are common in compiler code bases, for
example a desugaring pass that transforms a single term-former into
a composition of several simpler term-formers, and applies recursively to
its subterms.

Following the TMC transformation naively, the DPS version would propagate two different locations and performs two writes. We introduced ``constructor compression'', an optimization of the generated code that avoids creating intermediary destinations for nested constructors, leading to clearer generated code and better constant factors. Compare the naive translation of \ocaml|dup|, on the left, and our compressed translation on the right:

\begin{minipage}{0.5\linewidth}
\begin{Ocaml}
let rec dup_dps dst ofs = function
| [] -> dst.(ofs) <- []
| x :: xs ->
  let dst1 = x :: ? in
  dst.(ofs) <- dst1;
  let dst2 = x :: ? in
  dst1.(1) <- dst2;
  dup_dps dst2 1 xs
\end{Ocaml}
\end{minipage}
\hfill
\begin{minipage}{0.5\linewidth}
\begin{Ocaml}
let rec dup_dps dst ofs = function
| [] -> dst.(ofs) <- []
| x :: xs ->
  let dst2 = x :: ? in
  dst.(ofs) <- x :: dst2;
  dup_dps dst2 1 xs
\end{Ocaml}
\end{minipage}

This is implemented by passing a new transformation parameter: a stack of ``delayed'' constructor applications, that are in context and must be applied to the result of the subterm. When we encounter the final recursive call, we ``reify'' this stack: the last/innermost constructor in the stack becomes the new destination (\ocaml|dst2| in the example above), and the rest of the stack is applied to the new destination when we write to the old destination. There are two subtleties:
\begin{enumerate}
\item \ocaml|if p then e1 else e2| has two subterms which are transformed in DPS style, and naively passing the stack of delayed constructors to both subterms would duplicate code; instead we also reify the current stack when encountering such constructs. For example,\\
\ocaml{  1 :: 2 :: if p then (3 :: f ()) else (4 :: f ())}%
becomes:\\
\ocaml{  let dst1 = 1 :: 2 :: ? in}\\
\ocaml{  if p then (let dst2 = 3 :: ? in dst1.1 <- dst2; f dst2 1 ())}\\
\ocaml{       else (let dst3 = 4 :: ? in dst1.1 <- dst3; f dst3 1 ())}

\item This transformation may permute constructor applications after effectful subterms. If the constructor application context frame contains possibly-effectful subterms (for example \ocaml{f x :: _? } instead of \ocaml{x :: _?}), the compiler must \ocaml|let|-bind them at their original position to avoid changing the evaluation order. For example,\\
\ocaml{x () :: (y (); f ())} does not become \ocaml{y (); (let dst = x () :: ? in ...)},
but instead \ocaml{let tmp = x () in (y (); let dst = tmp :: ? in ...)}.
\end{enumerate}

It is conceptually easy to extend our previous formalization of TMC as a rewriting relation to capture constructor compression, by indexing this relation on an additional list of constructor contexts. We do not present this here for lack of space, but included this change in our \Coq proofs, which establish correctness of TMC in presence of constructor compression.

\subsection{Implementation} \label{subsec:implementation} Implementing the TMC transform is not an obvious top-down or bottom-up traversal, because it relies on information flowing in two opposite directions: the transformation is \emph{possible} for subterms that are in tail-or-constructor position relative to the root of the function (top-down information), and it is \emph{desirable} for subterms whose leaves contain calls to TMC-transformed functions (bottom-up information). A naive approach is to perform a recursive traversal that tracks the top-down context and, on each subterm, recursively traverses the whole subterm to check desirability; this has quadratic time complexity, which is best avoided in production compilers.

Instead of trying to transform each subterm in a single pass, our implementation computes for each subterm a ``choice'', a summary of all the bottom-up information that is relevant to choose how to transform this subterm -- once we have the top-down information available. This includes (1) the direct-style transform of the subterm, (2) the DPS-style transform of the subterm, (3) metadata tracking whether the transformation is beneficial (if a TMC-function call was found in tail-modulo-cons position), the list of TMC calls it contains (for error diagnostics), and whether some were explicitly requested by the user (for disambiguation). Computing transformed terms is not compositional; computing choices is compositional. We compute a choice for the whole function body (in one pass), from which we can directly extract the direct-style and DPS-style transformations of the function.

\subsection{Evaluation: Benchmarks}

We measured the performance of \ocaml|List.map (fun n -> n + 1)| to validate our claims that the TMC transformation preserves program performance, and lets us replace complex hand-optimized tail-recursive implementations. \ocaml|List.map| is a worst-case: with most of the time spent in recursion and list construction, it is more sensitive to constant-factor overheads than other recursive functions.

\begin{figure}[tp]
\def\svgscale{0.8}
\graphicspath{{plots/}}
\input{plots/plot.5.pdf_tex}
\caption{\ocaml|List.map| benchmark on \OCaml~5.1}
\label{fig:bench5}
\end{figure}

\newcommand{\bench}[1]{\textbf{#1}}

The different versions we benchmark are the following. We measure the code size (in lines) of each version, as a reasonable approximation of its implementation complexity.
\begin{description}
\item[\bench{nontail}] (5 lines of code) The naive, non-tail-recursive implementation.
\item[\bench{tail}] (9 lines) The naive tail-recursive implementation,
  \ocaml{List.rev (List.rev_map f xs)}.
\item[\bench{base}] (78 lines) The implementation of Jane Street's
  \href{https://github.com/janestreet/base}{Base} library
  (version 0.14.0). It is heavily hand-optimized to compensate for the costs
  of being tail-recursive.
\item[\bench{containers}] (55 lines) Another standard-library extension by Simon
  Cruanes; it is the hand-optimized tail-recursive implementation we
  included in the Prologue.
\item[\bench{batteries}] (29 lines) The implementation of the community-maintained
  \href{https://github.com/ocaml-batteries-team/batteries-included/}{Batteries}
  library. It is actually written in destination-passing-style, using
  an unsafe encoding with \ocaml{Obj.magic} to unsafely cast a mutable
  record into a list cell. (The trick comes from the older Extlib
  library, was introduced by Brian Hurt in 2003, and has a comment crediting Jacques
  Garrigue for the particular encoding used.)
\item[\bench{tmc}] (5 lines) ``Our'' version, the last version of the Prologue: the
  result of applying our implementation of the TMC transformation to
  the simple, non-tail-recursive version.
\item[\bench{tmc-unrolled}] (18 lines) The result of manually unrolling the \bench{tmc} implementation three times, to be compared with \bench{base} and \bench{containers} that use manual unrolling as well.
\end{description}

The benchmarks reports the relative performance compared to the naive tail-recursive version as our baseline. They were run on \OCaml~5.1 in July 2024, on a Linux machine with an AMD Ryzen processor fixed at a 3Ghz frequency, looping each measurement for 5s (a single \ocaml|List.map| run takes between 7ns, for empty lists, and 89ms on lists with a million element).

Qualitatively we see that there are four groups:
\begin{itemize}
\item \bench{tmc}, \bench{batteries} perform very well on large lists, but they
  are slower than the baseline on small lists.
\item \bench{nontail} performs better than \bench{tmc}, \bench{batteries} on list sizes up to $10^4$, and much worse on larger lists.
\item \bench{base}, \bench{containers} perform noticeably better than \bench{nontail} at all sizes,
  but worse than the TMC versions above size $10^4$.
\item \bench{tmc-unrolled} is the best option: it performs as well as \bench{base} and \bench{containers} before $10^4$, and as well as \bench{tmc}, \bench{batteries} afterwards.
\end{itemize}
Our interpretation of the result is that some unrolling makes a noticeable performance difference for such a short function: \bench{tmc} is not good enough on smaller lists, but \bench{tmc-unrolled} is the best-performing, despite being much simpler than the \bench{base} and \bench{containers} versions.

\paragraph{Asymptotics of \bench{nontail}} The bad behavior of \bench{nontail} on large lists comes from a quadratic behavior on very large call stacks, coming from a repeated scan of the call stack during minor collections. (The \OCaml compiler and runtime could be tweaked to avoid this quadratic behavior, at the cost of some small constant overhead on function returns.)

% \paragraph{Memory usage} The \bench{tail} version allocates twice as many words as the \bench{tail} or TMC versions. The \bench{base} and \bench{containers} version allocate more as well, once they reach the cutoff where they move to their tail-recursive regime. This probably does not matter for most use-cases, and it does not show in the time measurements in the microbenchmark.

%%% Local Variables:
%%% mode: latex
%%% TeX-master: "main"
%%% End:

\separate{\clearpage}{}
\section{Specifying TMC}
\label{sec:specification}

In this section, we gradually introduce aspects of our relational separation logic, by introducing our specifications for the direct-style and destination-passing-style transformations of \cref{sec:formalization} in relational separation logic.
% In the next sections, we will unroll the proof of these specifications and extract a soundness theorem that lives outside of separation logic, depending only on the \DataLang semantics.

\subsection{Direct Transformation}

Intuitively, the direct transformation $\datalangExpr_s \tmcDir{\datalangRenaming} \datalangExpr_t$ preserves the behaviors of the source expression $\datalangExpr_s$.
Basically, $\datalangExpr_s$ and $\datalangExpr_t$ compute the same thing.
Using \emph{relational Hoare logic}\Xfrancois{TODO cite Benton 2004}, a extension of standard Hoare logic relating two expressions, we would write:
\[
    \iSimvHoare{
        \datalangExpr_s \tmcDir{\datalangRenaming} \datalangExpr_t
    }{
        \datalangVal_s, \datalangVal_t \ldotp
        \datalangVal_s \iSimilar \datalangVal_t
    }{
        \datalangExpr_s
    }{
        \datalangExpr_t
    }
\]
\Xfrancois{The $\datalangExpr_s \tmcDir{\datalangRenaming} \datalangExpr_t$ is pure and could be outside the triple.}

The informal meaning of this specification is that 1) $\datalangExpr_t$ refines $\datalangExpr_s$ in the sense that any behavior (converging, diverging or stuck execution) of $\datalangExpr_t$ is also a behavior of $\datalangExpr_s$ and 2) if $\datalangExpr_t$ converges to value $\datalangVal_t$, then $\datalangExpr_s$ also converges to some value $\datalangVal_s$ that is \emph{similar} to $\datalangVal_s$.
We will formalize the notion of \emph{behavior} in \cref{sec:simulation} and that of \emph{similarity} later in this section.
For the time being, the reader may assume similarity is just equality on values.

\subsection{DPS Transformation}

The DPS transformation $(\datalangLoc, \datalangIdx, \datalangExpr_s) \tmcDps{\datalangRenaming} \datalangExpr_t$ is parameterized by a destination $(\datalangLoc, \datalangIdx)$ pointing to an uninitialized field of some block.
Intuitively, $\datalangExpr_t$ computes the same thing as $\datalangExpr_s$ but writes it into the destination instead of returning it. This can be expressed concisely in \emph{relational separation logic}~\citep*{yang-07}, a further extension of relational Hoare logic:
\[
    \iSimvHoare{
        (\datalangLoc, \datalangIdx, \datalangExpr_s) \tmcDps{\datalangRenaming} \datalangExpr_t \iSep
        (\datalangLoc + \datalangIdx) \iPointsto_t \datalangHole
    }{
        \datalangVal_s, \datalangUnit \ldotp
        \exists \datalangVal_t \ldotp
        (\datalangLoc + \datalangIdx) \iPointsto_t \datalangVal_t \iSep
        \datalangVal_s \iSimilar \datalangVal_t
    }{
        \datalangExpr_s
    }{
        \datalangExpr_t
    }
\]

In words: if $\datalangExpr_s$ transforms into $\datalangExpr_t$, and if we uniquely own the destination location $\datalangLoc + \datalangIdx$, we can transfer ownership to $\datalangExpr_t$ and run the two programs, whose execution must be related. When they reduce to values, $\datalangExpr_s$ reduces to a source value $\datalangVal_s$ and $\datalangExpr_t$ to the unit value $\datalangUnit$, and we recover the unique ownership of the destination, which now contains a target value $\datalangVal_t$ similar to $\datalangVal_s$.

\subsection{Heap Bijection}

Defining value similarity as just syntactic equality is not sufficient: corresponding source and target block allocations are not done in lockstep, so the resulting locations may differ.
For example, consider the \datalang{map} function and its DPS transform from \cref{subsec:tmc_example}.
In the source program, the cons cell \datalang{y :: @map (fn, xs)} is allocated after the recursive call.
In the transformed program, the corresponding block is allocated before the call.

To deal with this, we introduce a \emph{heap bijection} as in \Simuliris~\citep*{simuliris-2022}.
This is a partial bijection (some destination locations have no source counterpart) which grows over time.
Its usage is formalized by the \RefTirName{BijInsert} rule:

\begin{mathline}
    \inferrule*[lab=BijInsert]
        {
            \datalangLoc_s \iPointsto_s \datalangVal_s
        \and
            \datalangLoc_t \iPointsto_t \datalangVal_t
        \and
            \datalangVal_s \iSimilar \datalangVal_t
        }{
            \datalangLoc_s \iInBij \datalangLoc_t
        }
\end{mathline}
This is a \emph{ghost update} rule that mutates the logical state.
It can only be applied when the two locations $\datalangLoc_s$ and $\datalangLoc_t$ have similar content $\datalangVal_s \iSimilar \datalangVal_t$.
It consumes the ``private'' ownership of the source and target points-to $\datalangLoc_s \iPointsto_s \datalangVal_s$ and $\datalangLoc_t \iPointsto_t \datalangVal_t$, and produces a persistent proposition $\datalangLoc_s \iInBij \datalangLoc_t$ witnessing that the two locations are now in the ``public'' bijection.

We formally define value similarity $\datalangVal_s \iSimilar \datalangVal_t$ in \cref{fig:isimilar}.
It coincides with equality except on blocks, for which we require all fields to be registered in the bijection.

\begin{figure}[tp]
    \centering
    \begin{mathparpagebreakable}
        \inferrule*
            {}{
                \lambdUnit \iSimilar \lambdUnit
            }
        \and
        \inferrule*
            {}{
                i \iSimilar i
            }
        \and
        \inferrule*
            {}{
                t \iSimilar t
            }
        \and
        \inferrule*
            {}{
                b \iSimilar b
            }
        \and
        \inferrule*
            {
                \forall i \in \{0,1,2\} \ldotp
                (\loc_s + i) \iInBij (\loc_t + i)
            }{
                \loc_s \iSimilar \loc_t
            }
        \and
        \inferrule*
            {
                f \in \dom{p_s}
            }{
                \lambdFunc{f} \iSimilar \lambdFunc{f}
            }
    \end{mathparpagebreakable}
    \caption{Similarity in $\iProp$}
    \label{fig:isimilar}
\end{figure}

%%% Local Variables:
%%% mode: latex
%%% TeX-master: "main"
%%% End:


\separate{\clearpage}{}
\section{Program logic}

\cref{fig:sim_rules} defines the reasoning rules for our relational program logic. We omitted some standard rules for brevity.

\paragraph{Language-independent rules} The following rules are
independent of \DataLang and could be reused as-is in further works.

\RefTirName{SimPost} states that two terms are related when they are in the relational postcondition.

\RefTirName{SimBind} is a standard bind rule sequencing computations on both sides.

\RefTirName{SimSrcPure} and \RefTirName{SimTgtPure} let us take pure reduction steps in either the source or target and remain related. Pure steps (omitted for space) are the reduction steps that are deterministic and do not depend on the state.

\paragraph{Abstract protocols}
\RefTirName{SimApplyProtocol} is the reasoning rule for external transitions specified by the protocol parameter $\iProt$. The protocol is a relation that axiomatizes the ability of certain pairs of expression $\datalangExpr_s$ and $\datalangExpr_t$ to evolve in an abstract way specified by a relational postcondition $\iPredTwo$ -- they may reduce to arbitrary expressions $\datalangExpr_s'$ and $\datalangExpr_t'$ related by $\iPredTwo$. To conclude that $\datalangExpr_s$ and $\datalangExpr_t$ are related, one must prove that any such $\datalangExpr_s'$ and $\datalangExpr_t'$ remain related.

Here is a simple protocol to relate calls to the same functions in the source and target expressions:

\begin{center}
\begin{tabular}{rcl}
        $\iProt_\mathrm{dir} (\iPredTwo, \datalangExpr_s, \datalangExpr_t)$
        & $\coloneqq$ &
        $\exists \datalangFn, \datalangVal_s, \datalangVal_t \ldotp$
    \\
        &&
        $\datalangFn \in \dom{\datalangProg_s} \iSep
        \datalangExpr_s = \datalangCall{\datalangFnptr{\datalangFn}}{\datalangVal_s} \iSep
        \datalangExpr_t = \datalangCall{\datalangFnptr{\datalangFn}}{\datalangVal_t} \iSep
        \datalangVal_s \iSimilar \datalangVal_t \iSep {}$
    \\
        &&
        $\forall \datalangVal_s', \datalangVal_t' \ldotp
        \datalangVal_s' \iSimilar \datalangVal_t' \iSepImp
        \iPredTwo (\datalangVal_s', \datalangVal_t')$
\end{tabular}
\end{center}

This protocol relates two calls of the same function $\datalangFnptr{\datalangFn}$
with related arguments $\datalangVal_s \iSimilar \datalangVal_t$. Informally, it then
guarantees that the results of the calls will be related values
$\datalangVal_s' \iSimilar \datalangVal_t'$; precisely, it accepts any choice of post-condition $\iPredTwo$ contains the similarity relation on values.

If we use this protocol in the reasoning rule
\RefTirName{SimApplyProtocol} and instantiate the postcondition $\iPred$ to be the value similarity relation $(\iSimilar)$ itself, we get the following specialized rule:
\begin{mathpar}
\infer
{
  f \in \dom{\datalangProg_s}
  \and
  \datalangVal_s \iSimilar \datalangVal_t
}{
  \iSim[\iProt]{\iSimilar}
    {\datalangCall{\datalangFnptr{f}}{\datalangVal_s}}
    {\datalangCall{\datalangFnptr{f}}{\datalangVal_t}}
}
\end{mathpar}
This is exactly the \TirName{Sim-Call} rule of \Simuliris~\citep*{TODO-simuliris}. Our generalized rule \RefTirName{SimApplyProtocol} is a contribution of our work, inspired by \citet*{TODO-paulo}. The extra expressivity is required to support program transformations such as TMC that change function calling conventions.

\paragraph{Language-specific rules: non-determinism}
%
Our relation $\iSim[\iProt]{\iPred}{\datalangExpr_s}{\datalangExpr_t}$ asserts that the target expression $\datalangExpr_t$ refines the source expression $\datalangExpr_s$: any behavior of $\datalangExpr_t$ is a valid behavior of $\datalangExpr_s$, all reductions of $\datalangExpr_t$ can be matched by some reduction of $\datalangExpr_s$. Consequently, non-determinism is treated differently in the source and target: we treat non-determinism as \emph{angelic} in source reductions and \emph{demonic} in target reductions.

Our operational semantics uses non-deterministic rule for reduction of constructors: the non-deterministic constructor $\datalangBlock{c}{\datalangExpr_1}{\datalangExpr_2}$ reduces to a deterministic constructor $\datalangBlockDet{c}{\datalangVar_1}{\datalangVar_2}$, where the $\datalangVar_1$ and $\datalangVar_2$ are bound to $\datalangExpr_1$ and $\datalangExpr_2$ in some non-determistic order.
%
In the program logic, the user may \emph{choose} an order for the source reduction, by using one of the rules \RefTirName{SimSrcBlock1} or \RefTirName{SimSrcBlock2}, and it has to prove that the expressions are related against \emph{any} target order, by proving the two premises of the rule \RefTirName{SimTgtBlock}.

Once we have a deterministic constructor $\datalangBlockDet{c}{\datalangVal_1}{\datalangVal_2}$, the source and target reasoning rules \RefTirName{SimSrcBlockDet} and \RefTirName{SimTgtBlockDet} are symmetric.
%
They allocate the block at a new location in the source or target state; the user can assume a points-to predicate and has to prove that the location is related to the other expression.

\paragraph{Language-specific rules: private locations}
%
If the user owns a points-to assertion $(\datalangLoc + \datalangIdx) \iPointsto \datalangVal$ on the source or target side, they can use it to reason about loads and stores on the location $\datalangLoc$ at index $\datalangIdx$. The rules \RefTirName{SimSrcLoad} and \RefTirName{SimTgtLoad} let us replace the load expression $\datalangLoad{\datalangLoc}{\datalangIdx}$ by the pointed value $\datalangVal$, and the rules \RefTirName{SimSrcStore} and \RefTirName{SimTgtStore} let us store a new value $\datalangVal'$, replacing the points-to assertion by $(\datalangLoc + \datalangIdx) \iPointsto \datalangVal'$.

\paragraph{Language-specific rules: locations in the bijection}
%
In addition to points-to predicate on source and target locations, we maintain a heap bijection formed by the resource assertions $\datalangLoc_s \iInBij \datalangLoc_t$.

At any point the user can add two locations to the heap bijection using the rule \RefTirName{SimBijInsert}. They need to provide private points-to predicates $\datalangLoc_s \iPointsto_s \datalangVal_s$ and $\datalangLoc_t \iPointsto_t \datalangVal_t$, and prove that the content of the locations are related: $\datalangVal_s \iSimilar \datalangVal_t$.
%
In exchange they receive a heap bijection assertion $\datalangLoc_s \iInBij \datalangLoc_t$. These locations will remain in the heap bijection forever ($\datalangLoc_s \iInBij \datalangLoc_t$ is a persistent assertion), and we will preserve the invariant that their values are related.

To add locations $\datalangLoc_s, \datalangLoc_t$ to the bijections, we \emph{consume} their points-to assertions. At this point we cannot use the one-sided reasoning rules for load and store: we lost ownership of $\datalangLoc_s \iPointsto_s \datalangVal_s$, $\datalangLoc_t \iPointsto_t \datalangVal_t$.
%
The user must instead use the synchronized rules \RefTirName{SimLoad} and \RefTirName{SimStore}, which relate two loads or two stores on $\datalangLoc_s$ and $\datalangLoc_t$.
%
\RefTirName{SimLoad} proves that two loads $\datalangLoad{\datalangLoc_s}{\datalangIdx}$ and $\datalangLoad{\datalangLoc_t}{\datalangIdx}$ are related. The user must prove that $\datalangLoc_s \iSimilar \datalangLoc_t$, that is, that all three pairs of fields $(\datalangLoc_s + \datalangIdx), (\datalangLoc_t + \datalangIdx)$ are in the bijection; it knows nothing of the current values $\datalangVal_s, \datalangVal_t$ at these locations, except that they are related by the bijection invariant.
%
\RefTirName{SimStore} proves that two stores $\datalangStore{\datalangLoc_s}{\datalangIdx}{\datalangVal_s}$ and $\datalangStore{\datalangLoc_t}{\datalangIdx}{\datalangVal_t}$ are related. The user must show that $\datalangLoc_s, \datalangLoc_t$ are in the bijection, and preserve the bijection invariant by proving that the values to be stored are related.

\begin{figure}[tp]
    \centering
    \begin{mathparpagebreakable}
        \inferrule*
            {}{
                \lambdUnit \iSimilar \lambdUnit
            }
        \and
        \inferrule*
            {}{
                i \iSimilar i
            }
        \and
        \inferrule*
            {}{
                t \iSimilar t
            }
        \and
        \inferrule*
            {}{
                b \iSimilar b
            }
        \and
        \inferrule*
            {
                \forall i \in \{0,1,2\} \ldotp
                (\loc_s + i) \iInBij (\loc_t + i)
            }{
                \loc_s \iSimilar \loc_t
            }
        \and
        \inferrule*
            {
                f \in \dom{p_s}
            }{
                \lambdFunc{f} \iSimilar \lambdFunc{f}
            }
    \end{mathparpagebreakable}
    \caption{Similarity in $\iProp$}
    \label{fig:isimilar}
\end{figure}
\begin{figure}[tp]
    \begin{mathparpagebreakable}
        \inferrule*[lab=SimPost]
            {
                \iPred (\datalangExpr_s, \datalangExpr_t)
            }{
                \iSim[\iProt]{\iPred}{\datalangExpr_s}{\datalangExpr_t}
            }
        \and
        \inferrule*[lab=SimBind]
            {
                \iSim[\iProt]{\lambdaAbs (\datalangExpr_s', \datalangExpr_t') \ldotp \iSim[\iProt]{\iPred}{\datalangEctx_s [\datalangExpr_s']}{\datalangEctx_t [\datalangExpr_t']}}{\datalangExpr_s}{\datalangExpr_t}
            }{
                \iSim[\iProt]{\iPred}{\datalangEctx_s [\datalangExpr_s]}{\datalangEctx_t [\datalangExpr_t]}
            }
        \\
        \inferrule*[lab=SimSrcPure]
            {
                \datalangExpr_s \pureStep{\datalangProg_s} \datalangExpr_s'
            \and
                \iSim[\iProt]{\iPred}{\datalangExpr_s'}{\datalangExpr_t}
            }{
                \iSim[\iProt]{\iPred}{\datalangExpr_s}{\datalangExpr_t}
            }
        \and
        \inferrule*[lab=SimTgtPure]
            {
                \datalangExpr_t \pureStep{\datalangProg_t} \datalangExpr_t'
            \and
                \iSim[\iProt]{\iPred}{\datalangExpr_s}{\datalangExpr_t'}
            }{
                \iSim[\iProt]{\iPred}{\datalangExpr_s}{\datalangExpr_t}
            }
        \\
        \inferrule*[lab=SimApplyProtocol]
            {
              \iProt (\iPredTwo, \datalangExpr_s, \datalangExpr_t)
            \and
               \forall \datalangExpr_s', \datalangExpr_t' \ldotp
               \iPredTwo (\datalangExpr_s', \datalangExpr_t') \iSepImp
               \iSim[\iProt]{\iPred}{\datalangExpr_s'}{\datalangExpr_t'}
            }{
                \iSim[\iProt]{\iPred}{\datalangExpr_s}{\datalangExpr_t}
            }
        \\
        \inferrule*[lab=SimSrcBlock1]
            {
                \iSim[\iProt]{\iPred}{
                    \begin{array}{l}
                        \datalangLet{\datalangVar_1}{\datalangExpr_{s1}}{\\
                        \datalangLet{\datalangVar_2}{\datalangExpr_{s2}}{\\
                        \datalangBlockDet{\datalangTag}{\datalangVar_1}{\datalangVar_2}}}
                    \end{array}
                }{\datalangExpr_t}
            }{
                \iSim[\iProt]{\iPred}{\datalangBlock{\datalangTag}{\datalangExpr_{s1}}{\datalangExpr_{s2}}}{\datalangExpr_t}
            }
        \and
        \inferrule*[lab=SimSrcBlock2]
            {
                \iSim[\iProt]{\iPred}{
                    \begin{array}{l}
                        \datalangLet{\datalangVar_2}{\datalangExpr_{s2}}{\\
                        \datalangLet{\datalangVar_1}{\datalangExpr_{s1}}{\\
                        \datalangBlockDet{\datalangTag}{\datalangVar_1}{\datalangVar_2}}}
                    \end{array}
                }{\datalangExpr_t}
            }{
                \iSim[\iProt]{\iPred}{\datalangBlock{\datalangTag}{\datalangExpr_{s1}}{\datalangExpr_{s2}}}{\datalangExpr_t}
            }
        \and
        \inferrule*[lab=SimTgtBlock]
            {
                \iSim[\iProt]{\iPred}{\datalangExpr_s}{
                    \begin{array}{l}
                        \datalangLet{\datalangVar_1}{\datalangExpr_{t1}}{\\
                        \datalangLet{\datalangVar_2}{\datalangExpr_{t2}}{\\
                        \datalangBlockDet{\datalangTag}{\datalangVar_1}{\datalangVar_2}}}
                    \end{array}
                }
            \and
                \iSim[\iProt]{\iPred}{\datalangExpr_s}{
                    \begin{array}{l}
                        \datalangLet{\datalangVar_2}{\datalangExpr_{t2}}{\\
                        \datalangLet{\datalangVar_1}{\datalangExpr_{t1}}{\\
                        \datalangBlockDet{\datalangTag}{\datalangVar_1}{\datalangVar_2}}}
                    \end{array}
                }
            }{
                \iSim[\iProt]{\iPred}{\datalangExpr_s}{\datalangBlock{\datalangTag}{\datalangExpr_{t1}}{\datalangExpr_{t2}}}
            }
        \\
        \inferrule*[lab=SimSrcBlockDet]
            {
                \forall \datalangLoc_s \ldotp
                \datalangLoc_s \iPointsto_s (\datalangTag, \datalangVal_{s1}, \datalangVal_{s2}) \iSepImp
                \iSim[\iProt]{\iPred}{\datalangLoc_s}{\datalangExpr_t}
            }{
                \iSim[\iProt]{\iPred}{\datalangBlockDet{\datalangTag}{\datalangVal_{s1}}{\datalangVal_{s2}}}{\datalangExpr_t}
            }
        \and
        \inferrule*[lab=SimTgtBlockDet]
            {
                \forall \datalangLoc_t \ldotp
                \datalangLoc_t \iPointsto_t (\datalangTag, \datalangVal_{t1}, \datalangVal_{t2}) \iSepImp
                \iSim[\iProt]{\iPred}{\datalangExpr_s}{\datalangLoc_t}
            }{
                \iSim[\iProt]{\iPred}{\datalangExpr_s}{\datalangBlockDet{\datalangTag}{\datalangVal_{t1}}{\datalangVal_{t2}}}
            }
        \\
        \inferrule*[lab=SimSrcLoad]
            {
                (\datalangLoc_s + \datalangIdx) \iPointsto_s \datalangVal_s
            \\\\
                (\datalangLoc_s + \datalangIdx) \iPointsto_s \datalangVal_s \iSepImp
                \iSim[\iProt]{\iPred}{\datalangVal_s}{\datalangExpr_t}
            }{
                \iSim[\iProt]{\iPred}{\datalangLoad{\datalangLoc_s}{\datalangIdx}}{\datalangExpr_t}
            }
        \and
        \inferrule*[lab=SimSrcStore]
            {
                (\datalangLoc_s + \datalangIdx) \iPointsto_s \datalangVal_s
            \\\\
                (\datalangLoc_s + \datalangIdx) \iPointsto_s \datalangVal_s' \iSepImp
                \iSim[\iProt]{\iPred}{\datalangUnit}{\datalangExpr_t}
            }{
                \iSim[\iProt]{\iPred}{\datalangStore{\datalangLoc_s}{\datalangIdx}{\datalangVal_s'}}{\datalangExpr_t}
            }
        \\
        \inferrule*[lab=SimTgtLoad]
            {
                (\datalangLoc_t + \datalangIdx) \iPointsto_t \datalangVal_t
            \\\\
                (\datalangLoc_s + \datalangIdx) \iPointsto_t \datalangVal_t \iSepImp
                \iSim[\iProt]{\iPred}{\datalangExpr_s}{\datalangVal_t}
            }{
                \iSim[\iProt]{\iPred}{\datalangExpr_s}{\datalangLoad{\datalangLoc_t}{\datalangIdx}}
            }
        \and
        \inferrule*[lab=SimTgtStore]
            {
                (\datalangLoc_t + \datalangIdx) \iPointsto_t \datalangVal_t
                \\\\
                (\datalangLoc_t + \datalangIdx) \iPointsto_t \datalangVal_t' \iSepImp
                \iSim[\iProt]{\iPred}{\datalangExpr_s}{\datalangUnit}
            }{
                \iSim[\iProt]{\iPred}{\datalangExpr_s}{\datalangStore{\datalangLoc_t}{\datalangIdx}{\datalangVal_t'}}
            }
        \\
        \inferrule*[lab=SimBijInsert]
            {
                \datalangLoc_s \iPointsto_s \datalangVal_s
            \and
                \datalangLoc_t \iPointsto_t \datalangVal_t
            \and
                \datalangVal_s \iSimilar \datalangVal_t
            \and
                \datalangLoc_s \iInBij \datalangLoc_t \iSepImp
                \iSim[\iProt]{\iPred}{\datalangExpr_s}{\datalangExpr_t}
            }{
                \iSim[\iProt]{\iPred}{\datalangExpr_s}{\datalangExpr_t}
            }
        \\
        \inferrule*[lab=SimLoad]
            {
                \datalangLoc_s \iSimilar \datalangLoc_t
            \and
                \forall \datalangVal_s, \datalangVal_t \ldotp
                \datalangVal_s \iSimilar \datalangVal_t
                \iSepImp
                \iPred (\datalangVal_s, \datalangVal_t)
            }{
                \iSim[\iProt]{\iPred}{\datalangLoad{\datalangLoc_s}{\datalangIdx}}{\datalangLoad{\datalangLoc_t}{\datalangIdx}}
            }
        \and
        \inferrule*[lab=SimStore]
            {
                \datalangLoc_s \iSimilar \datalangLoc_t
            \and
                \datalangVal_s \iSimilar \datalangVal_t
            \and
                \iPred (\datalangUnit, \datalangUnit)
            }{
                \iSim[\iProt]{\iPred}{\datalangStore{\datalangLoc_s}{\datalangIdx}{\datalangVal_s}}{\datalangStore{\datalangLoc_t}{\datalangIdx}{\datalangVal_t}}
            }
    \end{mathparpagebreakable}
    \caption{Reasoning rules for simulation (excerpt, omitting freshness conditions)}
    \label{fig:sim_rules}
\end{figure}

\separate{\clearpage}{}
\section{Abstract protocols} \label{sec:protocols} \label{subsec:protocols}

\begin{figure}[tp]
    \begin{tabular}{rcl}
            $\iProt_\mathrm{dir} (\iPredTwo, \datalangExpr_s, \datalangExpr_t)$
            & $\coloneqq$ &
            $\exists \datalangFn, \datalangVal_s, \datalangVal_t \ldotp$
        \\
            &&
            $\datalangFn \in \dom{\datalangProg_s} \iSep
            \datalangExpr_s = \datalangCall{\datalangFnptr{\datalangFn}}{\datalangVal_s} \iSep
            \datalangExpr_t = \datalangCall{\datalangFnptr{\datalangFn}}{\datalangVal_t} \iSep {}$
        \\
            &&
            $\datalangVal_s \iSimilar \datalangVal_t \iSep {}$
        \\
            &&
            $\forall \datalangValTwo_s, \datalangValTwo_t \ldotp
            \datalangValTwo_s \iSimilar \datalangValTwo_t \iWand
            \iPredTwo (\datalangValTwo_s, \datalangValTwo_t)$
        \\
            $\iProt_\mathrm{DPS} (\iPredTwo, \datalangExpr_s, \datalangExpr_t)$
            & $\coloneqq$ &
            $\exists \datalangFn, \datalangFn_\mathit{dps}, \datalangVal_s, \datalangLoc_1, \datalangLoc_2, \datalangLoc, \datalangIdx, \datalangVal_t \ldotp$
        \\
            &&
            $\datalangFn \in \dom{\datalangProg_s} \iSep
            \datalangRenaming [\datalangFn] = \datalangFn_\mathit{dps} \iSep
            \datalangExpr_s = \datalangCall{\datalangFnptr{\datalangFn}}{\datalangVal_s} \iSep
            \datalangExpr_t = \datalangCall{\datalangFnptr{\datalangFn_\mathit{dps}}}{\datalangLoc_1} \iSep {}$
        \\
            &&
            $(\datalangLoc_1 + 1) \iPointsto_t (\datalangLoc_2, \datalangVal_t) \iSep
            (\datalangLoc_2 + 1) \iPointsto_t (\datalangLoc, \datalangIdx) \iSep
            (\datalangLoc + \datalangIdx) \iPointsto \datalangHole \iSep
            \datalangVal_s \iSimilar \datalangVal_t$
        \\
            &&
            $\forall \datalangValTwo_s, \datalangValTwo_t \ldotp
            (\datalangLoc + \datalangIdx) \iPointsto \datalangValTwo_t \iSep
            \datalangValTwo_s \iSimilar \datalangValTwo_t \iWand
            \iPredTwo (\datalangValTwo_s, \datalangUnit)$
        \\
            $\iProt_\mathrm{TMC}$
            & $\coloneqq$ &
            $\iProt_\mathrm{dir} \sqcup \iProt_\mathrm{DPS}
            =
            \lambdaAbs (\iPredTwo, \datalangExpr_s, \datalangExpr_t) \ldotp \iProt_\mathrm{dir} (\iPredTwo, \datalangExpr_s, \datalangExpr_t) \iOr \iProt_\mathrm{DPS} (\iPredTwo, \datalangExpr_s, \datalangExpr_t)$
    \end{tabular}
    \caption{TMC protocol ($\iProt_\mathrm{TMC}$)}
    \label{fig:protocol}
\end{figure}

In \cref{sec:simulation}, we explain how our relation is defined coinductively and the first step of the proof essentially amounts to coinduction.
To internalize the coinduction hypothesis into the program logic, we introduce an additional parameter $\iProt$, a \emph{protocol}~\citep*{protocols-2021}, which are general proof-state transformers of type
\begin{mathline}
(\datalangExpr[] \to \datalangExpr[] \to \iProp) \to \datalangExpr[] \to \datalangExpr[] \to \iProp
\end{mathline}

Protocols are used in the logic via the $\RefTirName{RelProtocol}$ rule.
A pair of expressions $\datalangExpr_s$ and $\datalangExpr_t$ is supported by the protocol when it relates them to a postcondition $\iPredTwo$, capturing the possible results of an abstract/axiomatic transition from $\datalangExpr_s$ and $\datalangExpr_t$.
To conclude that $\datalangExpr_s$ and $\datalangExpr_t$ are related, one must prove that any two $\datalangExpr_s'$ and $\datalangExpr_t'$ accepted by this postcondition $\iPredTwo$ remain related.

\subsection{TMC protocols}

In our correctness proof for the TMC transformation, we use a sepecific protocol $\iProt_\mathrm{TMC}$ defined in \cref{fig:protocol} by combining two sub-protocols $\iProt_\mathrm{dir}$ and $\iProt_\mathrm{DPS}$ for the direct-style and DPS-style functions. Our coinduction hypothesis assumes toplevel function calls to be compatible with the direct and DPS specifications that we want to prove, and allows to reason about recursive calls to those functions inside the function bodies we are trying to relate.

$\iProt_\mathrm{dir}$ specifies the \emph{direct calling convention} induced by the direct transformation.
It requires $\datalangExpr_s$ and $\datalangExpr_t$ to be function calls to the same function with similar arguments.
To apply it, users can choose any postcondition implied by value similarity.
This rule is equivalent to the \TirName{Sim-Call} rule of \Simuliris, and most useful protocols are formed by combining $\iProp_\mathrm{dir}$ with other, more specialized protocols.

$\iProt_\mathrm{DPS}$ specifies the \emph{DPS calling convention} induced by the DPS transformation.
It requires $\datalangExpr_s$ to be a function call to a TMC-transformed function $\datalangFn$ and $\datalangExpr_t$ to be a function call to the DPS transform of $\datalangFn$.
As in the DPS specification, ownership of the destination must be passed to the protocol.
To apply it, users can choose any postcondition implied by the postcondition of the DPS specification, including the recovered ownership of the modified destination.

% In the instantiated program logic, the application of the $\iProt_\mathrm{TMC}$ protocol is equivalent to the following rules:
% \begin{mathpar}
%     \inferrule*[lab=RelDir]
%         {
%             \datalangFn \in \dom{\datalangProg_s}
%         \\\\
%             \datalangVal_s \iSimilar \datalangVal_t
%         \\\\
%             \forall \datalangValTwo_s, \datalangValTwo_t \ldotp
%             \datalangValTwo_s \iSimilar \datalangValTwo_t \iWand
%             \iPred (\datalangValTwo_s, \datalangValTwo_t)
%         }{
%             \iSimv{
%                 \iPred
%             }{
%                 \datalangCall{\datalangFnptr{\datalangFn}}{\datalangVal_s}
%             }{
%                 \datalangCall{\datalangFnptr{\datalangFn}}{\datalangVal_t}
%             }
%         }
%     \and
%     \inferrule*[lab=RelDPS]
%         {
%             \datalangRenaming [\datalangFn] = \datalangFn_\mathit{dps}
%         \\\\
%             (\datalangLoc_1 + 1) \iPointsto_t (\datalangLoc_2, \datalangVal_t)
%         \\\\
%             (\datalangLoc_2 + 1) \iPointsto_t (\datalangLoc, \datalangIdx)
%         \\\\
%             (\datalangLoc + \datalangIdx) \iPointsto_t \datalangHole
%         \\\\
%             \datalangVal_s \iSimilar \datalangVal_t
%         \\\\
%             \forall \datalangValTwo_s, \datalangValTwo_t \ldotp
%             (\datalangLoc + \datalangIdx) \iPointsto_t \datalangValTwo_t \iWand
%             \datalangValTwo_s \iSimilar \datalangValTwo_t \iWand
%             \iPred (\datalangValTwo_s, \datalangUnit)
%         }{
%             \iSimv{
%                 \iPred
%             }{
%                 \datalangCall{\datalangFnptr{\datalangFn}}{\datalangVal_s}
%             }{
%                 \datalangCall{\datalangFnptr{\datalangFn_\mathit{dps}}}{\datalangLoc_1}
%             }
%         }
% \end{mathpar}

\subsection{Other examples of protocols}

Our program logic can be instantiated with other protocols to reason other program transformations.
To demonstrate this generality, we have also verified an inlining and an accumulator-passing-style (APS) transformation --- both included in our mechanization.

\paragraph{Inlining:} Here, the relation $\datalangExpr_s \rightsquigarrow \datalangExpr_t$ allows $\datalangExpr_t$ to recursively inline functions in $\datalangExpr_s$.
As with TMC, it captures all possible inlining strategies.

This relation can be proved correct by using a fairly simple protocol (combined with $\iProt_\mathrm{dir}$) relating a source function and its body:

\begin{tabular}{rcl}
        $\iProt_\mathrm{inline} (\iPredTwo, \datalangExpr_s, \datalangExpr_t)$
        & $\coloneqq$ &
        $\exists \datalangFn, \datalangVar, \datalangExpr_s', \datalangExpr_t', \datalangVal_s, \datalangVal_t \ldotp$
    \\
        &&
        $
        \datalangExpr_s = \datalangCall{\datalangFnptr{\datalangFn}}{\datalangVal_s}
        \ \iSep\ %
        \datalangVal_s \iSimilar \datalangVal_t
        \ \iSep {}$
    \\
        &&
        $\datalangProg_s [\datalangFn] = (\datalangRec{\datalangVar}{\datalangExpr_s'})
        \ \iSep\ %
        \datalangExpr_s' \rightsquigarrow \datalangExpr_t'
        \ \iSep\ %
        \datalangExpr_t = (\datalangLet{\datalangVar}{\datalangVal_t}{\datalangExpr_t'})
        \ \iSep {}$
    \\
        &&
        $\forall \datalangValTwo_s, \datalangValTwo_t \ldotp
        \datalangValTwo_s \iSimilar \datalangValTwo_t \iWand
        \iPredTwo (\datalangValTwo_s, \datalangValTwo_t)$
\end{tabular}
\medskip

\paragraph{Accumulator-passing style}: The APS transformation is a variant of the TMC transformation where the contexts that are made tail-recursive are applications of associative arithmetic operators, typically of the form $(\datalangExpr + \datalangCtxHole)$ or $\datalangExpr_1 + (\datalangExpr_2 \times \datalangCtxHole)$. (See the discussion in \citet*{tmc-koka-2023}.)

We defined an APS transformation, after extending \DataLang with integers and arithmetic operations. We verify it with a protocol similar to $\iProt_\mathrm{DPS}$ that allows calling the APS transform of a source function with an integer accumulator:

\medskip
\begin{tabular}{rcl}
        $\iProt_\mathrm{APS} (\iPredTwo, \datalangExpr_s, \datalangExpr_t)$
        & $\coloneqq$ &
        $\exists \datalangFn, \datalangFn_\mathit{aps}, \datalangVal_s, \datalangVal_\mathit{acc}, \datalangVal_t \ldotp$
    \\
        &&
        $\datalangFn \in \dom{\datalangProg_s}
        \ \iSep\ %
        \datalangRenaming [\datalangFn] = \datalangFn_\mathit{aps}
        \iSep {}$
    \\
        &&
        $\datalangVal_s \iSimilar \datalangVal_t
        \ \iSep\ %
        \datalangExpr_s = \datalangCall{\datalangFnptr{\datalangFn}}{\datalangVal_s}
        \ \iSep\ %
        \datalangExpr_t = \datalangCall{\datalangFnptr{\datalangFn_\mathit{aps}}}{\datalangPair{\datalangVal_\mathit{acc}}{\datalangVal_t}}
        \iSep {}$
    \\
        &&
        $\forall \datalangVal_s', \datalangExpr_t' \ldotp$
    \\
        &&
        $\bm{\mathrm{match}}\ \datalangVal_s'\ \bm{\mathrm{with}}\ \datalangNat \Rightarrow \datalangExpr_t' = \datalangVal_\mathit{acc} + \datalangNat \mid \_ \Rightarrow \mathrm{strongly \mathhyphen stuck}_{\datalangProg_t} (\datalangExpr_t')\ \bm{\mathrm{end}} \iWand$
    \\
        &&
        $\iPredTwo (\datalangVal_s', \datalangExpr_t')$
\end{tabular}
\medskip

One subtlety is that our \DataLang language is untyped, so arithmetic operations (here addition) may get stuck on non-integer values. If the function call $\datalangCall {\datalangFnptr \datalangFn} {\datalangVal_s}$ in the source program returns a non-integer value, then the outer context $\datalangVar_\mathit{acc} + \datalangCtxHole$ gets stuck. But in the transformed program, this failure happens inside the body of the APS-transformed function $\datalangFnptr {\datalangFn_\mathit{aps}}$. To represent this failure case in our protocol, the postcondition $\iPredTwo$ relates a non-integer source return value $\datalangVal_s'$ with any strongly stuck expression $\datalangExpr_t'$ in the target. This relies on the generality of our protocols being predicate transformers on expressions, not just values.

%%% Local Variables:
%%% mode: latex
%%% TeX-master: "main"
%%% End:


\separate{\clearpage}{}
\section{Proof of the specification}
\label{sec:proof}

In this section, we prove the specifications of \cref{sec:specification}.

As mentioned in \cref{sec:protocols}, we instantiate our program logic with a specific protocol $\iProt_\mathrm{TMC}$ defined in \cref{fig:protocol}. We define a shorthand notation for this instantiation:
\begin{mathline}
    \iSimv{
        \iPred
    }{
        \datalangExpr_s
    }{
        \datalangExpr_t
    }
    \coloneqq
    \iSimv[
        \iProt_\mathrm{TMC}
    ]{
        \iPred
    }{
        \datalangExpr_s
    }{
        \datalangExpr_t
    }
\end{mathline}

So far, we worked with \emph{closed} expressions, that have no free variables.\Xfrancois{This was never said.}
We need to generalize the specifications to \emph{open} expressions that may have free variables, as is standard.
To do so, we introduce a \emph{runtime relation} $\iRsimv{\iPred}{\datalangExpr_s}{\datalangExpr_t}$ in \cref{fig:rsim}.
It requires $\datalangBisubst_s (\datalangExpr_s)$ and $\datalangBisubst_t (\datalangExpr_t)$ to be related for any \emph{well-formed closing bisubstitution} $\datalangBisubst \in \datalangVar[] \rightarrow \datalangVal[] \times \datalangVal[]$.
In practice, $\datalangBisubst$ contains $\datalangLetKeyword$-bound variables that have been $\beta$-reduced, and their substitute source and target values.

In addition, we will only consider \emph{valid source expressions} --- denoted by $\wf{\datalangExpr_s}$ ---, \ie those that do not involve any location, deterministic block expressions, or undefined source function.

\begin{lemma}[Specification of direct transformation] \label{thm:dir}
    \[
        \iRsimvHoare{
            \wf{\datalangExpr_s} \iSep
            \datalangExpr_s \tmcDir{\datalangRenaming} \datalangExpr_t
        }{
            \datalangVal_s, \datalangVal_t \ldotp
            \datalangVal_s \iSimilar \datalangVal_t
        }{
            \datalangExpr_s
        }{
            \datalangExpr_t
        }
    \]
\end{lemma}
\Xfrancois{Here and below the pure assumptions should be put outside, with an implication, they are ``static'' (compile-time) assumptions.}

\begin{lemma}[Specification of DPS transformation] \label{thm:dps}
    \[
        \iRsimvHoare{
            \wf{\datalangExpr_s} \iSep
            (\datalangLoc, \datalangIdx, \datalangExpr_s) \tmcDps{\datalangRenaming} \datalangExpr_t \iSep
            (\datalangLoc + \datalangIdx) \iPointsto_t \datalangHole
        }{
            \datalangVal_s, \datalangUnit \ldotp
            \exists \datalangVal_t \ldotp
            (\datalangLoc + \datalangIdx) \iPointsto_t \datalangVal_t \iSep
            \datalangVal_s \iSimilar \datalangVal_t
        }{
            \datalangExpr_s
        }{
            \datalangExpr_t
        }
    \]
\end{lemma}
\Xfrancois{TODO point back to the beginning of the paper for an informal explanation of these statements.}

Both proofs proceed by induction over $\datalangExpr_s$ and mutual induction over $\datalangExpr_s \tmcDir{\datalangRenaming} \datalangExpr_t$ and $(\datalangLoc, \datalangIdx, \datalangExpr_s) \tmcDps{\datalangRenaming} \datalangExpr_t$.% -- we refer to our mechanization for the details.
    % We only sketch the cases shown in \cref{fig:tmc_dir} and \cref{fig:tmc_dps} and refer to our mechanization for the complete proof.
    % \begin{itemize}[align=left, leftmargin=*]
    %     \item[\RefTirName{DirVal}:] Follows from $\wf{\datalangVal}$.
    %     \item[\RefTirName{DirVar}:] Follows from $\wf{\datalangBisubst}$.
    %     \item[\RefTirName{DirLet}:] Follows from IH for $\datalangExpr_{s1}$ and IH for $\datalangExpr_{s2}$.
    %     \item[\RefTirName{DirBlockDPS1}:] Follows from \RefTirName{RelSrcBlock1}, \RefTirName{RelTgtBlock}, IH for $\datalangExpr_{s1}$, \RefTirName{RelTgtBlockDet}, IH for $\datalangExpr_{s2}$, \RefTirName{RelSrcBlockDet} and \RefTirName{BijInsert}.
    %     \item[\RefTirName{DirBlockDPS2}:] Similar to \RefTirName{DirBlockDPS1}.
    %     \item[\RefTirName{DPSBase}:] Follows from IH for $\datalangExpr_s \tmcDir{\datalangRenaming} \datalangExpr_t$ and \RefTirName{RelTgtStore}.
    %     \item[\RefTirName{DPSLet}:] Follows from IH for $\datalangExpr_{s1}$ and IH for $\datalangExpr_{s2}$.
    %     \item[\RefTirName{DPSCall}:] Follows from \RefTirName{SimTgtBlock}, \RefTirName{SimTgtBlockDet}, IH for $\datalangExpr_s$ and \RefTirName{RelDPS}.
    %     \item[\RefTirName{DPSIf}:] Follows from IH for $\datalangExpr_{s0}$, IH for $\datalangExpr_{s1}$ and IH for $\datalangExpr_{s2}$.
    %     \item[\RefTirName{DPSBlock1}:] Follows from \RefTirName{RelSrcBlock1}, \RefTirName{RelTgtBlock}, IH for $\datalangExpr_{s1}$, \RefTirName{RelTgtBlockDet}, \RefTirName{RelTgtStore} IH for $\datalangExpr_{s2}$, \RefTirName{RelSrcBlockDet} and \RefTirName{BijInsert}.
    %     \item[\RefTirName{DPSBlock2}:] Similar to \RefTirName{DPSBlock1}. \qedhere
    % \end{itemize}

\begin{figure}[tp]
    \begin{align*}
    		\wf{\datalangBisubst}
    		&\coloneqq
    		\forall \datalangVar \ldotp
    		\exists \datalangVal_s, \datalangVal_t \ldotp
    		\datalangBisubst (\datalangVar) = (\datalangVal_s, \datalangVal_t) \iSep
    		\datalangVal_s \iSimilar \datalangVal_t
    	\\
    		\iRsimv[\iProt]{\iPred}{\datalangExpr_s}{\datalangExpr_t}
    		&\coloneqq
    		\forall \datalangBisubst \ldotp
    		\wf{\datalangBisubst} \iSepImp
    		\iSimv[\iProt]{\iPred}{\datalangBisubst_s (\datalangExpr_s)}{\datalangBisubst_t (\datalangExpr_t)}
    	\\
    	   \iRsimvHoare[\iProt]{P}{\iPred}{\datalangExpr_s}{\datalangExpr_t}
    	   &\coloneqq
    	   \iPersistent \left( P \iSepImp \iRsimv[\iProt]{\iPred}{\datalangExpr_s}{\datalangExpr_t} \right)
    \end{align*}
    \caption{Logical relation on open expressions}
    \label{fig:rsim}
\end{figure}

%%% Local Variables:
%%% mode: latex
%%% TeX-master: "main"
%%% End:


\separate{\clearpage}{}
\section{Simulation}
\label{sec:simulation}

% ---------------------------------------------------------

%\subsection{The quest for the right notion of simulation}
%\label{sec:howto-relation}
%
%The final theorem we want to establish is a behaviour refinement $\datalangProg_s \refined \datalangProg_t$: the behaviours of the transformed program are included in the behaviour of the source program.
%%
%We propose a notion of behaviour refinement that gives a form of total correctness.
%%
%It is termination-preserving, but also (this is less common in the body of work around Iris) safety-preserving and divergence-preserving.
%
%We prove this using a backward simulation argument for our transformations $\datalangExpr_s \tmc \datalangExpr_t$.
%%
%A direct approach, ultimately unsucessful, would be to prove that the relation is in the simulation by induction, on $\datalangExpr_s$ for example.
%%
%This works well for all constructions except for function calls $\datalangCall{\datalangFnptr{\datalangFn}}{\datalangExpr}$: to reason on the relation between a call to $\datalangFnptr{\datalangFn}$ and its transformation (in direct or DPS style), we would need to inline the function definition, which is not structurally decreasing.
%
%This is a common problem to establish simulations, and the solution is to use a form of co-induction.
%%
%In the Iris logic, a natural way to do this would be to use a ``later'' modality.
%%
%But defining simulations using a ``later'' modality is in fact non-trivial; simple definitions do not give the excepted notion of simulation, due to non-intuitive aspects of the step-indexing semantics of ``later'' in the Iris model. This problem is discussed in details in the previous work on Transfinite Iris~\citep*{TODO-transfinite-Iris}, an alternative logic with different axioms and models.
%%
%Sticking to the main Iris logic, a good definition of simulation has been worked out in Simuliris~\citep*{TODO-Simuliris}.
%%
%It avoids using the ``later'' modality by using the notion of coinduction native to Coq. In Simuliris, coinductive reasoning steps correspond to an ``abstract'' case in the simulation relation, used for function applications; an ``open simulation'' allows these abstract steps, while a ``closed simulation'' does not allow them.
%%
%The coinduction principle is encapsulated in a ``simulation closure'' theorem, which states if two terms are related in the open simulation, they are in fact also related in the closed simulation if we (coinductively) assume that foreign function bodies are correct.
%
%Our proof is heavily inspired by the Simuliris work, but we cannot reuse their results directly because the notion of simulation that they provide is not expressive enough for our proof: we need to support transformations that change the calling conventions between source and target programs.
%%
%(Along the way we also simplified the simulation relation by removing proof rules related to concurrency, which we do not consider in this work.)
%
%To support different calling conventions, we make our simulation relation $\iSim[\iProt]{\iPred}{\datalangExpr_s}{\datalangExpr_t}$ parametric over abstract protocols $\iProt$. This extends the technique of \citet*{TODO-paulo} from the unary setting of safety predicates to the binary setting of a relational separation logic.
%
%\subsection{Proof outline}
%
%Once we have decided to reuse and extend the Simuliris work, our argument can be split in three separate parts:
%
%\begin{enumerate}
%
%\item Define our notion of simulation parametric over protocols (calling conventions), prove its adequacy (it implies a behaviour refinement) and establish a ``closure theorem'' as in the Simuliris work.
%%
%  This technical contribution is independent of the TMC language or transformation, and could be reused in other works.
%
%\item Build a relational program logic for our programming language, as a body of lemmas to establish simulations between expressions.
%
%\item Use this program logic to establish the soundness of the TMC transformation.
%%
%The key ingredients are two specifications for the direct and DPS transformations that are proved in a mutually-inductive way, and are used to prove the hypothesis of the closure theorem of our simulation.
%\end{enumerate}

% ---------------------------------------------------------

%Simuliris~\citep*{TODO-simuliris} suggests a definition of simulation relations in Iris. As we explained in \cref{sec:howto-relation}, it carefully avoids using the ``later'' modality by using Coq's native notion of coinduction. Simuliris has several desirable properties for us: it can be defined in the un-modified Iris base logic (unlike Transfinite Iris~\citep*{transfinite-iris}), it is defined in a way that makes it easy to specialize to other programming languages, and it supports showing the preservation of termination. On the other hand, the original definition is complex as it supports concurrency and notions of fairness. We reused the definition of Simuliris, simplified for the sequential setting. We then extended the resulting definition to support our treatment of stuck states as failures, and generalize the notion of external calls to abstract protocols $\iProt$. Most of the ideas below come from the Simuliris work directly, we will explicitly point out the parts that differ.
%
%\paragraph{Relating states} The \emph{state interpretation} $I(\datalangState_s, \datalangState_t)$ relates source and target states. It extends the unary notion of state interpretation predicate $I(\datalangState)$. Recall that in separation logic, formulas contain both pure propositions and ressources, which include in particular logical points-to assertions $\datalangLoc \iPointsto v$. $I(\datalangState)$ enforces the consistency between those logical assertions and the physical state $\datalangState$.
%
%In the relational setting, the state interpretation becomes a relation $I(\datalangState_s, \datalangState_t)$ tracking points-to predicates on either states $\datalangLoc_s \iPointsto_s \datalangVal_s$ and $\datalangLoc_t \iPointsto_t \datalangVal_t$. It is additionally in charge of maintaining the heap bijection mentioned in \cref{TODO} by tracking predicates $\datalangLoc_s \iInBij \datalangLoc_t$. This assertion is persistent: one cannot remove locations from the bijection.
%
%\paragraph{Relating values} TODO relation between values.
%
%\paragraph{Relating expressions} The definition of the expression relation $\iSim[\iProt]{\iPred}{\datalangExpr_s}{\datalangExpr_t}$ is shown in \cref{fig:sim}; it reads as ``$\datalangExpr_s$ simulates $\datalangExpr_t$ under the postcondition $\iPred$ and protocol $\iProt$''. It expresses that the source $\datalangExpr_s$ can simulate any computation of the target $\datalangExpr_t$ until they both get stuck, both diverge, or reach two expressions in the relational post-condition $\iPred$. The parameter $\iProt$ is the relational protocol that must be followed by function calls in $\datalangExpr_s$ and $\datalangExpr_t$ -- generalizing the Simuliris rule for external calls.
%
%\paragraph{Mixed induction and coinduction}
%The definition of the relation itself starts with a double fixpoint, a greatest fixpoint $\nuAbs sim .$ (coinduction) of smallest fixpoint $\muAbs {sim \mathhyphen inner} .$ (induction) of a relation $\mathrm{sim \mathhyphen body}_\iProt$. Inside the relation definition, the cases that use the inductive $sim \mathhyphen inner$ in their recursive occurrence can only be repeated finitely many times, while the cases that use the coinductive $sim$ in their recursive occurrence can be repeated infinitely, but can only do so under a form of guardedness condition.
%
%\paragraph{Simulation clauses} The relation between $\datalangExpr_s$ and $\datalangExpr_t$, at a given post-condition $\iPred$, must be parametric over all physical states in the state interpretation relation $I(\datalangState_s, \datalangState_t)$. It is established by one of the following clauses:
%
%\begin{enumerate}
%\item[\circled{1}] Halting clause: $\datalangExpr_s$, $\datalangExpr_t$ can stop if they are already in the post-condition $\iPred$ -- and the states are still related.
%\item[\circled{2}]
%\item[\circled{3}]
%\item[\circled{4}]
%\item[\circled{5}]
%\end{enumerate}
%
%% sim-body:
%% cas 1, 2, 3, 4: on raconte
%% "ouverte" comprend un autre cas.
%% cas 5: les appels externes, protocoles (on raconte)
%
%\begin{theorem}[Close simulation]
%    \begin{align*}
%            &
%            \iPersistent \left(
%                \forall \iPredTwo, \datalangExpr_s, \datalangExpr_t \ldotp
%                \iProt (\iPredTwo, \datalangExpr_s, \datalangExpr_t) \iSepImp
%                \mathrm{sim \mathhyphen inner}_\bot (\lambdaAbs (\_, \datalangExpr_s', \datalangExpr_t') \ldotp \iSim[\iProt]{\iPredTwo}{\datalangExpr_s'}{\datalangExpr_t'}) (\bot, \datalangExpr_s, \datalangExpr_t)
%            \right) \iSepImp
%        \\
%            &
%            \iSim[\iProt]{\iPred}{\datalangExpr_s}{\datalangExpr_t} \iSepImp
%            \iSim[\bot]{\iPred}{\datalangExpr_s}{\datalangExpr_t}
%    \end{align*}
%\end{theorem}

%\subsection{Soundness theorem}
%
%The TMC transformation preserves the behaviours of programs in a strong sense.
%We expect terminating programs to remain terminating, diverging programs to remain diverging and failing programs (those that get stuck) to remain failing.
%
%To formalize it, we first define the notion of behaviour.
%
%we define in \cref{fig:refinement} the notions of behaviours and behaviour refinement as in \Simuliris~\cite{DBLP:journals/pacmpl/GaherSSJDKKD22}.
%TODO
%
%We define the set $\mathrm{behaviours}_{\datalangProg} (\datalangExpr)$ of behaviours of an expression $\datalangExpr$ within a program $\datalangProg$, starting from an empty store.
%The behaviour $\constr{Conv}(\datalangVal_s)$ indicates that $\datalangExpr$ can evaluate to $\datalangVal_s$.
%The behaviour $\constr{Conv}(\datalangExpr')$, when $\datalangExpr'$ is not a value, indicates that $\datalangExpr$ can reduce to a stuck configuration $(\datalangExpr', \datalangState)$.
%Finally, $\constr{Div}$ indicates that $\datalangProg$ can diverge.
%
%The refinement relation on behaviours $b_s \refined b_t$ then follows in a natural way: divergence refines divergence, any stuck expression refines any other stuck expression, and a value $\datalangVal_t$ refines $\datalangVal_s$ when they are related for a ground equivalence $\datalangVal_s \similar \datalangVal_t$, which is the equality for all value kinds, except for locations where it gives no information, as done in Simuliris.
%This equivalence of ground values could be used, for example, to argue for the correctness of a \texttt{main} function that takes and returns integer arguments.
%
%Note a difference to Simuliris: our refinement between stuck/failing behaviours is $\constr{Div} \refined \constr{Div}$, whereas Simuliris uses $\constr{Div} \refined b_t$: the Simuliris relation assumes that the input program is safe, never gets stuck, and the behaviour refinement thus allows a stuck source term to be refined by any target program. With our definition, a source program that only gets stuck must be transformed into a target program that only gets stuck.
%This property would not be desirable for C compilers that want to optimize aggressively assuming the absence of undefined behaviors, but it does capture a finer-grained property of our program transformation.
%
%Finally, an expression $\datalangExpr_t$ refines $\datalangExpr_s$ when all its behaviours are refined by a behaviour of $\datalangExpr_s$, and a program $\datalangProg_t$ refines $\datalangProg_s$ when calls to source functions on equivalent inputs are in the refinement relation.
%
%We can now state the main soundness theorem of our work, whose proof is spread over the rest of the present \cref{sec:soundness}.
%
%\begin{theorem}[Soundness]
%    $
%        \wf{\datalangProg_s} \wedge \datalangProg_s \tmc \datalangProg_t \implies
%        \datalangProg_s \refined \datalangProg_t
%    $
%\end{theorem}
%
%The condition $\wf{\datalangProg_s}$ guarantees that the input program $\datalangProg_s$ is well-scoped. Under this condition, we can consider the target programs $\datalangProg_t$ that are TMC transformations $\datalangProg_s \tmc \datalangProg_s$, and our theorem states that they refine $\datalangProg_s$ as expected.

%\begin{theorem}[Adequacy]
%    $
%        \left( \vdash \iSimv{\iSimilar}{\datalangExpr_s}{\datalangExpr_t} \right) \implies
%        \datalangExpr_s \refined \datalangExpr_t
%    $
%\end{theorem}

% ---------------------------------------------------------

% I(sigma, sigma´) is as in Simuliris, not given explicitly in the current document.
% the \approx^bij relation refers to a component of I (a bijection between addresses)
% that is an implicit parameter of the definitions.

% ---------------------------------------------------------

% If two expressions refine internally (thy are in the
% simulation relation), and go to the internal equivalence between
% values, then they refine externally (they are in the denotational
% refinement relation).

% This is similar to the Simuliris proof, simplified thanks to the absence
% of concurrency.

% ---------------------------------------------------------

\begin{figure}[tp]
    \begin{align*}
    		\mathrm{sim \mathhyphen inner}_\Chi
    		&\coloneqq
    		\begin{array}{l}
    				\lambdaAbs sim \ldotp
    				\muAbs sim \mathhyphen inner \ldotp
    				\lambdaAbs{(\Phi, e_s, e_t)} \ldotp
    				\forall \sigma_s, \sigma_s \ldotp
    				\mathrm{I} (\sigma_s, \sigma_t)
    				\iSepImp \iBupd
    			\\
    				\bigvee \left[ \begin{array}{ll}
    							\circled{1}
    						&
    							\Phi (e_s, e_t) \iSep \mathrm{I} (\sigma_s, \sigma_t)
    					\\
    							\circled{2}
    						&
    							\exists e_s', \sigma_s' \ldotp
    							(e_s, \sigma_s) \step{p_s}^+ (e_s', \sigma_s') \iSep
    							\mathrm{I} (\sigma_s', \sigma_t) \iSep
    							sim \mathhyphen inner (\Phi, e_s', e_t)
    					\\
    							\circled{3}
    						&
								\mathrm{reducible} (e_t, \sigma_t) \iSep
								\forall e_t', \sigma_t' \ldotp
								(e_t, \sigma_t) \step{p_t} (e_t', \sigma_t')
								\iSepImp \iBupd
						\\
                            &
								\bigvee \left[ \begin{array}{ll}
											\circled{A}
										&
											\mathrm{I} (\sigma_s, \sigma_t') \iSep
											sim \mathhyphen inner (\Phi, e_s, e_t')
									\\
											\circled{B}
										&
											\exists e_s', \sigma_s' \ldotp
											(e_s, \sigma_s) \step{p_s}^+ (e_s', \sigma_s') \iSep
											\mathrm{I} (\sigma_s', \sigma_t') \iSep
											sim (\Phi, e_s', e_t')
								\end{array} \right.
    					\\
    							\circled{4}
    						&
    							\exists K_s, e_s', K_t, e_t', \Psi \ldotp
    					\\
    					    &
    					        e_s = K_s [e_s'] \iSep
    							e_t = K_t [e_t'] \iSep
    						    \Chi (\Psi, e_s', e_t') \iSep
    						    \mathrm{I} (\sigma_s, \sigma_t) \iSep {}
    					\\
                            &
								\forall e_s'', e_t'' \ldotp
								\Psi (e_s'', e_t'') \iSepImp
								sim (\Phi, K_s [e_s''], K_t [e_t''])
    				\end{array} \right.
    		\end{array}
    	\\
    		\mathrm{sim}_\Chi
    		&\coloneqq
    		\nuAbs{\mathrm{sim \mathhyphen inner}_\Chi}
    	\\
    		\iSim[\Chi]{\Phi}{e_s}{e_t}
    		&\coloneqq
    		\mathrm{sim}_\Chi (\Phi, e_s, e_t)
    	\\
    	   \iSimv[\Chi]{\Phi}{e_s}{e_t}
    	   &\coloneqq
    	   \iSim[\Chi]{\lambdaAbs (e_s', e_t') \ldotp \exists v_s, v_t \ldotp e_s' = v_s \iSep e_t' = v_t \iSep \Phi (v_s, v_t)}{e_s}{e_t}
    \end{align*}
    \caption{Simulation with protocol}
    \label{fig:sim}
\end{figure}
\begin{figure}[tp]
    \centering
    \begin{mathparpagebreakable}
        \inferrule*
            {}{
                \datalangUnit \similar \datalangUnit
            }
        \and
        \inferrule*
            {}{
                \datalangIdx \similar \datalangIdx
            }
        \and
        \inferrule*
            {}{
                \datalangTag \similar \datalangTag
            }
        \and
        \inferrule*
            {}{
                \datalangBool \similar \datalangBool
            }
        \and
        \inferrule*
            {}{
                \datalangLoc_s \similar \datalangLoc_t
            }
        \and
        \inferrule*
            {}{
                \datalangFnptr{\datalangFn} \similar \datalangFnptr{\datalangFn}
            }
    \end{mathparpagebreakable}
    \caption{Similarity in $\Prop$}
    \label{fig:similarity}
\end{figure}
\begin{figure}[tp]
    \centering
    \begin{mathparpagebreakable}
        \inferrule*
            {
                v_s \similar v_t
            }{
                \constr[v_s]{Conv} \refined \constr[v_t]{Conv}
            }
        \and
        \inferrule*
            {
                e_s \notin \nterm{Val}
            \and
                e_t \notin \nterm{Val}
            }{
                \constr[e_s]{Conv} \refined \constr[e_t]{Conv}
            }
        \and
        \inferrule*
            {}{
                \constr{Div} \refined \constr{Div}
            }
    \end{mathparpagebreakable}
    \\
    \begin{align*}
        	\mathrm{behaviours}_p (e)
        	&\coloneqq
        	\begin{array}{l}
        			\{ \constr[e']{Conv} \mid \exists \sigma \ldotp (e, \emptyset) \step{p}^* (e', \sigma) \wedge \mathrm{irreducible}_p (e', \sigma) \}\ \uplus
        		\\
        			\{ \constr{Div} \mid\ \diverges{p}{(e, \emptyset)} \}
            \end{array}
        \\
            e_s \refined e_t
            &\coloneqq
            \forall b_t \in \mathrm{behaviours}_{p_t} (e_t) \ldotp
            \exists b_s \in \mathrm{behaviours}_{p_s} (e_s) \ldotp
            b_s \refined b_t
        \\
            p_s \refined p_t
            &\coloneqq
            \forall f \in \dom{p_t}, v_s, v_t \ldotp
            \wf{v_s} \wedge v_s \similar v_t \implies
            \lambdCall{\lambdFunc{f}}{v_s} \refined \lambdCall{\lambdFunc{f}}{v_t}
    \end{align*}
    \caption{Program refinement}
    \label{fig:refinement}
\end{figure}

\separate{\clearpage}{}
\section{Related work}

\subsection{TMC support in compilers}

Tail-recursion modulo cons was well-known in the Lisp community as
early as the 1970s. For example the REMREC system~\citep*{risch-73}
would automatically transform recursive functions into loops, and
supports modulo-cons tail recursion. It also supports tail-recursion
modulo associative arithmetic operators, which is outside the scope
of our work, but supported by the GCC compiler for example. The TMC
fragment is precisely described (in prose) in \citet*{friedman-wise-75}.

In the Prolog community it is a common pattern to implement
destination-passing style through unification variables; in particular
``difference lists'' are a common representation of lists with a final
hole. Unification variables are first-class values, in particular they
can be passed as function arguments. This makes it easy to write the
destination-passing-style equivalent of a context of the form
\ocaml{List.append li _?}, as the difference list
\ocaml{(List.append li X, X)}. In constrast, we only support direct
constructor applications. However, this expressivity comes at
a performance cost, and there is no static checking that the data is
fully initialized at the end of computation.

\subsection{Reasoning about destination-passing-style}

In general, if we think of non-tail recursive functions as having an
``evaluation context'' left for after the recursive call, then the
techniques to turn classes of calls into tail-calls correspond to
different reified representations of non-tail contexts, as long as
they support efficient composition and hole-plugging. TMC comes from
representing data-construction contexts as the partial data itself,
with hole-plugging by mutation. Associative-operator transformations
represent the context \ocaml|1 + (4 + _?)| as the number \ocaml|5|
directly. (Sometimes it suffices to keep around an abstraction of the
context; this is a key idea in John Clements' work on stack-based
security in presence of tail calls.)

\cite*{minamide-98} gives a ``functional'' interface to
destination-passing-style program, by presenting a partial
data-constructor composition \ocaml{Foo(x,Bar(_?))} as a use-once,
linear-typed function \ocaml{linfun h -> Foo(x,Bar(h))}. Those special
linear functions remain implemented as partial data, but they expose
a referentially-transparent interface to the programmer, restricted by
a linear type discline. This is a beautiful way to represent
destination-passing style, orthogonal to our work: users of Minamide's
system would still have to write the transformed version by hand, and
we could implement a transformation into destination-passing style
expressed in his system. \citet*{mezzo-2016}
supports a more general-purpose type system based on separation logic,
which can directly express uniquely-owned partially-initialized data,
and its implicit transformation into immutable, duplicable
results. (See the
\href{https://protz.github.io/mezzo/code_samples/list.mz.html}{List}
module of the Mezzo standard library, and in particular \ocaml{cell},
\ocaml{freeze} and \ocaml{append} in destination-passing-style).

\Xgabriel{TODO: Leijen and Lorenzen}

\subsection{Relational reasoning in separation logic}

% + ReLoC Reloaded: A Mechanized Relational Logic for Fine-Grained Concurrency and Logical Atomicity
% + A Higher-Order Logic for Concurrent Termination-Preserving Refinement
% + Transfinite Iris: Resolving an Existential Dilemma of Step-Indexed Separation Logic
% + Simuliris: A Separation Logic Framework for Verifying Concurrent Program Optimizations

\subsection{Protocols}

% + Compiler Verification Meets Cross-Language Linking via Data Abstraction
%   axiomatic semantics are close to protocols in spirit, they include in the
%   syntactic reasoning rules an "axiomatic" step that non-deterministically
%   reduces a function call following a (P,Q) pre/post
% + A Separation Logic for Effect Handlers
%   we were inspired by Paulo: Χ allows do v { Φ }
%     + specification of an effect
%     + protocol Χ allows the program to perform the effect v and garantees that every permitted reply satisfies postcondition Φ
%     + "we would like programmers to think about performing an effect essentially in the same way as they think about calling a function"
%     + wp parameterized by protcol: ewp e < Χ > { Φ }
% + Melocoton: A Program Logic for Verified Interoperability Between OCaml and C


%%% Local Variables:
%%% mode: latex
%%% TeX-master: "main"
%%% End:


% \separate{\clearpage}{}
% \section{Conclusion and Future Work}

concurrency

effect handlers

compression of constructors

prove other program transformation using this framework (CPS)

% To give another example of simulation with a different protocol,
% we could look at accumulator-passing-style transformations
% (tail-modulo-context for associative arithmetic operators). Prove
% that `factorial` is equivalent to the tmc-optimized factorial
% (two versions, one in accumulator-passing-style and the "direct
% style" version that calls it with accumulator 1).
%    .. et donc il faut garder les entiers!
%
% Clément: j'ai commencé à bosser sur ça. Ça demande de travailler
% sur des expressions plutôt que des valeurs (l'accumulateur peut
% être un entier mais aussi une variable). Il faut regénéraliser
% des bouts du développment Coq qui était spécialisé aux
% valeurs. Il se passe un truc intéressant sur les échecs si
% l'accumulateur est une variable qui est substituée par un
% non-entier (échec pendant le calcul d'un côté, après le calcul
% de l'autre).
% La formulation où on accumule des opérations sur des expressions
% arbitraires et pas juste des constantes entières est un peu plus
% générale que celle de Daan Leijen (que des constantes). Elle permet
% de faire une transformation APS sur List.sum.

% From the Author Instruction:
%
% > POPL-specific: The sections at the end of the paper should be ordered as follows:
% > first appendices, then acknowledgments, and last references.
% > The acknowledgments and references will not count towards the page limit.
%
% by email, Conference Publishing in fact clarifies that there cannot be appendices,
% so our appendices will have to be published in a separate "full" version
% of the paper, and it is more convenient to include them at the end.

%% Acknowledgments
\begin{acks}
  This work was initiated by Frédéric Bour who implemented TMC in an
  experimental variant of the OCaml compiler and submitted his work
  for inclusion in 2015, but the proposal remained stalled due to lack
  of review and integration effort among maintainers. The broad lines
  of the final implementation, in term of generated code, were already
  in Frédéric Bour's initial version, as well as the idea to have an
  opt-in transformation controlled by an attribute
  (\cref{subsec:optin}). This experimental also generated the
  performance data that motivated us to push further. (One notable
  technical difference is that instead of each destination-passing
  style taking two parameters \ocaml|dst| and |ofs|, it would generate
  several versions specializing the \ocaml|ofs| variable to
  a constant. Frédéric Bour found that this did not noticeably improve
  performance, and changed back to a parameter to simplify the
  implementation and reduce code size.)
  
  In 2020, Gabriel Scherer restarted Frédéric Bour's effort with
  a review followed by a partial reimplementation of the
  transformation that introduced the applicative style discussed in
  \cref{subsec:applicative-impl}, and much of the current
  attribute-based user interface to control the transformation
  (\cref{subsec:user-control}). Basile Clément in turn reviewed
  Gabriel's version, and introduced constructor compression
  (\cref{subsec:constructor-compression}). Xavier Leroy implemented
  a change to the \OCaml calling convention to remove parameter-number
  restrictions on tail calls on some architecture
  (\cref{subsec:PR-history}). The work was finally reviewed by Pierre
  Chambart and merged in the upstream \OCaml compiler in November
  2021.

  Clément Allain started working on a mechanized soundness proof for
  the TMC transformation in Iris in Summer 2022, as a master
  internship supervised by François Pottier. They discovered that the
  question of defining simulations in Iris is surprisingly
  interesting, and that correctness proofs of transformations of
  general recursive functions require coinductive reasoning. Clément
  Allain wrote the bulk of the correctness proof at this point, and
  finished the mechanization work over 2023.

  Finally, the present research article itself was written by Clément
  Allain and Gabriel Scherer in Summer 2023 and Spring 2024, with
  excellent review comments from François Pottier, the POPL'25
  anonymous reviewers, and Anton Lorenzen.
\end{acks}

%%% Local Variables:
%%% mode: latex
%%% TeX-master: "main"
%%% End:


%% Bibliography/References

\clearpage
\bibliography{tmc}

%% Appendices

\begin{version}{\Appendices}
\clearpage
\appendix

\begin{version}{\FALSE}
\section{Extended republication}
\label{app:extended-republication}

A preliminary, work-in-progress version of this work was published at
a national conference -- all talks given in a non-English
language. This makes the present submission to a PACMPL-published
conference a partial republication.

This is unusual, so we decided to write this appendix section to
explain this choice. Our explanation is carefully worded to maintain
anonymity. (Note: there is overlap between the authors of both
presentations, but they are not equal sets, so guessing which previous
version we are talking about would be only partially de-anonimizing.)

\subsection{The letter of the law.}

We believe that partial republication fully satisfies the
\href{http://www.sigplan.org/Resources/Policies/Republication/}{SIGPLAN
  Republication Policy}, which we quote below in full
(with our emphasis):

\begin{quote}
  \textbf{Prior Publication of Closely Related Work.}  If a closely
  related paper was previously accepted at a conference or a workshop
  with proceedings of any kind, the original paper must be cited, its
  relationship to the current paper explained, and the program chair
  must be explicitly informed. The second paper will be judged on the
  additional value of its publication in the new venue. Such value
  might come from the \textbf{wider audience for the new venue} or
  from subjecting the work to a \textbf{higher standard of
    review}. Program committees should consider factors such as the
  following in deciding whether to publish the second paper:
    \begin{itemize}
    \item The call for papers for the first venue clearly states that publication in the venue is not intended to preclude later publication.
    \item The original proceedings are not easily accessible to the SIGPLAN community.
    \item The first venue \textbf{targeted a geographically limited audience}.
    \end{itemize}
\end{quote}

The present submission will be subject to higher standards
(the acceptance rate of the national venue is in the
80-100\% range). If accepted, it would reach a wider audience. The
past submission targeted a geographically limited audience. (If we
remember correctly, everyone attending the conference that year was
working in the same country it was located in.)

\subsection{The spirit of the law.}

We believe that a partial republication mirrors the established
practice of publishing a first version of a paper in conference
proceedings, and then an extended version in a journal. Here the
situation is different but similar, we published a work-in-progress
presentation of our work in a minor/national conference, and we are
submitting a full presentation of the same work
(and additional contributions) to a major/international
conference-journal hybrid.

Common/accepted conference-to-journal republication typically expect
``enhanced versions'' with, according to the Journal of Functional
Programming guidelines for example, ``should contain additional
discussion, examples, or proofs''. We believe that our new
presentation of our work is well above this bar, as it includes major
new scientific contributions.

\subsection{Comparing the two versions.}

The work-in-progress paper was written in October 2020, after we
implemented TMC for OCaml, but before it was integrated into the OCaml
compiler. It focuses on the implementation in OCaml, and does not
contain any correctness argument for the program transformation --
a major missing piece for a high-quality publication in an
international conference.

The majority of the presentation in the present paper is new. The
following parts are reused from the previous paper --
integrated by the same author who originally wrote them:
\begin{itemize}
\item Two thirds of the introduction (the first two pages).
\item \cref{subsec:ocaml-examples} on examples of TRMC functions in
  OCaml (1.5 pages).
\end{itemize}
We thus estimate the reused content at about 3.5 pages, the rest of
the content was written from scratch for the present submission.

The major scientific difference between the two papers is the
correctness proof in Iris, which is a significant contribution of its
own. To our knowledge, it is the first mechanized proof of correctness
for the tail-modulo-constructor program transformation, and the first
verification of a program transformation (for all inputs) on top of
\Simuliris. This constitutes the majority of the content of the
current presentation.

More minor differences include an analysis of the adoption of TMC in
OCaml libraries post-upstreaming, and a comparison with the Koka
work~\citep*{tmc-koka-2023} which was developed simultaneously with
ours.

\subsection{Motivating our choice.}

An alternative that we of course considered would be to present only
the correctness proof for TMC in the present paper, without reusing
any contributions and presentation from the work-in-progress,
implementation-focused paper.

This approach would however result in a ``small'' publication about
the (non-trivial) implementation work in the OCaml compiler, and
a more prestigious international submission for the correctness proof
alone. It would deprive from symbolic/academic reward the contributors
who lead the implementation and upstreaming work, without which none
of the present work would exist in the first place. (This approach
would also weaken national conferences, by making researchers think
twice about submitting implementation-focused work in progress.)

This is in fact a common dynamic in programming-language research that
combines delicate implementation work with delicate meta-theory
work. The latter is more easily identified as challenging research,
easier to evaluate for quality and novelty, and tends to gather more
scientific credit. The implementation work also typically comes
before, and it is natural to do the verification work separately, with
a different (sometimes entirely disjoint) group of contributors. We
are trying to avoid this dynamic by grouping the two sorts of
contributions together in the same publication.

%%% Local Variables:
%%% mode: latex
%%% TeX-master: "main"
%%% End:

\end{version}

\section{More on TMC in \OCaml}
\label{app:more-ocaml}

\subsection{Alternatives}
\label{subsec:alternative-impls}

We mention other approaches than TMC to solve the problem of overflowing the call stack, and explain why they were less suitable for \OCaml. We also discuss a significant implementation change between \OCaml 4 and \OCaml 5.

\subsubsection{Doing a CPS transformation}

Instead of a program transformation in \emph{destination-passing} style, we could perform a more general program transformation that can make more functions tail-recursive, for example a generic \emph{continuation-passing} style (CPS) transformation.

We have three arguments for implementing the TMC transformation:
\begin{itemize}
\item The TMC transformation generates more efficient code, using mutation instead of function calls. On the \OCaml runtime, the difference is a large constant factor.\footnote{On a toy benchmark with large-sized lists, the CPS version is 100\% slower and has 130\% more allocations than the non-tail-recursive version.}

\item The CPS transformation can be expressed at the source level, and can be made reasonably nice-looking using some monadic-binding syntactic sugar. TMC can only be done by the compiler, or using safety-breaking features.
\end{itemize}

\subsubsection{Lazy data}

Lazy (call-by-need) languages will also often avoid running into stack overflows: as soon as a lazy datastructure is returned, which is the default, functions such as \ocaml{map} will return immediately, with recursive calls frozen in a lazy thunk, waiting to be evaluated on-demand as the user traverses the result structure.
User still need to worry about tail-recursivity for their strict functions; strict functions are often preferred when writing performant code.

\subsubsection{Unlimiting the system stack}

Some operating systems can provide an unlimited system stack; such as \texttt{ulimit -s unlimited} on Linux systems -- the system stack is then resized on-demand.
Frustratingly, unlimited stacks are not available on all systems, and not the default on any system in wide use.
Convincing all users to setup their system in a non-standard way would be \emph{much} harder than performing a program transformation or accepting the CPS overhead for some programs.

\subsubsection{Using another stack}

Using the native system stack is a choice of the \OCaml~4 implementation.
Some other implementations of functional languages, such as SML/NJ, use a different stack (the \OCaml bytecode interpreter does this), or directly allocate stack frames on their GC-managed heap.
This approach can make ``stack overflow'' go away completely, and it also makes it very simple to implement stack-capture control operators, such as continuations, or other stack operations such as continuation marks.

On the other hand, using the native stack brings compatibility benefits (coherent stack traces for mixed \OCaml+C programs), and seems to improve the performance of function calls (on benchmarks that are only testing function calls and return, such as Ackermann or the naive Fibonacci, \OCaml can be 4x, 5x faster than SML/NJ.)

\paragraph{\OCaml~5}

\OCaml~5.0.0 was released in December 2022, about a year after we landed our TMC implementation in the \OCaml~4 compiler.
\OCaml~5 uses a different runtime to support multicore programming, and a different calling convention to support algebraic effects.
In particular, it only uses the system stack for C calls, and a ``cactus stack'' for \OCaml calls.

\OCaml~5 manages its own stack, but the implementors still decided to have a stack limit.
It is sensibly higher by default (1Gio at the time of writing) than the system stack (8Mio), so many ``small'' can now use non-tail-recursive functions without fears of stack overflows, but overflows remain possible and likely in practice on large inputs.
The reason to keep a (user-settable) limit for the \OCaml call stack is usability in the face of buggy programs.
When programmers write recursive functions, they occasionally go through incorrect versions with a faulty base case that fail to terminate.
In this case, we want the system to fail quickly with an error, instead of remaining silent for a few minutes, crashing once all the machine memory has been consumed.

We were curious about whether users would still see a need for TMC in \OCaml~5, or whether they would stop using the feature in practice.
The feedback we got from expert users is that stack overflows is still an issue they worry about.
We observe that TMC adoption in \OCaml codebases keep growing after the transition to \OCaml~5.

\subsection{Implementation history}
\label{subsec:PR-history}

We first proposed adding TMC as an optional program transformation to
the \OCaml compiler in
May 2015.\footnote{\nonanon{URL}{\url{https://github.com/ocaml/ocaml/pull/181}}}
The proposal was received favorably, but it never received an in-depth
review and a detailed performance evaluation and remained unmerged for
years.

We restarted the implementation in 2020, resulting in a new pull July 2020 request with a modified implementation, and a careful performance
evaluation\footnote{\nonanon{URL}{\url{https://github.com/ocaml/ocaml/pull/9760}}}. The
new implementation put a larger focus on producing readable code,
giving more control to the user through annotations. It also removed
an optimization of the previous implementation that would specialize
the DPS version, generating a distinct definition for each block
offset. The new design was carefully reviewed by Basile Clément and
Pierre Chambart, and was finally merged in November 2021, available to
users with \OCaml~4.14, released in March 2022.

The TMC transformation is that it adds extra parameters to functions
(two parameters, the block and the offset). At the time the \OCaml
compiler had a limitation on some supported architectures, where it
would not optimze tail calls above a certain number of arguments
(enough that they cannot be all passed by registers), breaking the
tail-call promises of TMC on those systems. Xavier Leroy implemented
a change to the \OCaml calling convention for those architectures in
May 2021,\footnote{\url{https://github.com/ocaml/ocaml/pull/10595}},
which was motivated by the TMC work.

\subsection{Implementation: an applicative functor}
\label{subsec:applicative-impl}

The computation of choices described in \cref{subsec:implementation} can be expressed elegantly. Instead of a monomorphic type \ocaml|Choice.t| that represents a choice of how to transform a term, we use a polymorphic type \ocaml|'a Choice.t| that represents how to transform zero, one or several subterms that occur in the source; for example transforming two subterms in the same context relies on a choice of type \ocaml|(lambda * lambda) Choice.t|, where \ocaml|lambda| is the type of terms in the intermediate representation we are working on. This \ocaml|'a Choice.t| type can be equipped with an applicative functor interface:

\begin{minipage}{0.4\linewidth}
\begin{Ocaml}
module Choice : sig
  type 'a t = {
    dps : 'a Dps.t;
    direct : unit -> 'a;
    tmc_calls : tmc_call_info list;
    benefits_from_dps : bool;
    explicit_tailcall_request : bool;
  }





end
\end{Ocaml}
\end{minipage}
\hfill
\begin{minipage}{0.5\linewidth}
\begin{Ocaml}
  (* construct a choice from an arbitrary term *)
  val lambda : lambda -> lambda t

  (* applicative functor interface *)
  val unit : unit t
  val map : ('a -> 'b) -> 'a t -> 'b t
  val pair : 'a t * 'b t -> ('a * 'b) t

  (* extract the direct-style and DPS transforms *)
  val direct : lambda t -> lambda
  val dps :
    lambda t -> tail:bool -> dst:offset dst ->
    lambda
\end{Ocaml}
\end{minipage}

With this interface, the easy cases of the transformation can be expressed compactly:
\begin{Ocaml}
  let rec choice ctx ~tail t =
    match t with
    [...]
    | Lifthenelse (l1, l2, l3) ->
        let l1 = traverse ctx l1 in
        let+ l2 = choice ctx ~tail l2
        and+ l3 = choice ctx ~tail l3
        in Lifthenelse (l1, l2, l3)
    [...]
\end{Ocaml}
The binding operators \ocaml|let+|, \ocaml|and+| are standard syntax for applicative-like structures in \OCaml: \ocaml|let+| desugars to \ocaml|Choice.map| and \ocaml|and+| desugars to \ocaml|Choice.pair|. \ocaml|traverse| performs a direct-style translation, and \ocaml|traverse_letrec| may also transform \ocaml|let|-bound functions marked with \ocaml|[@tail_mod_constr]| and add them to the local transformation environment.

\subsection{Evaluation: adoption}
\label{subsec:adoption}

% [@tail_mod_cons] usage on Github

% Search URL:
%   https://github.com/search?q=tail_mod_cons+lang%3Aocaml&type=code

% Results (November 13th 2023):
%   - in the OCaml stdlib
%     + Melange
%       (reused from stdlib)
%     + janestreet/base
%       (with manual unfolding)
%   - in other general utility modules
%     https://github.com/RedPRL/asai/blob/ac523674579772e5929c8d912d407b8b7db89c74/src/Utils.ml#L33
%     https://github.com/jonathan-laurent/KaTie/blob/2db0d2cf223fae003b26bdf4c82a9788b5804bc7/src/utils.ml#L95
%     https://github.com/jamsidedown/ocaml-cll/blob/394effa65b5d3f883e2de8322ed60fb7aa495bec/examples/crab_cups.ml#L61
%   - in other user code
%     at list type:
%       https://github.com/brendanzab/language-garden/blob/9cc21e2f57228556cf0b867ca2874ceb4a4250ec/compile-arith/lib/StackLang.ml#L63
%       https://github.com/abella-prover/abella/blob/ceee13822001137898ca5de9c76a315da3d1f0e3/src/extensions.ml#L421
%       https://github.com/bikallem/spring/blob/6d10fef0b21caea3f975ff13132f5994e7c37db5/lib_spring/response.ml#L99
%       https://github.com/bikallem/spring/blob/6d10fef0b21caea3f975ff13132f5994e7c37db5/lib_spring/headers.ml#L150
%       https://github.com/polytypic/io/blob/dbe4d37a98ef958178fe18bb4dae396a5537392e/src/io/io.ml#L8
%       https://github.com/avsm/eeww/blob/5b6c0617306f4c9d8b44bf07ef4d45ec9668dd0c/lib/cohttp/cohttp-eio/src/rwer.ml#L43
%       https://github.com/just-max/less-power/blob/96f8e88f72275246b6bc175e44c732aa6e08ce7f/src/common/path_util.ml#L18
%       https://github.com/dalps/ocaml-challenge/blob/8fe115023457b6b4359ef736c78ba980cd871717/3/extract/solve.ml#L4
%       https://github.com/Octachron/ocaml-changelog-analyzer/blob/d6757b9576b5070ea3cfe6d4f4f79aaa3cc6e5d4/parse/release.ml#L4
%       https://github.com/Skyb0rg007/PL-Reading-Group/blob/7002ae35ba69fb9010e5049eee88ab475bc79ec3/list-optimizations/lib/adt.ml#L44
%       https://github.com/jfeser/symetric/blob/f49a57afc90a2c1e38f1097f98bd74725c074b9b/lib/tower.ml#L75
%       https://github.com/verse-lab/sisyphus/blob/a08addae06bbccb1e6ea68bac3d7f9eeea4197be/lib/dynamic/utils.ml#L65
%       https://github.com/jaymody/ocaml-tokenizers/blob/06cbce75d2865690cb39017222efff5c284bc721/lib/utils.ml#L18
%       https://github.com/Bannerets/ocaml-kdl/blob/687c404ebeceef7fd905935c23665294133f99e6/src/lens.ml#L83
%       https://github.com/OCADml/OCADml/blob/f5dfbf55f9c19ad808daa0df4d76a55e58756e5f/lib/poly3.ml#L33
%     other type
%       https://github.com/johnridesabike/acutis/blob/4113fb516d9c5dd6f2dbfd9657f52c5ae7a5dcae/lib/matching.ml#L161
%         (on-demand stream as a record of functions)
%       https://github.com/RedPRL/ocaml-bwd/blob/fbf496b29532085b38073eaa62ba3d22ac619d5d/src/BwdNoLabels.ml#L60
%         (snoc list)
%       https://github.com/Skyb0rg007/PL-Reading-Group/blob/7002ae35ba69fb9010e5049eee88ab475bc79ec3/effects/lib/free/tseq.ml#L44
%         (difference list gadt)

%   - Misuse
%     https://github.com/chengsun/simd-sexp/commit/e714a8205c6df512d920a31a4dd2448975703c55
%       (already tail-recursive)
%     https://github.com/gridbugs/llama/blob/96d1c96c31ff136f465b5f37c981fff591bac6fd/src/midi/byte_array_parser.ml#L50
%       (needless complexity)
%     https://github.com/LimitEpsilon/lambda-interpreter/blob/ab8e81de7f896aa64eba14e23d964b9705e94161/lib/evaluate.ml#L46
%       (already tail-recursive)

The TMC transformation was first made available to users in \OCaml~4.14.0, released in March 2022.
At the time there were no users of the feature.
Notably, the \OCaml standard library had intentionally not been modified to start using it: before the release, we did not want the stdlib codebase to depend on a not-available-yet feature, and after the release we decided not to push for adoption (we may be biased about the benefits/importance of TMC), and let other contributors propose its use, after doing their own evaluation of performance impact and code-clarity benefits.

Over time, \ocaml|[@tail_mod_cons]| was adopted in a few places in the \OCaml standard library, either to make some functions tail-recursive that previously were not, or simplify some complex tail-recursive implementations. As of spring 2024, the feature is now used in \ocaml|Hashtbl.find_all| and in \ocaml|List| module (\ocaml|init, map, mapi, map2, find_all, filteri, filter_map, take, take_while, of_seq|). \ocaml|init| and the \ocaml|map*| functions, which are the most heavily used, were unrolled once -- the minimal amount observed to preserve the non-tailrec performance for small lists. Other functions are written in the simplest possible way. (All \texttt{stdlib} uses so far build lists rather than another datatype.)

We performed a systematic search for \ocaml|[@tail_mod_cons]| usage on November 2023, and we found that it was also used
\begin{itemize}
\item in a few utility modules (general-purpose functions designed to extend or replace the standard library), notably in Jane Street's \texttt{base} library.
\item in the middle of domain-specific user code, to build lists, in 14 different projects
\item in the middle of user code, to build other types than lists, in three projects:
  \begin{itemize}
  \item in \texttt{RedPRL}\footnote{\url{https://github.com/RedPRL/ocaml-bwd/blob/fbf496b29532085b38073eaa62ba3d22ac619d5d/src/BwdNoLabels.ml\#L60}},
    it is used to build snoc lists
  \item in a project named \texttt{PL-reading-group}\footnote{\url{https://github.com/Skyb0rg007/PL-Reading-Group/blob/7002ae35ba69fb9010e5049eee88ab475bc79ec3/effects/lib/free/tseq.ml\#L44}},
    it is used to build a GADT of difference lists
  \item in \texttt{acutis}\footnote{\url{https://github.com/johnridesabike/acutis/blob/4113fb516d9c5dd6f2dbfd9657f52c5ae7a5dcae/lib/matching.ml\#L161}},
    in a group of mutually-recursive functions that builds a complex AST structure used options and nested records
  \end{itemize}
\end{itemize}

We also found three cases where the annotation is (in our opinion) misused: in two cases, the function is already tail-recursive, so there is no need for the annotation, and in a third case\footnote{\url{https://github.com/gridbugs/llama/blob/96d1c96c31ff136f465b5f37c981fff591bac6fd/src/midi/byte_array_parser.ml\#L50}} we consider that the use is gratuitous and can be replaced by already-available standard library functions.

%%% Local Variables:
%%% mode: latex
%%% TeX-master: "main"
%%% End:


\section{Benchmark results on \OCaml~4}

The benchmark results for \OCaml~4 are given in \cref{fig:bench4}. The
results are very close to \cref{fig:bench5}, and in particular the
qualitative analysis of the results is mostly unchanged. (We used
\texttt{ulimit -s unlimited} to disable the system stack limit, to run
the non-tail-recursive version on large lists.)

\begin{figure}[tp]
\def\svgscale{0.8}
\graphicspath{{plots/}}
\input{plots/plot.4.pdf_tex}
\caption{\ocaml|List.map| benchmark on \OCaml~4.14}
\label{fig:bench4}
\end{figure}

%%% Local Variables:
%%% mode: latex
%%% TeX-master: "main"
%%% End:

\end{version}
\end{document}
