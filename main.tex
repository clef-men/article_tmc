\documentclass[acmsmall,screen,review,anonymous]{acmart}
\settopmatter{printfolios=true,printccs=false,printacmref=false}
\renewcommand\footnotetextcopyrightpermission[1]{} % removes footnote with conference information in first column

% ------------------------------------------------------------------------------

\usepackage[T1]{fontenc}
\usepackage[utf8]{inputenc}
\usepackage[scaled=0.8]{beramono}

% ------------------------------------------------------------------------------

\usepackage{mathtools}
\usepackage{amsthm}

\usepackage{mathpartir}
\let\TirName\textsc
\renewcommand{\DefTirName}[1]{\hypertarget{#1}{\TirName {#1}}}
\newcommand{\RefTirName}[1]{\hyperlink{#1}{\TirName {#1}}}

\let\oldexists\exists
\let\exists\relax\DeclareMathOperator{\exists}{\oldexists}
\let\oldforall\forall
\let\forall\relax\DeclareMathOperator{\forall}{\oldforall}

\DeclareMathOperator{\lambdaAbs}{\lambda}
\DeclareMathOperator{\muAbs}{\mu}
\DeclareMathOperator{\nuAbs}{\nu}

% ------------------------------------------------------------------------------

\usepackage{macros}

\usepackage{listings}
\usepackage{listings/ocaml}
\usepackage{listings/datalang}

% ------------------------------------------------------------------------------

\usepackage[nameinlink,noabbrev]{cleveref}
\crefname{section}{\S\!}{\S}
\Crefname{section}{Sec.}{Sec.}
\crefname{figure}{Fig.}{Fig.}

% ------------------------------------------------------------------------------

%% Journal information
%% Supplied to authors by publisher for camera-ready submission;
%% use defaults for review submission.
\acmJournal{PACMPL}
\acmVolume{1}
\acmNumber{PLDI}
\acmArticle{1}
\acmYear{2023}
\acmMonth{11}
\acmDOI{} % \acmDOI{10.1145/nnnnnnn.nnnnnnn}
\startPage{1}

%% Copyright information
%% Supplied to authors (based on authors' rights management selection;
%% see authors.acm.org) by publisher for camera-ready submission;
%% use 'none' for review submission.
\setcopyright{none}
%\setcopyright{acmcopyright}
%\setcopyright{acmlicensed}
%\setcopyright{rightsretained}
%\copyrightyear{2018}           %% If different from \acmYear

%% Bibliography style
\bibliographystyle{ACM-Reference-Format}
%% Citation style
%% Note: author/year citations are required for papers published as an
%% issue of PACMPL.
\citestyle{acmnumeric}   %% For author/year citations

% ------------------------------------------------------------------------------
% ------------------------------------------------------------------------------

\begin{document}

\title{Verifying Tail Modulo Cons using Relational Separation Logic}

\author{Clément Allain}
\affiliation{
  \institution{Inria}
  \city{Paris}
  \country{France}
}
\email{clement.allain@inria.fr}

\author{François Pottier}
\affiliation{
  \institution{Inria}
  \city{Paris}
  \country{France}
}
\email{francois.pottier@inria.fr}

\author{Gabriel Scherer}
\affiliation{
  \institution{Inria}
  \city{Saclay}
  \country{France}
}
\email{gabriel.scherer@inria.fr}

\begin{abstract}
  Common functional languages incentivize tail-recursive functions, as opposed to general recursive functions that consume stack space and may not scale to large inputs.
%
This distinction occasionally requires writing functions in a tail-recursive style that may be more complex and slower than the natural, non-tail-recursive definition.

This work describes our implementation of the \emph{tail modulo constructor} (TMC) transformation in the \OCamlLang compiler, an optimization that provides stack-efficiency for a larger class of functions --- tail-recursive \emph{modulo constructors} --- which includes in particular the natural definition of \ocaml{List.map} and many similar recursive data-constructing functions.

We prove the correctness of this program transformation in a simplified setting --- a small untyped calculus --- that captures the salient aspects of the \OCamlLang implementation. Our proof is mechanized in the \Coq proof assistant, using the \Iris base logic.
%
An independent contribution of our work is an extension of the \Simuliris approach to define simulation relations that support different calling conventions. To our knowledge, this is the first use of \Simuliris to prove the correctness of a compiler transformation.

\end{abstract}

\maketitle

\keywords{compilation, separation logic, program verification}

\begin{CCSXML}
<ccs2012>
<concept>
<concept_id>10003752.10003790.10011742</concept_id>
<concept_desc>Theory of computation~Separation logic</concept_desc>
<concept_significance>500</concept_significance>
</concept>
<concept>
<concept_id>10003752.10010124.10010138.10010142</concept_id>
<concept_desc>Theory of computation~Program verification</concept_desc>
<concept_significance>500</concept_significance>
</concept>
</ccs2012>
\end{CCSXML}

\ccsdesc[500]{Theory of computation~Separation logic}
\ccsdesc[500]{Theory of computation~Program verification}

% ------------------------------------------------------------------------------

\section{Introduction}

\subsection{Prologue}

``OCaml'', we teach our students, ``is a functional programming language. We can write the beautiful function \ocaml{List.map} as follows:''
\begin{Ocaml}
let rec map f = function
| [] -> []
| x :: xs -> f x :: map f xs
\end{Ocaml}

``Well, actually'', we continue, ``OCaml is an effectful language, so we need to be careful about the evaluation order. We want \ocaml{map} to process elements from the beginning to the end of the input list, and the evaluation order of \ocaml{f x :: map f xs} is unspecified. So we write:''
\begin{Ocaml}
let rec map f = function
| [] -> []
| x :: xs ->
  let y = f x in
  y :: map f xs
\end{Ocaml}

``Well, actually, this version fails with a \ocaml{Stack_overflow}
exception on large input lists. If you want your \ocaml{map} to behave
correctly on all inputs, you should write a \emph{tail-recursive}
version. For this you can use the accumulator-passing style:''
\begin{Ocaml}
let map f li =
  let rec map_ acc = function
  | [] -> List.rev acc
  | x :: xs -> map_ (f x :: acc) xs
  in map_ [] f li
\end{Ocaml}

``Well, actually, this version works fine on large lists, but it is
less efficient than the original version. It is noticeably slower on
small lists, which are the most common inputs for most programs. We
measured it 35\% slower on lists of size 10. If you want to write
a robust function for a standard library, you may want to support both
use-cases as well as possible. One approach is to start with
a non-tail-recursive version, and switch to a tail-recursive version
for large inputs; even there you can use some manual optimizations to
reduce the overhead of the accumulator. For example, the nice
\href{https://github.com/c-cube/ocaml-containers}{Containers} library
does it as follows:''.

%\hspace{-3em}
\begin{minipage}{0.6\linewidth}
\begin{Ocaml}[basicstyle=\ttfamily\tiny]
let tail_map f l =
  (* Unwind the list of tuples, reconstructing the full list front-to-back.
     @param tail_acc a suffix of the final list; we append tuples' content
     at the front of it *)
  let rec rebuild tail_acc = function
    | [] -> tail_acc
    | (y0, y1, y2, y3, y4, y5, y6, y7, y8) :: bs ->
      rebuild (y0 :: y1 :: y2 :: y3 :: y4 :: y5 :: y6 :: y7 :: y8 :: tail_acc) bs
  in
  (* Create a compressed reverse-list representation using tuples
     @param tuple_acc a reverse list of chunks mapped with [f] *)
  let rec dive tuple_acc = function
    | x0 :: x1 :: x2 :: x3 :: x4 :: x5 :: x6 :: x7 :: x8 :: xs ->
      let y0 = f x0 in let y1 = f x1 in let y2 = f x2 in
      let y3 = f x3 in let y4 = f x4 in let y5 = f x5 in
      let y6 = f x6 in let y7 = f x7 in let y8 = f x8 in
      dive ((y0, y1, y2, y3, y4, y5, y6, y7, y8) :: tuple_acc) xs
    | xs ->
      (* Reverse direction, finishing off with a direct map *)
      let tail = List.map f xs in
      rebuild tail tuple_acc
  in
  dive [] l
\end{Ocaml}
\end{minipage}
\hfill
\begin{minipage}{0.4\linewidth}
\begin{Ocaml}[basicstyle=\ttfamily\tiny]
let direct_depth_default_ = 1000

let map f l =
  let rec direct f i l = match l with
    | [] -> []
    | [x] -> [f x]
    | [x1;x2] -> let y1 = f x1 in [y1; f x2]
    | [x1;x2;x3] ->
      let y1 = f x1 in let y2 = f x2 in [y1; y2; f x3]
    | _ when i=0 -> tail_map f l
    | x1::x2::x3::x4::l' ->
      let y1 = f x1 in
      let y2 = f x2 in
      let y3 = f x3 in
      let y4 = f x4 in
      y1 :: y2 :: y3 :: y4 :: direct f (i-1) l'
  in
  direct f direct_depth_default_ l
\end{Ocaml}
\end{minipage}
\lstset{basicstyle=\small\ttfamily}

At this point, unfortunately, some students leave the class and never
come back.

We propose a new feature for the OCaml compiler, an explicit, opt-in
``Tail Modulo Cons'' transformation, to retain our students. After the
first version (or maybe, if we are teaching an advanced class, after
the second version), we could show them the following version:
\begin{Ocaml}
let[@tail_mod_cons] rec map f = function
| [] -> []
| x :: xs -> f x :: map f xs
\end{Ocaml}

This version is as fast as the simple implementation, tail-recursive,
and easy to write.

The catch, of course, is to teach when this \ocaml{[@tail_mod_cons]}
annotation can be used. Maybe we would not show it at all, and pretend
that the direct \ocaml{map} version with \ocaml{let y} is fine. This
would be a much smaller lie than it currently is,
a \ocaml{[@tail_mod_cons]}-sized lie.

Finally, experts should be very happy. They know about all these
versions, but they do not have to write them by hand anymore. Have
a program perform (some of) the program transformations that they are
currently doing manually.

\subsection{TMC transformation example}

A function call is in \emph{tail position} within a function
definition if the definition has ``nothing to do'' after evaluating
the function call -- the result of the call is the result of the whole
function at this point of the program. (A precise definition will be
given in Section~\ref{sec:tcmc}.) A function is \emph{tail recursive}
if all its recursive calls are tail calls.

In the definition of \ocaml{map}, the recursive call is not in tail
position: after computing the result of \ocaml{map f xs} we still have
to compute the final list cell, \ocaml{y :: ?}. We say that a call is
\emph{tail modulo cons} when the work remaining is formed of data
\emph{constructors} only, such as \ocaml{(::)} here.

\begin{Ocaml}
let[@tail_mod_cons] rec map f = function
| [] -> []
| x :: xs ->
  let y = f x in
  y :: map f xs
\end{Ocaml}

Other datatype constructors may also be used; the following example is
also tail-recursive \emph{modulo cons}:

\begin{Ocaml}
let[@tail_mod_cons] rec tree_of_list = function
| [] -> Empty
| x :: xs -> Node(Empty, x, tree_of_list xs)
\end{Ocaml}

The TMC transformation returns an equivalent function in
\emph{destination-passing} style where the calls in \emph{tail modulo
  cons} position have been turned into \emph{tail} calls. In
particular, for \ocaml{map} it gives a tail-recursive function, which
runs in constant stack space; many other list functions also become
tail-recursive. The transformed code of \ocaml{map} can be described as
follows:

\hspace{-1.6em}
\begin{minipage}{0.5\linewidth}
\begin{Ocaml}
let rec map f = function
| [] -> []
| x::xs ->
  let y = f x in
  let dst = y :: Hole in
  map_dps dst 1 f xs;
  dst
\end{Ocaml}
\end{minipage}
\hfill
\begin{minipage}{0.5\linewidth}
\begin{Ocaml}
and map_dps dst i f = function
| [] ->
  dst.i <- []
| x::xs ->
  let y = f x in
  let dst' = y :: Hole in
  dst.i <- dst';
  map_dps dst' 1 f xs
\end{Ocaml}
\end{minipage}

The transformed code has two variants of the \ocaml{map} function. The
\ocaml{map_dps} variant is in \emph{destination-passing style}, it
expects additional parameters that specify a memory location,
a \emph{destination}, and will write its result to this
\emph{destination} instead of returning it. It is tail-recursive. The
\ocaml{map} variant provides the same interface as the non-transformed
function, and internally calls \ocaml{map_dps} on non-empty lists. It
is not tail-recursive, but it does not call itself recursively, it
jumps to the tail-recursive \ocaml{map_dps} after one call.

The key idea of the transformation is that the expression \ocaml{y :: map f xs}, which contained a non-tail-recursive call, is transformed into first the computation of a \emph{partial} list cell, written \ocaml{y :: ?}, followed by a call to \ocaml{map_dps} that is asked
to write its result in the position of the \ocaml{?}. The recursive
call thus happens after the cell creation (instead of before), in
tail-recursive position in the \ocaml{map_dps} variant. In the direct
variant, the value of the destination \ocaml{dst} has to be returned
after the call.

The transformed code is in a pseudo-OCaml, it is not a valid OCaml
program: we use a magical \ocaml{?} constant, and our notation
\ocaml{dst.i <- ...} to update constructor parameters in-place is also
invalid in source programs. The transformation is implemented on
a lower-level, untyped intermediate representation of the OCaml
compiler (Lambda), where those operations do exist. The OCaml type
system is not expressive enough to type-check the transformed program:
the list cell is only partially-initialized at first, each partial
cell is mutated exactly once, and in the end the whole result is
returned as an \emph{immutable} list. Some type system are expressive
enough to represent this transformed code, notably Mezzo
\citep*{mezzo}.


TODO: what we have above is a re-import of the JFLA introduction.
What is below is a copy of the previous introduction-in-progress.
What should reread it, and more importantly include a new subsection on the proof,
which can be mostly reused from what is below.

\section{Introduction}

In \OCaml, the natural implementation of the \ocaml|map| function on polymorphic lists is the following:

\begin{Ocaml}
let rec map f xs =
  match xs with
  | [] ->
      []
  | x :: xs ->
      let y = f x in
      y :: map f xs
\end{Ocaml}

Yet, calling \ocaml|map| on a large input list may crash with the dreaded \ocaml|Stack_overflow| exception.
%
Indeed, as functions calls consume stack space, the stack usage of this implementation is linear in the size of the input list.
%
This is the reason why \OCaml programmers have to be careful, when they write recursive functions, to remain in the so-called \emph{tail-recursive} fragment: every recursive call must be in tail position, that is the last thing the function does before returning.
%
As there is nothing to do after a tail call, the stack frame can be directly reused.
%
Therefore, tail calls do not consume stack space.
%
Knowing this, we can provide another tail-recursive implementation of \ocaml|map|:

\begin{minipage}{0.5\linewidth}
\begin{Ocaml}
let rec rev_map f ys xs =
  match xs with
  | [] ->
      ys
  | x :: xs ->
      let y = f x in
      rev_map f (y :: ys) xs
\end{Ocaml}
\end{minipage}
\hfill
\begin{minipage}{0.5\linewidth}
\begin{Ocaml}
let map f xs =
  let ys = rev_map f [] xs in
  List.rev ys
\end{Ocaml}
\end{minipage}

This second version computes the reverse result list with \ocaml|rev_map| in the accumulator-passing style and applies \ocaml|List.rev|.
%
As both \ocaml|rev_map| and \ocaml|rev| are tail-recursive, it is stack-safe.
%
However, it is not as efficient on small lists since it does two traversals instead of one.
%
There is yet another implementation:

\begin{minipage}{0.65\linewidth}
\begin{Ocaml}
/* [dst] is a partially initialized list cell. */
let rec map_dps dst f xs =
  match xs with
  | [] ->
      /* Write result empty list to [dst]. */
      set_field dst 1 []
  | x :: xs ->
      let y = f x in
      /* New partially initialized list cell. */
      let dst' = y :: [] in
      /* Write result [dst'] to [dst]. */
      set_field dst 1 dst' ;
      /* Continue with [dst'] as new destination. */
      map_dps dst' f xs
\end{Ocaml}
\end{minipage}
\hfill
\begin{minipage}{0.45\linewidth}
\begin{Ocaml}
let map f xs =
  match xs with
  | [] ->
      []
  | x :: xs ->
      let y = f x in
      let dst = y :: [] in
      map_dps dst f xs ;
      dst
\end{Ocaml}
\end{minipage}

This third implementation is much less natural.
%
It takes advantage of the memory representation of lists in \OCaml: some constant integer for the empty list and a heap-allocated cell containing one tag followed by two fields for the list constructor.
%
The \ocaml|set_field| function allows us to modify these fields.
%
It can be implemented using the unsafe \ocaml|Obj| module.

The core of this implementation is the tail-recursive \ocaml|map_dps| auxiliary function.
%
It is the \emph{destination-passing style} version of \ocaml|map|: it writes its result to the tail field of destination \ocaml|dst| instead of returning it.
%
On the empty list, it just writes the empty list.
%
On non-empty lists, it allocates a new partially initialized list cell \ocaml|dst'|, writes \ocaml|dst'| to \ocaml|dst| and calls itself recursively with \ocaml|dst|' as its new destination.
%
\ocaml|map| calls \ocaml|map_dps| on non-empty lists.

What happens is \ocaml|map_dps| constructs the final transformed list by modifying the last cell of the current partial transformed list.
%
Not only is it stack-safe but it also does only one traversal, as opposed the previous stack-safe implementation.
%
Unfortunately, it cannot be implemented using only safe \OCaml features, as list cells are normally immutable.
%
In fact, most strongly typed languages reject it.
%
A notable exception is \Mezzo \cite{DBLP:journals/toplas/BalabonskiPP16}, which is based on a permission system inspired by \emph{separation logic}~\cite{DBLP:journals/cacm/OHearn19}.

We can do better.
%
If we cannot write it, the compiler may be able to do so.
%
In fact, the manual transformation of the natural implementation of \ocaml|map| we have performed is an instance of a more general transformation known as \emph{tail call optimization modulo constructor} (TMC)~\cite{risch-73,friedman-wise-75}.
%
TMC has been implemented in Lisp~[TODO], Prolog~[TODO] and Koka~\cite{DBLP:journals/pacmpl/LeijenL23}.
%
The first main contribution of this work is an implementation of TMC in the \OCaml compiler as an optional optimization.
%
Various functions in the standard library and third-party codebases have been rewritten to use it.
%
For instance, the efficient stack-safe version of \ocaml|map| is obtained just by annotating the natural definition with the attribute \ocaml|[@tail_mod_cons]|:

\begin{Ocaml}
let[@tail_mod_cons] rec map f xs =
  match xs with
  | [] -> []
  | x :: xs -> f x :: map f xs
\end{Ocaml}

The second main contribution of this work is a mechanized proof of correctness for the core of this transformation on a small untyped calculus.
%
We established a \emph{behavioral refinement} between the source program and the transformed program: any behavior of the transformed program, be it converging, diverging or stuck, is a behavior of the source program.
%
Our proof relies on \Simuliris~\cite{DBLP:journals/pacmpl/GaherSSJDKKD22}, a separation logic framework built on top of the \Iris base logic.
%
We found that their proof technique, a simulation relation, is not expressive enough to reason about program transformations that introduce new function calling conventions.
%
We had to generalize the \Simuliris handling of function calls by parameterizing the simulation relation over an abstract protocol inspired by the work of \citeauthor{DBLP:journals/pacmpl/VilhenaP21}~\cite{DBLP:journals/pacmpl/VilhenaP21}.
%
This contribution of our work is independent from the TMC transformation and our small calculus.
%
It could be reused in future work on verifying compiler transformations.

Interestingly, the core of the soundness proof is the specification of the two variants of each TMC-transformed function --- direct style and destination-passing style.
%
It conveys the essence of destination-passing style: computing the same thing and writing it to an owned destination.
%
For instance, delaying the definition of the simulation relation $\iSimv{\Phi}{e_s}{e_t}$ and the similarity relation $\datalangVal_s \iSimilar \datalangVal_t$, the specification of the variants of \ocaml|map| is:
%
\begin{align*}
        \iSimvHoare{
            \iTrue
        }{
            \iSimilar
        }{
            & \datalangCall{\datalangFnptr{\ocamlText{map}}}{\datalangVal_s}
        }{
            \datalangCall{\datalangFnptr{\ocamlText{map}}}{\datalangVal_t}
        }
    \\
        \iSimvHoare{
            (\datalangLoc + \datalangIdx) \iPointsto \datalangHole
        }{
            \datalangVal_s, \datalangUnit \ldotp
            \exists \datalangVal_t \ldotp
            (\datalangLoc + \datalangIdx) \iPointsto \datalangVal_t \iSep
            \datalangVal_s \iSimilar \datalangVal_t
        }{
            & \datalangCall{\datalangFnptr{\ocamlText{map}}}{\datalangVal_s}
        }{
            \datalangCall{\datalangFnptr{\ocamlText{map_dps}}}{\datalangPair{\datalangPair{\datalangLoc}{\datalangIdx}}{\datalangVal_t}}
        }
\end{align*}

To sum up, our main contributions are:
\begin{enumerate}
    \item an implementation of the TMC transformation in the \OCaml compiler;
    \item a mechanized proof of soundness for a simplified transformation on a small calculus;
    \item a generalization of \Simuliris handling of function calls to reason about different calling conventions.
\end{enumerate}

%%% Local Variables:
%%% mode: latex
%%% TeX-master: "main"
%%% End:

\clearpage

\section{\OCaml Implementation}
\label{sec:implementation}

\subsection{Implementation history}

We first proposed adding TMC as an optional program transformation to
the OCaml compiler in
May 2015.\footnote{\nonanon{URL}{\url{https://github.com/ocaml/ocaml/pull/181}}}
The proposal was received favorably, but it never received an in-depth
review and a detailed performance evaluation and remained unmerged for
years.

We re-activated the discussion in July 2020 with a modified
implementation, and a careful performance
evaluation\footnote{\nonanon{URL}{\url{https://github.com/ocaml/ocaml/pull/9760}}}. The
new implementation put a larger focus on producing readable code,
giving more control to the user through annotations. It also removed
an optimization of the previous implementation that would specialize
the DPS version, generating a distinct definition for each block
offset. The new design was carefully reviewed by Basile Clément and
Pierre Chambart, and was finally merged in November 2021, available to
users with OCaml 4.14, released in March 2022.

The TMC transformation is that it adds extra parameters to functions
(two parameters, the block and the offset). At the time the OCaml
compiler had a limitation on some supported architectures, where it
would not optimze tail calls above a certain number of arguments
(enough that they cannot be all passed by registers), breaking the
tail-call promises of TMC on those systems. Xavier Leroy implemented
a change to the OCaml calling convention for those architectures in
May 2021,\footnote{\url{https://github.com/ocaml/ocaml/pull/10595}},
which was motivated by the TMC work.

\subsection{Examples}

Many functions that consume and produce lists are
tail-recursive-modulo-cons, in the sense that all they have a TMC
decomposition where all recursive calls are in TMC position. Notable
functions include \ocaml{map}, as already discussed, but also for
example:

\begin{Ocaml}
let[@tail_mod_cons] rec filter p = function
| [] -> []
| x :: xs -> if p x then x :: filter p xs else filter p xs

let[@tail_mod_cons] rec merge cmp l1 l2 =
  match l1, l2 with
  | [], l | l, [] -> l
  | h1 :: t1, h2 :: t2 ->
      if cmp h1 h2 <= 0
      then h1 :: merge cmp t1 l2
      else h2 :: merge cmp l1 t2
\end{Ocaml}

TMC is not useful only for lists or other ``linear'' data types, with
at most one recursive occurrence of the datatype in each
constructor.

\paragraph{A non-example} Consider a \ocaml{map} function on binary
trees:
\begin{Ocaml}
let[@tail_mod_cons] rec map f = function
| Leaf v -> Leaf (f v)
| Node(t1, t2) -> Node(map f t1, (map[@tailcall]) f t2)
\end{Ocaml}
In this function, there are two recursive calls, but only one of them
can be optimized; we used the \ocaml{[@tailcall]} attribute to direct
our implementation to optimize the call to the right child, as we will
discuss later. This is a \emph{bad} example of TMC usage in most
cases, given that
\begin{itemize}
\item If the tree is arbitrary, there is no reason that it would be
  right-leaning rather than left-leaning. Making only the right-child
  calls tail-calls does not protect us from stack overflows.
\item If the tree is known to be balanced, then in practice the depth
  is probably very small in both directions, so the TMC transformation
  is not necessary to have a well-behaved function.
\end{itemize}

\paragraph{Interesting non-linear examples} There \emph{are} interesting
examples of TMC-transformation on functions operating on tree-like
data structures, when there are natural assumptions about which child
is likely to contain a deep subtree. The OCaml compiler itself
contains a number of them; consider for example the following function
from the \ocaml{Cmm} module, one of its lower-level program
representations:
\begin{Ocaml}
let[@tail_mod_cons] rec map_tail f = function
  | Clet(id, exp, body) ->
      Clet(id, exp, map_tail f body)
  | Cifthenelse(cond, ifso, ifnot) ->
      Cifthenelse(cond, map_tail f ifso, (map_tail[@tailcall]) f ifnot)
  | Csequence(e1, e2) ->
      Csequence(e1, map_tail f e2)
  | Cswitch(e, tbl, el) ->
      Cswitch(e, tbl, Array.map (map_tail f) el)
  [...]
  | Cexit _ | Cop (Craise _, _, _) as cmm ->
      cmm
  | Cconst_int _ | Cvar _ | Ctuple _ | Cop _ as c ->
      f c
\end{Ocaml}

This function is traversing the ``tail'' context of an arbitrary
program term -- a meta-example! The \ocaml{Cifthenelse} node acts as
our binary-node constructor, we do not know which side is likely to be
larger, so TMC is not so interesting. The recursive calls for
\ocaml{Cswitch} are not in TMC position. But on the other hand the
\ocaml{Clet}, \ocaml{Csequence} cases are very beneficial to have in
TMC: while they have several recursive subtrees, they are in practice
only deeply nested in the direction that is turned into a tailcall by
the transformation. The OCaml compiler does sometimes encounter
machine-generated programs with a unusually long sequence of either
constructions, and the TMC transformation may very well avoid a stack
overflow in this case.

Another example would be
\href{https://github.com/ocaml/ocaml/pull/9636}{\#9636}, a patch to
the OCaml compiler proposed in June 2020 by Mark Shinwell, to get
a partially-tail-recursive implementation of the ``Common
Subexpression Elimination'' (CSE) pass through a manual
contiation-passing-style transform. Xavier Leroy remarked that the
existing implementation in fact fits the TMC fragment. Not all
recursive calls become tail-calls (this would require a more powerful
transformation or a longer, less readable patch), but the behavior of
TMC on the unchanged code matches the tail-call-ness proposed in the
human-written patch.

\subsection{Design choices}

\subsubsection{Resolving non-determinism}

The main question faced by an implementation is how to resolve the
inherent non-determinism. We identified fairly different kinds of
non-determinism.

\paragraph{Choices with an obviously better alternative.} Some choices
have an obviously better alternative, because they allow to strictly
improve more programs. For example, some implementations of TMC limit
themselves to calls that are immediately inside a constructor: they
would optimize the recursive call to \ocaml|f| in \ocaml|Cons(x, f y)|
but not, for example, in \ocaml|Cons(x1, Cons(x2, f y))|, or in
\ocaml|Cons (x, if p then f y else z)|.

Our implementation optimizes calls to functions under an arbitrary
nesting of tail-recursive contexts and constructor contexts, which is
the most general approach allowed by our relation. This comes at the
cost of some implementation complexity, but is otherwise an easy
choice to make when deciding on a specification.

\paragraph{Choices with a subtly better alternative.} In some cases it
is possible to put a bit more work in the choice heuristics, to
generate slightly better code, in terms of performance or
readability. For example, some natural way to simplify the
implementation would result in more subterms being transformed in
destination-passing-style, only to finally use the
\RefTirName{DPSBase} rule without any DPS function call. The generated
code is slightly less pleasant to read, and in our experience it can
be noticeably slower.

Code readability is important in our context of an on-demand program
transformation used by performance experts: those experts will often
read the intermediate representations produced by the compiler to
check that their performance assumptions hold.

\paragraph{Incomparable choices: force the user to decide} The
remaining choices are between incomparable alternatives, that could
each be better than the other depending on the specific program or
programmer intent. Consider again our binary tree example,
\ocaml|Node(map f left, map f right)|: we could make either the first
or the second argument a tail-call.

Our policy in such cases is to let the user decide, by providing
control over the transformation choices using \ocaml{[@tailcall]}
program annotations, and to force the user to decide by raising an
error in case of ambiguity:
\begin{Ocaml}
  | Cifthenelse(cond, ifso, ifnot) ->
      Cifthenelse(cond, map_tail f ifso, map_tail f ifnot)
      /[^^^^^^^^^^^^^^^^^^^^^^^^^^^^^^^^^^^^^^^^^^^^^^^^^^^^]/
/[Error]/: "[@tail_mod_cons]": this constructor application may be TMC-transformed
       in several different ways. Please disambiguate by adding an explicit
       "[@tailcall]" attribute to the call that should be made tail-recursive,
       or a "[@tailcall false]" attribute on calls that should not be
       transformed.
\end{Ocaml}

This approach would be incompatible with a view of the TMC
transformation as an implicit optimization, applied whenever
possible -- in that case one would rather have the compiler make
arbitrary choices rather than fail. We rather view TMC as a tool for
expert users to better reason about program performance. It should be
predictable and flexible (let the user express
their intent). Failing on the lack of annotations is an acceptable way
to drive user interaction.

\subsubsection{First-order implementation} A notable limitation of the
OCaml implementation is that it is first-order and non-modular in
nature: only direct calls to known functions can be converted into DPS
style, and the availability of a DPS variant is not exposed through
module abstractions.

It would be possible to allow DPS calls through external modules or
higher-order function parameters, by annotating interfaces and
function arguments with a \ocaml{[@tail_mod_cons]} annotation, and
elaborating this into a pair of function, the direct-style and the DPS
version.

\subsection{Constructor compression}

\subsection{Implementation}

\subsection{Evaluation: benchmarks}

\subsection{Evaluation: adoption}

% [@tail_mod_cons] usage on Github

% Search URL:
%   https://github.com/search?q=tail_mod_cons+lang%3Aocaml&type=code

% Results (November 13th 2023):
%   - in the OCaml stdlib
%     + Melange
%       (reused from stdlib)
%     + janestreet/base
%       (with manual unfolding)
%   - in other general utility modules
%     https://github.com/RedPRL/asai/blob/ac523674579772e5929c8d912d407b8b7db89c74/src/Utils.ml#L33
%     https://github.com/jonathan-laurent/KaTie/blob/2db0d2cf223fae003b26bdf4c82a9788b5804bc7/src/utils.ml#L95
%     https://github.com/jamsidedown/ocaml-cll/blob/394effa65b5d3f883e2de8322ed60fb7aa495bec/examples/crab_cups.ml#L61
%   - in other user code
%     at list type:
%       https://github.com/brendanzab/language-garden/blob/9cc21e2f57228556cf0b867ca2874ceb4a4250ec/compile-arith/lib/StackLang.ml#L63
%       https://github.com/abella-prover/abella/blob/ceee13822001137898ca5de9c76a315da3d1f0e3/src/extensions.ml#L421
%       https://github.com/bikallem/spring/blob/6d10fef0b21caea3f975ff13132f5994e7c37db5/lib_spring/response.ml#L99
%       https://github.com/bikallem/spring/blob/6d10fef0b21caea3f975ff13132f5994e7c37db5/lib_spring/headers.ml#L150
%       https://github.com/polytypic/io/blob/dbe4d37a98ef958178fe18bb4dae396a5537392e/src/io/io.ml#L8
%       https://github.com/avsm/eeww/blob/5b6c0617306f4c9d8b44bf07ef4d45ec9668dd0c/lib/cohttp/cohttp-eio/src/rwer.ml#L43
%       https://github.com/just-max/less-power/blob/96f8e88f72275246b6bc175e44c732aa6e08ce7f/src/common/path_util.ml#L18
%       https://github.com/dalps/ocaml-challenge/blob/8fe115023457b6b4359ef736c78ba980cd871717/3/extract/solve.ml#L4
%       https://github.com/Octachron/ocaml-changelog-analyzer/blob/d6757b9576b5070ea3cfe6d4f4f79aaa3cc6e5d4/parse/release.ml#L4
%       https://github.com/Skyb0rg007/PL-Reading-Group/blob/7002ae35ba69fb9010e5049eee88ab475bc79ec3/list-optimizations/lib/adt.ml#L44
%       https://github.com/jfeser/symetric/blob/f49a57afc90a2c1e38f1097f98bd74725c074b9b/lib/tower.ml#L75
%       https://github.com/verse-lab/sisyphus/blob/a08addae06bbccb1e6ea68bac3d7f9eeea4197be/lib/dynamic/utils.ml#L65
%       https://github.com/jaymody/ocaml-tokenizers/blob/06cbce75d2865690cb39017222efff5c284bc721/lib/utils.ml#L18
%       https://github.com/Bannerets/ocaml-kdl/blob/687c404ebeceef7fd905935c23665294133f99e6/src/lens.ml#L83
%       https://github.com/OCADml/OCADml/blob/f5dfbf55f9c19ad808daa0df4d76a55e58756e5f/lib/poly3.ml#L33
%     other type
%       https://github.com/johnridesabike/acutis/blob/4113fb516d9c5dd6f2dbfd9657f52c5ae7a5dcae/lib/matching.ml#L161
%         (on-demand stream as a record of functions)
%       https://github.com/RedPRL/ocaml-bwd/blob/fbf496b29532085b38073eaa62ba3d22ac619d5d/src/BwdNoLabels.ml#L60
%         (snoc list)
%       https://github.com/Skyb0rg007/PL-Reading-Group/blob/7002ae35ba69fb9010e5049eee88ab475bc79ec3/effects/lib/free/tseq.ml#L44
%         (difference list gadt)

%   - Misuse
%     https://github.com/chengsun/simd-sexp/commit/e714a8205c6df512d920a31a4dd2448975703c55
%       (already tail-recursive)
%     https://github.com/gridbugs/llama/blob/96d1c96c31ff136f465b5f37c981fff591bac6fd/src/midi/byte_array_parser.ml#L50
%       (needless complexity)
%     https://github.com/LimitEpsilon/lambda-interpreter/blob/ab8e81de7f896aa64eba14e23d964b9705e94161/lib/evaluate.ml#L46
%       (already tail-recursive)

\subsection{TMC and OCaml 5}

%%% Local Variables:
%%% mode: latex
%%% TeX-master: "main"
%%% End:
\clearpage

\section{Language}

To formalize the TMC transformation, we define the \DataLang language.
Its syntax and semantics are given in \cref{fig:syntax} and \cref{fig:semantics}.
We also introduce syntactic sugar in \cref{fig:sugar}, including some for lists.
For instance, we can define the \datalang|map| function on lists as in \cref{fig:map}.

\DataLang is an untyped sequential post closure conversion lambda calculus with mutable state.
That is, it does not feature lambda expressions, which are already compiled to toplevel closed functions referred to using function pointers ($\datalangFnptr{\datalangFn}$).

There is one noteworthy trait.
To express constructors and closures, \DataLang features tagged mutable memory blocks with two fields.
One can allocate a block with $\datalangBlock{\datalangTag}{\datalangExpr_1}{\datalangExpr_2}$, access its fields with $\datalangLoad{\datalangExpr_1}{\datalangExpr_2}$ and modify them with $\datalangStore{\datalangExpr_1}{\datalangExpr_2}{\datalangExpr_3}$.
As in \OCaml, the evaluation order of subexpressions $\datalangExpr_1$ and $\datalangExpr_2$ in $\datalangBlock{\datalangTag}{\datalangExpr_1}{\datalangExpr_2}$ is unspecified.
To model this in the semantics, such an expression first nondeterministically reduces to either $\datalangLet{\datalangVar_1}{\datalangExpr_1}{\datalangLet{\datalangVar_2}{\datalangExpr_2}{\datalangBlockDet{\datalangTag}{\datalangVar_1}{\datalangVar_2}}}$ through \RefTirName{StepBlock1} or $\datalangLet{\datalangVar_2}{\datalangExpr_2}{\datalangLet{\datalangVar_1}{\datalangExpr_1}{\datalangBlockDet{\datalangTag}{\datalangVar_1}{\datalangVar_2}}}$ through \RefTirName{StepBlock2} and $\datalangBlockDet{\datalangTag}{\datalangVal_1}{\datalangVal_2}$ performs the allocation  through \RefTirName{StepBlockDet}.
This $\datalangBlockDet{\datalangTag}{\datalangExpr_1}{\datalangExpr_2}$ construction cannot appear in source programs.

%Another choice is to take an interleaving semantics.

Obviously, \DataLang does not support all features of the \LambdaLang intermediate language in which the original TMC transformation is written.
It is only meant to study and verify the core of the transformation.

As a side note, we use named expression variables in this article but the \Coq mechanization actually adopts de Bruijn syntax better suited to define transformations involving binders.
More precisely, we rely on the \Autosubst library~\cite{DBLP:conf/itp/SchaferTS15}.

\begin{figure}[tp]
    \begin{tabular}{rclclcl}
            $\nterm{Index}$
            & $\ni$ &
            i
            & $\Coloneqq$ &
            $\lambdZero \mid \lambdOne \mid \lambdTwo$
            &&
            index
        \\
            $\nterm{Tag}$
            & $\ni$ &
            $t$
            &&
            &&
            tag
        \\
            $\mathbb{B}$
            & $\ni$ &
            $b$
            &&
            &&
            boolean
    	\\
    		$\mathbb{L}$
    		& $\ni$ &
    		$\loc$
    		&&
    		&&
    		location
        \\
            $\mathbb{F}$
            & $\ni$ &
            $f$
            &&
            &&
            function
        \\
            $\mathbb{X}$
            & $\ni$ &
            $x, y, z$
            &&
            &&
            variable
    	\\
            $\nterm{Val}$
            & $\ni$ &
            $v$
            & $\Coloneqq$ &
            $\lambdUnit \mid i \mid t \mid b \mid \ell \mid \lambdFunc{f}$
            &&
            value
        \\
            $\nterm{Expr}$
            & $\ni$ &
            $e$
            & $\Coloneqq$ &
            $v$
            &&
            expression
        \\
            &&
            & | &
            $x \mid \lambdLet{x}{e_1}{e_2} \mid \lambdCall{e_1}{e_2}$
        \\
            &&
            & | &
            $\lambdEq{e_1}{e_2}$
        \\
            &&
            & | &
            $\lambdIf{e_0}{e_1}{e_2}$
        \\
            &&
            & | &
            $\lambdConstr{t}{e_1}{e_2} \mid \lambdConstrDet{t}{e_1}{e_2}$
        \\
            &&
            & | &
            $\lambdLoad{e_1}{e_2}$
        \\
            &&
            & | &
            $\lambdStore{e_1}{e_2}{e_3}$
        \\
            $\nterm{Ectx}$
            & $\ni$ &
            $K$
            & $\Coloneqq$ &
            $\Box$
            &&
            evaluation context
        \\
            &&
            & | &
            $\lambdLet{x}{K}{e_2} \mid \lambdCall{e_1}{K} \mid \lambdCall{K}{v_2}$
        \\
            &&
            & | &
            $\lambdEq{e_1}{K} \mid \lambdEq{K}{v_2}$
        \\
            &&
            & | &
            $\lambdIf{K}{e_1}{e_2}$
        \\
            &&
            & | &
            $\lambdLoad{e_1}{K} \mid \lambdLoad{K}{v_2}$
        \\
            &&
            & | &
            $\lambdStore{e_1}{e_2}{K} \mid \lambdStore{e_1}{K}{v_3} \mid \lambdStore{K}{v_2}{v_3}$
        \\
            $\nterm{Def}$
            & $\ni$ &
            $d$
            & $\Coloneqq$ &
            $\lambdDef{x}{e}$
            &&
            definition
        \\
            $\nterm{Prog}$
            & $\ni$ &
            $p$
            & $\coloneqq$ &
            $\mathbb{F} \finmap \nterm{Def}$
            &&
            program
        \\
            $\nterm{State}$
            & $\ni$ &
            $\sigma$    
            & $\coloneqq$ &
            $\mathbb{L} \finmap \nterm{Val}$
            &&
            state
        \\
            $\nterm{Config}$
            & $\ni$ &
            $\rho$
            & $\coloneqq$ &
            $\nterm{Expr} \times \nterm{State}$
            &&
            configuration
        \\
            $\nterm{Bisubst}$
            & $\ni$ &
            $\Gamma$
            & $\coloneqq$ &
            $\mathbb{X} \rightarrow \nterm{Expr} \times \nterm{Expr}$
            &&
            bisubstitution
    \end{tabular}
    \caption{Syntax of \LambdaLang}
    \label{fig:syntax}
\end{figure}
\begin{figure}[tp]
    \begin{tabular}{rcl}
            $\datalangSeq{\datalangExpr_1}{\datalangExpr_2}$
            & $\coloneqq$ &
            $\datalangLet{\datalangVar}{\datalangExpr_1}{\datalangExpr_2}$
        \\
            $\datalangPair{\datalangExpr_1}{\datalangExpr_2}$
            & $\coloneqq$ &
            $\datalangBlock{\mathrm{PAIR}}{\datalangExpr_1}{\datalangExpr_2}$
        \\
            $\datalangLet{\datalangPair{\datalangVar_1}{\datalangVar_2}}{\datalangExpr_1}{\datalangExpr_2}$
            & $\coloneqq$ &
            $\datalangLet{\datalangVarTwo}{\datalangExpr_1}{$
        \\
            &&
            $\datalangLet{\datalangVar_1}{\datalangLoad{\datalangVarTwo}{1}}{$
        \\
            &&
            $\datalangLet{\datalangVar_2}{\datalangLoad{\datalangVarTwo}{2}}{$
        \\
            &&
            $\datalangExpr_2}}}$
        \\
            $\datalangRec{\datalangPair{\datalangVar_1}{\datalangVar_2}}{\datalangExpr}$
            & $\coloneqq$ &
            $\datalangRec{\datalangVarTwo}{
            \datalangLet{\datalangPair{\datalangVar_1}{\datalangVar_2}}{\datalangVarTwo}{\datalangExpr}}$
        \\
            $\datalangNil$
            & $\coloneqq$ &
            $\datalangUnit$
        \\
            $\datalangCons{\datalangExpr_1}{\datalangExpr_2}$
            & $\coloneqq$ &
            $\datalangBlock{\mathrm{CONS}}{\datalangExpr_1}{\datalangExpr_2}$
        \\
            $\datalangMatchOneline[]{\datalangExpr_0}{\datalangExpr_1}{\datalangVar}{\mathit{\datalangVar s}}{\datalangExpr_2}$
            & $\coloneqq$ &
            $\datalangLet{\datalangVarTwo}{\datalangExpr_0}{$
        \\
            &&
            $\datalangIf{\datalangEq{\datalangVarTwo}{\datalangNil}$
        \\
            &&
            $}{\datalangExpr_1$
        \\
            &&
            $}{\datalangLet{\datalangPair{\datalangVar}{\mathit{\datalangVar s}}}{\datalangVarTwo}{\datalangExpr_2}}}$
        \\
            $\datalangHole$
            & $\coloneqq$ &
            $\datalangUnit$
    \end{tabular}
    \caption{\DataLang syntactic sugar (omitting freshness conditions)}
    \label{fig:sugar}
\end{figure}
\begin{figure}[tp]
    \begin{mathparpagebreakable}
        \inferrule*
            {}{
                \boxed{- \headStep{\datalangProg} - : \datalangConfig[] \rightarrow \datalangConfig[] \rightarrow \Prop}
            }
        \and
        \inferrule*
            {}{
                \boxed{- \step{\datalangProg} - : \datalangConfig[] \rightarrow \datalangConfig[] \rightarrow \Prop}
            }
        \\
        \inferrule*[lab=StepLet]
            {}{
                \left(
                    \datalangLet{\datalangVar}{\datalangVal}{\datalangExpr},
                    \datalangState
                \right)
                \headStep{\datalangProg}
                \left(
                    \datalangExpr{} [\datalangVar \backslash \datalangVal],
                    \datalangState
                \right)
            }
        \and
        \inferrule*[lab=StepCall]
            {
                \datalangProg{} [\datalangFn] = (\datalangRec{\datalangVar}{\datalangExpr})
            }{
                \left(
                    \datalangCall{\datalangFnptr{\datalangFn}}{\datalangVal},
                    \datalangState
                \right)
                \headStep{\datalangProg}
                \left(
                    \datalangExpr{} [\datalangVar \backslash \datalangVal],
                    \datalangState
                \right)
            }
        \\
        \inferrule*[lab=StepBlock1]
            {}{
                \left(
                    \datalangBlock{\datalangTag}{\datalangExpr_1}{\datalangExpr_2},
                    \datalangState
                \right)
                \headStep{\datalangProg}
                \left(
                    \begin{array}{l}
                        \datalangLet{\datalangVar_1}{\datalangExpr_1}{\\
                        \datalangLet{\datalangVar_2}{\datalangExpr_2}{\\
                        \datalangBlockDet{\datalangTag}{\datalangVar_1}{\datalangVar_2}}}
                    \end{array},
                    \datalangState
                \right)
            }
        \and
        \inferrule*[lab=StepBlock2]
            {}{
                \left(
                    \datalangBlock{\datalangTag}{\datalangExpr_1}{\datalangExpr_2},
                    \datalangState
                \right)
                \headStep{\datalangProg}
                \left(
                    \begin{array}{l}
                        \datalangLet{\datalangVar_2}{\datalangExpr_2}{\\
                        \datalangLet{\datalangVar_1}{\datalangExpr_1}{\\
                        \datalangBlockDet{\datalangTag}{\datalangVar_1}{\datalangVar_2}}}
                    \end{array},
                    \datalangState
                \right)
            }
        \\
        \inferrule*[lab=StepBlockDet]
            {
                \forall \datalangIdx \in \datalangIdx[], \datalangLoc + \datalangIdx \notin \dom{\datalangState}
            }{
                \left(
                    \datalangBlockDet{\datalangTag}{\datalangVal_1}{\datalangVal_2},
                    \datalangState
                \right)
                \headStep{\datalangProg}
                \left(
                    \datalangLoc,
                    \datalangState{} [\datalangLoc \mapsto \datalangTag, \datalangVal_1, \datalangVal_2]
                \right)
            }
        \and
        \inferrule*[lab=StepLoad]
            {
                \datalangState{} [\datalangLoc] = \datalangVal
            }{
                \left(
                    \datalangLoad{\datalangLoc}{\datalangIdx},
                    \datalangState
                \right)
                \headStep{\datalangProg}
                \left(
                    \datalangVal,
                    \datalangState
                \right)
            }
        \and
        \inferrule*[lab=StepStore]
            {
                \datalangLoc + \datalangIdx \in \dom{\datalangState}
            }{
                \left(
                    \datalangStore{\datalangLoc}{\datalangIdx}{\datalangVal},
                    \datalangState
                \right)
                \headStep{\datalangProg}
                \left(
                    \datalangUnit,
                    \datalangState{} [\datalangLoc + \datalangIdx \mapsto \datalangVal]
                \right)
            }
        \and
        \inferrule*[lab=StepEctx]
            {
                \left(
                    \datalangExpr,
                    \datalangState
                \right)
                \headStep{\datalangProg}
                \left(
                    \datalangExpr',
                    \datalangState'
                \right)
            }{
                \left(
                    \datalangEctx{} [\datalangExpr],
                    \datalangState
                \right)
                \step{\datalangProg}
                \left(
                    \datalangEctx{} [\datalangExpr'],
                    \datalangState'
                \right)
            }
    \end{mathparpagebreakable}
    \caption{\DataLang semantics (excerpt)}
    \label{fig:semantics}
\end{figure}
\begin{figure}[tp]
\begin{tabular}{c}
\begin{Datalang}
map |-> rec λ(f, xs) = match xs with
                       | [] -> []
                       | x :: xs -> let y = f x in y :: @map (f, xs)
\end{Datalang}
\end{tabular}
\caption{Natural implementation of \datalang|map| in \DataLang}
\label{fig:map}
\end{figure}
\clearpage

\section{Transformation}

Lambda language

syntactic sugar

well-formedness, closed

TMC relation

example: map

\begin{figure}[tp]
    \begin{tabular}{rclclcl}
            $\nterm{Index}$
            & $\ni$ &
            i
            & $\Coloneqq$ &
            $\lambdZero \mid \lambdOne \mid \lambdTwo$
            &&
            index
        \\
            $\nterm{Tag}$
            & $\ni$ &
            $t$
            &&
            &&
            tag
        \\
            $\mathbb{B}$
            & $\ni$ &
            $b$
            &&
            &&
            boolean
    	\\
    		$\mathbb{L}$
    		& $\ni$ &
    		$\loc$
    		&&
    		&&
    		location
        \\
            $\mathbb{F}$
            & $\ni$ &
            $f$
            &&
            &&
            function
        \\
            $\mathbb{X}$
            & $\ni$ &
            $x, y, z$
            &&
            &&
            variable
    	\\
            $\nterm{Val}$
            & $\ni$ &
            $v$
            & $\Coloneqq$ &
            $\lambdUnit \mid i \mid t \mid b \mid \ell \mid \lambdFunc{f}$
            &&
            value
        \\
            $\nterm{Expr}$
            & $\ni$ &
            $e$
            & $\Coloneqq$ &
            $v$
            &&
            expression
        \\
            &&
            & | &
            $x \mid \lambdLet{x}{e_1}{e_2} \mid \lambdCall{e_1}{e_2}$
        \\
            &&
            & | &
            $\lambdEq{e_1}{e_2}$
        \\
            &&
            & | &
            $\lambdIf{e_0}{e_1}{e_2}$
        \\
            &&
            & | &
            $\lambdConstr{t}{e_1}{e_2} \mid \lambdConstrDet{t}{e_1}{e_2}$
        \\
            &&
            & | &
            $\lambdLoad{e_1}{e_2}$
        \\
            &&
            & | &
            $\lambdStore{e_1}{e_2}{e_3}$
        \\
            $\nterm{Ectx}$
            & $\ni$ &
            $K$
            & $\Coloneqq$ &
            $\Box$
            &&
            evaluation context
        \\
            &&
            & | &
            $\lambdLet{x}{K}{e_2} \mid \lambdCall{e_1}{K} \mid \lambdCall{K}{v_2}$
        \\
            &&
            & | &
            $\lambdEq{e_1}{K} \mid \lambdEq{K}{v_2}$
        \\
            &&
            & | &
            $\lambdIf{K}{e_1}{e_2}$
        \\
            &&
            & | &
            $\lambdLoad{e_1}{K} \mid \lambdLoad{K}{v_2}$
        \\
            &&
            & | &
            $\lambdStore{e_1}{e_2}{K} \mid \lambdStore{e_1}{K}{v_3} \mid \lambdStore{K}{v_2}{v_3}$
        \\
            $\nterm{Def}$
            & $\ni$ &
            $d$
            & $\Coloneqq$ &
            $\lambdDef{x}{e}$
            &&
            definition
        \\
            $\nterm{Prog}$
            & $\ni$ &
            $p$
            & $\coloneqq$ &
            $\mathbb{F} \finmap \nterm{Def}$
            &&
            program
        \\
            $\nterm{State}$
            & $\ni$ &
            $\sigma$    
            & $\coloneqq$ &
            $\mathbb{L} \finmap \nterm{Val}$
            &&
            state
        \\
            $\nterm{Config}$
            & $\ni$ &
            $\rho$
            & $\coloneqq$ &
            $\nterm{Expr} \times \nterm{State}$
            &&
            configuration
        \\
            $\nterm{Bisubst}$
            & $\ni$ &
            $\Gamma$
            & $\coloneqq$ &
            $\mathbb{X} \rightarrow \nterm{Expr} \times \nterm{Expr}$
            &&
            bisubstitution
    \end{tabular}
    \caption{Syntax of \LambdaLang}
    \label{fig:syntax}
\end{figure}
\begin{figure}[tp]
    \begin{tabular}{rcl}
            $\datalangSeq{\datalangExpr_1}{\datalangExpr_2}$
            & $\coloneqq$ &
            $\datalangLet{\datalangVar}{\datalangExpr_1}{\datalangExpr_2}$
        \\
            $\datalangPair{\datalangExpr_1}{\datalangExpr_2}$
            & $\coloneqq$ &
            $\datalangBlock{\mathrm{PAIR}}{\datalangExpr_1}{\datalangExpr_2}$
        \\
            $\datalangLet{\datalangPair{\datalangVar_1}{\datalangVar_2}}{\datalangExpr_1}{\datalangExpr_2}$
            & $\coloneqq$ &
            $\datalangLet{\datalangVarTwo}{\datalangExpr_1}{$
        \\
            &&
            $\datalangLet{\datalangVar_1}{\datalangLoad{\datalangVarTwo}{1}}{$
        \\
            &&
            $\datalangLet{\datalangVar_2}{\datalangLoad{\datalangVarTwo}{2}}{$
        \\
            &&
            $\datalangExpr_2}}}$
        \\
            $\datalangRec{\datalangPair{\datalangVar_1}{\datalangVar_2}}{\datalangExpr}$
            & $\coloneqq$ &
            $\datalangRec{\datalangVarTwo}{
            \datalangLet{\datalangPair{\datalangVar_1}{\datalangVar_2}}{\datalangVarTwo}{\datalangExpr}}$
        \\
            $\datalangNil$
            & $\coloneqq$ &
            $\datalangUnit$
        \\
            $\datalangCons{\datalangExpr_1}{\datalangExpr_2}$
            & $\coloneqq$ &
            $\datalangBlock{\mathrm{CONS}}{\datalangExpr_1}{\datalangExpr_2}$
        \\
            $\datalangMatchOneline[]{\datalangExpr_0}{\datalangExpr_1}{\datalangVar}{\mathit{\datalangVar s}}{\datalangExpr_2}$
            & $\coloneqq$ &
            $\datalangLet{\datalangVarTwo}{\datalangExpr_0}{$
        \\
            &&
            $\datalangIf{\datalangEq{\datalangVarTwo}{\datalangNil}$
        \\
            &&
            $}{\datalangExpr_1$
        \\
            &&
            $}{\datalangLet{\datalangPair{\datalangVar}{\mathit{\datalangVar s}}}{\datalangVarTwo}{\datalangExpr_2}}}$
        \\
            $\datalangHole$
            & $\coloneqq$ &
            $\datalangUnit$
    \end{tabular}
    \caption{\DataLang syntactic sugar (omitting freshness conditions)}
    \label{fig:sugar}
\end{figure}
\begin{figure}[tp]
    \begin{mathparpagebreakable}
        \inferrule*
            {}{
                \boxed{- \headStep{\datalangProg} - : \datalangConfig[] \rightarrow \datalangConfig[] \rightarrow \Prop}
            }
        \and
        \inferrule*
            {}{
                \boxed{- \step{\datalangProg} - : \datalangConfig[] \rightarrow \datalangConfig[] \rightarrow \Prop}
            }
        \\
        \inferrule*[lab=StepLet]
            {}{
                \left(
                    \datalangLet{\datalangVar}{\datalangVal}{\datalangExpr},
                    \datalangState
                \right)
                \headStep{\datalangProg}
                \left(
                    \datalangExpr{} [\datalangVar \backslash \datalangVal],
                    \datalangState
                \right)
            }
        \and
        \inferrule*[lab=StepCall]
            {
                \datalangProg{} [\datalangFn] = (\datalangRec{\datalangVar}{\datalangExpr})
            }{
                \left(
                    \datalangCall{\datalangFnptr{\datalangFn}}{\datalangVal},
                    \datalangState
                \right)
                \headStep{\datalangProg}
                \left(
                    \datalangExpr{} [\datalangVar \backslash \datalangVal],
                    \datalangState
                \right)
            }
        \\
        \inferrule*[lab=StepBlock1]
            {}{
                \left(
                    \datalangBlock{\datalangTag}{\datalangExpr_1}{\datalangExpr_2},
                    \datalangState
                \right)
                \headStep{\datalangProg}
                \left(
                    \begin{array}{l}
                        \datalangLet{\datalangVar_1}{\datalangExpr_1}{\\
                        \datalangLet{\datalangVar_2}{\datalangExpr_2}{\\
                        \datalangBlockDet{\datalangTag}{\datalangVar_1}{\datalangVar_2}}}
                    \end{array},
                    \datalangState
                \right)
            }
        \and
        \inferrule*[lab=StepBlock2]
            {}{
                \left(
                    \datalangBlock{\datalangTag}{\datalangExpr_1}{\datalangExpr_2},
                    \datalangState
                \right)
                \headStep{\datalangProg}
                \left(
                    \begin{array}{l}
                        \datalangLet{\datalangVar_2}{\datalangExpr_2}{\\
                        \datalangLet{\datalangVar_1}{\datalangExpr_1}{\\
                        \datalangBlockDet{\datalangTag}{\datalangVar_1}{\datalangVar_2}}}
                    \end{array},
                    \datalangState
                \right)
            }
        \\
        \inferrule*[lab=StepBlockDet]
            {
                \forall \datalangIdx \in \datalangIdx[], \datalangLoc + \datalangIdx \notin \dom{\datalangState}
            }{
                \left(
                    \datalangBlockDet{\datalangTag}{\datalangVal_1}{\datalangVal_2},
                    \datalangState
                \right)
                \headStep{\datalangProg}
                \left(
                    \datalangLoc,
                    \datalangState{} [\datalangLoc \mapsto \datalangTag, \datalangVal_1, \datalangVal_2]
                \right)
            }
        \and
        \inferrule*[lab=StepLoad]
            {
                \datalangState{} [\datalangLoc] = \datalangVal
            }{
                \left(
                    \datalangLoad{\datalangLoc}{\datalangIdx},
                    \datalangState
                \right)
                \headStep{\datalangProg}
                \left(
                    \datalangVal,
                    \datalangState
                \right)
            }
        \and
        \inferrule*[lab=StepStore]
            {
                \datalangLoc + \datalangIdx \in \dom{\datalangState}
            }{
                \left(
                    \datalangStore{\datalangLoc}{\datalangIdx}{\datalangVal},
                    \datalangState
                \right)
                \headStep{\datalangProg}
                \left(
                    \datalangUnit,
                    \datalangState{} [\datalangLoc + \datalangIdx \mapsto \datalangVal]
                \right)
            }
        \and
        \inferrule*[lab=StepEctx]
            {
                \left(
                    \datalangExpr,
                    \datalangState
                \right)
                \headStep{\datalangProg}
                \left(
                    \datalangExpr',
                    \datalangState'
                \right)
            }{
                \left(
                    \datalangEctx{} [\datalangExpr],
                    \datalangState
                \right)
                \step{\datalangProg}
                \left(
                    \datalangEctx{} [\datalangExpr'],
                    \datalangState'
                \right)
            }
    \end{mathparpagebreakable}
    \caption{\DataLang semantics (excerpt)}
    \label{fig:semantics}
\end{figure}
\begin{figure}[tp]
    \begin{mathparpagebreakable}
        \inferrule*
            {}{
                \boxed{- \tmcDir{\datalangRenaming} - : \datalangDef[] \rightarrow \datalangDef[] \rightarrow \Prop}
            }
        \and
        \inferrule*
            {}{
                \boxed{- \tmcDir{\datalangRenaming} - : \datalangExpr[] \rightarrow \datalangExpr[] \rightarrow \Prop}
            }
        \\
        \inferrule*[lab=DirDef]
            {
                \datalangExpr_s \tmcDir{\datalangRenaming} \datalangExpr_t
            }{
                \datalangRec{\datalangVar}{\datalangExpr_s}
                \tmcDir{\datalangRenaming}
                \datalangRec{\datalangVar}{\datalangExpr_t}
            }
        \\
        \inferrule*[lab=DirVal]
            {}{
                \datalangVal
                \tmcDir{\datalangRenaming}
                \datalangVal
            }
        \and
        \inferrule*[lab=DirVar]
            {}{
                \datalangVar
                \tmcDir{\datalangRenaming}
                \datalangVar
            }
        \and
        \inferrule*[lab=DirLet]
            {
                \datalangExpr_{s1} \tmcDir{\datalangRenaming} \datalangExpr_{t1}
            \and
                \datalangExpr_{s2} \tmcDir{\datalangRenaming} \datalangExpr_{t2}
            }{
                \datalangLet{\datalangVar}{\datalangExpr_{s1}}{\datalangExpr_{s2}}
                \tmcDir{\datalangRenaming}
                \datalangLet{\datalangVar}{\datalangExpr_{t1}}{\datalangExpr_{t2}}
            }
        \\
        \inferrule*[lab=DirCall]
            {
                \datalangExpr_{s1} \tmcDir{\datalangRenaming} \datalangExpr_{t1}
            \and
                \datalangExpr_{s2} \tmcDir{\datalangRenaming} \datalangExpr_{t2}
            }{
                \datalangCall {\datalangExpr_{s1}} {\datalangExpr_{s2}}
                \tmcDir{\datalangRenaming}
                \datalangCall {\datalangExpr_{t1}} {\datalangExpr_{t2}}
            }
        \and
        \inferrule*[lab=DirBlock]
            {
                \datalangExpr_{s1} \tmcDir{\datalangRenaming} \datalangExpr_{t1}
            \and
                \datalangExpr_{s2} \tmcDir{\datalangRenaming} \datalangExpr_{t2}
            }{
                \datalangBlock \datalangTag {\datalangExpr_{s1}} {\datalangExpr_{s2}}
                \tmcDir{\datalangRenaming}
                \datalangBlock \datalangTag {\datalangExpr_{t1}} {\datalangExpr_{t2}}
            }
        \\
        \inferrule*[lab=DirBlockDPS1]
            {
                (\datalangVar, \datalangOne, \datalangExpr_{s1}) \tmcDps{\datalangRenaming} \datalangExpr_{t1}
            \and
                \datalangExpr_{s2} \tmcDir{\datalangRenaming} \datalangExpr_{t2}
            }{
                \datalangBlock{\datalangTag}{\datalangExpr_{s1}}{\datalangExpr_{s2}}
                \tmcDir{\datalangRenaming}
                \begin{array}{l}
                    \datalangLet{\datalangVar}{\datalangBlock{\datalangTag}{\datalangHole}{\datalangExpr_{t2}}}{\\
                    \datalangSeq{\datalangExpr_{t1}}{\datalangVar}}
                \end{array}
            }
        \and
        \inferrule*[lab=DirBlockDPS2]
            {
                \datalangExpr_{s1} \tmcDir{\datalangRenaming} \datalangExpr_{t1}
            \and
                (\datalangVar, \datalangTwo, \datalangExpr_{s2}) \tmcDps{\datalangRenaming} \datalangExpr_{t2}
            }{
                \datalangBlock{\datalangTag}{\datalangExpr_{s1}}{\datalangExpr_{s2}}
                \tmcDir{\datalangRenaming}
                \begin{array}{l}
                    \datalangLet{\datalangVar}{\datalangBlock{\datalangTag}{\datalangExpr_{t1}}{\datalangHole}}{\\
                  \datalangSeq{\datalangExpr_{t2}}{\datalangVar}}
                \end{array}
            }
        \and
        \inferrule*[lab=DirBlockDPS12]
            {
                (\datalangVar, \datalangOne, \datalangExpr_{s1}) \tmcDps{\datalangRenaming} \datalangExpr_{t1}
            \and
                (\datalangVar, \datalangTwo, \datalangExpr_{s2}) \tmcDps{\datalangRenaming} \datalangExpr_{t2}
            }{
                \datalangBlock{\datalangTag}{\datalangExpr_{s1}}{\datalangExpr_{s2}}
                \tmcDir{\datalangRenaming}
                \begin{array}{l}
                    \datalangLet{\datalangVar}{\datalangBlock{\datalangTag}{\datalangHole}{\datalangHole}}{\\
                    \datalangSeq{\datalangExpr_{t1}}{
                      \datalangSeq{\datalangExpr_{t2}}{
                        \datalangVar}}}
                \end{array}
            }
    \end{mathparpagebreakable}
    \caption{Direct TMC transformation (omitting congruence rules similar to \TirName{DirCall})}
    \label{fig:tmc_dir}
\end{figure}

%%% Local Variables:
%%% mode: latex
%%% TeX-master: "../main"
%%% End:

\begin{figure}[tp]
    \begin{mathparpagebreakable}
        \inferrule*
            {}{
                \boxed{- \tmcDps{\datalangRenaming} - : \datalangDef[] \rightarrow \datalangDef[] \rightarrow \Prop}
            }
        \and
        \inferrule*
            {}{
                \boxed{- \tmcDps{\datalangRenaming} - : \datalangExpr[] \times \datalangExpr[] \times \datalangExpr[] \rightarrow \datalangExpr[] \rightarrow \Prop}
            }
        \\
        \inferrule*[lab=DPSDef]
            {
                (\datalangVar_\mathit{dst}, \datalangVar_\mathit{idx}, \datalangExpr_s) \tmcDps{\datalangRenaming} \datalangExpr_t
            }{
                \datalangRec{\datalangVar}{\datalangExpr_s}
                \tmcDps{\datalangRenaming}
                \datalangRec{\datalangPair{\datalangPair{\datalangVar_\mathit{dst}}{\datalangVar_\mathit{idx}}}{\datalangVar}}{\datalangExpr_t}
            }
        \\
        \inferrule*[lab=DPSLet]
            {
                \datalangExpr_{s1} \tmcDir{\datalangRenaming} \datalangExpr_{t1}
            \and
                (\datalangExpr_\mathit{dst}, \datalangExpr_\mathit{idx}, \datalangExpr_{s2}) \tmcDps{\datalangRenaming} \datalangExpr_{t2}
            }{
                \left(
                  {\begin{array}{l}
                    \datalangExpr_\mathit{dst},
                    \datalangExpr_\mathit{idx}, \\
                    \datalangLet{\datalangVar}{\datalangExpr_{s1}}{\datalangExpr_{s2}}
                  \end{array}}
                \right)
                \tmcDps{\datalangRenaming}
                \datalangLet{\datalangVar}{\datalangExpr_{t1}}{\datalangExpr_{t2}}
            }
        \quad
        \inferrule*[lab=DPSIf]
            {
                \datalangExpr_{s0} \tmcDir{\datalangRenaming} \datalangExpr_{t0}
            \and
                (\datalangExpr_\mathit{dst}, \datalangExpr_\mathit{idx}, \datalangExpr_{s1}) \tmcDps{\datalangRenaming} \datalangExpr_{s2}
            \and
                (\datalangExpr_\mathit{dst}, \datalangExpr_\mathit{idx}, \datalangExpr_{s2}) \tmcDps{\datalangRenaming} \datalangExpr_{t2}
            }{
                \left(
                  {\begin{array}{l}
                    \datalangExpr_\mathit{dst},
                    \datalangExpr_\mathit{idx}, \\
                    \datalangIf{\datalangExpr_{s0}}{\datalangExpr_{s1}}{\datalangExpr_{s2}}
                  \end{array}}
                \right)
                \tmcDps{\datalangRenaming}
                \datalangIf{\datalangExpr_{t0}}{\datalangExpr_{t1}}{\datalangExpr_{t2}}
            }
        \\
        \inferrule*[lab=DPSBlock1]
            {
                (\datalangVar, \datalangOne, \datalangExpr_{s1}) \tmcDps{\datalangRenaming} \datalangExpr_{t1}
            \and
                \datalangExpr_{s2} \tmcDir{\datalangRenaming} \datalangExpr_{t2}
            }{
                \left(
                  {\begin{array}{l}
                    \datalangExpr_\mathit{dst},
                    \datalangExpr_\mathit{idx}, \\
                    \datalangBlock{\datalangTag}{\datalangExpr_{s1}}{\datalangExpr_{s2}}
                  \end{array}}
                \right)
                \tmcDps{\datalangRenaming}
                \begin{array}{l}
                    \datalangLet{\datalangVar}{\datalangBlock{\datalangTag}{\datalangHole}{\datalangExpr_{t2}}}{\\
                    \datalangSeq{\datalangStore{\datalangExpr_\mathit{dst}}{\datalangExpr_\mathit{idx}}{\datalangVar}}{\datalangExpr_{t1}}}
                \end{array}
            }
        \quad
        \inferrule*[lab=DPSBlock2]
            {
                \datalangExpr_{s1} \tmcDir{\datalangRenaming} \datalangExpr_{t1}
            \and
                (\datalangVar, \datalangTwo, \datalangExpr_{s2}) \tmcDps{\datalangRenaming} \datalangExpr_{t2}
            }{
                \left(
                  {\begin{array}{l}
                    \datalangExpr_\mathit{dst},
                    \datalangExpr_\mathit{idx},
                    \\
                    \datalangBlock{\datalangTag}{\datalangExpr_{s1}}{\datalangExpr_{s2}}
                  \end{array}}
                \right)
                \tmcDps{\datalangRenaming}
                \begin{array}{l}
                    \datalangLet{\datalangVar}{\datalangBlock{\datalangTag}{\datalangExpr_{t1}}{\datalangHole}}{\\
                    \datalangSeq{\datalangStore{\datalangExpr_\mathit{dst}}{\datalangExpr_\mathit{idx}}{\datalangVar}}{\datalangExpr_{t2}}}
                \end{array}
            }
        % \and
        % \inferrule*[lab=DPSBlock21]
        %     {
        %         (\datalangVar, \datalangTwo, \datalangExpr_{s1}) \tmcDps{\datalangRenaming} \datalangExpr_{t1}
        %     \and\datalangExpr_{t1}
        %         (\datalangVar, \datalangTwo, \datalangExpr_{s2}) \tmcDps{\datalangRenaming} \datalangExpr_{t2}
        %     }{
        %         \left(
        %             \datalangExpr_\mathit{dst},
        %             \datalangExpr_\mathit{idx},
        %             \datalangBlock{\datalangTag}{\datalangExpr_{s1}}{\datalangExpr_{s2}}
        %         \right)
        %         \tmcDps{\datalangRenaming}
        %         \begin{array}{l}
        %             \datalangLet{\datalangVar}{\datalangBlock{\datalangTag}{\datalangHole}{\datalangHole}}{\\
        %             \datalangSeq{\datalangStore{\datalangExpr_\mathit{dst}}{\datalangExpr_\mathit{idx}}{\datalangVar}}{\\
        %             \datalangSeq{\datalangExpr_{t2}}{\datalangExpr_{t1}}}}
        %         \end{array}
        %     }
        % \;
        % \inferrule*[lab=DPSBlock12]
        %     {
        %         (\datalangVar, \datalangTwo, \datalangExpr_{s1}) \tmcDps{\datalangRenaming} \datalangExpr_{t1}
        %     \and\datalangExpr_{t1}
        %         (\datalangVar, \datalangTwo, \datalangExpr_{s2}) \tmcDps{\datalangRenaming} \datalangExpr_{t2}
        %     }{
        %         \left(
        %             \datalangExpr_\mathit{dst},
        %             \datalangExpr_\mathit{idx},
        %             \datalangBlock{\datalangTag}{\datalangExpr_{s1}}{\datalangExpr_{s2}}
        %         \right)
        %         \tmcDps{\datalangRenaming}
        %         \begin{array}{l}
        %             \datalangLet{\datalangVar}{\datalangBlock{\datalangTag}{\datalangHole}{\datalangHole}}{\\
        %             \datalangSeq{\datalangStore{\datalangExpr_\mathit{dst}}{\datalangExpr_\mathit{idx}}{\datalangVar}}{\\
        %             \datalangSeq{\datalangExpr_{t1}}{\datalangExpr_{t2}}}}
        %         \end{array}
        %     }
        \\
        \inferrule*[lab=DPSCall]
            {
                \datalangFn \in \dom{\datalangRenaming}
            \and
                \datalangExpr_s \tmcDir{\datalangRenaming} \datalangExpr_t
            }{
                \left(
                    \datalangExpr_\mathit{dst},
                    \datalangExpr_\mathit{idx},
                    \datalangCall{\datalangFnptr{\datalangFn}}{\datalangExpr_s}
                \right)
                \tmcDps{\datalangRenaming}
                \datalangCall{
                    \datalangFnptr{\datalangRenaming [\datalangFn]}
                }{
                    \datalangPair{\datalangPair{\datalangExpr_\mathit{dst}}{\datalangExpr_\mathit{idx}}}{\datalangExpr_t}
                }
            }
        \and
        \inferrule*[lab=DPSBase]
            {
                \datalangExpr_s \tmcDir{\datalangRenaming} \datalangExpr_t
            }{
                \left(
                    \datalangExpr_\mathit{dst},
                    \datalangExpr_\mathit{idx},
                    \datalangExpr_s
                \right)
                \tmcDps{\datalangRenaming}
                \datalangStore{\datalangExpr_\mathit{dst}}{\datalangIdx}{\datalangExpr_t}
            }
    \end{mathparpagebreakable}
    \caption{Destination-passing style TMC transformation of definitions and expressions (in full)}
    \label{fig:tmc_dps}
\end{figure}

%%% Local Variables:
%%% mode: latex
%%% TeX-master: "../main"
%%% End:

\begin{figure}[tp]
    \[
        p_s \tmc p_t
        \coloneqq
        \exists \xi \ldotp
        \bigwedge \left[ \begin{array}{l}
                \dom{\xi} \subseteq \dom{p_s}
            \\
                \dom{p_t} = \dom{p_s} \cup \codom{\xi}
            \\
                \forall f, d_s \ldotp
                p_s [f] = d_s \implies
                \exists d_t \ldotp
                p_t [f] = d_t \wedge d_s \tmcDir{\xi} d_t
            \\
                \forall f, f_\mathit{dps}, d_s \ldotp
                p_s [f] = d_s \wedge \xi [f] = f_{\mathit{dps}} \implies
                \exists d_t \ldotp
                p_t [f_\mathit{dps}] = d_t \wedge d_s \tmcDps{\xi} d_t
        \end{array} \right.
    \]
    \caption{TMC transformation}
    \label{fig:tmc}
\end{figure}
\begin{figure}[tp]
\begin{tabular}{c}
\begin{Lambd}
map |-> rec arg =
  let fn = arg.(1) in
  let xs = arg.(2) in
  match xs with
  | [] ==> 
      []
  | x :: xs ==>
      let y = fn x in
      y :: (@map (fn, xs))
\end{Lambd}
\end{tabular}
\caption{\lambd|map| in direct style (\LambdaLang)}
\label{fig:lambda_map}
\end{figure}
\begin{figure}[tp]
\begin{minipage}{.45\columnwidth}
\begin{Lambd}
map |-> rec arg =
  let fn = arg.(1) in
  let xs = arg.(2) in
  match xs with
  | [] ==> 
      []
  | x :: xs ==>
      let y = fn x in
      let dst = y :: () in
      @map_dps ((dst, 2), (fn, xs)) ;
      dst
\end{Lambd}
\end{minipage}
\hfill
\begin{minipage}{.45\columnwidth}
\begin{Lambd}
map_dps |-> rec arg' =
  let dst_idx = arg'.(1) in
  let dst = dst_idx.(1) in
  let idx = dst_idx.(2) in
  let arg = arg'.(2) in
  let fn = arg.(1) in
  let xs = arg.(2) in
  match xs with
  | [] ==> 
      dst.(idx) <- []
  | x :: xs ==>
      let y = fn x in
      let dst' = y :: () in
      dst.(idx) <- dst' ;
      @map_dps ((dst', 2), (fn, xs))
\end{Lambd}
\end{minipage}
\caption{\lambd|map| in destination-passing style (\LambdaLang)}
\label{fig:lambda_map_dps}
\end{figure}
\begin{figure}[tp]
	\centering
	\begin{tabular}{|c|ccc||c|ccc|}
		\hline
				\multicolumn{4}{|c||}{\bf{Source program}}
			&
				\multicolumn{4}{c|}{\bf{Transformed program}}
		\\ \hline
				\bf{Heap}
			&
				\multicolumn{3}{c||}{\bf{Position}}
			&
				\bf{Heap}
			&
				\multicolumn{3}{c|}{\bf{Position}}
		\\ \hline
				$\emptyset$
			&
				$\rightarrow$
			&
				$\lambdCall{\lambdText{map}}{\loc_1}$
			&
			&
				$\emptyset$
			&
				$\rightarrow$
			&
				$\lambdCall{\lambdText{map}}{\loc_1}$
			&
		\\ \hline
				$\emptyset$
			&
				$\rightarrow$
			&
				$\lambdCall{\lambdText{map}}{\loc_2}$
			&
			&
				$\loc_1' \mapsto (\mathrm{CONS}, v_1', \lambdHole)$
			&
				$\rightarrow$
			&
				$\lambdCall{\lambdText{map\_dps}}{\lambdPair{\lambdPair{\loc_1'}{\lambdTwo}}{\loc_2}}$
			&
		\\ \hline
				$\emptyset$
			&
				$\rightarrow$
			&
				$\lambdCall{\lambdText{map}}{\loc_3}$
			&
			&
				$\begin{array}{l}
						\loc_1' \mapsto (\mathrm{CONS}, v_1', \loc_2') \iSep {}
					\\
						\loc_2' \mapsto (\mathrm{CONS}, v_2', \lambdHole)
				\end{array}$
			&
				$\rightarrow$
			&
				$\lambdCall{\lambdText{map_dps}}{\lambdPair{\lambdPair{\loc_2'}{\lambdTwo}}{\loc_3}}$
			&
		\\ \hline
				$\textcolor{teal}{\loc_3'} \mapsto (\mathrm{NIL}, \lambdUnit, \lambdUnit)$
			&
			&
				$\lambdCall{\lambdText{map}}{\loc_3}$
			&
				$\rightarrow$
			&
				$\begin{array}{l}
						\loc_1' \mapsto (\mathrm{CONS}, v_1', \loc_2') \iSep {}
					\\
						\loc_2' \mapsto (\mathrm{CONS}, v_2', \loc_3') \iSep {}
					\\
                        \textcolor{teal}{\loc_3'} \mapsto (\mathrm{NIL}, \lambdUnit, \lambdUnit)
				\end{array}$
			&
			&
				$\lambdCall{\lambdText{map\_dps}}{\lambdPair{\lambdPair{\loc_2'}{\lambdTwo}}{\loc_3}}$
			&
				$\rightarrow$
		\\ \hline
				$\begin{array}{l}
				        \textcolor{teal}{\loc_3'} \mapsto (\mathrm{NIL}, \lambdUnit, \lambdUnit) \iSep {}
                    \\
				        \textcolor{blue}{\loc_2'} \mapsto (\mathrm{CONS}, v_2', \loc_3')
				\end{array}$
			&
			&
				$\lambdCall{\lambdText{map}}{\loc_2}$
			&
				$\rightarrow$
			&
				$\begin{array}{l}
						\loc_1' \mapsto (\mathrm{CONS}, v_1', \loc_2') \iSep {}
					\\
						\textcolor{blue}{\loc_2'} \mapsto (\mathrm{CONS}, v_2', \loc_3')
					\\
                        \textcolor{teal}{\loc_3'} \mapsto (\mathrm{NIL}, \lambdUnit, \lambdUnit)
				\end{array}$
			&
			&
				$\lambdCall{\lambdText{map\_dps}}{\lambdPair{\lambdPair{\loc_1'}{\lambdTwo}}{\loc_2}}$
			&
				$\rightarrow$
		\\ \hline
				$\begin{array}{l}
				        \textcolor{teal}{\loc_3'} \mapsto (\mathrm{NIL}, \lambdUnit, \lambdUnit) \iSep {}
                    \\
						\textcolor{blue}{\loc_2'} \mapsto (\mathrm{CONS}, v_2', \loc_3') \iSep {}
					\\
						\textcolor{red}{\loc_1'} \mapsto (\mathrm{CONS}, v_1', \loc_2')
				\end{array}$
			&
			&
				$\lambdCall{\lambdText{map}}{\loc_1}$
			&
				$\rightarrow$
			&
				$\begin{array}{l}
						\textcolor{red}{\loc_1'} \mapsto (\mathrm{CONS}, v_1', \loc_2') \iSep {}
					\\
						\textcolor{blue}{\loc_2'} \mapsto (\mathrm{CONS}, v_2', \loc_3')
					\\
                        \textcolor{teal}{\loc_3'} \mapsto (\mathrm{NIL}, \lambdUnit, \lambdUnit)
				\end{array}$
			&
			&
				$\lambdCall{\lambdText{map}}{\loc_1}$
			&
				$\rightarrow$
		\\ \hline
	\end{tabular}
	\captionsetup{format=hang}
	\caption{Execution of \lambd{map} in the source program and the transformed program from the heap ${\loc_1 \mapsto (\mathrm{CONS}, v_1, \loc_2) \iSep \loc_2 \mapsto (\mathrm{CONS}, v_2, \loc_3) \iSep \loc_3 \mapsto (\mathrm{NIL}, \lambdUnit, \lambdUnit)}$}
	\label{fig:heap}
\end{figure}
\clearpage

\section{Sketch of soundness proof}
\label{sec:sketch}

In this section, we give the soundness theorem for the \DataLang TMC transformation and sketch its proof.
The main idea is to use a \emph{relational separation logic} in the style of \Simuliris~\cite{DBLP:journals/pacmpl/GaherSSJDKKD22} that supports \emph{multiple calling conventions} in the transformed program.

\subsection{Soundness theorem}

The TMC transformation preserves the behaviours of programs in a strong sense.
We expect terminating programs to remain terminating, diverging programs to remain diverging and failing programs (those that get stuck) to remain failing.

To formalize it, we define in \cref{fig:refinement} the notions of behaviours and behaviour refinement as in \Simuliris~\cite{DBLP:journals/pacmpl/GaherSSJDKKD22}.
TODO

We define the set $\mathrm{behaviours}_{\datalangProg} (\datalangExpr)$ of behaviours of an expression $\datalangExpr$ within a program $\datalangProg$, starting from an empty store.
The behaviour $\constr{Conv}(\datalangVal_s)$ indicates that $\datalangExpr$ can evaluate to $\datalangVal_s$.
The behaviour $\constr{Conv}(\datalangExpr')$, when $\datalangExpr'$ is not a value, indicates that $\datalangExpr$ can reduce to a stuck configuration $(\datalangExpr', \datalangState)$.
Finally, $\constr{Div}$ indicates that $\datalangProg$ can diverge.

The refinement relation on behaviours $b_s \refined b_t$ then follows in a natural way: divergence refines divergence, any stuck expression refines any other stuck expression, and a value $\datalangVal_t$ refines $\datalangVal_s$ when they are related for a ground equivalence $\datalangVal_s \similar \datalangVal_t$, which is the equality for all value kinds, except for locations where it gives no information, as done in Simuliris.
This equivalence of ground values could be used, for example, to argue for the correctness of a \texttt{main} function that takes and returns integer arguments.

Note a difference to Simuliris: our refinement between stuck/failing behaviours is $\constr{Div} \refined \constr{Div}$, whereas Simuliris uses $\constr{Div} \refined b_t$: the Simuliris relation assumes that the input program is safe, never gets stuck, and the behaviour refinement thus allows a stuck source term to be refined by any target program. With our definition, a source program that only gets stuck must be transformed into a target program that only gets stuck.
This property would not be desirable for C compilers that want to optimize aggressively assuming the absence of undefined behaviors, but it does capture a finer-grained property of our program transformation.

Finally, an expression $\datalangExpr_t$ refines $\datalangExpr_s$ when all its behaviours are refined by a behaviour of $\datalangExpr_s$, and a program $\datalangProg_t$ refines $\datalangProg_s$ when calls to source functions on equivalent inputs are in the refinement relation.

We can now state the main soundness theorem of our work, whose proof is spread over the rest of the present \cref{sec:soundness}.

\begin{theorem}[Soundness]
    $
        \wf{\datalangProg_s} \wedge \datalangProg_s \tmc \datalangProg_t \implies
        \datalangProg_s \refined \datalangProg_t
    $
\end{theorem}

The condition $\wf{\datalangProg_s}$ guarantees that the input program $\datalangProg_s$ is well-scoped. Under this condition, we can consider the target programs $\datalangProg_t$ that are TMC transformations $\datalangProg_s \tmc \datalangProg_s$, and our theorem states that they refine $\datalangProg_s$ as expected.

\begin{figure}[tp]
    \centering
    \begin{mathparpagebreakable}
        \inferrule*
            {}{
                \datalangUnit \similar \datalangUnit
            }
        \and
        \inferrule*
            {}{
                \datalangIdx \similar \datalangIdx
            }
        \and
        \inferrule*
            {}{
                \datalangTag \similar \datalangTag
            }
        \and
        \inferrule*
            {}{
                \datalangBool \similar \datalangBool
            }
        \and
        \inferrule*
            {}{
                \datalangLoc_s \similar \datalangLoc_t
            }
        \and
        \inferrule*
            {}{
                \datalangFnptr{\datalangFn} \similar \datalangFnptr{\datalangFn}
            }
    \end{mathparpagebreakable}
    \caption{Similarity in $\Prop$}
    \label{fig:similarity}
\end{figure}
\begin{figure}[tp]
    \centering
    \begin{mathparpagebreakable}
        \inferrule*
            {
                v_s \similar v_t
            }{
                \constr[v_s]{Conv} \refined \constr[v_t]{Conv}
            }
        \and
        \inferrule*
            {
                e_s \notin \nterm{Val}
            \and
                e_t \notin \nterm{Val}
            }{
                \constr[e_s]{Conv} \refined \constr[e_t]{Conv}
            }
        \and
        \inferrule*
            {}{
                \constr{Div} \refined \constr{Div}
            }
    \end{mathparpagebreakable}
    \\
    \begin{align*}
        	\mathrm{behaviours}_p (e)
        	&\coloneqq
        	\begin{array}{l}
        			\{ \constr[e']{Conv} \mid \exists \sigma \ldotp (e, \emptyset) \step{p}^* (e', \sigma) \wedge \mathrm{irreducible}_p (e', \sigma) \}\ \uplus
        		\\
        			\{ \constr{Div} \mid\ \diverges{p}{(e, \emptyset)} \}
            \end{array}
        \\
            e_s \refined e_t
            &\coloneqq
            \forall b_t \in \mathrm{behaviours}_{p_t} (e_t) \ldotp
            \exists b_s \in \mathrm{behaviours}_{p_s} (e_s) \ldotp
            b_s \refined b_t
        \\
            p_s \refined p_t
            &\coloneqq
            \forall f \in \dom{p_t}, v_s, v_t \ldotp
            \wf{v_s} \wedge v_s \similar v_t \implies
            \lambdCall{\lambdFunc{f}}{v_s} \refined \lambdCall{\lambdFunc{f}}{v_t}
    \end{align*}
    \caption{Program refinement}
    \label{fig:refinement}
\end{figure}

\subsection{Relational separation logic}

cellules partiellement initialisées
logique de séparation (relationnelle)
bijection entre tas source et tas cible

spécifications directe/DPS

% ----------

The separation logic in which our reasoning rules are formulated is relational: it describes resources from a source and target programs and relations between them. We use the following primite assertions:

\begin{itemize}
\item $\datalangLoc_s \iPointsto_s \datalangVal_s$, a points-to assertion for the source program
\item $\datalangLoc_t \iPointsto_t \datalangVal_t$, for the target program
\item $\datalangLoc_s \iInBij \datalangLoc_t$, which asserts that the source location
  $\datalangLoc_s$ and the target location $\datalangLoc_t$ are in the ``heap
  bijection'' mentioned in \cref{TODO}. This assertion is
  persistent: one cannot remove locations from the bijection.
\end{itemize}

On top of these assertions we define a \emph{value similarity} relation $\datalangVal_s \iSimilar \datalangVal_t$, given in Figure~\ref{fig:isimilar}.  Except for locations, this is exactly the equality. For locations, notice that locations appearing during the reduction of our programs are locations of valid blocks (always of size 3 in our language). We relate them by asking their field locations to be in the heap bijection.

% ----------

\begin{figure}[tp]
	\centering
	\begin{tabular}{|c|ccc||c|ccc|}
		\hline
				\multicolumn{4}{|c||}{\bf{Source program}}
			&
				\multicolumn{4}{c|}{\bf{Transformed program}}
		\\ \hline
				\bf{Heap}
			&
				\multicolumn{3}{c||}{\bf{Position}}
			&
				\bf{Heap}
			&
				\multicolumn{3}{c|}{\bf{Position}}
		\\ \hline
				$\emptyset$
			&
				$\rightarrow$
			&
				$\datalangCall{\datalangText{map}}{\datalangLoc_1}$
			&
			&
				$\emptyset$
			&
				$\rightarrow$
			&
				$\datalangCall{\datalangText{map}}{\datalangLoc_1}$
			&
		\\ \hline
				$\emptyset$
			&
				$\rightarrow$
			&
				$\datalangCall{\datalangText{map}}{\datalangLoc_2}$
			&
			&
				$\datalangLoc_1' \iPointsto (\mathrm{CONS}, \datalangVal_1', \datalangHole)$
			&
				$\rightarrow$
			&
				$\datalangCall{\datalangText{map\_dps}}{\datalangPair{\datalangPair{\datalangLoc_1'}{\datalangTwo}}{\datalangLoc_2}}$
			&
		\\ \hline
				$\emptyset$
			&
				$\rightarrow$
			&
				$\datalangCall{\datalangText{map}}{\datalangLoc_3}$
			&
			&
				$\begin{array}{l}
						\datalangLoc_1' \iPointsto (\mathrm{CONS}, \datalangVal_1', \datalangLoc_2') \iSep {}
					\\
						\datalangLoc_2' \iPointsto (\mathrm{CONS}, \datalangVal_2', \datalangHole)
				\end{array}$
			&
				$\rightarrow$
			&
				$\datalangCall{\datalangText{map_dps}}{\datalangPair{\datalangPair{\datalangLoc_2'}{\datalangTwo}}{\datalangLoc_3}}$
			&
		\\ \hline
				$\textcolor{teal}{\datalangLoc_3'} \iPointsto (\mathrm{NIL}, \datalangUnit, \datalangUnit)$
			&
			&
				$\datalangCall{\datalangText{map}}{\datalangLoc_3}$
			&
				$\rightarrow$
			&
				$\begin{array}{l}
						\datalangLoc_1' \iPointsto (\mathrm{CONS}, \datalangVal_1', \datalangLoc_2') \iSep {}
					\\
						\datalangLoc_2' \iPointsto (\mathrm{CONS}, \datalangVal_2', \datalangLoc_3') \iSep {}
					\\
                        \textcolor{teal}{\datalangLoc_3'} \iPointsto (\mathrm{NIL}, \datalangUnit, \datalangUnit)
				\end{array}$
			&
			&
				$\datalangCall{\datalangText{map\_dps}}{\datalangPair{\datalangPair{\datalangLoc_2'}{\datalangTwo}}{\datalangLoc_3}}$
			&
				$\rightarrow$
		\\ \hline
				$\begin{array}{l}
				        \textcolor{teal}{\datalangLoc_3'} \iPointsto (\mathrm{NIL}, \datalangUnit, \datalangUnit) \iSep {}
                    \\
				        \textcolor{blue}{\datalangLoc_2'} \iPointsto (\mathrm{CONS}, \datalangVal_2', \datalangLoc_3')
				\end{array}$
			&
			&
				$\datalangCall{\datalangText{map}}{\datalangLoc_2}$
			&
				$\rightarrow$
			&
				$\begin{array}{l}
						\datalangLoc_1' \iPointsto (\mathrm{CONS}, \datalangVal_1', \datalangLoc_2') \iSep {}
					\\
						\textcolor{blue}{\datalangLoc_2'} \iPointsto (\mathrm{CONS}, \datalangVal_2', \datalangLoc_3')
					\\
                        \textcolor{teal}{\datalangLoc_3'} \iPointsto (\mathrm{NIL}, \datalangUnit, \datalangUnit)
				\end{array}$
			&
			&
				$\datalangCall{\datalangText{map\_dps}}{\datalangPair{\datalangPair{\datalangLoc_1'}{\datalangTwo}}{\datalangLoc_2}}$
			&
				$\rightarrow$
		\\ \hline
				$\begin{array}{l}
				        \textcolor{teal}{\datalangLoc_3'} \iPointsto (\mathrm{NIL}, \datalangUnit, \datalangUnit) \iSep {}
                    \\
						\textcolor{blue}{\datalangLoc_2'} \iPointsto (\mathrm{CONS}, \datalangVal_2', \datalangLoc_3') \iSep {}
					\\
						\textcolor{red}{\datalangLoc_1'} \iPointsto (\mathrm{CONS}, \datalangVal_1', \datalangLoc_2')
				\end{array}$
			&
			&
				$\datalangCall{\datalangText{map}}{\datalangLoc_1}$
			&
				$\rightarrow$
			&
				$\begin{array}{l}
						\textcolor{red}{\datalangLoc_1'} \iPointsto (\mathrm{CONS}, \datalangVal_1', \datalangLoc_2') \iSep {}
					\\
						\textcolor{blue}{\datalangLoc_2'} \iPointsto (\mathrm{CONS}, \datalangVal_2', \datalangLoc_3')
					\\
                        \textcolor{teal}{\datalangLoc_3'} \iPointsto (\mathrm{NIL}, \datalangUnit, \datalangUnit)
				\end{array}$
			&
			&
				$\datalangCall{\datalangText{map}}{\datalangLoc_1}$
			&
				$\rightarrow$
		\\ \hline
	\end{tabular}
	\captionsetup{format=hang}
	\caption{Execution of \datalang{map} in the source program and the transformed program from the heap ${\datalangLoc_1 \iPointsto (\mathrm{CONS}, \datalangVal_1, \datalangLoc_2) \iSep \datalangLoc_2 \iPointsto (\mathrm{CONS}, \datalangVal_2, \datalangLoc_3) \iSep \datalangLoc_3 \iPointsto (\mathrm{NIL}, \datalangUnit, \datalangUnit)}$}
	\label{fig:heap_bij}
\end{figure}
\begin{figure}[tp]
    \centering
    \begin{mathparpagebreakable}
        \inferrule*
            {}{
                \lambdUnit \iSimilar \lambdUnit
            }
        \and
        \inferrule*
            {}{
                i \iSimilar i
            }
        \and
        \inferrule*
            {}{
                n \iSimilar n
            }
        \and
        \inferrule*
            {}{
                b \iSimilar b
            }
        \and
        \inferrule*
            {
                \forall i \in \{0,1,2\} \ldotp
                (\loc_s + i) \iInBij (\loc_t + i)
            }{
                \loc_s \iSimilar \loc_t
            }
        \and
        \inferrule*
            {
                f \in \dom{p_s}
            }{
                \lambdFunc{f} \iSimilar \lambdFunc{f}
            }
    \end{mathparpagebreakable}
    \caption{Similarity in $\iProp$}
    \label{fig:isimilar}
\end{figure}

\subsection{Coinduction with multiple calling conventions}

coinduction pour gérer cas appels (directs et DPS)
protocoles

\begin{figure}[tp]
    \begin{tabular}{rcl}
            $\iProt_\mathrm{dir} (\iPredTwo, \datalangExpr_s, \datalangExpr_t)$
            & $\coloneqq$ &
            $\exists \datalangFn, \datalangVal_s, \datalangVal_t \ldotp$
        \\
            &&
            $\datalangFn \in \dom{\datalangProg_s} \iSep
            \datalangExpr_s = \datalangCall{\datalangFnptr{\datalangFn}}{\datalangVal_s} \iSep
            \datalangExpr_t = \datalangCall{\datalangFnptr{\datalangFn}}{\datalangVal_t} \iSep {}$
        \\
            &&
            $\datalangVal_s \iSimilar \datalangVal_t \iSep {}$
        \\
            &&
            $\forall \datalangValTwo_s, \datalangValTwo_t \ldotp
            \datalangValTwo_s \iSimilar \datalangValTwo_t \iWand
            \iPredTwo (\datalangValTwo_s, \datalangValTwo_t)$
        \\
            $\iProt_\mathrm{DPS} (\iPredTwo, \datalangExpr_s, \datalangExpr_t)$
            & $\coloneqq$ &
            $\exists \datalangFn, \datalangFn_\mathit{dps}, \datalangVal_s, \datalangLoc_1, \datalangLoc_2, \datalangLoc, \datalangIdx, \datalangVal_t \ldotp$
        \\
            &&
            $\datalangFn \in \dom{\datalangProg_s} \iSep
            \datalangRenaming [\datalangFn] = \datalangFn_\mathit{dps} \iSep
            \datalangExpr_s = \datalangCall{\datalangFnptr{\datalangFn}}{\datalangVal_s} \iSep
            \datalangExpr_t = \datalangCall{\datalangFnptr{\datalangFn_\mathit{dps}}}{\datalangLoc_1} \iSep {}$
        \\
            &&
            $(\datalangLoc_1 + 1) \iPointsto_t (\datalangLoc_2, \datalangVal_t) \iSep
            (\datalangLoc_2 + 1) \iPointsto_t (\datalangLoc, \datalangIdx) \iSep
            (\datalangLoc + \datalangIdx) \iPointsto \datalangHole \iSep
            \datalangVal_s \iSimilar \datalangVal_t$
        \\
            &&
            $\forall \datalangValTwo_s, \datalangValTwo_t \ldotp
            (\datalangLoc + \datalangIdx) \iPointsto \datalangValTwo_t \iSep
            \datalangValTwo_s \iSimilar \datalangValTwo_t \iWand
            \iPredTwo (\datalangValTwo_s, \datalangUnit)$
        \\
            $\iProt_\mathrm{TMC}$
            & $\coloneqq$ &
            $\iProt_\mathrm{dir} \sqcup \iProt_\mathrm{DPS}
            =
            \lambdaAbs (\iPredTwo, \datalangExpr_s, \datalangExpr_t) \ldotp \iProt_\mathrm{dir} (\iPredTwo, \datalangExpr_s, \datalangExpr_t) \iOr \iProt_\mathrm{DPS} (\iPredTwo, \datalangExpr_s, \datalangExpr_t)$
    \end{tabular}
    \caption{TMC protocol ($\iProt_\mathrm{TMC}$)}
    \label{fig:protocol}
\end{figure}
\clearpage

\section{Relational separation logic}
\label{sec:program_logic}

\begin{figure}[tp]
    \begin{mathparpagebreakable}
        \inferrule*[lab=RelPost]
            {
                \iPred (\datalangVal_s, \datalangVal_t)
            }{
                \iSimv[\iProt]{\iPred}{\datalangVal_s}{\datalangVal_t}
            }
        \and
        \inferrule*[lab=RelStuck]
            {
                \mathrm{strongly \mathhyphen stuck}_{\datalangProg_s} (\datalangExpr_s)
            \and
                \mathrm{strongly \mathhyphen stuck}_{\datalangProg_t} (\datalangExpr_t)
            }{
                \iSimv[\iProt]{\iPred}{\datalangExpr_s}{\datalangExpr_t}
            }
        \\
        \inferrule*[lab=RelBind]
            {
                \iSimv[\iProt]{\lambdaAbs (\datalangVal_s, \datalangVal_t) \ldotp \iSimv[\iProt]{\iPred}{\datalangEctx_s [\datalangVal_s]}{\datalangEctx_t [\datalangVal_t]}}{\datalangExpr_s}{\datalangExpr_t}
            }{
                \iSimv[\iProt]{\iPred}{\datalangEctx_s [\datalangExpr_s]}{\datalangEctx_t [\datalangExpr_t]}
            }
        \\
        \inferrule*[lab=RelSrcPure]
            {
                \datalangExpr_s \pureStep{\datalangProg_s} \datalangExpr_s'
            \and
                \iSimv[\iProt]{\iPred}{\datalangExpr_s'}{\datalangExpr_t}
            }{
                \iSimv[\iProt]{\iPred}{\datalangExpr_s}{\datalangExpr_t}
            }
        \and
        \inferrule*[lab=RelTgtPure]
            {
                \datalangExpr_t \pureStep{\datalangProg_t} \datalangExpr_t'
            \and
                \iSimv[\iProt]{\iPred}{\datalangExpr_s}{\datalangExpr_t'}
            }{
                \iSimv[\iProt]{\iPred}{\datalangExpr_s}{\datalangExpr_t}
            }
        \\
        \inferrule*[lab=RelSrcBlock1]
            {
                \iSimv[\iProt]{\iPred}{
                    \begin{array}{l}
                        \datalangLet{\datalangVar_1}{\datalangExpr_{s1}}{\\
                        \datalangLet{\datalangVar_2}{\datalangExpr_{s2}}{\\
                        \datalangBlockDet{\datalangTag}{\datalangVar_1}{\datalangVar_2}}}
                    \end{array}
                }{\datalangExpr_t}
            }{
                \iSimv[\iProt]{\iPred}{\datalangBlock{\datalangTag}{\datalangExpr_{s1}}{\datalangExpr_{s2}}}{\datalangExpr_t}
            }
        \and
        \inferrule*[lab=RelSrcBlock2]
            {
                \iSimv[\iProt]{\iPred}{
                    \begin{array}{l}
                        \datalangLet{\datalangVar_2}{\datalangExpr_{s2}}{\\
                        \datalangLet{\datalangVar_1}{\datalangExpr_{s1}}{\\
                        \datalangBlockDet{\datalangTag}{\datalangVar_1}{\datalangVar_2}}}
                    \end{array}
                }{\datalangExpr_t}
            }{
                \iSimv[\iProt]{\iPred}{\datalangBlock{\datalangTag}{\datalangExpr_{s1}}{\datalangExpr_{s2}}}{\datalangExpr_t}
            }
        \and
        \inferrule*[lab=RelTgtBlock]
            {
                \iSimv[\iProt]{\iPred}{\datalangExpr_s}{
                    \begin{array}{l}
                        \datalangLet{\datalangVar_1}{\datalangExpr_{t1}}{\\
                        \datalangLet{\datalangVar_2}{\datalangExpr_{t2}}{\\
                        \datalangBlockDet{\datalangTag}{\datalangVar_1}{\datalangVar_2}}}
                    \end{array}
                }
            \and
                \iSimv[\iProt]{\iPred}{\datalangExpr_s}{
                    \begin{array}{l}
                        \datalangLet{\datalangVar_2}{\datalangExpr_{t2}}{\\
                        \datalangLet{\datalangVar_1}{\datalangExpr_{t1}}{\\
                        \datalangBlockDet{\datalangTag}{\datalangVar_1}{\datalangVar_2}}}
                    \end{array}
                }
            }{
                \iSimv[\iProt]{\iPred}{\datalangExpr_s}{\datalangBlock{\datalangTag}{\datalangExpr_{t1}}{\datalangExpr_{t2}}}
            }
        \\
        \inferrule*[lab=RelSrcBlockDet]
            {
                \forall \datalangLoc_s \ldotp
                \datalangLoc_s \iPointsto_s (\datalangTag, \datalangVal_{s1}, \datalangVal_{s2}) \iSepImp
                \iSimv[\iProt]{\iPred}{\datalangLoc_s}{\datalangExpr_t}
            }{
                \iSimv[\iProt]{\iPred}{\datalangBlockDet{\datalangTag}{\datalangVal_{s1}}{\datalangVal_{s2}}}{\datalangExpr_t}
            }
        \and
        \inferrule*[lab=RelTgtBlockDet]
            {
                \forall \datalangLoc_t \ldotp
                \datalangLoc_t \iPointsto_t (\datalangTag, \datalangVal_{t1}, \datalangVal_{t2}) \iSepImp
                \iSimv[\iProt]{\iPred}{\datalangExpr_s}{\datalangLoc_t}
            }{
                \iSimv[\iProt]{\iPred}{\datalangExpr_s}{\datalangBlockDet{\datalangTag}{\datalangVal_{t1}}{\datalangVal_{t2}}}
            }
        \\
        \inferrule*[lab=RelSrcLoad]
            {
                (\datalangLoc_s + \datalangIdx) \iPointsto_s \datalangVal_s
            \\\\
                (\datalangLoc_s + \datalangIdx) \iPointsto_s \datalangVal_s \iSepImp
                \iSimv[\iProt]{\iPred}{\datalangVal_s}{\datalangExpr_t}
            }{
                \iSimv[\iProt]{\iPred}{\datalangLoad{\datalangLoc_s}{\datalangIdx}}{\datalangExpr_t}
            }
        \and
        \inferrule*[lab=RelSrcStore]
            {
                (\datalangLoc_s + \datalangIdx) \iPointsto_s \datalangVal_s
            \\\\
                (\datalangLoc_s + \datalangIdx) \iPointsto_s \datalangVal_s' \iSepImp
                \iSimv[\iProt]{\iPred}{\datalangUnit}{\datalangExpr_t}
            }{
                \iSimv[\iProt]{\iPred}{\datalangStore{\datalangLoc_s}{\datalangIdx}{\datalangVal_s'}}{\datalangExpr_t}
            }
        \\
        \inferrule*[lab=RelTgtLoad]
            {
                (\datalangLoc_t + \datalangIdx) \iPointsto_t \datalangVal_t
            \\\\
                (\datalangLoc_s + \datalangIdx) \iPointsto_t \datalangVal_t \iSepImp
                \iSimv[\iProt]{\iPred}{\datalangExpr_s}{\datalangVal_t}
            }{
                \iSimv[\iProt]{\iPred}{\datalangExpr_s}{\datalangLoad{\datalangLoc_t}{\datalangIdx}}
            }
        \and
        \inferrule*[lab=RelTgtStore]
            {
                (\datalangLoc_t + \datalangIdx) \iPointsto_t \datalangVal_t
                \\\\
                (\datalangLoc_t + \datalangIdx) \iPointsto_t \datalangVal_t' \iSepImp
                \iSimv[\iProt]{\iPred}{\datalangExpr_s}{\datalangUnit}
            }{
                \iSimv[\iProt]{\iPred}{\datalangExpr_s}{\datalangStore{\datalangLoc_t}{\datalangIdx}{\datalangVal_t'}}
            }
%        \\
%        \inferrule*[lab=RelBijInsert]
%            {
%                \datalangLoc_s \iPointsto_s \datalangVal_s
%            \and
%                \datalangLoc_t \iPointsto_t \datalangVal_t
%            \and
%                \datalangVal_s \iSimilar \datalangVal_t
%            \and
%                \datalangLoc_s \iInBij \datalangLoc_t \iSepImp
%                \iSimv[\iProt]{\iPred}{\datalangExpr_s}{\datalangExpr_t}
%            }{
%                \iSimv[\iProt]{\iPred}{\datalangExpr_s}{\datalangExpr_t}
%            }
        \\
        \inferrule*[lab=RelLoad]
            {
                \datalangLoc_s \iSimilar \datalangLoc_t
            \and
                \forall \datalangVal_s, \datalangVal_t \ldotp
                \datalangVal_s \iSimilar \datalangVal_t
                \iSepImp
                \iPred (\datalangVal_s, \datalangVal_t)
            }{
                \iSimv[\iProt]{\iPred}{\datalangLoad{\datalangLoc_s}{\datalangIdx}}{\datalangLoad{\datalangLoc_t}{\datalangIdx}}
            }
        \and
        \inferrule*[lab=RelStore]
            {
                \datalangLoc_s \iSimilar \datalangLoc_t
            \and
                \datalangVal_s \iSimilar \datalangVal_t
            \and
                \iPred (\datalangUnit, \datalangUnit)
            }{
                \iSimv[\iProt]{\iPred}{\datalangStore{\datalangLoc_s}{\datalangIdx}{\datalangVal_s}}{\datalangStore{\datalangLoc_t}{\datalangIdx}{\datalangVal_t}}
            }
        \\
        \inferrule*[lab=RelApplyProtocol]
            {
              \iProt (\iPredTwo, \datalangExpr_s, \datalangExpr_t)
            \and
               \forall \datalangExpr_s', \datalangExpr_t' \ldotp
               \iPredTwo (\datalangExpr_s', \datalangExpr_t') \iSepImp
               \iSimv[\iProt]{\iPred}{\datalangExpr_s'}{\datalangExpr_t'}
            }{
                \iSimv[\iProt]{\iPred}{\datalangExpr_s}{\datalangExpr_t}
            }
    \end{mathparpagebreakable}
    \caption{Reasoning rules for simulation (excerpt, omitting freshness conditions)}
    \label{fig:program_logic}
\end{figure}

In this section, we describe the inference rules of our relational program logic,
presented in \cref{fig:program_logic}.
% In the next section, we use this logic to prove the specifications.
% In section \cref{sec:simulation}, we show how it is actually defined and what notion of soundness supports it.
We omitted some standard rules for brevity.\Xgabriel{Which ones?}
Compared to the specifications of \cref{sec:specification}, we introduced an additional protocol parameter $\iProt$.
We explain it together with the \RefTirName{RelProtocol} rule in \cref{subsec:protocols}.

\Xgabriel{There are two judgments lying around, one with a precondition and one without. I would expect here some explanation about how those two judgments relate. TODO for Clément.}

\subsection{Language-independent rules}
The following rules are independent of \DataLang and could be reused as is in further works.

\RefTirName{RelPost} states that two expressions are related when they are in the relational postcondition.

\RefTirName{RelStuck} relates \emph{strongly stuck} expressions.
An expression is strongly stuck when it is stuck for any heap state.

\RefTirName{RelBind} is a standard bind rule sequencing computations on both sides.

\RefTirName{RelSrcPure} and \RefTirName{RelTgtPure} let us take pure reduction steps in either the source or target.
Pure steps (omitted for brevity) are the reduction steps that are deterministic and do not depend on the state.

\subsection{Language-specific rules: non-determinism}
$\iSimv[\iProt]{\iPred}{\datalangExpr_s}{\datalangExpr_t}$ asserts that $\datalangExpr_t$ refines $\datalangExpr_s$: any behavior of $\datalangExpr_t$ is also a behavior of $\datalangExpr_s$.
Consequently, non-determinism is treated differently in the source and target: we treat non-determinism as \emph{angelic} in source reductions and \emph{demonic} in target reductions.

Our operational semantics uses non-determinism in the reduction of constructors: $\datalangBlock{\datalangTag}{\datalangExpr_1}{\datalangExpr_2}$ reduces to $\datalangBlockDet{\datalangTag}{\datalangVar_1}{\datalangVar_2}$, where $\datalangVar_1$ and $\datalangVar_2$ are bound to $\datalangExpr_1$ and $\datalangExpr_2$ in some non-deterministic order.
In the program logic, the user may \emph{choose} an order for the source reduction, by using one of the rules \RefTirName{RelSrcBlock1} or \RefTirName{RelSrcBlock2}, and it has to prove that the expressions are related against \emph{any} target order, by proving the two premises of the rule \RefTirName{RelTgtBlock}.

\subsection{Language-specific rules: private locations.}
We can reason on points-to assertions --- that we interpret as locations private to the source or target --- in a standard way.
From a deterministic constructor $\datalangBlockDet{\datalangTag}{\datalangVal_1}{\datalangVal_2}$, we can apply \RefTirName{RelSrcBlockDet} or \RefTirName{RelTgtBlockDet}, yielding a points-to assertion for the allocated block.
The rules \RefTirName{RelSrcLoad} and \RefTirName{RelTgtLoad} let us load the pointed value while \RefTirName{RelSrcStore} and \RefTirName{RelTgtStore} let us update it with a new value.

\subsection{Language-specific rules: locations in the bijection.}
Corresponding source and target locations registered in the bijection through \RefTirName{BijInsert} have given up their respective points-to assertions but can still be accessed using the rules \RefTirName{RelLoad} and \RefTirName{RelStore}.

\RefTirName{RelLoad} states that simultaneously loading from two corresponding blocks yields similar values.

\RefTirName{RelStore} let us simultaneously store similar values into the same field of two corresponding blocks.

Together, these two rules enforce the bijection invariant: the contents of corresponding blocks are always similar.

\section{Abstract protocols} \label{sec:protocols} \label{subsec:protocols}

\begin{figure}[tp]
    \begin{tabular}{rcl}
            $\iProt_\mathrm{dir} (\iPredTwo, \datalangExpr_s, \datalangExpr_t)$
            & $\coloneqq$ &
            $\exists \datalangFn, \datalangVal_s, \datalangVal_t \ldotp$
        \\
            &&
            $\datalangFn \in \dom{\datalangProg_s} \iSep
            \datalangExpr_s = \datalangCall{\datalangFnptr{\datalangFn}}{\datalangVal_s} \iSep
            \datalangExpr_t = \datalangCall{\datalangFnptr{\datalangFn}}{\datalangVal_t} \iSep {}$
        \\
            &&
            $\datalangVal_s \iSimilar \datalangVal_t \iSep {}$
        \\
            &&
            $\forall \datalangValTwo_s, \datalangValTwo_t \ldotp
            \datalangValTwo_s \iSimilar \datalangValTwo_t \iWand
            \iPredTwo (\datalangValTwo_s, \datalangValTwo_t)$
        \\
            $\iProt_\mathrm{DPS} (\iPredTwo, \datalangExpr_s, \datalangExpr_t)$
            & $\coloneqq$ &
            $\exists \datalangFn, \datalangFn_\mathit{dps}, \datalangVal_s, \datalangLoc_1, \datalangLoc_2, \datalangLoc, \datalangIdx, \datalangVal_t \ldotp$
        \\
            &&
            $\datalangFn \in \dom{\datalangProg_s} \iSep
            \datalangRenaming [\datalangFn] = \datalangFn_\mathit{dps} \iSep
            \datalangExpr_s = \datalangCall{\datalangFnptr{\datalangFn}}{\datalangVal_s} \iSep
            \datalangExpr_t = \datalangCall{\datalangFnptr{\datalangFn_\mathit{dps}}}{\datalangLoc_1} \iSep {}$
        \\
            &&
            $(\datalangLoc_1 + 1) \iPointsto_t (\datalangLoc_2, \datalangVal_t) \iSep
            (\datalangLoc_2 + 1) \iPointsto_t (\datalangLoc, \datalangIdx) \iSep
            (\datalangLoc + \datalangIdx) \iPointsto \datalangHole \iSep
            \datalangVal_s \iSimilar \datalangVal_t$
        \\
            &&
            $\forall \datalangValTwo_s, \datalangValTwo_t \ldotp
            (\datalangLoc + \datalangIdx) \iPointsto \datalangValTwo_t \iSep
            \datalangValTwo_s \iSimilar \datalangValTwo_t \iWand
            \iPredTwo (\datalangValTwo_s, \datalangUnit)$
        \\
            $\iProt_\mathrm{TMC}$
            & $\coloneqq$ &
            $\iProt_\mathrm{dir} \sqcup \iProt_\mathrm{DPS}
            =
            \lambdaAbs (\iPredTwo, \datalangExpr_s, \datalangExpr_t) \ldotp \iProt_\mathrm{dir} (\iPredTwo, \datalangExpr_s, \datalangExpr_t) \iOr \iProt_\mathrm{DPS} (\iPredTwo, \datalangExpr_s, \datalangExpr_t)$
    \end{tabular}
    \caption{TMC protocol ($\iProt_\mathrm{TMC}$)}
    \label{fig:protocol}
\end{figure}

In \cref{sec:simulation}, we explain how our relation is defined coinductively and the first step of the proof essentially amounts to coinduction.
To internalize the coinduction hypothesis into the program logic, we introduce an additional parameter $\iProt$, a \emph{protocol}~\citep*{protocols-2021}, which are general proof-state transformers of type
\begin{mathline}
(\datalangExpr[] \to \datalangExpr[] \to \iProp) \to \datalangExpr[] \to \datalangExpr[] \to \iProp
\end{mathline}

Protocols are used in the logic via the $\RefTirName{RelProtocol}$ rule.
A pair of expressions $\datalangExpr_s$ and $\datalangExpr_t$ is supported by the protocol when it relates them to a postcondition $\iPredTwo$, capturing the possible results of an abstract/axiomatic transition from $\datalangExpr_s$ and $\datalangExpr_t$.
To conclude that $\datalangExpr_s$ and $\datalangExpr_t$ are related, one must prove that any two $\datalangExpr_s'$ and $\datalangExpr_t'$ accepted by this postcondition $\iPredTwo$ remain related.

\subsection{TMC protocols}

In our correctness proof for the TMC transformation, we use a sepecific protocol $\iProt_\mathrm{TMC}$ defined in \cref{fig:protocol} by combining two sub-protocols $\iProt_\mathrm{dir}$ and $\iProt_\mathrm{DPS}$ for the direct-style and DPS-style functions. Our coinduction hypothesis assumes toplevel function calls to be compatible with the direct and DPS specifications that we want to prove, and allows to reason about recursive calls to those functions inside the function bodies we are trying to relate.

$\iProt_\mathrm{dir}$ specifies the \emph{direct calling convention} induced by the direct transformation.
It requires $\datalangExpr_s$ and $\datalangExpr_t$ to be function calls to the same function with similar arguments.
To apply it, users can choose any postcondition implied by value similarity.
This rule is equivalent to the \TirName{Sim-Call} rule of \Simuliris, and most useful protocols are formed by combining $\iProp_\mathrm{dir}$ with other, more specialized protocols.

$\iProt_\mathrm{DPS}$ specifies the \emph{DPS calling convention} induced by the DPS transformation.
It requires $\datalangExpr_s$ to be a function call to a TMC-transformed function $\datalangFn$ and $\datalangExpr_t$ to be a function call to the DPS transform of $\datalangFn$.
As in the DPS specification, ownership of the destination must be passed to the protocol.
To apply it, users can choose any postcondition implied by the postcondition of the DPS specification, including the recovered ownership of the modified destination.

In the instantiated program logic, the application of the $\iProt_\mathrm{TMC}$ protocol is equivalent to the following rules:
\begin{mathpar}
    \inferrule*[lab=RelDir]
        {
            \datalangFn \in \dom{\datalangProg_s}
        \\\\
            \datalangVal_s \iSimilar \datalangVal_t
        \\\\
            \forall \datalangValTwo_s, \datalangValTwo_t \ldotp
            \datalangValTwo_s \iSimilar \datalangValTwo_t \iWand
            \iPred (\datalangValTwo_s, \datalangValTwo_t)
        }{
            \iSimv{
                \iPred
            }{
                \datalangCall{\datalangFnptr{\datalangFn}}{\datalangVal_s}
            }{
                \datalangCall{\datalangFnptr{\datalangFn}}{\datalangVal_t}
            }
        }
    \and
    \inferrule*[lab=RelDPS]
        {
            \datalangRenaming [\datalangFn] = \datalangFn_\mathit{dps}
        \\\\
            (\datalangLoc_1 + 1) \iPointsto_t (\datalangLoc_2, \datalangVal_t)
        \\\\
            (\datalangLoc_2 + 1) \iPointsto_t (\datalangLoc, \datalangIdx)
        \\\\
            (\datalangLoc + \datalangIdx) \iPointsto_t \datalangHole
        \\\\
            \datalangVal_s \iSimilar \datalangVal_t
        \\\\
            \forall \datalangValTwo_s, \datalangValTwo_t \ldotp
            (\datalangLoc + \datalangIdx) \iPointsto_t \datalangValTwo_t \iWand
            \datalangValTwo_s \iSimilar \datalangValTwo_t \iWand
            \iPred (\datalangValTwo_s, \datalangUnit)
        }{
            \iSimv{
                \iPred
            }{
                \datalangCall{\datalangFnptr{\datalangFn}}{\datalangVal_s}
            }{
                \datalangCall{\datalangFnptr{\datalangFn_\mathit{dps}}}{\datalangLoc_1}
            }
        }
\end{mathpar}

\subsection{Other examples of protocols}

Our program logic can be instantiated with other protocols to reason other program transformations.
To demonstrate this generality, we have also verified an inlining and an accumulator-passing-style (APS) transformation --- both included in our mechanization.

\paragraph{Inlining:} Here, the relation $\datalangExpr_s \rightsquigarrow \datalangExpr_t$ allows $\datalangExpr_t$ to recursively inline functions in $\datalangExpr_s$.
As with TMC, it captures all possible inlining strategies.

This relation can be proved correct by using a fairly simple protocol (combined with $\iProt_\mathrm{dir}$) relating a source function and its body:

\begin{tabular}{rcl}
        $\iProt_\mathrm{inline} (\iPredTwo, \datalangExpr_s, \datalangExpr_t)$
        & $\coloneqq$ &
        $\exists \datalangFn, \datalangVar, \datalangExpr_s', \datalangExpr_t', \datalangVal_s, \datalangVal_t \ldotp$
    \\
        &&
        $
        \datalangExpr_s = \datalangCall{\datalangFnptr{\datalangFn}}{\datalangVal_s}
        \ \iSep\ %
        \datalangVal_s \iSimilar \datalangVal_t
        \ \iSep {}$
    \\
        &&
        $\datalangProg_s [\datalangFn] = (\datalangRec{\datalangVar}{\datalangExpr_s'})
        \ \iSep\ %
        \datalangExpr_s' \rightsquigarrow \datalangExpr_t'
        \ \iSep\ %
        \datalangExpr_t = (\datalangLet{\datalangVar}{\datalangVal_t}{\datalangExpr_t'})
        \ \iSep {}$
    \\
        &&
        $\forall \datalangValTwo_s, \datalangValTwo_t \ldotp
        \datalangValTwo_s \iSimilar \datalangValTwo_t \iWand
        \iPredTwo (\datalangValTwo_s, \datalangValTwo_t)$
\end{tabular}
\medskip

\paragraph{Accumulator-passing style}: The APS transformation is a variant of the TMC transformation where the contexts that are made tail-recursive are applications of associative arithmetic operators, typically of the form $(\datalangExpr + \datalangCtxHole)$ or $\datalangExpr_1 + (\datalangExpr_2 \times \datalangCtxHole)$. (See the discussion in \citet*{tmc-koka-2023}.)

We defined an APS transformation, after extending \DataLang with integers and arithmetic operations. We verify it with a protocol similar to $\iProt_\mathrm{DPS}$ that allows calling the APS transform of a source function with an integer accumulator:

\medskip
\begin{tabular}{rcl}
        $\iProt_\mathrm{APS} (\iPredTwo, \datalangExpr_s, \datalangExpr_t)$
        & $\coloneqq$ &
        $\exists \datalangFn, \datalangFn_\mathit{aps}, \datalangVal_s, \datalangVal_\mathit{acc}, \datalangVal_t \ldotp$
    \\
        &&
        $\datalangFn \in \dom{\datalangProg_s}
        \ \iSep\ %
        \datalangRenaming [\datalangFn] = \datalangFn_\mathit{aps}
        \iSep {}$
    \\
        &&
        $\datalangVal_s \iSimilar \datalangVal_t
        \ \iSep\ %
        \datalangExpr_s = \datalangCall{\datalangFnptr{\datalangFn}}{\datalangVal_s}
        \ \iSep\ %
        \datalangExpr_t = \datalangCall{\datalangFnptr{\datalangFn_\mathit{aps}}}{\datalangPair{\datalangVal_\mathit{acc}}{\datalangVal_t}}
        \iSep {}$
    \\
        &&
        $\forall \datalangVal_s', \datalangExpr_t' \ldotp$
    \\
        &&
        $\bm{\mathrm{match}}\ \datalangVal_s'\ \bm{\mathrm{with}}\ \datalangNat \Rightarrow \datalangExpr_t' = \datalangVal_\mathit{acc} + \datalangNat \mid \_ \Rightarrow \mathrm{strongly \mathhyphen stuck}_{\datalangProg_t} (\datalangExpr_t')\ \bm{\mathrm{end}} \iWand$
    \\
        &&
        $\iPredTwo (\datalangVal_s', \datalangExpr_t')$
\end{tabular}
\medskip

\Xgabriel{TODO Clément: read this.}
One subtlety is that our \DataLang language is untyped, so arithmetic operations (here addition) may get stuck on non-integer values. If the function call $\datalangCall {\datalangFnptr \datalangFn} {\datalangVal_s}$ in the source program returns a non-integer value, then the outer context $\datalangVar_\mathit{acc} + \datalangCtxHole$ gets stuck. But in the transformed program, this failure happens inside the body of the APS-transformed function $\datalangFnptr {\datalangFn_\mathit{aps}}$. To represent this failure case in our protocol, the postcondition $\iPredTwo$ relates non-integer source return value $\datalangVal_s'$ with any strongly stuck expression $\datalangExpr_t'$ in the target. This relies on the generality of our protocols being predicate transformers on expressions, not just values.

%%% Local Variables:
%%% mode: latex
%%% TeX-master: "main"
%%% End:

\clearpage

\section{Soundness}
\label{sec:soundness}

(This paragraph below probably rather belongs to the introduction. Abort.)

The formal contribution of our work is a mechanized soundness proof for the TMC transformation on a simplified programming language, exhibiting the salient parts of the actual transformation for OCaml.
%
The destination-passing style used by the TMC transformation critically relies on uniquely-owned mutable state.
%
It is thus natural to use separation logic as our specification language.

\subsection{The quest for the right notion of simulation}
\label{sec:howto-relation}

The final theorem we want to establish is a behaviour refinement $p_s \refined p_t$: the behaviours of the transformed program are included in the behaviour of the source program.
%
We propose a notion of behaviour refinement that gives a form of total correctness.
%
It is termination-preserving, but also (this is less common in the body of work around Iris) safety-preserving and divergence-preserving.

We prove this using a backward simulation argument for our transformations $e_s \tmc e_t$.
%
A direct approach, ultimately unsucessful, would be to prove that the relation is in the simulation by induction, on $e_s$ for example.
%
This works well for all constructions except for function calls $\lambdCall{\lambdFunc{f}}{e}$: to reason on the relation between a call to $\lambdFunc{f}$ and its transformation (in direct or DPS style), we would need to inline the function definition, which is not structurally decreasing.

This is a common problem to establish simulations, and the solution is to use a form of co-induction.
%
In the Iris logic, a natural way to do this would be to use a ``later'' modality.
%
But defining simulations using a ``later'' modality is in fact non-trivial; simple definitions do not give the excepted notion of simulation, due to non-intuitive aspects of the step-indexing semantics of ``later'' in the Iris model. This problem is discussed in details in the previous work on Transfinite Iris~\citep*{TODO-transfinite-Iris}, an alternative logic with different axioms and models.
%
Sticking to the main Iris logic, a good definition of simulation has been worked out in Simuliris~\citep*{TODO-Simuliris}.
%
It avoids using the ``later'' modality by using the notion of coinduction native to Coq. In Simuliris, coinductive reasoning steps correspond to an ``abstract'' case in the simulation relation, used for function applications; an ``open simulation'' allows these abstract steps, while a ``closed simulation'' does not allow them.
%
The coinduction principle is encapsulated in a ``simulation closure'' theorem, which states if two terms are related in the open simulation, they are in fact also related in the closed simulation if we (coinductively) assume that foreign function bodies are correct.

Our proof is heavily inspired by the Simuliris work, but we cannot reuse their results directly because the notion of simulation that they provide is not expressive enough for our proof: we need to support transformations that change the calling conventions between source and target programs.
%
(Along the way we also simplified the simulation relation by removing proof rules related to concurrency, which we do not consider in this work.)

To support different calling conventions, we make our simulation relation $\iSim[\Chi]{\Phi}{e_s}{e_t}$ parametric over abstract protocols $\Chi$. This extends the technique of \citet*{TODO-paulo} from the unary setting of safety predicates to the binary setting of a relational separation logic.

\subsection{Proof outline}

Once we have decided to reuse and extend the Simuliris work, our argument can be split in three separate parts:

\begin{enumerate}

\item Define our notion of simulation parametric over protocols (calling conventions), prove its adequacy (it implies a behaviour refinement) and establish a ``closure theorem'' as in the Simuliris work.
%
  This technical contribution is independent of the TMC language or transformation, and could be reused in other works.

\item Build a relational program logic for our programming language, as a body of lemmas to establish simulations between expressions.

\item Use this program logic to establish the soundness of the TMC transformation.
%
The key ingredients are two specifications for the direct and DPS transformations that are proved in a mutually-inductive way, and are used to prove the hypothesis of the closure theorem of our simulation.
\end{enumerate}

The rest of this section expands each step of the proof, providing the necessary technical details to follow the Coq formalization.

\subsection{End goal: behaviour refinement}

The TMC transformation preserves the behaviour of programs in a strong sense.
%
We expect terminating programs to remain terminating, diverging programs to remain diverging and failing programs (those that get 
stuck) to remain failing.

Our definitions of behaviours and behaviour refinement are given in \fref{fig:refinement}.
%
Our notion of behaviours follows the definitions of Simuliris~\citep*{TODO-simuliris}.
%
We define the set $\mathrm{behaviours}_p(e)$ of behaviours of an expression $e$ within a program $p$, starting from an empty store.
%
The behaviour $\constr{Conv}(v_s)$ indicates that $e$ can evaluate to $v_s$.
%
The behaviour $\constr{Conv}(e')$, when $e'$ is not a value, indicates that $e$ can reduce to a stuck configuration $(e', \sigma)$.
%
Finally, $\constr{Div}$ indicates that $e$ can diverge.

The refinement relation on behaviours $b_s \refined b_t$ then follows in a natural way: divergence refines divergence, any stuck expression refines any other stuck expression, and a value $v_t$ refines $v_s$ when they are related for a ground equivalence $v_s \similar v_t$, which is the equality for all value kinds, except for locations where it gives no information, as done in Simuliris.
%
This equivalence of ground values could be used, for example, to argue for the correctness of a \texttt{main} function that takes and returns integer arguments.

Note a difference to Simuliris: our refinement between stuck/failing behaviours is $\constr{Div} \refined \constr{Div}$, whereas Simuliris uses $\constr{Div} \refined b_t$: the Simuliris relation assumes that the input program is safe, never gets stuck, and the behaviour refinement thus allows a stuck source term to be refined by any target program. With our definition, a source program that only gets stuck must be transformed into a target program that only gets stuck.
%
This property would not be desirable for C compilers that want to optimize aggressively assuming the absence of undefined behaviors, but it does capture a finer-grained property of our program transformation.

Finally, an expression $e_t$ refines $e_s$ when all its behaviours are refined by a behaviour of $e_s$, and a program $p_t$ refines $p_s$ when calls to source functions on equivalent inputs are in the refinement relation.

\begin{figure}[tp]
    \centering
    \begin{mathparpagebreakable}
        \inferrule*
            {}{
                \datalangUnit \similar \datalangUnit
            }
        \and
        \inferrule*
            {}{
                \datalangIdx \similar \datalangIdx
            }
        \and
        \inferrule*
            {}{
                \datalangTag \similar \datalangTag
            }
        \and
        \inferrule*
            {}{
                \datalangBool \similar \datalangBool
            }
        \and
        \inferrule*
            {}{
                \datalangLoc_s \similar \datalangLoc_t
            }
        \and
        \inferrule*
            {}{
                \datalangFnptr{\datalangFn} \similar \datalangFnptr{\datalangFn}
            }
    \end{mathparpagebreakable}
    \caption{Similarity in $\Prop$}
    \label{fig:similarity}
\end{figure}
\begin{figure}[tp]
    \centering
    \begin{mathparpagebreakable}
        \inferrule*
            {
                v_s \similar v_t
            }{
                \constr[v_s]{Conv} \refined \constr[v_t]{Conv}
            }
        \and
        \inferrule*
            {
                e_s \notin \nterm{Val}
            \and
                e_t \notin \nterm{Val}
            }{
                \constr[e_s]{Conv} \refined \constr[e_t]{Conv}
            }
        \and
        \inferrule*
            {}{
                \constr{Div} \refined \constr{Div}
            }
    \end{mathparpagebreakable}
    \\
    \begin{align*}
        	\mathrm{behaviours}_p (e)
        	&\coloneqq
        	\begin{array}{l}
        			\{ \constr[e']{Conv} \mid \exists \sigma \ldotp (e, \emptyset) \step{p}^* (e', \sigma) \wedge \mathrm{irreducible}_p (e', \sigma) \}\ \uplus
        		\\
        			\{ \constr{Div} \mid\ \diverges{p}{(e, \emptyset)} \}
            \end{array}
        \\
            e_s \refined e_t
            &\coloneqq
            \forall b_t \in \mathrm{behaviours}_{p_t} (e_t) \ldotp
            \exists b_s \in \mathrm{behaviours}_{p_s} (e_s) \ldotp
            b_s \refined b_t
        \\
            p_s \refined p_t
            &\coloneqq
            \forall f \in \dom{p_t}, v_s, v_t \ldotp
            \wf{v_s} \wedge v_s \similar v_t \implies
            \lambdCall{\lambdFunc{f}}{v_s} \refined \lambdCall{\lambdFunc{f}}{v_t}
    \end{align*}
    \caption{Program refinement}
    \label{fig:refinement}
\end{figure}

We can now state the main soundness theorem of our work, whose proof is spread over the rest of the present Section~\ref{sec:soundness}.

\begin{theorem}[Soundness]
    $
        \wf{p_s} \wedge p_s \tmc p_t \implies
        p_s \refined p_t
    $
\end{theorem}

The condition $\wf{p_s}$ guarantees that the input program $p_s$ is well-scoped. Under this condition, we can consider the target programs $p_t$ that are TMC transformations $p_s \tmc p_s$, and our theorem states that they refine $p_s$ as expected.

\subsection{Proof technique: Simuliris with protocols}

Simuliris~\citep*{TODO-simuliris} suggests a definition of simulation relations in Iris. As we explained in Section~\ref{sec:howto-relation}, it carefully avoids using the ``later'' modality by using Coq's native notion of coinduction. Simuliris has several desirable properties for us: it can be defined in the un-modified Iris base logic (unlike Transfite Iris~\citep*{transfinite-iris}), it is defined in a way that makes it easy to specialize to other programming languages, and it supports showing the preservation of termination. On the other hand, the original definition is complex as it supports concurrency and notions of fairness. We reused the definition of Simuliris, simplified for the sequential setting. We then extended the resulting definition to support our treatment of stuck states as failures, and generalize the notion of external calls to abstract protocols $\Chi$. Most of the ideas below come from the Simuliris work directly, we will explicitly point out the parts that differ.

The definition of the expression relation $\iSim[\Chi]{\Phi}{e_s}{e_t}$ is shown in \fref{fig:sim}; it reads as ``$e_s$ simulates $e_t$ under the postcondition $\Phi$ and protocol $\Chi$''. It expresses that the source $e_s$ can simulate any computation of the target $e_t$ until they both get stuck, both diverge, or reach two expressions in the relational post-condition $\Phi$. The parameter $\Chi$ is the relational protocol that must be followed by function calls in $e_s$ and $e_t$ -- generalizing the Simuliris rule for external calls.

As is standard in Iris, we use a \emph{state interpretation} relation $I(\sigma_s, \sigma_t)$ between source and target states. Recall that in Separation Logic, formulas contain both pure propositions and ressources, which include in particular logical assertions $\loc \mapsto_s v_s$ and $\loc \mapsto_t v_t$ on the physical states. The relation $I(\sigma_s, \sigma_t)$ enforces the consistency of those logical assertions with the physical states $\sigma_s$ and $\sigma_t$.

The definition of the relation itself starts with a double fixpoint, a greatest fixpoint $\nuAbs sim .$ (coinduction) of smallest fixpoint $\muAbs {sim \mathhyphen inner} .$ (induction) of a relation $\mathrm{sim \mathhyphen body}_\Chi$. Inside the relation definition, the cases that use the inductive $sim \mathhyphen inner$ in their recursive occurrence can only be repeated finitely many times, while the cases that use the coinductive $sim$ in their recursive occurrence can be repeated infinitely, but can only do so under a form of guardedness condition.
%
The simulation relation $\mathrm{sim \mathhyphen body}_\Chi$ can relate a source expression $e_s$ and a target expression $e_t$ if, for any source and target heaps that respect some relatedness invariant $I(\sigma_s, \sigma_t)$, .

\clearpage
\begin{figure}[tp]
    \begin{align*}
    		\mathrm{sim \mathhyphen body}_\iProt
    		&\coloneqq
    		\begin{array}{l}
    				\lambdaAbs sim \ldotp
    				\lambdaAbs sim \mathhyphen inner \ldotp
    				\lambdaAbs {(\iPred, \datalangExpr_s, \datalangExpr_t)} \ldotp
    				\forall \datalangState_s, \datalangState_s \ldotp
    				\mathrm{I} (\datalangState_s, \datalangState_t)
    				\iSepImp \iBupd
    			\\
    				\bigvee \left[ \begin{array}{ll}
    							\circled{1}
    						&
    							\mathrm{I} (\datalangState_s, \datalangState_t) \iSep
    							\iPred (\datalangExpr_s, \datalangExpr_t)
    					\\
    					        \circled{2}
                            &
                                \mathrm{I} (\datalangState_s, \datalangState_t) \iSep
    							\mathrm{strongly \mathhyphen stuck}_{\datalangProg_s} (\datalangExpr_s) \iSep
    							\mathrm{strongly \mathhyphen stuck}_{\datalangProg_t} (\datalangExpr_s)
    					\\
    							\circled{3}
    						&
    							\exists \datalangExpr_s', \datalangState_s' \ldotp
    							(\datalangExpr_s, \datalangState_s) \step{\datalangProg_s}^+ (\datalangExpr_s', \datalangState_s') \iSep
    							\mathrm{I} (\datalangState_s', \datalangState_t) \iSep
    							sim \mathhyphen inner (\iPred, \datalangExpr_s', \datalangExpr_t)
    					\\
    							\circled{4}
    						&
								\mathrm{reducible}_{\datalangProg_t} (\datalangExpr_t, \datalangState_t) \iSep
								\forall \datalangExpr_t', \datalangState_t' \ldotp
								(\datalangExpr_t, \datalangState_t) \step{\datalangProg_t} (\datalangExpr_t', \datalangState_t')
								\iSepImp \iBupd
						\\
                            &
								\bigvee \left[ \begin{array}{ll}
											\circled{A}
										&
											\mathrm{I} (\datalangState_s, \datalangState_t') \iSep
											sim \mathhyphen inner (\iPred, \datalangExpr_s, \datalangExpr_t')
									\\
											\circled{B}
										&
											\exists \datalangExpr_s', \datalangState_s' \ldotp
											(\datalangExpr_s, \datalangState_s) \step{\datalangProg_s}^+ (\datalangExpr_s', \datalangState_s') \iSep
											\mathrm{I} (\datalangState_s', \datalangState_t') \iSep
											sim (\iPred, \datalangExpr_s', \datalangExpr_t')
								\end{array} \right.
    					\\
    							\circled{5}
    						&
    							\exists \datalangEctx_s, \datalangExpr_s', \datalangEctx_t, \datalangExpr_t', \iPredTwo \ldotp
    					\\
    					    &
    					        \datalangExpr_s = \datalangEctx_s [\datalangExpr_s'] \iSep
    							\datalangExpr_t = \datalangEctx_t [\datalangExpr_t'] \iSep
    						    \iProt (\iPredTwo, \datalangExpr_s', \datalangExpr_t') \iSep
    						    \mathrm{I} (\datalangState_s, \datalangState_t) \iSep {}
    					\\
                            &
								\forall \datalangExpr_s'', \datalangExpr_t'' \ldotp
								\iPredTwo (\datalangExpr_s'', \datalangExpr_t'') \iSepImp
								sim (\iPred, \datalangEctx_s [\datalangExpr_s''], \datalangEctx_t [\datalangExpr_t''])
    				\end{array} \right.
    		\end{array}
    	\\
    	    \mathrm{sim \mathhyphen inner}_\iProt
    	    &\coloneqq
    	    \lambdaAbs sim \ldotp
    	    \muAbs sim \mathhyphen inner \ldotp
    	    \mathrm{sim \mathhyphen body}_\iProt (sim, sim \mathhyphen inner)
    	\\
    		\mathrm{sim}_\iProt
    		&\coloneqq
    		\nuAbs sim \ldotp
    		\mathrm{sim \mathhyphen inner}_\iProt (sim)
    	\\
    		\iSim[\iProt]{\iPred}{\datalangExpr_s}{\datalangExpr_t}
    		&\coloneqq
    		\mathrm{sim}_\iProt (\iPred, \datalangExpr_s, \datalangExpr_t)
    	\\
    	   \iSimv[\iProt]{\iPred}{\datalangExpr_s}{\datalangExpr_t}
    	   &\coloneqq
    	   \iSim[\iProt]{\lambdaAbs (\datalangExpr_s', \datalangExpr_t') \ldotp \exists \datalangVal_s, \datalangVal_t \ldotp \datalangExpr_s' = \datalangVal_s \iSep \datalangExpr_t' = \datalangVal_t \iSep \iPred (\datalangVal_s, \datalangVal_t)}{\datalangExpr_s}{\datalangExpr_t}
    \end{align*}
    \caption{Simulation with protocol}
    \label{fig:sim}
\end{figure}

The relation between $e_s$ and $e_t$, at a given post-condition $\Phi$, must be parametric over all physical states in the state interpretation relation $I(\sigma_s, \sigma_t)$. It is established by one of the following rules:

\begin{enumerate}
\item[\circled{1}]
\item[\circled{2}]
\item[\circled{3}]
\item[\circled{4}]
\item[\circled{5}]
\end{enumerate}

% sim-body:
% cas 1, 2, 3, 4: on raconte
% "ouverte" comprend un autre cas.
% cas 5: les appels externes, protocoles (on raconte)



TODO

\clearpage

\begin{theorem}[Close simulation]
    \begin{align*}
            &
            \iPersistent \left(
                \forall \Psi, e_s, e_t \ldotp
                \Chi (\Psi, e_s, e_t) \iSepImp
                \mathrm{sim \mathhyphen inner}_\bot (\lambdaAbs (\_, e_s', e_t') \ldotp \iSim[\Chi]{\Psi}{e_s'}{e_t'}) (\bot, e_s, e_t)
            \right) \iSepImp
        \\
            &
            \iSim[\Chi]{\Phi}{e_s}{e_t} \iSepImp
            \iSim[\bot]{\Phi}{e_s}{e_t}
    \end{align*}
\end{theorem}

\begin{theorem}[Adequacy]
    $
        \left( \vdash \iSimv{\iSimilar}{e_s}{e_t} \right) \implies
        e_s \refined e_t
    $
\end{theorem}

relational separation logic = simulation

problem: direct calling convention in Simuliris, we need two calling conventions (direct + DPS)

remedy: simulation parameterized by protocol, generalization of Simuliris (but: sequential, without source safety but closed at toplevel)

% X_tmc := X_dir \uplus X_dps

% 1. when we are done
% 2. target stuttering (inductive)
% 3. all one-step source movees
%    3a. no source change (inductive)
%    3b. at least one source reduction (coinductive / guarded)
% 4. "open" simulation; X (\Khi) characterizes admissible foreign calls

% A trick from Simuliris:
% sim-after-progress: instance of sim-inner with bottom in well-chosen places
% closed simulation theorem: if the protocol X is "closable",
%   then the simulation relation for X "collapses" into a closed simulation (bottom for X, no case 4)
%
% MYSTERY: from a high-level perspective this "closd simulation theorem" feels a bit obvious:
% its hypothesis says that your protocol X can always be replaced by making progress on both sides,
% so it should be "easy" to transform a simulation proof using 1-2-3-4 into a simulation proof using only 1-2-3.
%
% Question: if it is "easy", cannot you do this right away (in the proof of simulation for a specific X that satisfies
% this property), proving a closed simulation (without 4) directly?
%
% Clément thinks that the answer is that this would require a super
% tricky coinduction, just like the one used in the proof of this
% theorem.
%
% Clément: this theorem is a sort of coinduction principle. You can
% prove an open simulation by induction (no coinduction
% apparently required), and then applying this theorem gives a result
% for which a direct proof would require a coinduction.

DPS calling convention

TMC protocol

% The X for TMC is essentially a relation/first-order rephrasing of
% the relation specs for direct-style and destination-passing-style
% transformations.

heap bijection

% I(sigma, sigma´) is as in Simuliris, not given explicitly in the current document.
% the \approx^bij relation refers to a component of I (a bijection between addresses)
% that is an implicit parameter of the definitions.

adequacy

% If two expressions refine internally (thy are in the
% simulation relation), and go to the internal equivalence between
% values, then they refine externally (they are in the denotational
% refinement relation).

% This is similar to the Simuliris proof, simplified thanks to the absence
% of concurrency.

simulation rules

% Note: those inference rules are in fact Iris propositions
% (the separating conjunction of the premises magic-wands the conclusion)

% Pure reductions: those that do not need state (neither write nor read)

% Gabriel: we could introduc the non-deterministic pair rules into the "pure" relation
% if we changed the pure-target rule to quantify over all e'_t such that e_t -{pure}-> e'_t.
% This would please Gabriel (existential on the left, universal on the right) and be more compact,
% but Clément prefers the current presentation -- it is more standard for Iris folks -- and closer
% to the mechanization.

% SimBijInsert: \approx^bij: Simuliris calls this the escaped/public
% locations, those that have been published.

% Clément now mentions Figure 12, which shows how stores evolve on an
% example in the TMC case. The shape of the relation between the
% source and target stores is sort of implicit in the program logic
% simulations rules, but it is clear on this example.
%
% Clément would explain this to give an informal sense of the proof,
% before even talking about specification logic etc.

closing simulation

proof


\begin{theorem}[Specification of direct transformation]
    \[
        \iCsimvHoare{
            \wf{e_s} \iSep
            e_s \tmcDir{\xi} e_t
        }{
            \iSimilar
        }{
            e_s
        }{
            e_t
        }
    \]
\end{theorem}

\begin{theorem}[Specification of DPS transformation]
    \[
        \iCsimvHoare{
            \wf{e_s} \iSep
            (\loc, i, e_s) \tmcDps{\xi} e_t \iSep
            (\loc + i) \mapsto \lambdHole
        }{
            \lambdaAbs (v_s, w_t) \ldotp
            \exists v_t \ldotp
            w_t = \lambdUnit \iSep
            (\loc + i) \mapsto v_t \iSep
            v_s \iSimilar v_t
        }{
            e_s
        }{
            e_t
        }
    \]
\end{theorem}

% Those two statements are proven together.
% (First by induction on e_s, then on the two transformation relations mutually-inductively.)

% Note: the \geq here is a "parametric" simulation, we assume that all
% free variables of e_s, e_t (they must be the same) are replaced by
% externally-equivalent values.

% Taking a step back: to prove the contextual refinements between programs,
% we have to check each function,
%   @f vs refined by @f vt  (for vs externally-equiv. vt)
% for this we can use the adequacy lemma
%   f vs simulated by @f vt (for the empty protocol) into internal value equivalence
% and for this we use the simulation closure theorem
%   f vs simulated by @f vt (for the Xtmc protocol)
% assuming the Xtmc protocol is valid/adequate.
%
% But then this is trivial, just use case (4) right away.
% So the meat of the proof is in showing that the Xtmc protocol is valid.

% To prove that the Xtmc propocol is valid:
% - in the direct case, we have a function application on both sides,
%   so we make a pure step on both sides and we have to show that the function bodies
%   (with a bisubstitution (x mapsto (vs, vt)) applied) are in the relation.

\begin{figure}[tp]
    \begin{mathparpagebreakable}
        \inferrule*[lab=SimPost]
            {
                \Phi (e_s, e_t)
            }{
                \iSim[\Chi]{\Phi}{e_s}{e_t}
            }
        \and
        \inferrule*[lab=SimBind]
            {
                \iSim[\Chi]{\lambdaAbs (e_s', e_t') \ldotp \iSim[\Chi]{\Phi}{K_t [e_s']}{K_t [e_t']}}{e_s}{e_t}
            }{
                \iSim[\Chi]{\Phi}{K_t [e_s]}{K_t [e_t]}
            }
        \\
        \inferrule*[lab=SimSrcPure]
            {
                e_s \pureStep{p_s} e_s'
            \and
                \iSim[\Chi]{\Phi}{e_s'}{e_t}
            }{
                \iSim[\Chi]{\Phi}{e_s}{e_t}
            }
        \and
        \inferrule*[lab=SimTgtPure]
            {
                e_t \pureStep{p_t} e_t'
            \and
                \iSim[\Chi]{\Phi}{e_s}{e_t'}
            }{
                \iSim[\Chi]{\Phi}{e_s}{e_t}
            }
        \\
        \inferrule*[lab=SimBijInsert]
            {
                \loc_s \mapsto_s v_s
            \and
                \loc_t \mapsto_t v_t
            \and
                v_s \iSimilar v_t
            \and
                \loc_s \iInBij \loc_t \iSepImp
                \iSim[\Chi]{\Phi}{e_s}{e_t}
            }{
                \iSim[\Chi]{\Phi}{e_s}{e_t}
            }
        \\
        \inferrule*[lab=SimSrcConstr1]
            {
                \iSim[\Chi]{\Phi}{
                    \begin{array}{l}
                        \lambdLet{x_1}{e_{s1}}{\\
                        \lambdLet{x_2}{e_{s2}}{\\
                        \lambdConstrDet{t}{x_1}{x_2}}}
                    \end{array}
                }{e_t}
            }{
                \iSim[\Chi]{\Phi}{\lambdConstr{t}{e_{s1}}{e_{s2}}}{e_t}
            }
        \and
        \inferrule*[lab=SimSrcConstr2]
            {
                \iSim[\Chi]{\Phi}{
                    \begin{array}{l}
                        \lambdLet{x_2}{e_{s2}}{\\
                        \lambdLet{x_1}{e_{s1}}{\\
                        \lambdConstrDet{t}{x_1}{x_2}}}
                    \end{array}
                }{e_t}
            }{
                \iSim[\Chi]{\Phi}{\lambdConstr{t}{e_{s1}}{e_{s2}}}{e_t}
            }
        \and
        \inferrule*[lab=SimTgtConstr]
            {
                \iSim[\Chi]{\Phi}{e_s}{
                    \begin{array}{l}
                        \lambdLet{x_1}{e_{t1}}{\\
                        \lambdLet{x_2}{e_{t2}}{\\
                        \lambdConstrDet{t}{x_1}{x_2}}}
                    \end{array}
                }
            \and
                \iSim[\Chi]{\Phi}{e_s}{
                    \begin{array}{l}
                        \lambdLet{x_2}{e_{t2}}{\\
                        \lambdLet{x_1}{e_{t1}}{\\
                        \lambdConstrDet{t}{x_1}{x_2}}}
                    \end{array}
                }
            }{
                \iSim[\Chi]{\Phi}{e_s}{\lambdConstr{t}{e_{t1}}{e_{t2}}}
            }
        \\
        \inferrule*[lab=SimSrcConstrDet]
            {
                \forall \loc_s \ldotp
                \loc_s \mapsto_s (t, v_{s1}, v_{s2}) \iSepImp
                \iSim[\Chi]{\Phi}{\loc_s}{e_t}
            }{
                \iSim[\Chi]{\Phi}{\lambdConstrDet{t}{v_{s1}}{v_{s2}}}{e_t}
            }
        \and
        \inferrule*[lab=SimTgtConstrDet]
            {
                \forall \loc_t \ldotp
                \loc_t \mapsto_t (t, v_{t1}, v_{t2}) \iSepImp
                \iSim[\Chi]{\Phi}{e_s}{\loc_t}
            }{
                \iSim[\Chi]{\Phi}{e_s}{\lambdConstrDet{t}{v_{t1}}{v_{t2}}}
            }
        \\
        \inferrule*[lab=SimSrcStore]
            {
                (\loc_s + i) \mapsto_s w_s
            \and
                (\loc_s + i) \mapsto_s v_s \iSepImp
                \iSim[\Chi]{\Phi}{\lambdUnit}{e_t}
            }{
                \iSim[\Chi]{\Phi}{\lambdStore{\loc_s}{i}{v_s}}{e_t}
            }
        \and
        \inferrule*[lab=SimTgtStore]
            {
                (\loc_t + i) \mapsto_t w_t
            \and
                (\loc_t + i) \mapsto_t v_t \iSepImp
                \iSim[\Chi]{\Phi}{e_s}{\lambdUnit}
            }{
                \iSim[\Chi]{\Phi}{e_s}{\lambdStore{\loc_t}{i}{v_t}}
            }
        \\
        \inferrule*[lab=SimStore]
            {
                \loc_s \iSimilar \loc_t
            \and
                v_s \iSimilar v_t
            \and
                \Phi (\lambdUnit, \lambdUnit)
            }{
                \iSim[\Chi]{\Phi}{\lambdStore{\loc_s}{i}{v_s}}{\lambdStore{\loc_t}{i}{v_t}}
            }
    \end{mathparpagebreakable}
    \caption{Reasoning rules for simulation (excerpt, omitting freshness conditions)}
    \label{fig:sim_rules}
\end{figure}

% TODO add the rule that uses X

\begin{figure}[tp]
    \centering
    \begin{mathparpagebreakable}
        \inferrule*
            {}{
                \lambdUnit \iSimilar \lambdUnit
            }
        \and
        \inferrule*
            {}{
                i \iSimilar i
            }
        \and
        \inferrule*
            {}{
                n \iSimilar n
            }
        \and
        \inferrule*
            {}{
                b \iSimilar b
            }
        \and
        \inferrule*
            {
                \forall i \in \{0,1,2\} \ldotp
                (\loc_s + i) \iInBij (\loc_t + i)
            }{
                \loc_s \iSimilar \loc_t
            }
        \and
        \inferrule*
            {
                f \in \dom{p_s}
            }{
                \lambdFunc{f} \iSimilar \lambdFunc{f}
            }
    \end{mathparpagebreakable}
    \caption{Similarity in $\iProp$}
    \label{fig:isimilar}
\end{figure}
\begin{figure}[tp]
    \begin{tabular}{rcl}
            $\iProt_\mathrm{dir} (\iPredTwo, \datalangExpr_s, \datalangExpr_t)$
            & $\coloneqq$ &
            $\exists \datalangFn, \datalangVal_s, \datalangVal_t \ldotp$
        \\
            &&
            $\datalangFn \in \dom{\datalangProg_s} \iSep
            \datalangExpr_s = \datalangCall{\datalangFnptr{\datalangFn}}{\datalangVal_s} \iSep
            \datalangExpr_t = \datalangCall{\datalangFnptr{\datalangFn}}{\datalangVal_t} \iSep {}$
        \\
            &&
            $\datalangVal_s \iSimilar \datalangVal_t \iSep {}$
        \\
            &&
            $\forall \datalangValTwo_s, \datalangValTwo_t \ldotp
            \datalangValTwo_s \iSimilar \datalangValTwo_t \iWand
            \iPredTwo (\datalangValTwo_s, \datalangValTwo_t)$
        \\
            $\iProt_\mathrm{DPS} (\iPredTwo, \datalangExpr_s, \datalangExpr_t)$
            & $\coloneqq$ &
            $\exists \datalangFn, \datalangFn_\mathit{dps}, \datalangVal_s, \datalangLoc_1, \datalangLoc_2, \datalangLoc, \datalangIdx, \datalangVal_t \ldotp$
        \\
            &&
            $\datalangFn \in \dom{\datalangProg_s} \iSep
            \datalangRenaming [\datalangFn] = \datalangFn_\mathit{dps} \iSep
            \datalangExpr_s = \datalangCall{\datalangFnptr{\datalangFn}}{\datalangVal_s} \iSep
            \datalangExpr_t = \datalangCall{\datalangFnptr{\datalangFn_\mathit{dps}}}{\datalangLoc_1} \iSep {}$
        \\
            &&
            $(\datalangLoc_1 + 1) \iPointsto_t (\datalangLoc_2, \datalangVal_t) \iSep
            (\datalangLoc_2 + 1) \iPointsto_t (\datalangLoc, \datalangIdx) \iSep
            (\datalangLoc + \datalangIdx) \iPointsto \datalangHole \iSep
            \datalangVal_s \iSimilar \datalangVal_t$
        \\
            &&
            $\forall \datalangValTwo_s, \datalangValTwo_t \ldotp
            (\datalangLoc + \datalangIdx) \iPointsto \datalangValTwo_t \iSep
            \datalangValTwo_s \iSimilar \datalangValTwo_t \iWand
            \iPredTwo (\datalangValTwo_s, \datalangUnit)$
        \\
            $\iProt_\mathrm{TMC}$
            & $\coloneqq$ &
            $\iProt_\mathrm{dir} \sqcup \iProt_\mathrm{DPS}
            =
            \lambdaAbs (\iPredTwo, \datalangExpr_s, \datalangExpr_t) \ldotp \iProt_\mathrm{dir} (\iPredTwo, \datalangExpr_s, \datalangExpr_t) \iOr \iProt_\mathrm{DPS} (\iPredTwo, \datalangExpr_s, \datalangExpr_t)$
    \end{tabular}
    \caption{TMC protocol ($\iProt_\mathrm{TMC}$)}
    \label{fig:protocol}
\end{figure}
\begin{figure}[tp]
    \begin{align*}
    		\wf{\Gamma}
    		&\coloneqq
    		\forall x \ldotp
    		\exists v_s, v_t \ldotp
    		\Gamma (x) = (v_s, v_t) \iSep
    		v_s \iSimilar v_t
    	\\
    		\iCsimv{\Phi}{e_s}{e_t}
    		&\coloneqq
    		\forall \Gamma \ldotp
    		\wf{\Gamma} \iSepImp
    		\iSimv[\Chi_\mathrm{TMC}]{\Phi}{\Gamma_s (e_s)}{\Gamma_t (e_t)}
    	\\
    	   \iCsimvHoare{P}{\Phi}{e_s}{e_t}
    	   &\coloneqq
    	   \iPersistent \left( P \iSepImp \iCsimv{\Phi}{e_s}{e_t} \right)
    \end{align*}
    \caption{Closed simulation}
    \label{fig:csim}
\end{figure}

% François: the relational separation logic captures the right
% specification for functions. This is key to compositionality. We
% cannot do this with just a refinement relation. We should emphasize
% that this is a good approach.

%%% Local Variables:
%%% mode: latex
%%% TeX-master: "main"
%%% End:

\clearpage

\section{Simulation}

Simuliris~\citep*{TODO-simuliris} suggests a definition of simulation relations in Iris. As we explained in \cref{sec:howto-relation}, it carefully avoids using the ``later'' modality by using Coq's native notion of coinduction. Simuliris has several desirable properties for us: it can be defined in the un-modified Iris base logic (unlike Transfite Iris~\citep*{transfinite-iris}), it is defined in a way that makes it easy to specialize to other programming languages, and it supports showing the preservation of termination. On the other hand, the original definition is complex as it supports concurrency and notions of fairness. We reused the definition of Simuliris, simplified for the sequential setting. We then extended the resulting definition to support our treatment of stuck states as failures, and generalize the notion of external calls to abstract protocols $\iProt$. Most of the ideas below come from the Simuliris work directly, we will explicitly point out the parts that differ.

\paragraph{Relating states} The \emph{state interpertation} $I(\datalangState_s, \datalangState_t)$ relates source and target states. It extends the unary notion of state intereptation predicate $I(\datalangState)$. Recall that in separation logic, formulas contain both pure propositions and ressources, which include in particular logical points-to assertions $\datalangLoc \iPointsto v$. $I(\datalangState)$ enforces the consistency between those logical assertions and the physical state $\datalangState$.

In the relational setting, the state interpretation becomes a relation $I(\datalangState_s, \datalangState_t)$ tracking points-to predicates on either states $\datalangLoc_s \iPointsto_s \datalangVal_s$ and $\datalangLoc_t \iPointsto_t \datalangVal_t$. It is additionally in charge of maintaing the heap bijection mentioned in \cref{TODO} by tracking predicates $\datalangLoc_s \iInBij \datalangLoc_t$. This assertion is persistent: one cannot remove locations from the bijection.

\paragraph{Relating values} TODO relation between values.

\paragraph{Relating expressions} The definition of the expression relation $\iSim[\iProt]{\iPred}{\datalangExpr_s}{\datalangExpr_t}$ is shown in \cref{fig:sim}; it reads as ``$\datalangExpr_s$ simulates $\datalangExpr_t$ under the postcondition $\iPred$ and protocol $\iProt$''. It expresses that the source $\datalangExpr_s$ can simulate any computation of the target $\datalangExpr_t$ until they both get stuck, both diverge, or reach two expressions in the relational post-condition $\iPred$. The parameter $\iProt$ is the relational protocol that must be followed by function calls in $\datalangExpr_s$ and $\datalangExpr_t$ -- generalizing the Simuliris rule for external calls.

\paragraph{Mixed induction and coinduction}
The definition of the relation itself starts with a double fixpoint, a greatest fixpoint $\nuAbs sim .$ (coinduction) of smallest fixpoint $\muAbs {sim \mathhyphen inner} .$ (induction) of a relation $\mathrm{sim \mathhyphen body}_\iProt$. Inside the relation definition, the cases that use the inductive $sim \mathhyphen inner$ in their recursive occurrence can only be repeated finitely many times, while the cases that use the coinductive $sim$ in their recursive occurrence can be repeated infinitely, but can only do so under a form of guardedness condition.

\paragraph{Simulation clauses} The relation between $\datalangExpr_s$ and $\datalangExpr_t$, at a given post-condition $\iPred$, must be parametric over all physical states in the state interpretation relation $I(\datalangState_s, \datalangState_t)$. It is established by one of the following clauses:

\begin{enumerate}
\item[\circled{1}] Halting clause: $\datalangExpr_s$, $\datalangExpr_t$ can stop if they are already in the post-condition $\iPred$ -- and the states are still related.
\item[\circled{2}]
\item[\circled{3}]
\item[\circled{4}]
\item[\circled{5}]
\end{enumerate}

% sim-body:
% cas 1, 2, 3, 4: on raconte
% "ouverte" comprend un autre cas.
% cas 5: les appels externes, protocoles (on raconte)

\begin{figure}[tp]
    \begin{align*}
    		\mathrm{sim \mathhyphen body}_\iProt
    		&\coloneqq
    		\begin{array}{l}
    				\lambdaAbs sim \ldotp
    				\lambdaAbs sim \mathhyphen inner \ldotp
    				\lambdaAbs {(\iPred, \datalangExpr_s, \datalangExpr_t)} \ldotp
    				\forall \datalangState_s, \datalangState_s \ldotp
    				\mathrm{I} (\datalangState_s, \datalangState_t)
    				\iSepImp \iBupd
    			\\
    				\bigvee \left[ \begin{array}{ll}
    							\circled{1}
    						&
    							\mathrm{I} (\datalangState_s, \datalangState_t) \iSep
    							\iPred (\datalangExpr_s, \datalangExpr_t)
    					\\
    					        \circled{2}
                            &
                                \mathrm{I} (\datalangState_s, \datalangState_t) \iSep
    							\mathrm{strongly \mathhyphen stuck}_{\datalangProg_s} (\datalangExpr_s) \iSep
    							\mathrm{strongly \mathhyphen stuck}_{\datalangProg_t} (\datalangExpr_s)
    					\\
    							\circled{3}
    						&
    							\exists \datalangExpr_s', \datalangState_s' \ldotp
    							(\datalangExpr_s, \datalangState_s) \step{\datalangProg_s}^+ (\datalangExpr_s', \datalangState_s') \iSep
    							\mathrm{I} (\datalangState_s', \datalangState_t) \iSep
    							sim \mathhyphen inner (\iPred, \datalangExpr_s', \datalangExpr_t)
    					\\
    							\circled{4}
    						&
								\mathrm{reducible}_{\datalangProg_t} (\datalangExpr_t, \datalangState_t) \iSep
								\forall \datalangExpr_t', \datalangState_t' \ldotp
								(\datalangExpr_t, \datalangState_t) \step{\datalangProg_t} (\datalangExpr_t', \datalangState_t')
								\iSepImp \iBupd
						\\
                            &
								\bigvee \left[ \begin{array}{ll}
											\circled{A}
										&
											\mathrm{I} (\datalangState_s, \datalangState_t') \iSep
											sim \mathhyphen inner (\iPred, \datalangExpr_s, \datalangExpr_t')
									\\
											\circled{B}
										&
											\exists \datalangExpr_s', \datalangState_s' \ldotp
											(\datalangExpr_s, \datalangState_s) \step{\datalangProg_s}^+ (\datalangExpr_s', \datalangState_s') \iSep
											\mathrm{I} (\datalangState_s', \datalangState_t') \iSep
											sim (\iPred, \datalangExpr_s', \datalangExpr_t')
								\end{array} \right.
    					\\
    							\circled{5}
    						&
    							\exists \datalangEctx_s, \datalangExpr_s', \datalangEctx_t, \datalangExpr_t', \iPredTwo \ldotp
    					\\
    					    &
    					        \datalangExpr_s = \datalangEctx_s [\datalangExpr_s'] \iSep
    							\datalangExpr_t = \datalangEctx_t [\datalangExpr_t'] \iSep
    						    \iProt (\iPredTwo, \datalangExpr_s', \datalangExpr_t') \iSep
    						    \mathrm{I} (\datalangState_s, \datalangState_t) \iSep {}
    					\\
                            &
								\forall \datalangExpr_s'', \datalangExpr_t'' \ldotp
								\iPredTwo (\datalangExpr_s'', \datalangExpr_t'') \iSepImp
								sim (\iPred, \datalangEctx_s [\datalangExpr_s''], \datalangEctx_t [\datalangExpr_t''])
    				\end{array} \right.
    		\end{array}
    	\\
    	    \mathrm{sim \mathhyphen inner}_\iProt
    	    &\coloneqq
    	    \lambdaAbs sim \ldotp
    	    \muAbs sim \mathhyphen inner \ldotp
    	    \mathrm{sim \mathhyphen body}_\iProt (sim, sim \mathhyphen inner)
    	\\
    		\mathrm{sim}_\iProt
    		&\coloneqq
    		\nuAbs sim \ldotp
    		\mathrm{sim \mathhyphen inner}_\iProt (sim)
    	\\
    		\iSim[\iProt]{\iPred}{\datalangExpr_s}{\datalangExpr_t}
    		&\coloneqq
    		\mathrm{sim}_\iProt (\iPred, \datalangExpr_s, \datalangExpr_t)
    	\\
    	   \iSimv[\iProt]{\iPred}{\datalangExpr_s}{\datalangExpr_t}
    	   &\coloneqq
    	   \iSim[\iProt]{\lambdaAbs (\datalangExpr_s', \datalangExpr_t') \ldotp \exists \datalangVal_s, \datalangVal_t \ldotp \datalangExpr_s' = \datalangVal_s \iSep \datalangExpr_t' = \datalangVal_t \iSep \iPred (\datalangVal_s, \datalangVal_t)}{\datalangExpr_s}{\datalangExpr_t}
    \end{align*}
    \caption{Simulation with protocol}
    \label{fig:sim}
\end{figure}
\clearpage

\section{Related Work}

\subsection{TMC Support in Compilers}

Tail-recursion modulo cons was well-known in the Lisp community as
early as the 1970s. For example the REMREC system~\citep*{risch-73}
automatically transforms recursive functions into loops, and
supports modulo-cons tail recursion. It also supports tail-recursion
modulo associative arithmetic operators, which is outside the scope
of our work, but supported by the GCC compiler for example. The TMC
fragment is precisely described (in prose) by \citet*{friedman-wise-75}.

In the Prolog community it is a common pattern to implement destination-passing style through unification variables; in particular ``difference lists'' are a common representation of lists with a final hole.
Unification variables are first-class values: in particular they can be passed as function arguments, providing expressive, first-class support for destination-passing style in the source language.
For example, we do not support optimizing tail contexts of the form \ocaml{List.append li _?}, only direct constructor applications; this can be expressed in Prolog as just the difference list \ocaml{(List.append li X, X)} for a fresh destination variable \ocaml{X}.
But this expressiveness comes at a performance cost, and there is no static checking that the data is fully initialized at the end of computation.

\citet*{tmc-scala-2013} implement Ozma, an experimental backend for
the Scala programming language that targets the Oz virtual machine.
Oz~\citep*{oz-1994,oz-1995} is a language that integrates features
from logic- and functional-programming style, and in particular offers
pervasive ``dataflow values'', a generalization of Prolog unification
variables. Ozma brings to Scala an idiom of Oz, where Prolog-style
difference lists are used to represent potentially-infinite streams
that model synchronous concurrent agents. These lists must be
processed in constant stack space, so Ozma introduces
a tail-modulo-cons transformation in Prolog fashion: all data
constructor arguments (and some explicitly-annotated
function arguments) are transformed into dataflow values. The
motivation is expressivity, not performance, and the Ozma backend is
not competitive with the Scala JVM backend. Besides the obvious
engineering-effort differences, the pervasive use of dataflow values
may incur high constant overhead, making this approach unsuitable to
bring TMC to performance-conscious Scala users.

Independently of our work, Koka has implemented TMC starting in August
2020\footnote{\url{https://github.com/koka-lang/koka/commit/f6a343d31f486ea5edd44798dca7bca52d7b450c}}~\citep*{tmc-koka-2023}.
An interesting problem they had to solve, which does not occur in \OCaml,
is how to support TMC in presence of non-linear continuations. Our
correctness argument for TMC relies on the fact that the destination
is uniquely owned, and written exactly once; this property may not
hold in programs that use multishot continuations (\ocaml|call/cc|,
\ocaml|let/cc|, \ocaml|delim/cc|) or multishot effect handlers. The standard Koka runtime uses its
reference-counting machinery to determine that a destination is not
uniquely-owned anymore, and stores extra metadata in
partially-initialized blocks to be able to copy them on-demand in this
case. Its JavaScript back-end instead reverts to a CPS transformation
when non-linear control flow is detected.

\subsection{Reasoning About Destination-Passing-Style}

In general, if we think of non-tail recursive functions as having an ``evaluation context'' representing the continuation of the recursive call, then the techniques to turn classes of calls into tail-calls correspond to different reified representations of non-tail contexts, equipped with specific (efficient) implementations of context composition and hole-plugging.
TMC comes from representing data-construction contexts as the partial data itself, with hole-plugging by mutation.
Associative-operator transformations represent the context \ocaml|1 + (4 + _?)| as the number \ocaml|5| directly.
(Sometimes it suffices to keep around an abstraction of the context; see John Clements' work stack-based security.)

\citet*{minamide-98} gives a ``functional'' interface to destination-passing-style programs, by presenting a partial data-constructor composition \ocaml{Foo(x,Bar(_?))} as a use-once, linear-typed function \ocaml{linfun h -> Foo(x,Bar(h))}.
Those special linear functions remain implemented as partial data, but they expose a referentially-transparent interface to the programmer, restricted by a linear type discipline.
This is a beautiful way to represent destination-passing style, orthogonal to our work: users of Minamide's system would still have to write the transformed version by hand, and we could implement a transformation into destination-passing style expressed in his system.
\citet*{sobel-friedman-98}, inspired by Minamide's work, derive tree-traversal functions with a concrete representation of their continuations that implement pointer inversion to remain tail-recursive.
Mezzo~\citep*{mezzo-2016} supports a more general-purpose type system based on separation logic, which can directly express uniquely-owned partially-initialized data, and its transformation into immutable, duplicable results.
(See the \href{https://protz.github.io/mezzo/code_samples/list.mz.html}{List} module of the Mezzo standard library, and in particular \ocaml{cell}, \ocaml{freeze} and \ocaml{append} in destination-passing-style).

\subsection{Correctness Proof for TMC}

\citet*{tmc-koka-2023} provide a pen-and-paper correctness argument for TMC, or in fact a family of approaches based on optimized representations of classes of non-tail contexts, in the style of program calculation.
The clarity of their exposition is remarkable.

The implementation they prove correct, which corresponds to the approach described in \citet*{minamide-98}, results in a slightly less efficient generation where recursive calls to \ocaml{map_dps} are passed \emph{both} the start of the list and the destination to be written at its end.
The start of the list remains constant over all recursive calls, so our DPS version does not propagate it.

We were inspired by the generality of their presentation and verified that our proof technique can also be applied to some other TMC variants, by extending our mechanized development with a correctness proof for an
accumulator-passing-style transformation.

Finally, our correctness results are more precise and slightly stronger: they work in a well-typed setting where programs do not fail, whereas we use untyped terms and show preservation of failure; their proofs assume a deterministic, non-effectful language, whereas we use a more general non-deterministic, effectful language; finally, they have a very simple definition of program equivalence (reducing to the exact same value) that works well for semi-formal reasoning, but is unsuitable to scale the argument to other programming languages, whereas we use a standard notion of behavioral refinement that can scale to less idealized settings.
%
We get this extra generality mostly as a direct result of our methodology (relational program logic backed by a simulation); but showing preservation of failure requires some care when handling arithmetic operators in the accumulator-passing-style variant.

\subsection{Relational Reasoning in Separation Logic}

Defining a program logic to capture unary program properties is a typical usage of \Iris; relational properties are rarer. \citet*{tassarotti-2017} use (a linear variant of) \Iris to prove the correctness of a program transformation that implements communication channels using shared references. See the related work of ReLoC Reloaded~\citep*{reloc-2021}.
\Xfrancois{+ thèse de Friis Vindum.}
\Xfrancois{ReLoC a été utilisé surtout pour vérifier des algos concurrents.}

A relational program logic can be justified in \Iris by interpreting it as a unary relation on the target program, typically involving the \texttt{wp} predicate of the base language. This approach is inspired by CaReSL~\citep*{caresl-2013}. We follow a more direct, traditional approach of interpreting the program logic as a (binary) simulation relation (defined in the \Iris meta-logic) which is shown (adequacy) to imply a refinement between the program behaviors (denotations).

It is in fact surprisingly difficult to define simulations in \Iris, if we expect them to be adequate (to correspond to the usual notion of simulation outside the \Iris world). This is due to meta-theoretical difficulties around the ``later'' modality which led to the Transfinite \Iris variant~\citep*{transfinite-iris-2021}. We started by defining simulations in Transfinite \Iris, but later moved to the \Simuliris approach~\citep*{simuliris-2022}, where simulations are defined in standard \Iris without using the ``later'' modality, using coinduction instead (via its impredicative encoding).

As a minor technical point of comparison to \Simuliris, our definition of behaviors (denotations) includes non-termination, successfully evaluating to a value, but also failing with an error, and refinement preserves all three kind of behaviors. We do not model undefined behaviors.

We believe that our approach (relational program logics justified by a simulation) is showing promises for compiler verification. Verification of CompCert\Xgabriel{TODO citation} passes typically prove a forward simulation result, which is strengthened into a backward simulation thanks to a determinism assumption. We get the desired backward simulation directly, with compositional proofs.

\subsection{Protocols}

The function-call rule of \Simuliris only relates calls to the same function, so it is unsuitable for program transformations that also transform function definitions. We parameterize our program logic and notion of simulation on a \emph{protocol} $\iProt$, an arbitrary predicate transformer injected into the relation. This approach is reminiscent of the \emph{axiomatic semantics}~\citep*{axiomatic-semantics-2014} proposed to reason about foreign function calls. In the \Iris community, we were directly inspired by \emph{protocols} \citet*{protocols-2021}, and this approach was also reused recently, in a unary setting, by \citet*{melocoton-2023}. Our notion of protocol is slightly more general than in those two works, as it can relate arbitrary expressions in evaluation position (not just function calls), and ``return'' after an axiomatic transition with arbitrary expressions (not just values). We use this extra generality to reason about accumulator-passing-style transformation in presence of ill-typed programs -- see \cref{subsec:protocols}.

%%% Local Variables:
%%% mode: latex
%%% TeX-master: "main"
%%% End:

\clearpage

\section{Conclusion and future work}

concurrency

effect handlers

compression of constructors

prove other program transformation using this framework (CPS)
\clearpage

% Missing sections ?
% TMC in OCaml, for users
% TMC in OCaml, for implementors
% TMC in OCaml, evaluation (benchmarks + usage feedback)
%   note: OCaml 4 vs. OCaml 5

% TODO: decider si on clame "TMC dans OCaml" comme une contribution
% TODO: décider qui sont les auteurs (Gabriel ? Frédéric ?)
% TODO: indiquer qui a fait quoi
% ------------------------------------------------------------------------------

% Propositions sur la grammaire:
% - les constructeurs sont des valeurs
% - primitive d'égalité sur les constructeurs

% vu sim. sim-innerX(sim)


% unstructured TODO list

% - To give another example of simulation with a different protocol,
% we could look at accumulator-passing-style transformations
% (tail-modulo-context for associative arithmetic operators). Prove
% that `factorial` is equivalent to the tmc-optimized factorial
% (two versions, one in accumulator-passing-style and the "direct
% style" version that calls it with accumulator 1).
%    .. et donc il faut garder les entiers!
%
% Clément: j'ai commencé à bosser sur ça. Ça demande de travailler
% sur des expressions plutôt que des valeurs (l'accumulateur peut
% être un entier mais aussi une variable). Il faut regénéraliser
% des bouts du développment Coq qui était spécialisé aux
% valeurs. Il se passe un truc intéressant sur les échecs si
% l'accumulateur est une variable qui est substituée par un
% non-entier (échec pendant le calcul d'un côté, après le calcul
% de l'autre).
% La formulation où on accumule des opérations sur des expressions
% arbitraires et pas juste des constantes entières est un peu plus
% générale que celle de Daan Leijen (que des constantes). Elle permet
% de faire une transformation APS sur List.sum.

% Gabriel: should we consider adding a primitive that reduces non-deterministically to any integer?
% This can be a model of a "read from the standard input" primitive, without having to deal with I/O in the operational semantics.
% (And: it should be very easy to add as you already support non-determinism.)
%
% The context for this suggestion was to consider reducing the correctness of any function (say f : Nat -> Nat) to just
% "main" functions with no arguments, by reading the argument from the standard input instead.

\hbadness=10000
\bibliography{english}

\end{document}
