\documentclass[acmsmall,screen]{acmart}

\usepackage[T1]{fontenc}
\usepackage[utf8]{inputenc}
\usepackage[scaled=0.8]{beramono}

% ------------------------------------------------------------------------------

\usepackage{mathtools}
\usepackage{amsthm}

\usepackage{mathpartir}
\let\TirName\textsc

\let\oldexists\exists
\let\exists\relax\DeclareMathOperator{\exists}{\oldexists}
\let\oldforall\forall
\let\forall\relax\DeclareMathOperator{\forall}{\oldforall}

\DeclareMathOperator{\lambdaAbs}{\lambda}
\DeclareMathOperator{\muAbs}{\mu}
\DeclareMathOperator{\nuAbs}{\nu}

% ------------------------------------------------------------------------------

\usepackage{listings}
\usepackage{listings/ocaml}
\usepackage{listings/lambd}

% ------------------------------------------------------------------------------

\usepackage{macros}

% ------------------------------------------------------------------------------

%% Journal information
%% Supplied to authors by publisher for camera-ready submission;
%% use defaults for review submission.
\acmJournal{PACMPL}
\acmVolume{1}
\acmNumber{POPL} % CONF = POPL or ICFP or OOPSLA
\acmArticle{1}
\acmYear{2018}
\acmMonth{1}
\acmDOI{} % \acmDOI{10.1145/nnnnnnn.nnnnnnn}
\startPage{1}

%% Copyright information
%% Supplied to authors (based on authors' rights management selection;
%% see authors.acm.org) by publisher for camera-ready submission;
%% use 'none' for review submission.
\setcopyright{none}
%\setcopyright{acmcopyright}
%\setcopyright{acmlicensed}
%\setcopyright{rightsretained}
%\copyrightyear{2018}           %% If different from \acmYear

%% Bibliography style
\bibliographystyle{ACM-Reference-Format}
%% Citation style
%% Note: author/year citations are required for papers published as an
%% issue of PACMPL.
\citestyle{acmauthoryear}   %% For author/year citations

% ------------------------------------------------------------------------------
% ------------------------------------------------------------------------------

\begin{document}

\title{Verifying Tail Modulo Cons using Relational Separation Logic}

\author{Clément Allain}
\affiliation{
  \institution{Inria}
  \city{Paris}
  \country{France}
}
\email{clement.allain@inria.fr}

\author{François Pottier}
\affiliation{
  \institution{Inria}
  \city{Paris}
  \country{France}
}
\email{francois.pottier@inria.fr}

\author{Gabriel Scherer}
\affiliation{
  \institution{Inria}
  \city{Saclay}
  \country{France}
}
\email{gabriel.scherer@inria.fr}

\begin{abstract}
Common implementations of functional programming languages incentivize tail-recursive function definitions, as opposed to general recursive functions that consume stack space and may not scale to large inputs.
%
This distinction occasionally requires writing functions in a tail-recursive style that may be more complex and slower than the natural, non-tail-recursive definition.

This work describes our implementation of the \emph{tail modulo constructor} (TMC) transformation in the OCaml compiler, a compilation strategy that provides stack-efficiency for a larger class of functions -- tail-recursive \emph{modulo constructors} -- which include in particular the natural definition of \ocaml{List.map} and many similar recursive data-constructing functions.

We prove the correctness of this program transformation in a simplified setting -- a small untyped calculus -- that captures the salient aspects of the OCaml implementation. Our proof is mechanized in the Coq proof assistant, using the Iris base logic. An independent contribution of our work is a extension of the Simuliris approach to define simulation relations that support different calling conventions. To our knowledge, this is the first use of a simulation relation defined in Iris to prove the correctness of a compiler transformation.

\end{abstract}

\maketitle

\keywords{compilation, separation logic, program verification}

\begin{CCSXML}
<ccs2012>
<concept>
<concept_id>10003752.10003790.10011742</concept_id>
<concept_desc>Theory of computation~Separation logic</concept_desc>
<concept_significance>500</concept_significance>
</concept>
<concept>
<concept_id>10003752.10010124.10010138.10010142</concept_id>
<concept_desc>Theory of computation~Program verification</concept_desc>
<concept_significance>500</concept_significance>
</concept>
</ccs2012>
\end{CCSXML}

\ccsdesc[500]{Theory of computation~Separation logic}
\ccsdesc[500]{Theory of computation~Program verification}

% ------------------------------------------------------------------------------

\section{Introduction}

describe TMC

OCaml compiler

Daan Leijen

annoncer éléments décisifs (specs dir/dps) ?

contributions :
generalization of Simuliris simulation (more calling conventions)
mechanized proof of soundness in separation logic
intuitive specification (direct and DPS calling conventions)
global optimization

\begin{figure}[tp]
\begin{OCaml}
let rec map f xs =
  match xs with
  | [] ->
      []
  | x :: xs ->
      let y = f x in
      y :: map f xs
\end{OCaml}
\caption{\ocaml|map| in direct style (\OCamlLang)}
\label{fig:map}
\end{figure}
\begin{figure}[tp]
\begin{OCaml}
let rec rev_map f ys xs =
  match xs with
  | [] ->
      ys
  | x :: xs ->
      let y = f x in
      rev_map f (y :: ys) xs
      
let map f xs =
  let ys = rev_map f [] xs in
  rev ys
\end{OCaml}
\caption{\ocaml|map| in accumulator-passing style (\OCamlLang)}
\label{fig:rev_map}
\end{figure}
\begin{figure}[tp]
\begin{OCaml}
(* [dst] points to a partially initialized list cell. *)
let rec map_dps dst f xs =
  match xs with
  | [] ->
      (* Write our result, the empty list, to [dst]. *)
      set_field dst 1 []
  | x :: xs ->
      let y = f x in
      (* Allocate a new partially initialized list cell. *)
      let dst' = y :: [] in
      (* Write our result, [dst'], to [dst]. *)
      set_field dst 1 dst' ;
      (* Continue with [dst'] as new destination. *)
      map_dps dst' f xs

let map f xs =
  match xs with
  | [] ->
      []
  | x :: xs ->
      let y = f x in
      let dst = y :: [] in
      map_dps dst f xs ;
      dst
\end{OCaml}
\caption{\ocaml|map| in destination-passing style (\OCamlLang)}
\label{fig:map_dps}
\end{figure}

\section{Transformation}

Lambda language

% Functions (function pointers) as values: useful for the example of 'map'.

% 't' for tags, 'c' for constructors
% Clément prefers tags, but 't' looks like 'terms' (e) in many other works.

% Gabriel: pourquoi ne pas faire la sémantique small-step par
% interleaving pour avoir des paires non-déterministes ?
%
% Clément: en Iris la façon standard de définir des langages demande
% une décomposition contexte-terme maximale unique K[e]. Avec
% l'interleaving on a deux décompositions possibles pour (c, e1, e2):
% (c, [], e2) ou (c, e1, []). Avec mon approche on ne peut pas entrer
% à l'intérieur de (c, e1, e2), c'est un redex, donc on récupère
% l'unicité. (Ça correspond à une sémantique un poil plus faible avec
% moins d'entrelacements possibles.) Perdre l'unicité ne permettrait
% plus de prouver la règle "bind' (la règle unaire d'Iris, pas la
% règle de notre logique de programme relationnelle).

syntactic sugar

well-formedness, closed

TMC relation

example: map

\begin{figure}[tp]
    \begin{tabular}{lclcl}
            $\datalangIdx[]$
            & $\ni$ &
            $\datalangIdx$
            & $\Coloneqq$ &
            $\datalangZero \mid \datalangOne \mid \datalangTwo$
        \\
            $\datalangTag[]$
            & $\ni$ &
            $\datalangTag$
        \\
            $\datalangBool[]$
            & $\ni$ &
            $\datalangBool$
    	\\
    		$\datalangLoc[]$
    		& $\ni$ &
    		$\datalangLoc$
        \\
            $\datalangFn[]$
            & $\ni$ &
            $\datalangFn$
        \\
            $\datalangVar[]$
            & $\ni$ &
            $\datalangVar, \datalangVarTwo$
    	\\
            $\datalangVal[]$
            & $\ni$ &
            $\datalangVal$
            & $\Coloneqq$ &
            $\datalangUnit \mid \datalangIdx \mid \datalangTag \mid \datalangBool \mid \datalangLoc \mid \datalangFnptr{\datalangFn}$
        \\
            $\datalangExpr[]$
            & $\ni$ &
            $\datalangExpr$
            & $\Coloneqq$ &
            $\datalangVal$
        \\
            &&
            & | &
            $\datalangVar \mid \datalangLet{\datalangVar}{\datalangExpr_1}{\datalangExpr_2} \mid \datalangCall{\datalangExpr_1}{\datalangExpr_2}$
        \\
            &&
            & | &
            $\datalangEq{\datalangExpr_1}{\datalangExpr_2}$
        \\
            &&
            & | &
            $\datalangIf{\datalangExpr_0}{\datalangExpr_1}{\datalangExpr_2}$
        \\
            &&
            & | &
            $\datalangBlock{\datalangTag}{\datalangExpr_1}{\datalangExpr_2} \mid \datalangBlockDet{\datalangTag}{\datalangExpr_1}{\datalangExpr_2}$
        \\
            &&
            & | &
            $\datalangLoad{\datalangExpr_1}{\datalangExpr_2} \mid \datalangStore{\datalangExpr_1}{\datalangExpr_2}{\datalangExpr_3}$
        \\
            $\datalangEctx[]$
            & $\ni$ &
            $\datalangEctx$
            & $\Coloneqq$ &
            $\Box$
        \\
            &&
            & | &
            $\datalangLet{\datalangVar}{\datalangEctx}{\datalangExpr_2} \mid \datalangCall{\datalangExpr_1}{\datalangEctx} \mid \datalangCall{\datalangEctx}{\datalangVal_2}$
        \\
            &&
            & | &
            $\datalangEq{\datalangExpr_1}{\datalangEctx} \mid \datalangEq{\datalangEctx}{\datalangVal_2}$
        \\
            &&
            & | &
            $\datalangIf{\datalangEctx}{\datalangExpr_1}{\datalangExpr_2}$
        \\
            &&
            & | &
            $\datalangLoad{\datalangExpr_1}{\datalangEctx} \mid \datalangLoad{\datalangEctx}{\datalangVal_2}$
        \\
            &&
            & | &
            $\datalangStore{\datalangExpr_1}{\datalangExpr_2}{K} \mid \datalangStore{\datalangExpr_1}{\datalangEctx}{\datalangVal_3} \mid \datalangStore{\datalangEctx}{\datalangVal_2}{\datalangVal_3}$
        \\
            $\datalangDef[]$
            & $\ni$ &
            $\datalangDef$
            & $\Coloneqq$ &
            $\datalangRec{\datalangVar}{\datalangExpr}$
        \\
            $\datalangProg[]$
            & $\ni$ &
            $\datalangProg$
            & $\coloneqq$ &
            $\datalangFn[] \finmap \datalangDef[]$
        \\
            $\datalangState[]$
            & $\ni$ &
            $\datalangState$    
            & $\coloneqq$ &
            $\datalangLoc[] \finmap \datalangVal[]$
        \\
            $\datalangConfig[]$
            & $\ni$ &
            $\datalangConfig$
            & $\coloneqq$ &
            $\datalangExpr[] \times \datalangState[]$
    \end{tabular}
    \caption{\DataLang syntax}
    \label{fig:syntax}
\end{figure}
\begin{figure}[tp]
    \begin{tabular}{rcl}
            $\datalangSeq{\datalangExpr_1}{\datalangExpr_2}$
            & $\coloneqq$ &
            $\datalangLet{\datalangVar}{\datalangExpr_1}{\datalangExpr_2}$
        \\
            $\datalangPair{\datalangExpr_1}{\datalangExpr_2}$
            & $\coloneqq$ &
            $\datalangBlock{\mathrm{PAIR}}{\datalangExpr_1}{\datalangExpr_2}$
        \\
            $\datalangLet{\datalangPair{\datalangVar_1}{\datalangVar_2}}{\datalangExpr_1}{\datalangExpr_2}$
            & $\coloneqq$ &
            $\datalangLet{\datalangVarTwo}{\datalangExpr_1}{$
        \\
            &&
            $\datalangLet{\datalangVar_1}{\datalangLoad{\datalangVarTwo}{1}}{$
        \\
            &&
            $\datalangLet{\datalangVar_2}{\datalangLoad{\datalangVarTwo}{2}}{$
        \\
            &&
            $\datalangExpr_2}}}$
    \end{tabular}~%
    \begin{tabular}{rcl}
            $\datalangRec{\datalangPair{\datalangVar_1}{\datalangVar_2}}{\datalangExpr}$
            & $\coloneqq$ &
            $\datalangRec{\datalangVarTwo}{
            \datalangLet{\datalangPair{\datalangVar_1}{\datalangVar_2}}{\datalangVarTwo}{\datalangExpr}}$
        \\
            $\datalangNil$
            & $\coloneqq$ &
            $\datalangUnit$
        \\
            $\datalangCons{\datalangExpr_1}{\datalangExpr_2}$
            & $\coloneqq$ &
            $\datalangBlock{\mathrm{CONS}}{\datalangExpr_1}{\datalangExpr_2}$
    \end{tabular}

    \begin{tabular}{rcl}
            $\datalangMatchOneline[]{\datalangExpr_0}{\datalangExpr_1}{\datalangVar}{\mathit{\datalangVar s}}{\datalangExpr_2}$
            & $\coloneqq$ &
            $\datalangLet{\datalangVarTwo}{\datalangExpr_0}{$
        \\
            &&
            $\datalangIf{\datalangEq{\datalangVarTwo}{\datalangNil}$
        \\
            &&
            $}{\datalangExpr_1$
        \\
            &&
            $}{\datalangLet{\datalangPair{\datalangVar}{\mathit{\datalangVar s}}}{\datalangVarTwo}{\datalangExpr_2}}}$
        \\
            $\datalangHole$
            & $\coloneqq$ &
            $\datalangUnit$
    \end{tabular}
    \caption{\DataLang syntactic sugar}
    \label{fig:sugar}
\end{figure}
\begin{figure}[tp]
    \begin{mathparpagebreakable}
        \inferrule*
            {}{
                \boxed{- \headStep{\datalangProg} - : \datalangConfig[] \rightarrow \datalangConfig[] \rightarrow \Prop}
            }
        \and
        \inferrule*
            {}{
                \boxed{- \step{\datalangProg} - : \datalangConfig[] \rightarrow \datalangConfig[] \rightarrow \Prop}
            }
        \\
        \inferrule*[lab=StepLet]
            {}{
                \left(
                    \datalangLet{\datalangVar}{\datalangVal}{\datalangExpr},
                    \datalangState
                \right)
                \headStep{\datalangProg}
                \left(
                    \datalangExpr{} [\datalangVar \backslash \datalangVal],
                    \datalangState
                \right)
            }
        \and
        \inferrule*[lab=StepCall]
            {
                \datalangProg{} [\datalangFn] = (\datalangRec{\datalangVar}{\datalangExpr})
            }{
                \left(
                    \datalangCall{\datalangFnptr{\datalangFn}}{\datalangVal},
                    \datalangState
                \right)
                \headStep{\datalangProg}
                \left(
                    \datalangExpr{} [\datalangVar \backslash \datalangVal],
                    \datalangState
                \right)
            }
        \\
        \inferrule*[lab=StepBlock1]
            {}{
                \left(
                    \datalangBlock{\datalangTag}{\datalangExpr_1}{\datalangExpr_2},
                    \datalangState
                \right)
                \headStep{\datalangProg}
                \left(
                    \begin{array}{l}
                        \datalangLet{\datalangVar_1}{\datalangExpr_1}{\\
                        \datalangLet{\datalangVar_2}{\datalangExpr_2}{\\
                        \datalangBlockDet{\datalangTag}{\datalangVar_1}{\datalangVar_2}}}
                    \end{array},
                    \datalangState
                \right)
            }
        \and
        \inferrule*[lab=StepBlock2]
            {}{
                \left(
                    \datalangBlock{\datalangTag}{\datalangExpr_1}{\datalangExpr_2},
                    \datalangState
                \right)
                \headStep{\datalangProg}
                \left(
                    \begin{array}{l}
                        \datalangLet{\datalangVar_2}{\datalangExpr_2}{\\
                        \datalangLet{\datalangVar_1}{\datalangExpr_1}{\\
                        \datalangBlockDet{\datalangTag}{\datalangVar_1}{\datalangVar_2}}}
                    \end{array},
                    \datalangState
                \right)
            }
        \\
        \inferrule*[lab=StepBlockDet]
            {
                \forall \datalangIdx \in \datalangIdx[], \datalangLoc + \datalangIdx \notin \dom{\datalangState}
            }{
                \left(
                    \datalangBlockDet{\datalangTag}{\datalangVal_1}{\datalangVal_2},
                    \datalangState
                \right)
                \headStep{\datalangProg}
                \left(
                    \datalangLoc,
                    \datalangState{} [\datalangLoc \mapsto \datalangTag, \datalangVal_1, \datalangVal_2]
                \right)
            }
        \and
        \inferrule*[lab=StepLoad]
            {
                \datalangState{} [\datalangLoc + \datalangIdx] = \datalangVal
            }{
                \left(
                    \datalangLoad{\datalangLoc}{\datalangIdx},
                    \datalangState
                \right)
                \headStep{\datalangProg}
                \left(
                    \datalangVal,
                    \datalangState
                \right)
            }
        \and
        \inferrule*[lab=StepStore]
            {
                \datalangLoc + \datalangIdx \in \dom{\datalangState}
            }{
                \left(
                    \datalangStore{\datalangLoc}{\datalangIdx}{\datalangVal},
                    \datalangState
                \right)
                \headStep{\datalangProg}
                \left(
                    \datalangUnit,
                    \datalangState{} [\datalangLoc + \datalangIdx \mapsto \datalangVal]
                \right)
            }
        \and
        \inferrule*[lab=StepEctx]
            {
                \left(
                    \datalangExpr,
                    \datalangState
                \right)
                \headStep{\datalangProg}
                \left(
                    \datalangExpr',
                    \datalangState'
                \right)
            }{
                \left(
                    \datalangEctx{} [\datalangExpr],
                    \datalangState
                \right)
                \step{\datalangProg}
                \left(
                    \datalangEctx{} [\datalangExpr'],
                    \datalangState'
                \right)
            }
    \end{mathparpagebreakable}
    \caption{\DataLang semantics (excerpt)}
    \label{fig:semantics}
\end{figure}
\begin{figure}[tp]
    \begin{mathparpagebreakable}
        \inferrule*
            {}{
                \boxed{- \tmcDir{\datalangRenaming} - : \datalangExpr[] \rightarrow \datalangExpr[] \rightarrow \Prop}
            }
        \and
        \inferrule*
            {}{
                \boxed{- \tmcDir{\datalangRenaming} - : \datalangDef[] \rightarrow \datalangDef[] \rightarrow \Prop}
            }
        \\
        \inferrule*[lab=DirVal]
            {}{
                \datalangVal
                \tmcDir{\datalangRenaming}
                \datalangVal
            }
        \and
        \inferrule*[lab=DirVar]
            {}{
                \datalangVar
                \tmcDir{\datalangRenaming}
                \datalangVar
            }
        \and
        \inferrule*[lab=DirLet]
            {
                \datalangExpr_{s1} \tmcDir{\datalangRenaming} \datalangExpr_{t1}
            \and
                \datalangExpr_{s2} \tmcDir{\datalangRenaming} \datalangExpr_{t2}
            }{
                \datalangLet{\datalangVar}{\datalangExpr_{s1}}{\datalangExpr_{s2}}
                \tmcDir{\datalangRenaming}
                \datalangLet{\datalangVar}{\datalangExpr_{t1}}{\datalangExpr_{t2}}
            }
        \\
        \inferrule*[lab=DirBlockDPS1]
            {
                \datalangExpr_{s1} \tmcDir{\datalangRenaming} \datalangExpr_{t1}
            \and
                (\datalangVar, \datalangTwo, \datalangExpr_{s2}) \tmcDps{\datalangRenaming} \datalangExpr_{t2}
            }{
                \datalangBlock{\datalangTag}{\datalangExpr_{s1}}{\datalangExpr_{s2}}
                \tmcDir{\datalangRenaming}
                \begin{array}{l}
                    \datalangLet{\datalangVar}{\datalangBlock{\datalangTag}{\datalangExpr_{t1}}{\datalangHole}}{\\
                    \datalangSeq{\datalangExpr_{t2}}{\datalangVar}}
                \end{array}
            }
        \and
        \inferrule*[lab=DirBlockDPS2]
            {
                \datalangExpr_{s2} \tmcDir{\datalangRenaming} \datalangExpr_{t2}
            \and
                (\datalangVar, \datalangOne, \datalangExpr_{s1}) \tmcDps{\datalangRenaming} \datalangExpr_{t1}
            }{
                \datalangBlock{\datalangTag}{\datalangExpr_{s1}}{\datalangExpr_{s2}}
                \tmcDir{\datalangRenaming}
                \begin{array}{l}
                    \datalangLet{\datalangVar}{\datalangBlock{\datalangTag}{\datalangHole}{\datalangExpr_{t2}}}{\\
                    \datalangSeq{\datalangExpr_{t1}}{\datalangVar}}
                \end{array}
            }
        \\
        \inferrule*[lab=DirDef]
            {
                \datalangExpr_s \tmcDir{\datalangRenaming} \datalangExpr_t
            }{
                \datalangRec{\datalangVar}{\datalangExpr_s}
                \tmcDir{\datalangRenaming}
                \datalangRec{\datalangVar}{\datalangExpr_t}
            }
    \end{mathparpagebreakable}
    \caption{Direct TMC transformation (omitting congruence rules similar to \TirName{DirLet} and freshness conditions)}
    \label{fig:tmc_dir}
\end{figure}
\begin{figure}[tp]
    \begin{mathparpagebreakable}
        \inferrule*
            {}{
                \boxed{- \tmcDps{\xi} - : \nterm{Expr} \times \nterm{Expr} \times \nterm{Expr} \rightarrow \nterm{Expr} \rightarrow \Prop}
            }
        \and
        \inferrule*
            {}{
                \boxed{- \tmcDps{\xi} - : \nterm{Def} \rightarrow \nterm{Def} \rightarrow \Prop}
            }
        \\
        \inferrule*[lab=DPSBase]
            {
                e_s \tmcDir{\xi} e_t
            }{
                \left(
                    \mathit{dst},
                    \mathit{idx},
                    e_s
                \right)
                \tmcDps{\xi}
                \lambdStore{\mathit{dst}}{i}{e_t}
            }
        \and
        \inferrule*[lab=DPSLet]
            {
                e_{s1} \tmcDir{\xi} e_{t1}
            \and
                (\mathit{dst}, \mathit{idx}, e_{s2}) \tmcDps{\xi} e_{t2}
            }{
                \left(
                    \mathit{dst},
                    \mathit{idx},
                    \lambdLet{x}{e_{s1}}{e_{s2}}
                \right)
                \tmcDps{\xi}
                \lambdLet{x}{e_{t1}}{e_{t2}}
            }
        \and
        \inferrule*[lab=DPSCall]
            {
                f \in \dom{\xi}
            \and
                e_s \tmcDir{\xi} e_t
            }{
                \left(
                    \mathit{dst},
                    \mathit{idx},
                    \lambdCall{\lambdFunc{f}}{e_s}
                \right)
                \tmcDps{\xi}
                \lambdCall{\lambdFunc{\xi [f]}}{e_t}
            }
        \and
        \inferrule*[lab=DPSIf]
            {
                e_{s0} \tmcDir{\xi} e_{t0}
            \and
                (\mathit{dst}, \mathit{idx}, e_{s1}) \tmcDps{\xi} e_{s2}
            \and
                (\mathit{dst}, \mathit{idx}, e_{s2}) \tmcDps{\xi} e_{t2}
            }{
                \left(
                    \mathit{dst},
                    \mathit{idx},
                    \lambdIf{e_{s0}}{e_{s1}}{e_{s2}}
                \right)
                \tmcDps{\xi}
                \lambdIf{e_{t0}}{e_{t1}}{e_{t2}}
            }
        \\
        \inferrule*[lab=DPSConstr1]
            {
                e_{s1} \tmcDir{\xi} e_{t1}
            \and
                (\mathit{dst}, \lambdTwo, e_{s2}) \tmcDir{\xi} e_{t2}
            }{
                \left(
                    \mathit{dst},
                    \mathit{idx},
                    \lambdConstr{c}{e_{s1}}{e_{s2}}
                \right)
                \tmcDps{\xi}
                \begin{array}{l}
                    \lambdLet{\mathit{dst'}}{\lambdConstr{c}{e_{t1}}{\lambdHole}}{\\
                    \lambdSeq{\lambdStore{\mathit{dst}}{\mathit{idx}}{\mathit{dst'}}}{e_{t2}}
                    }
                \end{array}
            }
        \and
        \inferrule*[lab=DPSConstr2]
            {
                e_{s2} \tmcDir{\xi} e_{t2}
            \and
                (\mathit{dst}, \lambdOne, e_{s1}) \tmcDir{\xi} e_{t1}
            }{
                \left(
                    \mathit{dst},
                    \mathit{idx},
                    \lambdConstr{c}{e_{s1}}{e_{s2}}
                \right)
                \tmcDps{\xi}
                \begin{array}{l}
                    \lambdLet{\mathit{dst'}}{\lambdConstr{c}{\lambdHole}{e_{t2}}}{\\
                    \lambdSeq{\lambdStore{\mathit{dst}}{\mathit{idx}}{\mathit{dst'}}}{e_{t1}}
                    }
                \end{array}
            }
        \\
        \inferrule*[lab=DPSDef]
            {
                (\mathit{dst}, \mathit{idx}, e_s) \tmcDps{\xi} e_t
            }{
                \lambdDef{x}{e_s}
                \tmcDps{\xi}
                {\begin{array}{l}
                    \lambdDef{y}{\\ \ \begin{array}{l}
                        \lambdLet{z}{\lambdLoad{y}{\lambdOne}}{\\
                        \lambdLet{\mathit{dst}}{\lambdLoad{z}{\lambdOne}}{\\
                        \lambdLet{\mathit{idx}}{\lambdLoad{z}{\lambdTwo}}{\\
                        \lambdLet{x}{\lambdLoad{y}{\lambdTwo}}{\\
                        e_t
                        }}}}
                    \end{array}}
                \end{array}}
            }
    \end{mathparpagebreakable}
    \caption{Destination-passing style TMC transformation (omitting freshness conditions)}
    \label{fig:tmc_dps}
\end{figure}
\begin{figure}[tp]
    \[
        \datalangProg_s \tmc \datalangProg_t
        \coloneqq
        \exists \datalangRenaming \ldotp
        \bigwedge \left[ \begin{array}{l}
                \dom{\datalangRenaming} \subseteq \dom{\datalangProg_s}
            \\
                \dom{\datalangProg_t} = \dom{\datalangProg_s} \cup \codom{\datalangRenaming}
            \\
                \forall \datalangFn, \datalangDef_s \ldotp
                \datalangProg_s [\datalangFn] = \datalangDef_s \implies
                \exists \datalangDef_t \ldotp
                \datalangProg_t [\datalangFn] = \datalangDef_t \wedge \datalangDef_s \tmcDir{\datalangRenaming} \datalangDef_t
            \\
                \forall \datalangFn, \datalangFn_\mathit{dps}, \datalangDef_s \ldotp
                \datalangProg_s [\datalangFn] = \datalangDef_s \wedge \datalangRenaming [\datalangFn] = \datalangFn_{\mathit{dps}} \implies
                \exists \datalangDef_t \ldotp
                \datalangProg_t [\datalangFn_\mathit{dps}] = \datalangDef_t \wedge \datalangDef_s \tmcDps{\datalangRenaming} \datalangDef_t
        \end{array} \right.
    \]
    \caption{TMC transformation}
    \label{fig:tmc}
\end{figure}
\begin{figure}[tp]
\begin{tabular}{c}
\begin{Lambd}
map |-> rec arg =
  let fn = arg.(1) in
  let xs = arg.(2) in
  match xs with
  | [] ==> 
      []
  | x :: xs ==>
      let y = fn x in
      y :: (@map (fn, xs))
\end{Lambd}
\end{tabular}
\caption{\lambd|map| in direct style (\LambdaLang)}
\label{fig:lambda_map}
\end{figure}
\begin{figure}[tp]
\begin{minipage}{.45\columnwidth}
\begin{Lambd}
map |-> rec arg =
  let fn = arg.(1) in
  let xs = arg.(2) in
  match xs with
  | [] ==> 
      []
  | x :: xs ==>
      let y = fn x in
      let dst = y :: () in
      @map_dps ((dst, 2), (fn, xs)) ;
      dst
\end{Lambd}
\end{minipage}
\hfill
\begin{minipage}{.45\columnwidth}
\begin{Lambd}
map_dps |-> rec arg' =
  let dst_idx = arg'.(1) in
  let dst = dst_idx.(1) in
  let idx = dst_idx.(2) in
  let arg = arg'.(2) in
  let fn = arg.(1) in
  let xs = arg.(2) in
  match xs with
  | [] ==> 
      dst.(idx) <- []
  | x :: xs ==>
      let y = fn x in
      let dst' = y :: () in
      dst.(idx) <- dst' ;
      @map_dps ((dst', 2), (fn, xs))
\end{Lambd}
\end{minipage}
\caption{\lambd|map| in destination-passing style (\LambdaLang)}
\label{fig:lambda_map_dps}
\end{figure}
\begin{figure}[tp]
	\centering
	\begin{tabular}{|c|ccc||c|ccc|}
		\hline
				\multicolumn{4}{|c||}{\bf{Source program}}
			&
				\multicolumn{4}{c|}{\bf{Transformed program}}
		\\ \hline
				\bf{Heap}
			&
				\multicolumn{3}{c||}{\bf{Position}}
			&
				\bf{Heap}
			&
				\multicolumn{3}{c|}{\bf{Position}}
		\\ \hline
				$\emptyset$
			&
				$\rightarrow$
			&
				$\lambdCall{\lambdText{map}}{\loc_1}$
			&
			&
				$\emptyset$
			&
				$\rightarrow$
			&
				$\lambdCall{\lambdText{map}}{\loc_1}$
			&
		\\ \hline
				$\emptyset$
			&
				$\rightarrow$
			&
				$\lambdCall{\lambdText{map}}{\loc_2}$
			&
			&
				$\loc_1' \mapsto (\mathrm{CONS}, v_1', \lambdHole)$
			&
				$\rightarrow$
			&
				$\lambdCall{\lambdText{map\_dps}}{\lambdPair{\lambdPair{\loc_1'}{\lambdTwo}}{\loc_2}}$
			&
		\\ \hline
				$\emptyset$
			&
				$\rightarrow$
			&
				$\lambdCall{\lambdText{map}}{\loc_3}$
			&
			&
				$\begin{array}{l}
						\loc_1' \mapsto (\mathrm{CONS}, v_1', \loc_2') \iSep {}
					\\
						\loc_2' \mapsto (\mathrm{CONS}, v_2', \lambdHole)
				\end{array}$
			&
				$\rightarrow$
			&
				$\lambdCall{\lambdText{map_dps}}{\lambdPair{\lambdPair{\loc_2'}{\lambdTwo}}{\loc_3}}$
			&
		\\ \hline
				$\textcolor{teal}{\loc_3'} \mapsto (\mathrm{NIL}, \lambdUnit, \lambdUnit)$
			&
			&
				$\lambdCall{\lambdText{map}}{\loc_3}$
			&
				$\rightarrow$
			&
				$\begin{array}{l}
						\loc_1' \mapsto (\mathrm{CONS}, v_1', \loc_2') \iSep {}
					\\
						\loc_2' \mapsto (\mathrm{CONS}, v_2', \loc_3') \iSep {}
					\\
                        \textcolor{teal}{\loc_3'} \mapsto (\mathrm{NIL}, \lambdUnit, \lambdUnit)
				\end{array}$
			&
			&
				$\lambdCall{\lambdText{map\_dps}}{\lambdPair{\lambdPair{\loc_2'}{\lambdTwo}}{\loc_3}}$
			&
				$\rightarrow$
		\\ \hline
				$\begin{array}{l}
				        \textcolor{teal}{\loc_3'} \mapsto (\mathrm{NIL}, \lambdUnit, \lambdUnit) \iSep {}
                    \\
				        \textcolor{blue}{\loc_2'} \mapsto (\mathrm{CONS}, v_2', \loc_3')
				\end{array}$
			&
			&
				$\lambdCall{\lambdText{map}}{\loc_2}$
			&
				$\rightarrow$
			&
				$\begin{array}{l}
						\loc_1' \mapsto (\mathrm{CONS}, v_1', \loc_2') \iSep {}
					\\
						\textcolor{blue}{\loc_2'} \mapsto (\mathrm{CONS}, v_2', \loc_3')
					\\
                        \textcolor{teal}{\loc_3'} \mapsto (\mathrm{NIL}, \lambdUnit, \lambdUnit)
				\end{array}$
			&
			&
				$\lambdCall{\lambdText{map\_dps}}{\lambdPair{\lambdPair{\loc_1'}{\lambdTwo}}{\loc_2}}$
			&
				$\rightarrow$
		\\ \hline
				$\begin{array}{l}
				        \textcolor{teal}{\loc_3'} \mapsto (\mathrm{NIL}, \lambdUnit, \lambdUnit) \iSep {}
                    \\
						\textcolor{blue}{\loc_2'} \mapsto (\mathrm{CONS}, v_2', \loc_3') \iSep {}
					\\
						\textcolor{red}{\loc_1'} \mapsto (\mathrm{CONS}, v_1', \loc_2')
				\end{array}$
			&
			&
				$\lambdCall{\lambdText{map}}{\loc_1}$
			&
				$\rightarrow$
			&
				$\begin{array}{l}
						\textcolor{red}{\loc_1'} \mapsto (\mathrm{CONS}, v_1', \loc_2') \iSep {}
					\\
						\textcolor{blue}{\loc_2'} \mapsto (\mathrm{CONS}, v_2', \loc_3')
					\\
                        \textcolor{teal}{\loc_3'} \mapsto (\mathrm{NIL}, \lambdUnit, \lambdUnit)
				\end{array}$
			&
			&
				$\lambdCall{\lambdText{map}}{\loc_1}$
			&
				$\rightarrow$
		\\ \hline
	\end{tabular}
	\captionsetup{format=hang}
	\caption{Execution of \lambd{map} in the source program and the transformed program from the heap ${\loc_1 \mapsto (\mathrm{CONS}, v_1, \loc_2) \iSep \loc_2 \mapsto (\mathrm{CONS}, v_2, \loc_3) \iSep \loc_3 \mapsto (\mathrm{NIL}, \lambdUnit, \lambdUnit)}$}
	\label{fig:heap}
\end{figure}
\section{Soundness proof}

The proof technique we propose to verify program transformations such
as TMC is to capture a relational specification between inputs and
outputs of the translation using a relational separation logic.

We use a quadruple of the form $\iSim[\iProt]{\iPred}{\datalangExpr_s}{\datalangExpr_t}$, which can be read as ``$\datalangExpr_t$ refines $\datalangExpr_s$ under the postcondition $\iPred$ and protocol $\iProt$''. Intuitively, it states that $\datalangExpr_s$ and $\datalangExpr_t$ behave in a related way, and eventually (both get stuck or) reduce to expressions in the relational post-condition $\iPred$. The parameter $\iProt$ is a \emph{protocol} that specifies ``external'' transitions, typically used to model function calls.

Traditionally this quadruple would be a judgment in a bespoke system of inference rules, proved correct by establishing a soundness or adequacy theorem. Our work instead uses the standard Iris approach of defining it directly as an assertion in a separation logic. The inference rules that we will provide are admissible lemmas, and they use separation logic as their meta-logic.

We defer the definition of the assertion as a simulation to \cref{sec:adequacy}, where it will also be proved adequate with respect to a notion of behavior refinement expressed outside Iris.

\section{Adequacy}
\label{sec:soundness}
\label{sec:adequacy}

(This paragraph below probably rather belongs to the introduction. Abort.)

The formal contribution of our work is a mechanized soundness proof for the TMC transformation on a simplified programming language, exhibiting the salient parts of the actual transformation for OCaml.
%
The destination-passing style used by the TMC transformation critically relies on uniquely-owned mutable state.
%
It is thus natural to use separation logic as our specification language.

\subsection{The quest for the right notion of simulation}
\label{sec:howto-relation}

The final theorem we want to establish is a behaviour refinement $\datalangProg_s \refined \datalangProg_t$: the behaviours of the transformed program are included in the behaviour of the source program.
%
We propose a notion of behaviour refinement that gives a form of total correctness.
%
It is termination-preserving, but also (this is less common in the body of work around Iris) safety-preserving and divergence-preserving.

We prove this using a backward simulation argument for our transformations $\datalangExpr_s \tmc \datalangExpr_t$.
%
A direct approach, ultimately unsucessful, would be to prove that the relation is in the simulation by induction, on $\datalangExpr_s$ for example.
%
This works well for all constructions except for function calls $\datalangCall{\datalangFnptr{\datalangFn}}{\datalangExpr}$: to reason on the relation between a call to $\datalangFnptr{\datalangFn}$ and its transformation (in direct or DPS style), we would need to inline the function definition, which is not structurally decreasing.

This is a common problem to establish simulations, and the solution is to use a form of co-induction.
%
In the Iris logic, a natural way to do this would be to use a ``later'' modality.
%
But defining simulations using a ``later'' modality is in fact non-trivial; simple definitions do not give the excepted notion of simulation, due to non-intuitive aspects of the step-indexing semantics of ``later'' in the Iris model. This problem is discussed in details in the previous work on Transfinite Iris~\citep*{TODO-transfinite-Iris}, an alternative logic with different axioms and models.
%
Sticking to the main Iris logic, a good definition of simulation has been worked out in Simuliris~\citep*{TODO-Simuliris}.
%
It avoids using the ``later'' modality by using the notion of coinduction native to Coq. In Simuliris, coinductive reasoning steps correspond to an ``abstract'' case in the simulation relation, used for function applications; an ``open simulation'' allows these abstract steps, while a ``closed simulation'' does not allow them.
%
The coinduction principle is encapsulated in a ``simulation closure'' theorem, which states if two terms are related in the open simulation, they are in fact also related in the closed simulation if we (coinductively) assume that foreign function bodies are correct.

Our proof is heavily inspired by the Simuliris work, but we cannot reuse their results directly because the notion of simulation that they provide is not expressive enough for our proof: we need to support transformations that change the calling conventions between source and target programs.
%
(Along the way we also simplified the simulation relation by removing proof rules related to concurrency, which we do not consider in this work.)

To support different calling conventions, we make our simulation relation $\iSim[\iProt]{\iPred}{\datalangExpr_s}{\datalangExpr_t}$ parametric over abstract protocols $\iProt$. This extends the technique of \citet*{TODO-paulo} from the unary setting of safety predicates to the binary setting of a relational separation logic.

\subsection{Proof outline}

Once we have decided to reuse and extend the Simuliris work, our argument can be split in three separate parts:

\begin{enumerate}

\item Define our notion of simulation parametric over protocols (calling conventions), prove its adequacy (it implies a behaviour refinement) and establish a ``closure theorem'' as in the Simuliris work.
%
  This technical contribution is independent of the TMC language or transformation, and could be reused in other works.

\item Build a relational program logic for our programming language, as a body of lemmas to establish simulations between expressions.

\item Use this program logic to establish the soundness of the TMC transformation.
%
The key ingredients are two specifications for the direct and DPS transformations that are proved in a mutually-inductive way, and are used to prove the hypothesis of the closure theorem of our simulation.
\end{enumerate}

The rest of this section expands each step of the proof, providing the necessary technical details to follow the Coq formalization.

relational separation logic = simulation

problem: direct calling convention in Simuliris, we need two calling conventions (direct + DPS)

remedy: simulation parameterized by protocol, generalization of Simuliris (but: sequential, without source safety but closed at toplevel)

% X_tmc := X_dir \uplus X_dps

% 1. when we are done
% 2. target stuttering (inductive)
% 3. all one-step source movees
%    3a. no source change (inductive)
%    3b. at least one source reduction (coinductive / guarded)
% 4. "open" simulation; X (\iProt) characterizes admissible foreign calls

% A trick from Simuliris:
% sim-after-progress: instance of sim-inner with bottom in well-chosen places
% closed simulation theorem: if the protocol X is "closable",
%   then the simulation relation for X "collapses" into a closed simulation (bottom for X, no case 4)
%
% MYSTERY: from a high-level perspective this "closd simulation theorem" feels a bit obvious:
% its hypothesis says that your protocol X can always be replaced by making progress on both sides,
% so it should be "easy" to transform a simulation proof using 1-2-3-4 into a simulation proof using only 1-2-3.
%
% Question: if it is "easy", cannot you do this right away (in the proof of simulation for a specific X that satisfies
% this property), proving a closed simulation (without 4) directly?
%
% Clément thinks that the answer is that this would require a super
% tricky coinduction, just like the one used in the proof of this
% theorem.
%
% Clément: this theorem is a sort of coinduction principle. You can
% prove an open simulation by induction (no coinduction
% apparently required), and then applying this theorem gives a result
% for which a direct proof would require a coinduction.

DPS calling convention

TMC protocol

% The X for TMC is essentially a relation/first-order rephrasing of
% the relation specs for direct-style and destination-passing-style
% transformations.

heap bijection

% I(sigma, sigma´) is as in Simuliris, not given explicitly in the current document.
% the \approx^bij relation refers to a component of I (a bijection between addresses)
% that is an implicit parameter of the definitions.

adequacy

% If two expressions refine internally (thy are in the
% simulation relation), and go to the internal equivalence between
% values, then they refine externally (they are in the denotational
% refinement relation).

% This is similar to the Simuliris proof, simplified thanks to the absence
% of concurrency.

simulation rules

% Note: those inference rules are in fact Iris propositions
% (the separating conjunction of the premises magic-wands the conclusion)

% Pure reductions: those that do not need state (neither write nor read)

% Gabriel: we could introduce the non-deterministic pair rules into the "pure" relation
% if we changed the pure-target rule to quantify over all e'_t such that \datalangExpr_t -{pure}-> e'_t.
% This would please Gabriel (existential on the left, universal on the right) and be more compact,
% but Clément prefers the current presentation -- it is more standard for Iris folks -- and closer
% to the mechanization.

% SimBijInsert: \approx^bij: Simuliris calls this the escaped/public
% locations, those that have been published.

% Clément now mentions Figure 12, which shows how stores evolve on an
% example in the TMC case. The shape of the relation between the
% source and target stores is sort of implicit in the program logic
% simulations rules, but it is clear on this example.
%
% Clément would explain this to give an informal sense of the proof,
% before even talking about specification logic etc.

closing simulation

proof


\begin{theorem}[Specification of direct transformation]
    \[
        \iRsimvHoare{
            \wf{\datalangExpr_s} \iSep
            \datalangExpr_s \tmcDir{\datalangRenaming} \datalangExpr_t
        }{
            \iSimilar
        }{
            \datalangExpr_s
        }{
            \datalangExpr_t
        }
    \]
\end{theorem}

\begin{theorem}[Specification of DPS transformation]
    \[
        \iRsimvHoare{
            \wf{\datalangExpr_s} \iSep
            (\datalangLoc, \datalangIdx, \datalangExpr_s) \tmcDps{\datalangRenaming} \datalangExpr_t \iSep
            (\datalangLoc + \datalangIdx) \iPointsto \datalangHole
        }{
            \lambdaAbs (\datalangVal_s, \datalangVal_t') \ldotp
            \exists \datalangVal_t \ldotp
            \datalangVal_t' = \datalangUnit \iSep
            (\datalangLoc + \datalangIdx) \iPointsto \datalangVal_t \iSep
            \datalangVal_s \iSimilar \datalangVal_t
        }{
            \datalangExpr_s
        }{
            \datalangExpr_t
        }
    \]
\end{theorem}

% Those two statements are proven together.
% (First by induction on \datalangExpr_s, then on the two transformation relations mutually-inductively.)

% Note: the \geq here is a "parametric" simulation, we assume that all
% free variables of \datalangExpr_s, \datalangExpr_t (they must be the same) are replaced by
% externally-equivalent values.

% Taking a step back: to prove the contextual refinements between programs,
% we have to check each function,
%   @f vs refined by @f vt  (for vs externally-equiv. vt)
% for this we can use the adequacy lemma
%   f vs simulated by @f vt (for the empty protocol) into internal value equivalence
% and for this we use the simulation closure theorem
%   f vs simulated by @f vt (for the Xtmc protocol)
% assuming the Xtmc protocol is valid/adequate.
%
% But then this is trivial, just use case (4) right away.
% So the meat of the proof is in showing that the Xtmc protocol is valid.

% To prove that the Xtmc propocol is valid:
% - in the direct case, we have a function application on both sides,
%   so we make a pure step on both sides and we have to show that the function bodies
%   (with a bisubstitution (x mapsto (vs, vt)) applied) are in the relation.
\section{\OCaml Implementation}
\label{sec:implementation}

résolution de l'ambiguïté de la transformation (règles et annotations)

\subsection{Implementation history}

\subsection{Design choices}

\subsection{Implementation}

\subsection{Evaluation: benchmarks}

\subsection{Evaluation: adoption}

% [@tail_mod_cons] usage on Github

% Search URL:
%   https://github.com/search?q=tail_mod_cons+lang%3Aocaml&type=code

% Results (November 13th 2023):
%   - in the OCaml stdlib
%     + Melange
%       (reused from stdlib)
%     + janestreet/base
%       (with manual unfolding)
%   - in other general utility modules
%     https://github.com/RedPRL/asai/blob/ac523674579772e5929c8d912d407b8b7db89c74/src/Utils.ml#L33
%     https://github.com/jonathan-laurent/KaTie/blob/2db0d2cf223fae003b26bdf4c82a9788b5804bc7/src/utils.ml#L95
%     https://github.com/jamsidedown/ocaml-cll/blob/394effa65b5d3f883e2de8322ed60fb7aa495bec/examples/crab_cups.ml#L61
%   - in other user code
%     at list type:
%       https://github.com/brendanzab/language-garden/blob/9cc21e2f57228556cf0b867ca2874ceb4a4250ec/compile-arith/lib/StackLang.ml#L63
%       https://github.com/abella-prover/abella/blob/ceee13822001137898ca5de9c76a315da3d1f0e3/src/extensions.ml#L421
%       https://github.com/bikallem/spring/blob/6d10fef0b21caea3f975ff13132f5994e7c37db5/lib_spring/response.ml#L99
%       https://github.com/bikallem/spring/blob/6d10fef0b21caea3f975ff13132f5994e7c37db5/lib_spring/headers.ml#L150
%       https://github.com/polytypic/io/blob/dbe4d37a98ef958178fe18bb4dae396a5537392e/src/io/io.ml#L8
%       https://github.com/avsm/eeww/blob/5b6c0617306f4c9d8b44bf07ef4d45ec9668dd0c/lib/cohttp/cohttp-eio/src/rwer.ml#L43
%       https://github.com/just-max/less-power/blob/96f8e88f72275246b6bc175e44c732aa6e08ce7f/src/common/path_util.ml#L18
%       https://github.com/dalps/ocaml-challenge/blob/8fe115023457b6b4359ef736c78ba980cd871717/3/extract/solve.ml#L4
%       https://github.com/Octachron/ocaml-changelog-analyzer/blob/d6757b9576b5070ea3cfe6d4f4f79aaa3cc6e5d4/parse/release.ml#L4
%       https://github.com/Skyb0rg007/PL-Reading-Group/blob/7002ae35ba69fb9010e5049eee88ab475bc79ec3/list-optimizations/lib/adt.ml#L44
%       https://github.com/jfeser/symetric/blob/f49a57afc90a2c1e38f1097f98bd74725c074b9b/lib/tower.ml#L75
%       https://github.com/verse-lab/sisyphus/blob/a08addae06bbccb1e6ea68bac3d7f9eeea4197be/lib/dynamic/utils.ml#L65
%       https://github.com/jaymody/ocaml-tokenizers/blob/06cbce75d2865690cb39017222efff5c284bc721/lib/utils.ml#L18
%       https://github.com/Bannerets/ocaml-kdl/blob/687c404ebeceef7fd905935c23665294133f99e6/src/lens.ml#L83
%       https://github.com/OCADml/OCADml/blob/f5dfbf55f9c19ad808daa0df4d76a55e58756e5f/lib/poly3.ml#L33
%     other type
%       https://github.com/johnridesabike/acutis/blob/4113fb516d9c5dd6f2dbfd9657f52c5ae7a5dcae/lib/matching.ml#L161
%         (on-demand stream as a record of functions)
%       https://github.com/RedPRL/ocaml-bwd/blob/fbf496b29532085b38073eaa62ba3d22ac619d5d/src/BwdNoLabels.ml#L60
%         (snoc list)
%       https://github.com/Skyb0rg007/PL-Reading-Group/blob/7002ae35ba69fb9010e5049eee88ab475bc79ec3/effects/lib/free/tseq.ml#L44
%         (difference list gadt)

%   - Misuse
%     https://github.com/chengsun/simd-sexp/commit/e714a8205c6df512d920a31a4dd2448975703c55
%       (already tail-recursive)
%     https://github.com/gridbugs/llama/blob/96d1c96c31ff136f465b5f37c981fff591bac6fd/src/midi/byte_array_parser.ml#L50
%       (needless complexity)
%     https://github.com/LimitEpsilon/lambda-interpreter/blob/ab8e81de7f896aa64eba14e23d964b9705e94161/lib/evaluate.ml#L46
%       (already tail-recursive)

\subsection{TMC and OCaml 5}

%%% Local Variables:
%%% mode: latex
%%% TeX-master: "main"
%%% End:
\section{Related work}

\subsection{TMC support in compilers}

Tail-recursion modulo cons was well-known in the Lisp community as
early as the 1970s. For example the REMREC system~\citep*{risch-73}
would automatically transform recursive functions into loops, and
supports modulo-cons tail recursion. It also supports tail-recursion
modulo associative arithmetic operators, which is outside the scope
of our work, but supported by the GCC compiler for example. The TMC
fragment is precisely described (in prose) in \citet*{friedman-wise-75}.

In the Prolog community it is a common pattern to implement
destination-passing style through unification variables; in particular
``difference lists'' are a common representation of lists with a final
hole. Unification variables are first-class values, in particular they
can be passed as function arguments. This makes it easy to write the
destination-passing-style equivalent of a context of the form
\ocaml{List.append li _?}, as the difference list
\ocaml{(List.append li X, X)}. In constrast, we only support direct
constructor applications. However, this expressivity comes at
a performance cost, and there is no static checking that the data is
fully initialized at the end of computation.

\subsection{Reasoning about destination-passing-style}

In general, if we think of non-tail recursive functions as having an
``evaluation context'' left for after the recursive call, then the
techniques to turn classes of calls into tail-calls correspond to
different reified representations of non-tail contexts, as long as
they support efficient composition and hole-plugging. TMC comes from
representing data-construction contexts as the partial data itself,
with hole-plugging by mutation. Associative-operator transformations
represent the context \ocaml|1 + (4 + _?)| as the number \ocaml|5|
directly. (Sometimes it suffices to keep around an abstraction of the
context; this is a key idea in John Clements' work on stack-based
security in presence of tail calls.)

\cite*{minamide-98} gives a ``functional'' interface to
destination-passing-style program, by presenting a partial
data-constructor composition \ocaml{Foo(x,Bar(_?))} as a use-once,
linear-typed function \ocaml{linfun h -> Foo(x,Bar(h))}. Those special
linear functions remain implemented as partial data, but they expose
a referentially-transparent interface to the programmer, restricted by
a linear type discline. This is a beautiful way to represent
destination-passing style, orthogonal to our work: users of Minamide's
system would still have to write the transformed version by hand, and
we could implement a transformation into destination-passing style
expressed in his system. \citet*{mezzo-2016}
supports a more general-purpose type system based on separation logic,
which can directly express uniquely-owned partially-initialized data,
and its implicit transformation into immutable, duplicable
results. (See the
\href{https://protz.github.io/mezzo/code_samples/list.mz.html}{List}
module of the Mezzo standard library, and in particular \ocaml{cell},
\ocaml{freeze} and \ocaml{append} in destination-passing-style).

\Xgabriel{TODO: Leijen and Lorenzen}

\subsection{Relational reasoning in separation logic}

% + ReLoC Reloaded: A Mechanized Relational Logic for Fine-Grained Concurrency and Logical Atomicity
% + A Higher-Order Logic for Concurrent Termination-Preserving Refinement
% + Transfinite Iris: Resolving an Existential Dilemma of Step-Indexed Separation Logic
% + Simuliris: A Separation Logic Framework for Verifying Concurrent Program Optimizations

\subsection{Protocols}

% + Compiler Verification Meets Cross-Language Linking via Data Abstraction
%   axiomatic semantics are close to protocols in spirit, they include in the
%   syntactic reasoning rules an "axiomatic" step that non-deterministically
%   reduces a function call following a (P,Q) pre/post
% + A Separation Logic for Effect Handlers
%   we were inspired by Paulo: Χ allows do v { Φ }
%     + specification of an effect
%     + protocol Χ allows the program to perform the effect v and garantees that every permitted reply satisfies postcondition Φ
%     + "we would like programmers to think about performing an effect essentially in the same way as they think about calling a function"
%     + wp parameterized by protcol: ewp e < Χ > { Φ }
% + Melocoton: A Program Logic for Verified Interoperability Between OCaml and C


%%% Local Variables:
%%% mode: latex
%%% TeX-master: "main"
%%% End:

\section{Conclusion and Future Work}

concurrency

effect handlers

compression of constructors

prove other program transformation using this framework (CPS)

% To give another example of simulation with a different protocol,
% we could look at accumulator-passing-style transformations
% (tail-modulo-context for associative arithmetic operators). Prove
% that `factorial` is equivalent to the tmc-optimized factorial
% (two versions, one in accumulator-passing-style and the "direct
% style" version that calls it with accumulator 1).
%    .. et donc il faut garder les entiers!
%
% Clément: j'ai commencé à bosser sur ça. Ça demande de travailler
% sur des expressions plutôt que des valeurs (l'accumulateur peut
% être un entier mais aussi une variable). Il faut regénéraliser
% des bouts du développment Coq qui était spécialisé aux
% valeurs. Il se passe un truc intéressant sur les échecs si
% l'accumulateur est une variable qui est substituée par un
% non-entier (échec pendant le calcul d'un côté, après le calcul
% de l'autre).
% La formulation où on accumule des opérations sur des expressions
% arbitraires et pas juste des constantes entières est un peu plus
% générale que celle de Daan Leijen (que des constantes). Elle permet
% de faire une transformation APS sur List.sum.

% Missing sections ?
% TMC in OCaml, for users
% TMC in OCaml, for implementors
% TMC in OCaml, evaluation (benchmarks + usage feedback)
%   note: OCaml 4 vs. OCaml 5

% TODO: decider si on clame "TMC dans OCaml" comme une contribution
% TODO: décider qui sont les auteurs (Gabriel ? Frédéric ?)
% TODO: indiquer qui a fait quoi
% ------------------------------------------------------------------------------

% Propositions sur la grammaire:
% - les constructeurs sont des valeurs
% - primitive d'égalité sur les constructeurs

% vu sim. sim-innerX(sim)


% unstructured TODO list 

% - To give another example of simulation with a different protocol,
% we could look at accumulator-passing-style transformations
% (tail-modulo-context for associative arithmetic operators). Prove
% that `factorial` is equivalent to the tmc-optimized factorial
% (two versions, one in accumulator-passing-style and the "direct
% style" version that calls it with accumulator 1).
%    .. et donc il faut garder les entiers!
%
% Clément: j'ai commencé à bosser sur ça. Ça demande de travailler
% sur des expressions plutôt que des valeurs (l'accumulateur peut
% être un entier mais aussi une variable). Il faut regénéraliser
% des bouts du développment Coq qui était spécialisé aux
% valeurs. Il se passe un truc intéressant sur les échecs si
% l'accumulateur est une variable qui est substituée par un
% non-entier (échec pendant le calcul d'un côté, après le calcul
% de l'autre).
% La formulation où on accumule des opérations sur des expressions
% arbitraires et pas juste des constantes entières est un peu plus
% générale que celle de Daan Leijen (que des constantes). Elle permet
% de faire une transformation APS sur List.sum.

% Gabriel: should we consider adding a primitive that reduces non-deterministically to any integer?
% This can be a model of a "read from the standard input" primitive, without having to deal with I/O in the operational semantics.
% (And: it should be very easy to add as you already support non-determinism.)
%
% The context for this suggestion was to consider reducing the correctness of any function (say f : Nat -> Nat) to just
% "main" functions with no arguments, by reading the argument from the standard input instead.

\hbadness=10000
\bibliography{english}

\end{document}
